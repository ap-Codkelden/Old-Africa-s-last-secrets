\documentclass[12pt,a4paper,twoside,openany,svgnames]{memoir}
\usepackage[utf8]{inputenc}
\usepackage{cmap}
\usepackage[T2A]{fontenc}
\usepackage[russian]{babel}
% переносы 
\tolerance 1414
\hbadness 1414
\emergencystretch 1.5em
\hfuzz 0.3pt
\widowpenalty=10000
\vfuzz \hfuzz
\raggedbottom

\usepackage[top=2cm, bottom=2cm, left=2.6cm, right=2cm, a4paper]{geometry}
\usepackage{import}
\usepackage{graphicx}
\usepackage{setspace}
\usepackage{float}
\usepackage{lipsum}
\usepackage{xcolor}
\usepackage{parskip}
\usepackage{textcomp}
\usepackage[protrusion=true]{microtype} % Висячая пунктуация
\usepackage[labelformat=empty]{caption}

\makeatletter
\newcommand*{\rom}[1]{\expandafter\@slowromancap\romannumeral #1@}
\makeatother


\author{Лоуренс Грин} %\thanks{\href{mailto:my@alv.me}{my@alv.me}}}
\title{Последние тайны старой Африки}

\date{2013}

\begin{document}

\maketitle

%\clearpage
\begin{figure}[ht!]
\centering
\includegraphics[width=120mm]{000000.jpg}
\label{overflow}
\end{figure}

\cleardoublepage


\chapter{Крупинка правды}

Еще в школьные годы очертания Африки всегда напоминали мне вопросительный знак. Я мечтал побывать на Занзибаре и в Лагосе, в Каире и на развалинах Зимбабве, потому что уже тогда я был заворожен географическими названиями и атмосферой неизвестности, которую таила в себе карта Африки. Никакой другой континент не представлял для меня столько очарования. И даже теперь, повидав мир, я по-прежнему отдаю предпочтение Африке.

Я знаю Африку лучше, чем любой другой континент, и это помогает мне понять те ее уголки, где я чувствую себя как дома.

Нередко складывается впечатление, что Африка раскрывает исследователю одну из своих загадок лишь для того, чтобы столкнуть его с еще более глубокой тайной. Если вы, как и я, родились в Африке, то вам должно быть знакомо это чувство. Я предпочитаю искать разумное объяснение тайнам и вполне понимаю, что если ключ к некоторым из них будет найден при моей жизни, то другие так и останутся не разгаданными до конца моих дней.

В настоящее время считают, что Африка~--- колыбель человечества. По найденным черепам и костям ученые пытаются воссоздать картину первобытного мира. Как они самонадеянны в своих аргументах и утверждениях! А ведь это одна из величайших загадок, которая вполне может оказаться выше человеческого понимания.

Впервые с останками первобытного человека я столкнулся в ранней юности. Мой отец работал тогда редактором выходившей в Кимберли газеты, и мы жили в поселке Александерсфонтейн, расположенном на окраине города алмазов. По субботам все жители Кимберли устремлялись к этому оазису, чтобы провести время за чашкой чая на берегу красивого пруда. <<Де Беерс>>, компания по добыче алмазов, держала там гостиницу, которая была ей в убыток, но служила отдушиной для жителей города. Не будь этой гостиницы, люди в Кимберли могли бы потерять рассудок от духоты и пыли. Местом моих игр был берег высохшего озера неподалеку от поселка, и там я находил каменные ножи и много других предметов из бурой и желтой яшмы.

В те дни мало кто задумывался над такими находками. Гораздо интереснее были плавающие в пруду лебеди. Мне вспоминается поездка на стареньком <<Ланчестере>> на одну ферму, где на гладко отполированных ледниками скалах я видел изображения носорогов и слонов, высеченные первобытными художниками. Тогда я, кажется, подумал только о том, что сам смог бы сделать более вразумительные зарисовки в своем школьном альбоме.

В районе Кимберли сохранились еще настоящие бушмены, хотя они и не занимаются больше охотой. Некоторые даже стали работать на алмазных рудниках. Когда я был совсем маленьким, меня возили по рудничным компаундам, и я научился узнавать бушменов по их морщинистым лицам и пучкам курчавых волос на голове. Там же я встречал людей из племени корана~--- чистокровных готтентотов, теперь почти вымерших. Эти высокие, сильные и добродушные люди, страстно влюбленные в танцы, гибли постепенно от вина. Много лет спустя я снова прошелся по компаундам алмазных рудников и понял, что история некоторых народов Африки представляет собой чистые страницы в истории человечества. О происхождении многих из них ничего не известно. Их считают дикарями, однако они обладают такими качествами, которым могут позавидовать и цивилизованные народы.

Нам часто приходится слышать о тайнах Востока, но, на мой взгляд, ничто не может сравниться с загадками Африки. Где еще, кроме Африки, вы найдете на одном конце сфинкса, на другом~--- Зимбабве и тысячи всяких тайн по всему континенту? Я с трепетом смотрел на вершину Эвереста, сверкающую в лучах восходящего солнца, но вот что я страстно хочу увидеть снова, так это Килиманджаро, кратер Килиманджаро с его загадками, высочайшую вершину Африки. Как мало мы знаем о горных вершинах! Всего лишь десяток лет назад самым высоким пиком в Южной Африке считался Монт-о-Сурс. И вот оказалось, что топографы ошибались. В Басутоленде были открыты две более высокие вершины.

Индия славится своими замечательными факирами, но в Египте я встречал магов, намного более искусных, чем их индийские собратья. Если вас интересуют заклинатели змей, скажу, что в Каире вы найдете более способных заклинателей, чем в Мадрасе и Калькутте. Подполковник
Р.~ Г.~Элиот, служивший хирургом в Индии, занимался исследованием различных видов магии в Африке и в Индии. Он состоял членом Общества магии в Лондоне и сам был отменным факиром. Как-то в конце своей карьеры он заявил, что ему приходилось наблюдать множество фокусов, которые противоречили сложившимся у него взглядам, но что ни в одном из них законы природы не нарушались по прихоти бродячего исполнителя. Наиболее сенсационные представления, добавил Элиот, показывали арабы, выходцы из Африки. Некоторые из них насквозь прокалывали себе шею около позвоночника, не повреждая при этом ни кровеносных сосудов, ни нервов. И в этом не было никакого обмана. Элиот делал рентгеновские снимки, которые показывали, что круглая рапира действительно проходила рядом с позвоночником. Кроме того, втыкали острые кинжалы себе в голову. <<Это была не магия, и мне пришлось пересмотреть свои представления о человеческих возможностях,~--- заключил Элиот.~--- Человек везде остается человеком>>.

Мой каирский гид уверял меня, что так называемый индийский фокус с веревкой часто показывали в Египте, но теперь, кажется, такие фокусники вымерли. Вы, наверное, помните, как фараон <<велел созвать всех магов Египта, всех египетских мудрецов и поведал им свой сон>>. Египет~--- родина магии. О заклинателях змей упоминается в Священном писании. Монах Клеменс, который жил у подножия горы Синай, описал фокус с плодом манго, исполнявшийся чародеем еще во времена апостолов. Ибн-Батута, впервые описавший фокус с веревкой, был родом из Африки. Этот властелин из Танжера совершил в начале четырнадцатого века большое путешествие, добравшись до Китая. И он утверждает, что именно там (а не в Индии) он видел этот фокус, который превратился в неумирающую легенду~--- легенду, в которую каждому хотелось бы верить. Современные историки считают, что рассказы Ибн-Батуты глубоки и достоверны. Интересно, что Батута приводит многие подробности, о которых мы слышим и теперь: веревка разматывается сама по себе и поднимается вверх; по ней лезет мальчик, которого преследует волшебник; на землю падают отдельные части тела мальчика.

<<Я был поражен до такой степени, что у меня началось сильное сердцебиение,~--- писал Ибн-Батута.~--- Мне дали чего-то выпить, и я пришел в себя>>. Затем Батута услышал, как кто-то из зрителей произнес: <<Бог ты мой, ведь никто ' не взбирался и не спускался по веревке и не разрезал мальчика на части. Это всего-навсего фокус>>. Таким образом, этот оригинальный текст наводит на мысль о гипнозе. Такого мнения придерживаются многие исследователи. Другие же предполагают, что во время некоторых представлений используют жаровню, куда подкладывается какое-то зелье.

Элиот вообще в это не верит. <<Не вижу оснований предполагать, что этот фокус с веревкой пробовали когда-нибудь исполнять всерьез,~--- заявил он.~--- На распространенных ``фотографиях'' индийского фокуса с веревкой в действительности изображен хорошо известный акробатический трюк. Мужчина там держит тяжелый бамбуковый шест, и на верхнем его конце балансирует мальчик. Обратите внимание на кольцевые утолщения на бамбуковом шесте, когда вам еще раз покажут эту фотографию>>.

Говорят, что королева Виктория предлагала две тысячи фунтов стерлингов тому, кто покажет фокус с веревкой. Английский иллюзионист Джон Маскелин увеличил награду до пяти тысяч, а лорд Лонсдейл~--- до десяти. Но попытки вице-королей, принцев и прочих влиятельных особ отыскать какого-нибудь исполнителя фокуса с веревкой не увенчались успехом. Им удалось найти лишь произносящую приветствие утку (фокус производился с помощью тонкой нити, привязанной к ноге исполнителя) да примитивный трюк с плодом мангового дерева~--- жалкий пример обмана при помощи куска материи.

Так что в наши дни факты опровергают фокус с веревкой. И тем не менее Ибн-Батута~--- этот заслуживающий доверия старец из Африки~--- видел этот фокус. Я не верю, что все это выдумка, история, которую он сам сочинил, чтобы приукрасить свое повествование. Не такой он человек. Здесь должна быть крупинка правды. Ибн-Батуту заставили что-то увидеть, и по прошествии шести веков мы все еще стараемся представить, что же в действительности увидел Батута и как был проделан этот фокус. И теперь мне хотелось бы заявить следующее. Этот фокус исполнялся во многих странах, и он с таким же правом может быть назван африканским, как и индийским.

В официальных протоколах Танганьики есть ключ к разгадке этого удивительного фокуса. Как-то к одному чиновнику из окружного управления пришли расстроенные африканцы. Их племя постигло большое несчастье: по повелению колдуна был повален священный баобаб, и, если он не поднимется, все племя погибнет.

Когда чиновник приехал на место происшествия, он увидел старейшин племени, горестно сидящих вокруг целого и невредимого баобаба. Они были уверены, что дерево повалилось, и никакие уговоры не могли их в этом разубедить. Тогда чиновник поговорил с глазу на глаз с колдуном, угрожая повесить его на этом самом баобабе, если тот его не <<поднимет>>. После долгих уговоров колдун согласился. Был разложен костер, и в жертву принесли козла. С криками изумления и радости старики наблюдали, как <<упавшее>> дерево вновь занимает прежнее положение. Это был явный случай массового гипноза, захватившего всех присутствующих, за исключением чиновника. Как я уже говорил, случай этот зафиксирован в официальном протоколе. Человеку, способному совершить подобную вещь, ничего не стоит убедить немногих доверчивых зрителей в том, то они действительно видели фокус с веревкой.

Я люблю тайну, но она не дает мне покоя до тех пор, пока я не докопаюсь до ее основ и не найду какого-нибудь правдоподобного объяснения. По-видимому, не все к этому так относятся. Многим нравится сверхъестественное, и они хотят, чтобы тайна оставалась тайной. Это вполне разумно, когда речь идет о произведении искусства, например о загадочной улыбке Моны Лизы. Один французский художник как-то заметил: <<Когда вы даете определение или объяснение картины, вы подменяете действительность словами. И тайна ускользает от вас. Чтобы тайна сохраняла свою силу, ее нужно уважать>>.

Такие понятия, как бесконечность или происхождение жизни, слишком сложны для человеческого разума, и я предпочитаю их не касаться. Но во многом другом я проявляю любознательность. Меня интересует судьба пропавшего без вести путешественника, корабля, не вернувшегося из плавания, летчика, который отправился через пустыню и не оставил видимых следов своей гибели. Мои любимые тайны не выходят за пределы человеческого понимания. Это <<волшебство>>, которое поддается терпеливому исследованию.

Путешествуя по Африке, я натолкнулся на факты, свидетельствующие о существовании внечувственно-го восприятия, телепатии и ясновидения у первобытных народов. Врачи, долго проработавшие в тропической Африке, согласятся, что некоторые африканцы обладают такими сведениями в области медицины, каких пока еще нет в учебниках белых. Известно немало случаев, когда африканские колдуны использовали свои гипнотические способности во зло человеку, не останавливаясь даже перед смертью жертвы.

И однако, принимая во внимание все вышесказанное, я все же не считаю гипноз магией. Несмотря на все мое пристрастие к тайне, я не склонен верить тому, что первобытный африканец может повлиять на погоду. Вы со мной согласны? И тем не менее такой проницательный и трезвый исследователь, как Джон Гантер, заявляет в своем объемистом сочинении по Африке, что дождь, выпавший в 1922 году в пораженной засухой долине реки Замбези, был следствием магии. <<Как можно объяснить подобные явления?~--- спрашивает Гантер.~--- Стечением обстоятельств? Возможно>>. Я считаю, что это~--- чистое стечение обстоятельств, без всяких <<возможно>>. Я помню историю с королевой дождя и расскажу обо всем позже. Гантер упустил из виду последующие события, а они более поразительны, чем дождь в самый драматический момент.

Точку зрения Гантера разделяют многие образованные люди. Они относятся скептически к случайным стечениям обстоятельств и не отбрасывают полностью магию. <<Стечение обстоятельств? Возможно>>.

По ночам я часто размышлял о том, как много значат для загадок Африки жара и состояние атмосферы. Обычно у меня везде хороший сон, и живу я на юге Африки с его умеренным климатом. Но в моей жизни бывали моменты, когда жара становится материальной и угнетающей силой, она преследует вас днем и душит по ночам. Из месяца в месяц двенадцатичасовой палящий зной, из месяца в месяц бессонные ночи. Все это приводит к психическим расстройствам, которые во французских колониях известны под названием <<ка-фар>>. Мне самому пришлось испытать немалые страдания во время летней поездки в Египет, и я могу судить об этом странном состоянии по собственному опыту. Воображение мое было так перевозбуждено, что я не мог положиться на свою память, хотя в обычном состоянии она у меня отличная. Приходилось пользоваться записной книжкой. Никогда прежде я не испытывал ничего подобного. Но как только я вернулся в здоровый климат, я сразу же исцелился. Это одна из моих личных загадок, которую я не мог решить. На тайну можно натолкнуться и в царстве льда, но, видимо, африканский зной усиливает напряжение, и тайна начинает сверкать и пульсировать.

Однажды в южных районах Конго засуха унесла полмиллиона жизней. Ни в памяти живущих, ни в преданиях племен не сохранилось воспоминания ни об одном годе, когда жара так бы безжалостно терзала землю. Летом 1890 года великая засуха была в самом разгаре, и в это время английский охотник на слонов Гарольд Мартин пересекал район бедствия, отчаянно пытаясь разыскать воду. Мартину пришлось зарыть в землю три тонны слоновой кости, так как местные носильщики отказывались уходить из своих деревень. На много миль вокруг воздух был пропитан запахом разлагающихся трупов животных. Повсюду, словно грозовые тучи, носились огромные стаи грифов. У Мартина и его носильщиков почти кончался последний калебас воды, когда они остановились на берегу высохшего озера в том месте, куда слеталась масса грифов.

Теперь это была выжженная солнцем впадина с остатками влаги на дне. В зеленой тине мучительно умирали восьмидесятифунтовые усачи, которых пожирали сотни уже объевшихся крокодилов. Повсюду валялись трупы антилоп. Деревья были с корнем вырваны умирающими от жажды слонами. Мартин подстрелил слониху, которая превратилась просто в скелет. Такой Африки он еще никогда не видел. Мартин выжил и смог поведать эту историю только благодаря одному из своих носильщиков, зулусу, который вдруг упал на колени в сухом русле реки и воскликнул: <<Я вижу воду>>. Он принялся копать землю и обнаружил воду~--- целые галлоны свежей воды, которой можно было наполнить опустевшие калебасы. Что это? Простое совпадение? Думаю, что зулус действительно увидел воду сквозь толщу сверкающего на солнце песка. Наблюдательность и природное чутье выросшего среди вольных просторов человека~--- вот что дало ему возможность <<увидеть>> воду по какому-то едва уловимому признаку, который и сам зулус, возможно, не в состоянии объяснить. Быть может, по запаху. Но уж никак не с помощью магии.

В период между двумя мировыми войнами я познакомился в Лондоне со старым охотником Мартином. Казалось, что расстояние от Пикадилли до Португальской Восточной Африки только способствовало его воспоминаниям. Несколько десятилетий бродил Мартин в поисках слоновой кости по английской, бельгийской и португальской Африке, и теперь он рассказал мне множество историй, которые всякому, кто никогда не сталкивался с жизнью местных следопытов африканского буша, показались бы сверхъестественными.

Мартин во всем искал разумного объяснения. Он верил, что опытный африканский следопыт не может заблудиться. Такие люди, говорил он, не только помнят ориентиры, но часто способны мысленно представить себе лежащую перед ними местность, где они никогда до этого не были. Чутье отводит их от болот и труднопроходимых мест. И им удивительно помогает воображение.

Один из следопытов Мартина мог идти по следам слона через обширные каменистые пространства. Эта способность объяснялась его превосходным зрением.

Сам Мартин не мог разглядеть отпечатков ног, пока следопыт не посыпал каменную поверхность золой из костра, а делал он это не менее тонко, чем специалист по дактилоскопии. Однажды этот следопыт очень уж удивил Мартина. Во время погони за слоном он вдруг остановился и произнес: <<Бвана, нет смысла идти дальше~--- этот слон без бивней>>. Мартин не поверил ему, и погоня продолжалась. Наконец, после утомительного перехода, они увидели слона~--- старого самца без бивней.

Другой охотник не стал бы искать настоящего объяснения, и тогда бы еще одна тайна Африки заняла свое место в царстве магии. Но Мартин уговорил своего следопыта открыть ему секрет. По пути слон терся головой о ствол дерева, и там не осталось следов бивня. Но следопыт не стал говорить об этом Мартину, ведь у слона мог быть и один бивень. Позднее, проходя мимо махогониевого дерева, слон потерся об него другой стороной. И снова на коре не осталось никаких следов бивня.

Этот же следопыт выслеживал однажды людей, похитивших слоновую кость. Следы привели его к мелкой речушке, куда свернули воры. Чтобы избежать преследования, они шли по воде. Однако следопыт справился и с этим. Он уверенно пошел вверх по течению, точно указал место, где беглецы вышли из воды, и привел Мартина туда, где была спрятана похищенная слоновая кость. Сделано это было необычайно искусно, и тем не менее все объяснялось очень просто. Ступая по дну реки, грабители перевернули много камней. Как видите, никакого волшебства~--- лишь очень тонкая наблюдательность.

Зная мое пристрастие к тайнам, мне часто сообщают о них в письмах или приходят с ними прямо ко мне домой. К разряду великих тайн, которые особенно любит читатель, относятся тайны, связанные с исчезновением людей. Причем иногда они только считаются погибшими, а на самом деле оказываются живыми. В Африке есть свои полковники Фосетты~--- люди, которые отправились на поиски сокровищ, затерянных городов, пропавших без вести путешественников или неизвестных науке животных. Они уехали в Африку и исчезли без следа. И с каждым годом растет очарование неразгаданной тайны. Эти люди нашли там свою смерть, но романтически настроенная публика предпочитает верить слухам, что они живы.

Африка~--- прибежище многих загадочных личностей, которые хотели бы скрыть свое прошлое. Она дает приют и множеству любителей приключений, которым как дом родной ее вельд, леса и пустыни. <<Я не мог больше выносить Англию, мне хотелось отправиться туда, где я когда-то жил,~--- к дикарям и диким животным~--- и там умереть>>,~--- восклицает Аллан Куотермейн. И в жизни находятся люди, которые все еще идут по стопам героя романов Райдера Хаггарда.

К числу таких людей принадлежал полковник У.~Д.~Э.~Грант из Девоншира. Грант нашел в Африке все, что искал, а у него были весьма странные причуды. Дом его был набит чучелами полярных медведей и головами моржей. Его имя вы найдете на картах Арктики. Но самая странная вещь в его доме~--- забальзамированный ястреб, которого он приобрел в конце прошлого столетия в Египте. Говорили, что при приближении войны из груди этой птицы сочится кровь. Грант сам уверял меня, что темная жидкость дважды проступала на груди ястреба~--- незадолго до начала англо-бурской войны и в августе 1914 года. Грант не верил в магию и появление жидкости объяснял воздействием температуры на вещества, использованные египтянами при бальзамировании птицы.

Самая странная, на мой взгляд, экспедиция Гранта была предпринята им в конце прошлого века по просьбе одной французской графини. Ее непутевый сын отправился в Кейптаун и исчез. Через некоторое время появились смутные слухи, будто бы он умер в Родезии. Гранта просили выяснить истину и в случае, если сообщения окажутся верными, привезти тело на родину. Графиня была согласна на любые расходы.

После нескольких недель тщательных поисков Грант выяснил, что молодой француз записался в британскую полицию в Южной Африке и был направлен в Гвело. Он умер в лагере в пятнадцати милях от Гвело и там же был похоронен. Грант отыскал в буше могилу и вскрыл ее в надежде опознать труп.

Труп был опознан. Француз умер от так называемой родезийской лихорадки. В теле его содержалось значительное количество алкоголя, который сыграл роль бальзама. Грант положил труп в новый свинцовый гроб и перевез в Гвело. Почтовая компания отказалась доставить гроб в Булавайо, где была железнодорожная станция. Из-за эпидемии чумы, поразившей крупный рогатый скот, Грант не мог нанять и воловьей упряжки. Отличаясь изобретательностью, он приобрел упряжку из коз и смастерил телегу на четырех велосипедных колесах. Прибыв в Булавайо, Грант поместил гроб в большой ящик со штампом: <<Горное оборудование>>. Когда гроб был наконец водворен в фамильном склепе во Франции, Грант счел, что его миссия закончена. Но это еще не был конец истории. В начале первой мировой войны немецкая бомба разрушила склеп. Грант воспринял это как возмездие за свою попытку вырвать у Африки тайну, чтобы удовлетворить прихоть матери-графини.

Африка уже раскрыла много своих загадок. Перестали быть тайной истоки Нила. Нанесены на карту Большие озера и Лунные горы. И все же немало странных загадок таит еще экваториальный лес. Пустыни прячут и раскрывают следы прежних трагедий, а движущиеся пески несут удивительные звуки, если вы умеете слушать, приложив ухо к туго натянутой коже африканского барабана.

Особенно хорошо чувствуешь Африку на больших реках. Я люблю речные путешествия. Они напоминают поездку по железной дороге, но без неприятных рывков поезда. А когда судно останавливается, у путешественника есть время ознакомиться со многими любопытными прибрежными уголками и с людьми.

Мне вспоминаются пароходики, на которых я пробирался через самое сердце Африки. На верхней палубе белые пассажиры. Салон на открытом воздухе, столики расставлены вокруг дымовой трубы. По ночам она озаряет темноту тусклыми красными отблесками. Изо дня в день судно, словно пьяное, тащится своим зигзагообразным курсом. Это рулевой избегает встреч с банками и отмелями. Я, как сейчас, чувствую все запахи и звуки. Запах костра и запах песка. Вижу бегемотов и крокодилов, долбленые челноки. Черные, как смоль, тела с сетями или со щитами и копьями. Отдаленные звуки барабанов~--- непременный атрибут Африки, <<телеграф буша>>, звучащий по всему континенту от

Занзибара до Бомы. Треск ярких костров и шум судовых машин. Частоколы из бамбука, грифы на дереве, а под его кроной африканские девушки, перемалывающие зерно. Бесконечная панорама старой Африки. И опять все тот же салон под открытым небом, где я прислушиваюсь к бормотанию реки и к рассказам о малоисследованных странах.

На почтовых пароходах из Конго пожилые люди в пробковых шлемах ведут беседы об африканских животных. Они хорошо знают то, о чем говорят. Кое-что из этих разговоров прозвучало бы ересью для зоологов, особенно для тех скептиков, которые любят посмеяться над легендами об африканских животных. Однако земля, где в начале этого века было обнаружено несколько новых видов крупных млекопитающих, может таить и другие неожиданности в своих лесах и болотах. Столетие назад, объясняя некоторые из старых тайн, Уинвуд Рид писал: <<Люди с собачьими головами, о которых рассказывает Геродот,~--- это лающие бабуины. Сирены из арабских сказок~--- африканские ламантины, грудь которых напоминает женскую, а морда~--- человеческое лицо. Огромный змий, остановивший армию Регула,~--- не что иное, как питон>>. Мне нравится такое разрешение тайн, но Рид стоял лишь на пороге области, которая и теперь еще до конца не исследована.

Одни тайны поддаются разрешению, другие~--- нет. Когда я был мальчишкой, родители подарили мне очаровательную книгу под названием <<Распространенные заблуждения и их толкование>>. Эта книга избавила меня от многих наиболее глупых и опасных предубеждений. Я узнал, что человек, падающий с большой высоты, не умирает, пока не коснется земли; что нельзя доверять грибу только потому, что он легко чистится; что пшеница, найденная в египетских гробницах, не прорастает. Из той же книжки я вычитал, что экипаж судна <<Биркенхед>> при гибели вовсе не стоял по команде <<смирно>>. С тех пор прошло полвека, а я все еще продолжаю доискиваться до сути странных историй, с которыми меня сталкивает судьба.

Я разрешил несколько загадок ради собственного удовлетворения. Когда люди, знавшие правду, умирают или бесследно исчезают, остается только строить гипотезы. Мне понравилась история, случившаяся во времена первых европейских поселенцев в Северной Родезии, когда из-за виски, находившегося в купе проводника, был украден целый пассажирский вагон. Как можно украсть вагон? Можно подвести к дороге ветку через буш, а затем снова ее разобрать. Но те, кто затевал ограбление, к сожалению, не придут ко мне домой, чтобы восполнить недостающие детали.

Неподалеку от побережья Западной Африки есть заросший пальмами остров, где можно провести остаток жизни и не встретить ни одного знакомого прежде человека. На этом острове высадился англичанин, обладавший исключительной памятью на лица. И там, в лавке, он встретил свою соотечественницу, которую разыскивали в Лондоне по обвинению в отравлении своего мужа. Была ли она виновна? Это еще один утерянный след, и я никогда не узнаю правды.

Африка~--- земля пропавших следов, пропавших людей, пропавших городов и сокровищ, пропавших без вести самолетов. А вдоль побережья вы можете увидеть останки погибших кораблей. Легенды о сокровищах долго не умирают, в Африке же они обычно возрождаются снова. Во всех уголках огромных пустынь Северной Африки вы можете услышать о погребенных под дюнами сокровищах. Их уже так много было найдено, что не остается сомнений в существовании других, еще не раскрытых богатств. Когда я был в Египте, мне приходилось слышать немало подобных историй, рассказанных людьми, которые сами вели раскопки могил и пополняли музеи новыми экспонатами.

Однажды из Ливийской пустыни в Каир пришел араб и принес под своей джеллабой кусок металла. Один купец на базаре предложил за него такую цену, что араб согласился с радостью. <<Если у тебя есть еще эти медяшки, я дам тебе за них на пиастр больше рыночной цены>>,~--- сказал купец арабу на прощанье. Араб приходил к купцу еще не раз и приносил с собой по изломанному куску металла. Товар взвешивался и продавался. Когда араб принес наконец последний кусок, купец сложил обломки вместе и получил статую человека в натуральную величину~--- статую из золота, которую нашел в пустыне простой араб на месте развалин римского города. Это сокровище не в пример многим другим попало в Каирский музей.

Каир и Кейптаун~--- два африканских города, которые я хорошо знаю. Могу сказать, что Кейптаун я знаю гораздо лучше. Но в Каире у меня была подруга, которую я называл Шахразада. Нельзя сказать, чтобы ей можно было уж очень доверять, но она была умна и забавна, и к тому же отличная собеседница. Мы бродили с нею по сомнительным уголкам Каира, вдалеке от дворца Каср-эль-Нил, и по лабиринтам узких переулков. Улицы всегда меня увлекают. Их надо исходить пешком, чтобы хорошенько узнать. Старые улочки Каира, всегда людные и грязные, казались такими же оживленными и романтичными, какими они были в те времена, когда легендарная Шахразада рассказывала свою тысячу сказок.

Так я пытался впитать мудрость Египта. Я постигал ислам Африки в кварталах Аль-Азхара, этого шумливого мусульманского университета, над которым пронеслись века и где вновь оживает древний Каир. Я мог вдоволь наслаждаться бесплатным кофе на улице ювелиров, в окружении бокалов и сосудов для благовоний, среди блюд, инкрустированных золотом и серебром. Я слушал, как поют ковровщики за работой, и в закоулках Моски изучал ремесла и\ldots людей.

Там мне встретился исцелитель шейх Ибрагим. Я часто видел, как он бродил по улицам, выкрикивая <<ин-шалла аль-хамаля ва метвалли>>. Это он просил мусульманского святого отвести болезни. У него была крохотная <<приемная>>, где он лечил своих больных. Пристальный взгляд и успокаивающий голос составляли два основных секрета врачебного искусства шейха. Я почти дословно могу перевести поток тщательно подобранных им арабских фраз: <<Ваши глаза устали\ldots они закрываются\ldots вы отдыхаете\ldots головная боль исчезает\ldots боль не вернется>>.

Иногда мы с Шахразадой заходили в кафе в квартале Аль-Азхар, где вдыхали аромат тушеного барашка, приготовленного по-египетски~--- с персиками. Там подавали лучшие деликатесы: превосходнейшие плоды манго, выращенные египетскими пашами, грозди сладчайшего белого винограда, финики из оазиса Сива, ароматные арбузы, фаршированные рисом баклажаны, шашлыки. Я хотел бы знать Египет так же, как знал египетскую кухню.

Мой знакомый из полиции возил меня по Ваг-аль-Бирка-огромному кварталу красных фонарей, совсем по соседству с районом Шефердса, но без его романтики. Я заглядывал в публичные дома, откуда доносились звуки музыкальных автоматов и резкий смех женщин. Видел кафе, которые посещают содержатели притонов и торговцы наркотиками. Был свидетелем многочисленных убийств. Мой друг из полиции рассказывал мне об убийствах высокопоставленных чиновников в те ужасные времена, когда Каир пестрел оранжевыми афишами, обещавшими вознаграждение до десяти тысяч фунтов за голову неизвестных убийц. Он рассказывал о вымогателях и фальшивомонетчиках, о контрабандистах и порочных гидах, о раскрытых и нераскрытых преступлениях. В период между двумя мировыми войнами, когда в Египте особенно процветала торговля наркотиками, некоторым крупным торговцам были проданы отпечатки их пальцев из полицейского управления. Это давало им возможность скрыть прежние судимости.

Странные в Египте встречаются люди и с необычными занятиями. В одной деревне, которая называется Бирма, жители с незапамятных времен занимаются искусственным выведением цыплят. Я думаю, что это их искусство восходит еще ко временам фараонов, и никто в деревне никогда не выдал своего секрета. Когда наступает пора выведения цыплят, жители Бирмы разбредаются по всей стране и лепят небольшие домики из глины, которые используют как инкубаторы. Каждая такая каморка заполняется тысячами яиц, в ней разводится огонь, и внутри остается человек. Говорят, что он проверяет температуру каждого яйца, прикладывая его к веку. В конце он перестает поддерживать огонь и выходит из домика с тысячами цыплят. Ни один современный электрический инкубатор не дает таких результатов, какие получают жители Бирмы, пользуясь своими тайными методами вот уже тысячу или две тысячи лет.

Мне довелось узнать кое-что о пустынях и городах Египта. Каждое утро, просыпаясь в каирском отеле, я чувствовал, как слегка дрожит здание. Это первый трамвай появлялся на площади Оперы, и вожатый тормозил на повороте. Разбудит ли меня еще когда-нибудь скрежет этого трамвая? Не думаю. Ведь я редко возвращаюсь в те места, где уже побывал однажды.

В Африке тайна принимает странные обличил, и никогда не знаешь, где с нею столкнешься. И хотя мы редко находили то, что искали, все же эти поездки всегда вознаграждали нас за все трудности, лишения и постоянный риск. В Африке есть над чем поразмыслить.

\chapter{Тайны африканской медицины}

Африка дала медицине и хирургии значительно больше, чем думают врачи так называемого цивилизованного мира. Из века в век, пробуя и ошибаясь, знахари африканских племен порой делали сенсационные открытия, и это еще в те времена, когда врачи в Европе были всего лишь невежественными шарлатанами. Многое в африканской медицине перестало быть тайной. Интересно, сколько там еще остается неизвестного европейской науке?

Много лет назад, путешествуя по Бельгийскому Конго, я подружился с одним французским врачом, любознательным человеком, который использовал малейшую возможность узнать что-нибудь об африканской медицине. Однажды во время стоянки парохода он увидел собравшихся на берегу африканцев и пригласил меня посмотреть на необычную операцию. У больного в предплечье была глубокая рана. Его друзья ловили огромных и злющих черных муравьев и по очереди сажали их на раненое место. Муравьи тотчас же впивались в рану и постепенно стягивали ее края. Как только муравей сделает свое дело, его тут же выбрасывали, и в конце концов рана была стянута так плотно, словно ее сшил хирург. Антисептика? В Конго об этом заботится солнце.

Среди знахарей есть люди, которые по-настоящему разбираются в лекарственных травах, хирургии и гипнозе. Лишь немногим более полувека тому назад сэр Рональд Росс удивил медицинский мир довольно простым открытием, доказав, что комары являются переносчиками малярии. Я считаю, что такое открытие следовало бы сделать намного раньше, ведь население всей тропической Африки знало об этом уже много столетий. <<Не стройте хижин в тех местах, где водятся комары, потому что комары пагубны для человека, они возбуждают жар в крови>>,~--- говорят мудрецы многих племен. Если бы в Африке была хинная кора, знахари ее разыскали бы. Нашли же они корни аконита, которые вызывают обильное потоотделение и заглушают малярию. Они излечивали черную водянку еще в те времена, когда белые почти всегда от нее умирали.

До недавнего времени европейские врачи лечили прогрессивный паралич искусственно вызванной малярией. Рональд Росс объяснял: <<Плазмодии малярии вступают в смертельную борьбу с микробами, вызывающими паралич. Затем малярия излечивается несколькими дозами хинина>>. Африканские знахари могли бы давным-давно сказать нашим ученым то же самое, только другими словами, если бы кто-нибудь догадался спросить их об этом. Своих больных они относят к болотам, чтобы их искусали комары.

Возвратный тиф, возникающий от укуса клеща спи-риллума, был еще одной болезнью, которую первыми одолели знахари. В районах, пораженных этим клещом, туземцы всегда носят на себе собственных клещей~--- это обеспечивает им естественный иммунитет. Другими словами, они все время поддерживают слабую инфекцию. Ведь при этой болезни первые дни самые тяжелые, а потом состояние улучшается. Если бы они этого не делали, начало следующего приступа было бы очень тяжелым.

Доктор Т. X. Далримпл, занимавший незадолго до второй мировой войны должность санитарного инспектора британской медицинской службы в Камеруне, высоко ценил искусство местных врачевателей. Он встречал одного знахаря, который излечил душевнобольного, когда все европейские врачи признали его безнадежным. Местные медики любили присутствовать на операциях европейских хирургов. Но они заверяли доктора Далримпла, что могли бы справиться с такой же операцией без хлороформа, с простейшими инструментами и без обычной в таких случаях суеты. Доктор Далримпл рассказывал об одной необычайно искусной операции, когда из тела женщины, умершей от отравления, был полностью, от рта до основания брюшной полости, удален яд, так что нельзя было обнаружить никаких его следов.

Примерно в то же время доктор Сесилия Уильямс поместила в журнале <<Ланцет>> статью о знахарях, с которыми она сталкивалась за девять лет пребывания на Золотом Береге. <<В ценности некоторых их лекарств,~--- писала она,~--- не приходится сомневаться. Знахари, конечно, имеют какое-то эффективное средство от столбняка>>. Растительное масло чаульмогра, которое европейская медицина открыла в период между двумя мировыми войнами как средство от проказы, давно уже было в сумке колдуна.

И конечно, змеиные укусы привлекали внимание знахарей за много сотен лет до открытия наших современных сывороток. В этой области знахарь все еще опережает современную медицину, так как он умеет создать иммунитет у своих пациентов. Присмотритесь к ногам носильщика в лесах Золотого Берега, которому нередко случается наступать на змей, и вы увидите между большим и вторым пальцем обеих ног крестообразные шрамы. Раз в несколько лет он делает прививку у знахарей, и это его спасает. О средствах против змеиного яда в тропической Африке можно было бы написать целый фолиант, если бы удалось уговорить знахарей открыть свои секреты.

Задолго до того, как в Европе открыли свойства радия, африканцы с реки Конго лечили ревматизм черной речной грязью. Женщины многих племен этого района, а также во Французской Экваториальной Африке использовали грязь и для других целей. Говорят, что они носили ее в амулетах, если не хотели иметь детей. Ученые наконец исследовали эту грязь и нашли, что она радиоактивна. Радием не только лечат ревматизм, но и вызывают бесплодие.

Покойный сенатор У.~П.~Стинкэмп, легендарная фигура в Южной Африке, серьезно изучал медицинские средства готтентотов и бушменов. Несколько лет Стинкэмп был священником, но ему так часто приходилось выступать в роли врача, что он решил получить медицинское образование в Соединенных Штатах.

Мне запомнился один рассказ Стинкэмпа. Когда он был мальчиком, на ферме его отца работал пастух-африканец, по имени Уиллем Пренс. Этот человек считался прекрасным исцелителем. Стинкэмп исследовал его лекарства. Среди них был высушенный желудок дикобраза, и Пренс с успехом применял его при лечении язвы желудка. Спустя многие годы вытяжка из стенок желудка свиньи была признана медициной хорошим средством для лечения язвы желудка. Но каким образом африканец (или его предки) дошли до этого открытия на земле бушменов? Я думаю, что только ценой бесчисленных проб и ошибок.

Для лечения многих кожных заболеваний готтентоты использовали овечью шерсть. Известную современную мазь ланолин получают из выделений сальных желез овцы. Этот жир легко впитывается в кожу.

Если вы когда-нибудь страдали от гипертонии, вам должно быть знакомо лекарство серпазил, которое извлекают из африканского растения Rauwolfia serpentina. Миссионеры не раз сообщали, что африканские знахари используют это растение и добиваются блестящих результатов. Европейская медицина узнала о нем лишь спустя десятилетия.

Среди совсем необычных для нас средств можно назвать паутину. Многие века знахари делали из определенных видов паутины катышки и давали их больным малярией. В конце прошлого века испанский фармаколог Олива из разной паутины, применяемой знахарями, приготовил арахнидин~--- жаропонижающее средство, по своему действию равное хинину.

Знахари племени алур, живущего в верховьях Нила, лечат умопомешательство, на какое-то время закапывая больного по горло в муравейник. В последние годы муравьиная кислота, продукт выделения муравьев, использовалась европейскими врачами как тонизирующее средство, а также для лечения неврастении. Еще одно старое африканское средство~--- пчелиный яд. Знахари использовали живых пчел для лечения хронического ревматизма. Это средство было известно также и в средневековой Европе, но лишь в нашем веке официальная медицина признала его снова.

С незапамятных времен африканские знахари лечили болезни носа вытяжкой из улиток. Несколько лет назад европейские врачи стали применять для лечения болезней носа и ларингита клейкое вещество, названное муцином. Извлекается оно из улиток.

Очень мало лекарственных растений устояло перед чудесными свойствами пенициллина и сульфапрепаратов. Однако некоторые старые лекарственные травы Африки выдержали испытание временем и сейчас занимают важное место в фармакологии. Среди них~--- камомил. Это горькое тонизирующее средство все еще можно найти в современной аптеке, так же как и бучу готтентотов. Смола акации, известная в торговле под названием гуммиарабик,~--- еще одно африканское лекарство, которое до сих пор пользуется спросом как успокоительное. Размельченные и высушенные корни растения калумба из Мозамбика все еще ценятся за свои тонизирующие свойства. Наркотики, которые были известны в Африке за много столетий до того, как они появились в Европе, представляли собой темно-зеленые сморщенные листья мандрагоры и семена дурмана.

В Восточной Африке туземцы дают женщинам при родах замечательное лекарство миликила, которое обладает болеутоляющими свойствами. Некоторые племена в Танганьике делают себе прививку на лбу и на плечах противооспенной сывороткой собственного приготовления. Они могут также предотвращать возвратный тиф, переносимый клещом спириллумом. Их лечебные травы от головной боли, ставшие известными благодаря капитану У. Хиченсу, служившему несколько лет назад в административном управлении Танганьики, настолько эффективны, что ими стали пользоваться многие европейцы.

У каждого знахаря в Африке есть набор слабительных средств и сумка с разными растениями, обладающими мочегонными и болеутоляющими свойствами. Они применяют мужской папоротник как глистогонное. Расстройства желудка и кишечника легко отступают перед лекарствами, которые есть у знахаря. Можно было бы составить длинный список рвотных и слабительных, пришедших к нам из Африки. К сожалению, мы еще не знаем всех их ядов.

Африканские знахари более сильны в лекарствах, чем в хирургии. И тем не менее их хирургическая техника позволяет им делать операции гораздо более серьезные, чем кровопускание или вскрытие гнойников и нарывов. У кенийских масаи есть знахари-хирурги, которые могут вынуть поврежденный глаз, ампутировать конечности, удалить гланды.

Когда-то в Уганде некий доктор Фелкин наблюдал, как с помощью простейших средств было сделано кесарево сечение. Я не мог понять, почему Фелкин не провел операцию сам, теперь же спрашивать об этом слишком поздно. Возможно, у него не было с собой необходимых инструментов.

Это была двадцатилетняя женщина, рожавшая впервые. Роды были трудные, и потребовалось хирургическое вмешательство. Ее одурманили сильно действующим местным напитком -банановым вином. Прежде чем приступить к операции, туземный хирург вымыл руки в спирте и им же протер живот больной. Затем одним движением вскрыл брюшную полость и стенки матки. Кровотечение было остановлено раскаленным железом. Когда ребенок был извлечен, хирург стал массировать матку. Он делал это до тех пор, пока она не сократилась. Разрез был сшит хорошо отполированными гвоздиками с накрученными на них крепкими нитками. Гвоздики были вынуты через неделю, а на одиннадцатый день рана зажила. Всю эту операцию доктор Фелкин описал в журнале <<Эдинбург Медик Джорнэл>> за апрель 1884 года, где и можно найти технические подробности этой операции.

Еще в период неолита в Африке делали трепанацию при проломе черепа, чтобы освободить мозг от давления. И некоторые больные вполне оправлялись после операции. Об этом говорят найденные археологами черепа. Инструментами первобытных хирургов были, по-видимому, обеззараженные осколки кремня или обсидиана. В качестве антисептиков применялись травы, спирт и огонь. Вокруг собирались сочувствующие друзья, которые монотонно пели и били в барабаны. При всем своем искусстве хирург должен был обладать отвагой. Несомненно, когда больной поправлялся, устраивались грандиозные танцы.

Зубному врачу нечего ожидать от знахаря и, вероятно, почти нечему учиться. В конце концов, что может сделать зубной врач без своих щипцов? Знахарь не придумал такого способа выдергивать зубы, который шел бы в сравнение с металлическими инструментами цивилизованных врачей. Однако знахарь может избавить от зубной боли. Если надо удалить зуб, знахарь обычно кладет на него высушенные и растертые в порошок корни определенных растений, от которых зуб дробится на части и по кусочкам выпадает. Это долгий и утомительный процесс. Зулусские медики, а вероятно, и многие другие знают траву, которая убивает зубной нерв.

Зулусы весьма искусно извлекают занозы. И они умели накладывать шины еще задолго до того, как первый белый человек появился в их стране. К сломанной конечности они привязывали кости собак, а для ускорения сращивания перелома давали лекарственные травы. Именно среди зулусов встречаются самые искусные в Африке лекари. Вероятно, первым южноафриканским врачом, получившим диплом за океаном, был зулус Джон Нембула. В 1891 году он окончил медицинскую школу в Чикаго. Один европейский медик, знавший Нембулу, сообщал, что хотя тот и не был выдающимся врачом, но обладал не меньшими способностями, чем его белые коллеги.

Египет, колыбель многих современных наук, видел, возможно, и рождение медицинской науки. Врачи, практиковавшие на берегах Нила, намного превосходили древних знахарей остальной части континента. В папирусе 1560 года до нашей эры перечислены замечательные лекарства. Медицинские книги, найденные в гробницах, показывают, что древнеегипетские врачи предписывали мази и пластыри, пилюли и свечи. Среди применявшихся ими лекарств были мед и полынь, ягоды можжевельника и разные травы, винные ягоды, тмин и настурция. Мочегонное и отхаркивающее средство Древнего Египта, приготовленное из высушенной луковицы морского лука, применяется и до сих пор.

Одно за другим лекарства древней Африки становятся известными у нас и испытываются на практике. Трудно сказать, сколько еще предстоит открыть, но было бы неразумно отрицать возможность всяких сюрпризов. В начале нашего века преподобный А.~Т.~Бриан перечислил более двухсот растений, которые зулусы используют для своих лекарств. <<Надо согласиться,~--- замечает он,~--- что туземный лекарь нередко находит средство, иногда совершенно изумительное средство, там, где все усилия европейских врачей оказываются бесплодными. У него бесчисленное множество лекарств, и некоторые из них по-настоящему целебны и годятся против любых болезней~--- физических, психических, моральных и социальных, которые только могут быть у человека>>.

\begin{figure}[ht!]
\centering
\includegraphics[width=90mm]{000004.jpg}
\caption{Африканский знахарь и его пациент}
\label{overflow}
\end{figure}


Доктор Майкл Гельфанд, работавший среди народности машона в Родезии, встречался там со многими лекарями и высоко оценил их ум и искусство врачевания. Обычно сын нганга перенимает секреты у своего отца и наследует его ремесло. Знахари рассуждают здраво, говорит Гельфанд. Они очень сведущие ботаники и разумные судьи в житейских делах. Они верят в свое искусство и хотят помочь другим. Поэтому к ним и обращаются пациенты.

Однако доктор Гельфанд не мог найти у местных знахарей никаких лекарств от тех болезней, перед которыми все еще бессильна европейская медицина. И в заключение он говорит: <<Наши больницы забиты больными туберкулезом, раком, циррозом печени, диабетом, сердечными заболеваниями, хроническим нефритом, воспалением легких, гипертонией, ревматическим артритом и проказой. Все эти больные безуспешно лечились у нганга>>.

Видимо, времена знахарей клонятся к закату и медицинская наука давно превзошла их искусство. Только в одном отношении знахарь, вероятно, все еще стоит на высоте. К нему всегда относятся как к чародею, и его клиенты так верят в него, что он может излечивать больных благодаря их вере. А как он проницателен! Наши психиатры могли бы позавидовать его умению. <<Секрет знахаря заключается не в действии вещества на вещество, то есть лекарства на плоть, а в тех сокровенных сферах, где разум действует на разум и разум~--- на плоть>>,~--- говорил Бриан в Зулуленде.

Африканцам нужны видимые признаки облегчения. А знахарю это сделать нетрудно: он пользуется способами, которые можно найти в любом руководстве по магии. Он определяет, что недуг был вызван врагами, бросившими в свою жертву камешек или иной предмет. После соответствующей церемонии он показывает эти камешки, колючки или даже живых ящериц. Удовлетворение клиента можно сравнить лишь с созерцанием больным вырванного зуба или аппендикса в банке.

Многие больные~--- как белые, так и черные~--- предпочитают чудодейственное лечение научной медицине. Знахарь лишь использует те способы и предрассудки, которые были широко распространены и в Европе в прошлом веке и которые все еще не изжиты. Именно благодаря этому знахари так часто достигают успеха.

В исцелении больного, который верит в выздоровление, нет никакого чуда. Это просто пример господства разума над телом. Когда сознание больного больше не угнетено, его организм сам борется с болезнью. В известных пределах возможности такого метода огромны. И африканский знахарь знает это так же хорошо, как и его коллеги в Европе и Америке.

\chapter{Цитадель колдунов}

Колдовство появилось на заре человечества. Может быть, именно Африка видела зарождение этого жестокого верования, и она же остается цитаделью колдунов. Черти, оборотни, заклинания и чары в средневековой Европе, религиозные предрассудки и <<дурной глаз>>, которые все еще существуют в нашей жизни, пришли к нам тысячу лет тому назад с Черного континента.

Колдовство так и не потеряло своего влияния на африканца. Куда бы вы ни направились~--- от Алжира до Кейптауна и от Дакара до Занзибара, вам повсюду встретятся миллионы африканцев, которые все еще боятся джиннов и демонов, волшебников и колдунов, нгогве и токолоше. Во многих племенах смерть почти всегда считается результатом колдовства врага. Миллионы людей убеждены, что женщина может выпустить в мир лишь одну душу. Поэтому двойни представляют собой расщепленную душу. Они заколдованы, и дьявол найдет путь в каждого из них, так как в них есть <<место без души>>. Умерщвление близнецов было в Африке когда-то повсеместным обычаем. И никто не может утверждать, что этого обычая больше не существует.

В Африке колдовство занимает важное место во всей жизни африканцев, и лишь очень немногие (какое бы образование они ни получили) не верят в колдовство. Да и может ли быть иначе, если многие образованные люди в Западной Европе сохраняют странные предрассудки без малейшего на то основания? Сегодня Африка демонстрирует нам со всей очевидностью образ мышления наших предков, живших в те времена, когда в Англии и Западной Европе вешали тысячи людей, заподозренных в колдовстве, или сжигали их на кострах, привязывая к столбу.

Африканцы окружены духами. Днем и ночью ревностные духи следят за ними, и орда разгневанных призраков немедленно наказывает того, кто нарушает традиции своего племени. Это говорит о мощном веровании Африки, едином веровании, присущем всем африканцам, будь то формально христиане, мусульмане или язычники. Новообращенные используют свою веру для того, чтобы отвратить колдовство, и тут стихи из Корана превращаются у них в заклинания. Христианские миссионеры составляют специальные молитвы и церковные службы для тех, кто считает себя околдованным.

Европейцы, долгие годы прожившие в тропической Африке бок о бок с колдовством, не раз отмечали, что <<в колдовстве заключается нечто большее, чем кажется с первого взгляда>>. Иначе говоря, они невольно признаются в своей вере в черную магию. Если бы они серьезно разобрались в каждом загадочном случае, то вместо магии увидели бы жестокость. Я помню одного западноафриканского владыку, которого какой-то умный чиновник охарактеризовал как человека <<с сердцем леопарда и моралью крокодила>>. Таково колдовство. Оно восходит к первобытному безжалостному миру.

В африканском колдовстве вы не раз столкнетесь с чем-то непонятным, с каким-то явлением, которое вам нелегко будет объяснить, если вы отрезаны в этой глуши от остального мира. Решением вопросов, которые требуют объединенных усилий целой армии Спилсбери и Фрейдов, вынуждены заниматься обыкновенные врачи. Их неудачи горячо обсуждаются, и небольшая группа белых поселенцев принимает сверхъестественное объяснение того или иного факта просто потому, что они были не в состоянии выяснить истину. Так и возникли те таинственные истории, которые перекочевали в Европу в сильно приукрашенном виде и которые годятся лишь для того, чтобы рассказывать их перед удивленной и восхищенной аудиторией, никогда не знавшей африканского буша.

Конечно, первобытные народы знают такие способы убийства и самоубийства, которые в деталях еще не выяснены европейской наукой. Среди аборигенов Австралии существует загадка <<указующей кости>>. В давние времена австралийцы, видимо, обнаружили, что человек может умереть, если оцарапается костью. Они, конечно, ничего не знали о микробах, и колдуны решили, что вполне возможно избавиться от нежелательных людей, если указать на них костью. Но сила гипноза такова, что <<указующая кость>> достигала своей цели! Если люди верят в то, что с ними произойдет, задача гипнотизера облегчается.

Это прием, которым белые гипнотизеры еще не овладели. Африканские колдуны отлично понимают силу гипноза, и именно этим я объясняю ряд таинственных случаев смерти, где отравление исключается полностью и вскрытие трупа ничего не дает.

Один такой случай был сообщен сэром X. Р. Палмером, губернатором провинции Бенуэ в Нигерии. В 1921 году, объезжая провинцию, он услышал, что молодому человеку из племени юкуна, претенденту на пост вождя, угрожает подобная опасность. Палмер взял юношу к себе личным слугой. Через два года Палмер переехал в Майдугури на севере Нигерии. Юноша слуга сообщил ему, что узнал о болезни своей матери и хотел бы поехать домой в местечко Иби в Бенуэ.

Помня о споре из-за поста вождя, Палмер послал в администрацию Иби телеграмму с просьбой сообщить о положении дел. Ему ответили, что мать юноши чувствует себя хорошо, но вождь племени болен. Палмер отказал слуге в просьбе. Через месяц, однако, тот все же настоял на поездке в Иби и попрощался с хозяином. Палмер отмечает, что юноша был совершенно здоров и спокоен. А через тридцать минут со слугой случился припадок и он умер.

Палмер был уверен, что тут не обошлось без колдовства, и дал указание правительственному врачу У.~Е.~С.~Дигби вскрыть труп. Врач произвел вскрытие, но никаких следов яда или других признаков насильственной смерти не обнаружил. Палмер мог лишь только догадываться, что молодой человек умер от страха, вызванного гипнозом.

X. Л. Уорд Прайс, многие годы занимавший высокий административный пост в Нигерии, сам чуть не умер от таинственной болезни, которую так и не удалось определить. В 1935 году он был в Ибадане и вдруг почувствовал себя плохо, после того как получил ряд дружеских предостережений, что его хотят отравить. В мечети была заказана особая служба, и три тысячи мусульман молились за его здоровье. Для его выздоровления были принесены в жертву козы, овцы и коровы. Но ему не становилось лучше, несмотря на всякие предосторожности. Врач Прайса верил, что на его пациента действует какая-то злая сила. Тогда власти решили отправить Уорда Прайса в Лагос. Там он совсем поправился.

Иногда причиной смерти бывает самовнушение. Колдун добывает обрезки ногтей или прядь волос, принадлежащие намеченной жертве, и изыскивает способ сообщить ей, что они у него есть и он собирается воспользоваться ими, чтобы вызвать смерть. В мире, где господствуют суеверия, жертва своей глубокой верой в могущество колдуна сама помогает его злым деяниям. Один крупный чиновник из Сьерра-Леоне сообщал о болезни африканского юноши, который учился в городе Бо в школе для будущих вождей. Он оскорбил одного вождя, который за это <<наложил на него заклятье>>. Юноша сказал английским врачам, что тут уж ничего не поделаешь. Раз вождь пожелал, чтобы он умер, то он умрет. Состояние молодого человека стало настолько серьезным, что врачи телеграфировали о присылке специального поезда. Юношу привезли к вождю и попросили простить его. Прощение было дано, и вскоре больной выздоровел.

Во времена правления Мошеша, величайшего из вождей Басутоленда, за такое колдовство приговаривали к смерти. Несомненно, это обуздало злых колдунов, но их зловещее искусство не исчезло. В последние годы случались ритуальные убийства. Колдунам нужны были отдельные части тела жертвы, чтобы использовать их как <<колдовское лекарство>>.

Вскоре после второй мировой войны лейтенант М.~К.~Ван Стаатен, служивший в конной полиции Басутоленда, расследовал одно из таких убийств и сделал любопытное открытие. Он захватил местное лекарство маиме~--- нечто вроде басутского хлороформа. Убийцы давали его своей жертве, которая после этого послушно следовала за ними к месту убийства. Достаточно было одного вдоха или глотка маиме, чтобы человек превратился в послушный автомат и не смог оказать сопротивления. Однако до судебного процесса в 1946 году над Манапо Коенехо и тремя другими африканцами, обвиненными в ритуальном убийстве, это необычное лекарство оставалось неизвестным. Все четверо обвиняемых были повешены.

\begin{figure}[ht!]
\centering
\includegraphics[width=90mm]{000005.jpg}
\caption{Колдун}
\label{overflow}
\end{figure}


Исследуя вещества, широко применяемые африканцами, европейские ученые очень часто бывают сбиты с толку. Профессор Витватерсрандского университета Д.~М.~Уотт описал одно судебное разбирательство, где был подвергнут экспертизе кусок коры, которую использовал убийца-зулус. В лаборатории кору кипятили в воде, но отвар оказался неядовитым. Тайна раскрылась лишь после того, как сам убийца пришел на помощь.

Он добровольно сообщил, что кору надо растолочь в порошок, только тогда она действует как яд. И действительно, порошок оказался смертельным. Профессор Уотт также сообщает, что только через пять лет удалось определить породу дерева, с которого была снята кора. Оказалось, что этот вид до сих пор был ботаникам неизвестен.

Испытание <<судом божьим>>~--- обычное явление во многих районах Африки. И тут колдунам, которые обнаруживают виновного, также приписывается сверхъестественная сила. И в самом деле, колдуны иногда устраивают такие драматические представления, что вводят в заблуждение даже опытных колониальных чиновников.

Незадолго до второй мировой войны в глухом уголке Уганды был зарезан африканский носильщик, который сопровождал партию охотников-англичан. Полицейский участок был далеко, и охотники сами провели следствие, но оно не дало никаких результатов. Тогда глава партии Грей согласился, чтобы вождь деревни вызвал колдуна.

Всех жителей выстроили в один ряд, и колдун велел им войти в хижину, где лежало тело убитого носильщика. Они должны были входить по очереди и прикасаться к телу рукой. <<Когда до тела дотронется виновник, мертвец оживет и проклянет убийцу>>,~--- заявил колдун.

Все стояли в мертвом молчании. Казалось, будто колдун вовлек белых людей в глупую мистификацию. Однако он пристально смотрел на каждого жителя деревни, затем указал на одного из них и крикнул, что он виновен. Мужчина бросился бежать, но был тут же схвачен. Вскоре он во всем признался.

Грей отозвал колдуна в сторонку и спросил, как он это сделал. Сначала колдун пытался уверить его, что все это магия, но Грей настаивал на своем и добился правды. Колдун намазал тело убитого бесцветным веществом, которое при высыхании белеет. Он знал, что только человек с чистой совестью дотронется до тела. И единственный, у кого на руке не оказалось белой отметки, был убийца.

Даже такой известный знаток Африки, как Д. X. Дриберг, был поставлен в тупик, когда ему пришлось наблюдать подобное испытание на Ниле среди племени какуа. Восемь человек, подозревавшихся в убийстве, посадили в круг и перед каждым из них положили камень. В центре круга к колышку была привязана курица.

Колдун окропил курицу водой, пробормотал заклинание и приказал ей найти убийцу. Потом он быстро оторвал ей голову. Курица запрыгала по кругу и упала на один из камней. Вскоре стало ясно, что птица действительно указала убийцу. Дриберг уговорил колдуна повторить процедуру еще раз десять, и все время результат был один и тот же. <<Сомнений не оставалось>>,~--- сообщал Дриберг. Должно быть, это была хитрость, но, в чем она заключалась, Дриберг так и не понял.

Самоубийство в нашем понимании этого слова среди жителей Западной Африки почти неизвестно. Но многие африканцы могут заставить себя умереть, и этого наука еще не в силах толком объяснить. Примеров такой смерти в самых различных районах Африки найдено так много, что сомневаться тут не приходится.

Во флотилии лодок, шедшей по Нилу в Хартум на выручку Гордону, было несколько гребцов из племени кру. Сначала они работали хорошо, но потом стали тосковать по своему западноафриканскому побережью. <<Мы хотим домой>>,~--- заявили они. Наконец они улеглись на дно лодок и через несколько часов умерли.

По всей Западной Африке встречаются люди, которые обладают необъяснимой властью над животными. Некоторые старики, возможно, еще помнят заклинателя с реки Кросс, который игрой на камышовой дудочке вызывал из болот бегемотов. Причем он никогда их не кормил. Другие тоже пробовали это сделать, но безуспешно. Он же мог подзывать их, когда хотел. С такими фактами европейцы сталкивались уже давно. В 1837 году адмирал сэр Генри Кеппел встретился в местечке Диксков на Золотом Береге с одной старой знахаркой, которая могла вызывать из реки крокодилов. Она была очень стара и почти слепа, но, когда стояла иод деревом с живым цыпленком на конце палки и пела, крокодил подползал к ней и хватал цыпленка.

Армейский капитан Ф.~У.~Батт-Томпсон, долгое время пробывший в Западной Африке, специально занимался изучением этого рода магии. Он рассказывал мне, что во внутренних районах Сьерра-Леоне видел женщину, которая плавала среди крокодилов и заставляла их следовать за собой. Кроме того, она ныряла в реку совершенно обнаженной, а появлялась увешанная нитками бус. Этот капитан, очень знающий человек, автор научных трудов об африканской магии, рассказал мне о некоторых наиболее удивительных случаях, с которыми он сталкивался. Колдун одного нигерийского тайного общества набирал в рот воды из калебаса, а выплескивал изо рта дюжину живых рыбок, водящихся в окрестных болотах. А в калебасе была только вода. В Конго Батт-Томпсон наблюдал, как человек из племени нкимба тер себе нос и оттуда появлялась длинная вереница красных блестящих муравьев. И он знал одного сенегальца, который мог бы постыдить европейских шпагоглотателей. Он вонзал в горло широкий нож.

Но, как узнал Батт-Томпсон, настоящими мастерами магии были колдуны прошлого. В Сьерра-Леоне на коронации Георга Второго, короля Буллома, королевский колдун сделал совершенно необъяснимую вещь. (К сожалению, это случилось очень давно, в 1827 году. Однако об этом случае африканцы все еще рассказывают.) День был совершенно безветренный и необычайно жаркий. И вот колдун вызвал ветер, который закачал ветки деревьев и к удовольствию короля разметал цветы, устилавшие алтарь.

Естественно, что такой колдун может отлично предсказывать погоду. Он знает, когда выбрать подходящий момент. Без всяких приборов он может сделать правильные выводы. Засухи и дожди, ураганы и грозы~--- все случается в его местности. И если его предсказание сбывается, он знает, как этим воспользоваться. Он даст вам амулеты, чтобы охранять от всех несчастий со дня рождения и до смерти. Даст вам янтарь для гадания на <<магическом кристалле>> и шерсть подходящего животного, чтобы спасти от дурного глаза.

Тайн в Сьерра-Леоне хватает с избытком. В 1Э09 году губернатор этой колонии сэр Лесли Пробин вместе с молодым служащим Лэйксм объезжал северные границы. И всюду, куда бы он ни приезжал, вожди племен были в смятении. Они сообщили губернатору, что наступает плохое время. Тот потребовал объяснений, и они рассказали, что их предки были предупреждены мудрецами Тимбукту, искусными астрономами, что в определенный год в ночном небе появится огромная звезда, которая принесет большое несчастье. Раньше они об этом не беспокоились, потому что предсказание было сделано очень давно. Но сейчас этот год наступает. Предсказания сбудутся в следующем году.

Как я уже говорил, это было в 1909 году, а в 1910 году появилась комета Галлея. И Сьерра-Леоне пережила самый сильный неурожай риса, какой только там помнят. Умер король Эдуард Седьмой, а вскоре распространилась эпидемия желтой лихорадки. Я думаю, что все это совпадения. Однако астрономы Тимбукту, кажется, знали все о комете Галлея.

Почти все в Западной Африке верят, что некоторые избранные личности способны превращаться в леопардов и других хищников. Но примечательнее всего, что есть европейцы, которые разделяют эту веру.

Много лет назад в полицейском управлении во Фритауне мне показали отдельные вещи <<Общества леопарда>>. И это были страшные вещи: одежда из шкуры леопарда, ножи с тремя лезвиями, напоминающими когти леопарда, и знахарская сумка~--- борфима. Считается, что она приносит ее владельцу силу и богатство. <<Общество леопарда>>~--- кровожадная форма магии. И ни один европеец никогда не знает, где оно проявит свою деятельность снова и сколько будет жертв.

<<Общество леопарда>> настолько секретно, что о самом его существовании стало известно лишь сто лет назад. В тех местах, где многие люди действительно гибнут от когтей леопарда, конечно, трудно отличить действия леопардов от убийств, совершаемых членами общества.

Из официальных документов Сьерра-Леоне известно, что в 1854 году один африканец был сожжен живьем своими соплеменниками в Порт-Локко <<за то, что превратился в леопарда>>. Но подробные сведения об этом тайном обществе стали известны лишь в 1912 году, когда сэр Уильям Брэндфорд Гриффит, в то время главный судья на Золотом Береге, возглавил особый суд, который расследовал ряд убийств, совершенных <<Обществом леопарда>>.

<<Я бывал во многих лесах, но ни один из них не производил на меня такого жуткого впечатления, как буш Западной Африки,~--- писал Гриффит.~--- В этом лесу и его поселениях есть что-то такое, что заставляет человека содрогаться. Мне казалось, что весь лес наполнен чем-то сверхъестественным, каким-то духом, который старается соединить зверя и человека. Некий таинственный дух этих мест вошел в людей и определил их образ жизни. Люди здесь обладают изумительной способностью скрывать то, о чем не должны знать другие. Это~--- результат деятельности нескольких поколений тайных обществ>>.

Впервые <<люди-леопарды>> стали преследоваться законом в Сьерра-Леоне в 1892 году, а через некоторое время было запрещено держать атрибуты общества~--- одежду и когти. К концу прошлого века администрация раскрыла деятельность <<Общества крокодила>>, совершающего такие же убийства в районах, где леопарды встречаются редко. А позднее сообщалось об <<Обществе бабуина>> в северных районах страны. Но об истинных мотивах этих убийств почти ничего не было известно. Был ли это только каннибализм или нечто иное? Все, что можно было сделать в то время,~--- это запретить держать шкуры крокодилов и бабуинов и некоторые другие предметы.

Однажды полиция Сьерра-Леоне захватила примитивную подводную лодку, построенную членами <<Общества крокодила>> и по форме напоминающую крокодила. Нос ее был сделан в виде крокодильей головы, а в движение она приводилась лопастями, похожими на лапы. Кожа и пчелиный воск делали всю конструкцию водонепроницаемой. Команда состояла из шести человек. Один из них, <<хватающий>>, занимал место у пасти <<крокодила>>, чтобы иметь возможность дотянуться до жертвы у берега и втащить ее под воду. Судно это было построено тайно. Предполагают, что при спуске его на воду была принесена человеческая жертва. Когда этот искусственный <<крокодил>> плавал на поверхности, виднелась лишь его голова.

Между 1907 и 1912 годами смерть от леопардов случалась так часто, что был учрежден упоминавшийся уже мною особый суд. Арестовали более четырехсот человек, в том числе ряд вождей. Заключенные содержались в надежном месте под охраной частей западноафриканских пограничных сил.

Один вождь был обвинен в убийстве собственного сына. Мать другой жертвы побоялась давать показания.

Обвиняемые неизменно утверждали, что в убийстве повинны леопарды, а не люди. Гриффит отмечал, что в нескольких ярдах от места, где заседал суд, было поставлено несколько ловушек и что во время заседания были убиты два леопарда на расстоянии мили.

Некоторые свидетели, преодолев страх, рассказали о своем посвящении в члены <<Общества леопарда>>. Специальной иглой им выжигали отличительные знаки, напоминающие случайные царапины, обычные для жителей буша. Кроме того, члены общества узнавали друг друга, особым образом вращая глазами. Они описали и сумку борфима. В ней содержатся части человеческого тела, кровь петуха и несколько граммов риса. На этой магической сумке давалась клятва при вступлении в члены общества. Но для того чтобы сумка не теряла свойств обогащать и охранять членов общества, ее надо время от времени смазывать человеческой кровью и жиром. В таких случаях созывалось собрание и один из посвящаемых в члены <<Общества леопарда>> выбирался для совершения убийства и <<насыщения борфима>>. После <<насыщения>> труп убитого делился между членами общества. Свидетели показали также, что, если кто-нибудь из членов нарушал клятву <<борфима>>, он лишался не только земной жизни, но и жизни загробной.

Среди обвиняемых был африканец из племени шер-бро, Даниэль Уилберфорс, потомок правящей династии, который получил образование в Соединенных Штатах в миссионерской школе. Он был хорошим учеником и стал священником. Но, вернувшись в Сьерра-Леоне, он вернулся и к прежней жизни, став верховным вождем племени имперри. В это время~--- с 1899 по 1905 год~--- власти заметили, что в районе, где правил Уилберфорс, снова ожила деятельность <<Общества леопарда>>. Наконец Уилберфорс был уличен и отдан под суд по обвинению в соучастии в убийстве, совершенном <<человеком-леопардом>>.

Защитниками Уилберфорса были видные африканские адвокаты, которые хорошо знали формальную сторону дела. Они доказали, что Уилберфорс является американским гражданином, а посему окружной суд не имеет над ним юрисдикции. Затем судебный процесс был перенесен в Бонт, где африканские присяжные оправдали Уилберфорса. Единственное, что могли сделать власти,~--- то выдворить Уилберфорса из его вотчины на том основании, что он иностранец. После этого Уил-берфорс возобновил свою деятельность священника и отправился с лекциями в турне по Соединенным Штатам. Его пылкое ораторское искусство принесло ему огромную сумму денег на <<миссионерскую деятельность>> в Африке. Кроме того, он приобрел много друзей из высшего общества в Англии и был гостем парламента страны.

Его триумфальное турне за океаном длилось несколько лет, после чего Уилберфорс вернулся в Сьерра-Леоне и как проповедник разъезжал по стране с проекционным фонарем. Видимо, он не в силах был противиться зову Западной Африки, и это превратилось у него во всепобеждающее желание принять участие в тайных обрядах, которые однажды уже привели его к беде. Поэтому, когда сэр Уильям Брэндфорд Гриффит председательствовал на особом суде в 1912~-- 1913 годах, Уилберфорс вторично оказался на скамье подсудимых. Обвинение было связано с исчезновением девушки во время уборки риса, и подозревали, что Уилберфорс получил часть ее тела.

Уилберфорс был оправдан снова, но на сей раз выслан в Либерию. Его имя замалчивалось на суде и несколько лет после него, так что Уилберфорс избежал лишних толков и мог заниматься миссионерской деятельностью. Европейцы, видевшие его на суде, не могли понять, как человек с его образованием, неотразимый проповедник и настоящий любитель хорошей музыки мог быть членом <<Общества леопарда>>. А сомнений в его виновности не было. Но к удовлетворению суда, вина его не была доказана.

Остальные представшие перед особым судом преступники оказались менее счастливыми, чем Уилберфорс. Пять членов <<Общества леопарда>> были публично повешены, и еще многие приговорены к тюремному заключению. <<Деятельность ``Леопарда'' приостановлена, но я сомневаюсь, разгромлена ли эта организация>>,~--- сообщал губернатор Сьерра-Леоне сэр Эдвард Мереуэтер.

Несомненно, самые потрясающие убийства, совершенные <<Обществом леопарда>> за последние годы, были в округе Калабар в Нигерии в 1945~-- 1947 годах. Там в различных местах было найдено более восьмидесяти жертв со вспоротыми яремными венами. Об <<Обществе леопарда>> в Нигерии ничего не было слышно в течение многих лет, и все же оно вновь появилось на сцене.

Около каждого изувеченного трупа найдены отпечатки лап леопарда. Перед полицией снова встала труднейшая задача: как отличить жертвы настоящих леопардов от жертв общества? Три белых офицера и двести полицейских-африканцев начали энергичную деятельность против людей-леопардов. Были назначены крупные вознаграждения, введено осадное положение. После четырех часов дня жителям деревень запрещалось выходить из дома, так как обычно убийства совершались в сумерках. И все же люди-леопарды настигали свои жертвы под носом у полицейских патрулей и даже убили одного полицейского. Находили трупы с вырезанным сердцем и легкими или же со следами когтей леопарда. Среди убитых было много детей.

Полиция арестовала сотни людей, и в конце концов восемнадцать человек были приговорены к смерти и повешены. Вначале предлагали устроить публичную казнь и тем самым показать, что люди-леопарды не сверхъестественные существа. Однако власти решили пригласить на казнь лишь местных вождей. Да, удивительная и непонятная история. Европейцы, очень давно живущие в Западной Африке, вполне серьезно говорили мне, что между каждым новым членом <<Общества леопарда>> и настоящим леопардом во время церемонии посвящения возникает <<кровная связь>>. Когда умирает человек-леопард, находят и мертвого леопарда. И наоборот. Это уж совсем неправдоподобно, но не так-то легко не поверить этому, когда ты очевидец событий. Вспомните слова Гриффита, этого трезвого судьи: <<Я бывал во многих лесах, но ни один из них не производил на меня такого жуткого впечатления, как буш Западной Африки>>.

\chapter{Говорящий дым}

Можно ли говорить об умении африканцев читать мысли на расстоянии? На мой взгляд, можно. Среди удивительных рассказов о телепатии и ясновидении есть, как я полагаю, немало достоверных случаев. По-видимому, такие явления чаще встречаются у отсталых народов, чем в цивилизованном обществе. Факты подтверждают это со всей очевидностью.

Впервые с телепатией я столкнулся в пустыне Калахари, в небольшом поселении бушменов на границе Бечуаналенда. Обычно считают, что бесконечное безмолвие пустыни вызывает у людей особое состояние~--- психическую настроенность. Я путешествовал тогда вместе с покойным Дональдом Бейном, известным проводником по пустыне и другом бушменов. Однажды в полдень я увидел столб дыма и сказал Бейну, что в буше пожар.

---~Это не пожар,~--- ответил Бейн.

Он подозвал к нашей палатке одного из бушменов, старика, говорившего на языке африкаанс\footnote{Язык голландских переселенцев (буров) в Южной Африке.}, и мы спросили у него, откуда этот дым. Бушмен объяснил, что его соплеменники охотятся и что они упустили одну антилопу бейзу и убили двух южноафриканских газелей неподалеку от высохшего русла реки Носсоб. Они набрали также кореньев и меду.


---~Это хорошо, что они нашли мед,~--- заметил старик,~--- теперь можно будет приготовить крепкий напиток.

Я внимательно всматривался в столб дыма, но не мог различить ничего особенного. Дым непрерывно поднимался прямо вверх, в безветренное небо.

---~Как вы догадались?~--- спросил я.

Старый бушмен не знал, что ответить. Тогда Бейн объяснил мне, что так называемое <<радио Калахари>>~--- это не код типа азбуки Морзе, а нечто более загадочное.

---~Бушмены смотрят на дым и узнают новости,~--- сказал Бейн.

Я постарался поподробнее расспросить старого бушмена и из его слов понял, что дым костра служит скорее <<призывом>>, а не определенным сигналом. Он означает, что находящиеся на далеком расстоянии охотники хотят что-то сообщить. Завидев дым, бушмены сосредоточивают на нем свое внимание. Вскоре некоторые из них уже знают, что произошло у охотников, и рассказывают об этом остальным. Одни умеют <<читать дым>>, другие не умеют. Бейн полагал, что дым здесь заменяет кристалл ясновидца. Вглядываясь в него, бушмены настраивают себя на восприятие сообщений. Но это не дымовые сигналы, а чтение мыслей на расстоянии. По <<радио Калахари>> передаются очень сложные сообщения, и они передаются слишком быстро для примитивной сигнальной системы. Бейн также уверял меня, что сам дым не играет существенной роли и что бушмены нередко и без помощи костра передают сообщения на значительные расстояния. Много лет спустя я прочитал о таком же использовании дыма аборигенами Австралии. <<Я развожу костер для того, чтобы другие знали, что я уже начал думать,~--- объяснял мне один австралиец.~--- И они тоже начинают думать, пока наши мысли не совпадают>>. Это было поразительное подтверждение слов старого бушмена. На различных континентах люди поддерживают связь друг с другом совершенно одинаковым способом.

Вполне достоверных случаев передачи мыслей на расстояние зарегистрировано очень много, ни о каком случайном совпадении тут не может быть и речи. Мне бы только хотелось, чтобы я сумел дать объяснение этой сложной загадке. Пытались объяснить это <<волнами мысли>>. Но едва ли физическая энергия, какую бы форму она ни принимала, могла распространяться подобным образом и воздействовать на другого человека с такой силой, что в его мозгу возникают сходные мысли. Телепатические процессы происходят подсознательно. Это пассивная связь, которая может с таким же успехом возникать и во время сна. Английский философ и математик А.~Н.~Уайтхед считал, что события являются конечным компонентом действительности и что все во вселенной взаимосвязанно. Оспаривая эту теорию, Кеннет Уокер, хирург и знаток телепатии, утверждал, что <<телепатия~--- это такое явление, которое в своей простейшей форме, представляющей собой способность угадывать то, что происходит на расстоянии, свойственно любому живому организму>>. Американский физик профессор Банеш Гофман считает, что телепатия, как и гравитация, обходит любое препятствие. <<Она может быть явлением физическим или каким-то другим непонятным явлением, которое распространяется по законам, выходящим за пределы пространства и времени, и совершенно не поддается объяснению с помощью тех средств, которыми располагает теперь наука>>,~--- заявляет Гофман.

Если эта тайна и будет когда-нибудь разгадана, то, вероятно, не без помощи отсталых народов и, возможно, в одной из африканских пустынь. Кеннет Уокер обратил внимание на то, что <<экстрасенсорное восприятие>> (научный термин, обозначающий телепатию, ясновидение и другие сходные с ними явления) очень распространено среди отсталых народов. У этих людей какая-то часть мозга до сих пор еще интенсивно функционирует по-старому и способна воспринимать то, что происходит в других местах.

Цивилизованные люди доказали, что телепатия действительно существует, но пока они не в состоянии раскрыть ее сущность. Отсталые народы могут принести ответ на эту пленительную загадку. А пока я могу лишь рассказать несколько случаев, взятых из африканских легенд и африканской действительности.

Многие считают, что африканцы телепатическим путем узнают о многих событиях, особенно о войнах и других бедствиях. Говорят, что весть о победе короля Кечвайо при Исандлване, которую тот одержал восемьдесят лет назад, во время зулусской войны, над армией полковника Дэрнфорда, разнеслась по всему Наталю быстрее, чем позволял любой из существовавших в те времена видов связи. Когда позднее Кечвайо попал в плен, его заключили в тюрьму в Кейптауне, где с ним обращались довольно любезно. Р.~К.~А.~Самуэльсон, приставленный к Кечвайо в качестве переводчика, вел дневник, куда заносил сны и предсказания короля. Впоследствии Самуэльсон стал видным административным деятелем в Натале, так что его дневник~--- документ достоверный.

Однажды, в сентябре 1881 года, Кечвайо сказал Саму эльсону:

---~Прошлой ночью мне приснилось, что меня от везли обратно в Зулуленд. Отец и мать целовали меня так, что мои губы совсем онемели.

В другой раз Кечвайо показал на комету близ Столовой горы и произнес:

---~Это знак того, что королева вернет меня в Зулуленд.

Спустя два года Кечвайо действительно освободили, но, конечно, ни его сон, ни комета не имеют значения.

Но вот запись в дневнике Самуэльсона, на которую нельзя не обратить внимания: <<12 сентября 1881 года. Королю приснилось, что Масумпа сдался и что в Басутоленде наступил мир>>. Масумпа~--- третий сын басутского вождя Мошеша~--- восстал против капских властей, и на подавление восстания были брошены большие силы. С пятью тысячами воинов Масумпа напал на Масеру. К моменту, когда Кечвайо поведал Самуэльсону свой сон, он уже почти год вел борьбу. Не было никаких причин думать, что Масумпа вот-вот должен сдаться. Он отказался от предложения губернатора сэра Геркулеса Робинсона урегулировать вопрос мирным путем. Однако неожиданно наступила развязка. 13 сентября в Кейптаун пришли вести, что Масумпа принял предложение Робинсона. <<Странно, что сон приснился Кечвайо до того, как король и все остальные узнали о решении Масумпы сдаться>>,~--- записал в своем дневнике Самуэльсон.

Случай, связанный с гибелью транспорта <<Менди>> в Ла-Манше в годы первой мировой войны, часто приводят как пример телепатической передачи сведений среди африканцев. Судно это потерпело крушение и пошло ко дну. При этом утонули сотни южноафриканцев, завербованных на работы во Францию. В течение некоторого времени это событие держалось в тайне. Затем, когда был составлен полный список погибших, генерал Бота сделал первое заявление в парламенте. Согласно легенде, которая неоднократно публиковалась в печати, множество женщин из племен банту оплакивали своих мужей задолго до появления официального сообщения.

Подобные рассказы могут быть достоверными, а могут быть и просто выдумкой. Нетрудно быть умным задним числом. Поэтому всякий ученый, занимающийся телепатией, вправе потребовать авторитетного письменного свидетельства с указанием точного часа и даты, а также подробного заявления, сделанного человеком, который узнал о случившемся несчастье, находясь в это время за тысячи миль от места происшествия. Я не могу ручаться за достоверность истории с <<Менди>>, но я могу привести несколько менее сенсационных случаев. О них я узнал от очевидца, на которого я могу положиться вполне.

В апреле 1912 года в Порт-Хералде (Ньясаленд) мой друг майор П.~К.~Лоуренс выслеживал льва недалеко от казарм аскеров (отряды африканцев). Сразу после сигнала отбоя Лоуренс подстрелил льва. На следующий день майор встречал поезд из Блантайра. Из вагона вышел один его знакомый, владелец плантации.

---~Послушай, Лоуренс,~--- сказал он,~--- говорят, что вчера вечером ты подстрелил отличного льва.

Лоуренс очень удивился и стал расспрашивать своего знакомого, откуда он все знает. Тот ответил, что сегодня утром ему сказал об этом слуга-африканец еще в Блантайре до отхода поезда.

Блантайр находится в ста десяти милях от Порт-Хералда, и между ними существует только телеграфная связь. В Порт-Хералде телеграф прекращает работу в пять часов дня. Майор Лоуренс сейчас же обратился к начальнику железнодорожной станции в Порт-Хералде, индийцу, и выяснил, что в тот день телеграфная контора закрылась в обычное время. Владелец плантации спросил своего слугу, откуда он узнал эту новость, но слуга только пожал плечами и сказал:

---~Просто я знаю, бвана.

Другой случай, который мне рассказал майор Лоуренс, произошел во время рождественских каникул в 1912 году. Он в то время находился в Зомба~--- административном центре Ньясаленда. Вместе со своим другом он отправился однажды на охоту в Нкула-Хилл, в двадцати милях к северу от Зомба.

---~Мы вышли в пять часов утра,~--- начал свой рассказ майор.~--- Чтобы избежать всяких недоразумений, я пошел со своим следопытом-африканцем на восток, а мой друг Джек~--- на запад. В десять часов пятьдесят минут я присел отдохнуть на камне и стал наблюдать за бабуинами. Мой следопыт попросил у меня спички, так как нашел дупло с пчелами и хотел их оттуда выкурить. Через несколько минут он вернулся, протянул мне спички и сказал: <<Бвана, Джек застрелил самку куду>>. На мой вопрос, откуда он узнал об этом, охотник как всегда ответил: <<Просто я знаю>>. Мы были на высокой гряде, и я подумал, что он мог услышать отсюда выстрел и догадаться, что это была антилопа. Мы пошли дальше и в лагерь вернулись к пяти часам вечера. Через десять минут появился Джек. Представьте его удивление, когда я сказал ему, что он убил самку куду. Мы всегда старались не убивать самок. Джек пояснил, что он целился в самца, но в тот момент, когда спускал курок, вперед выпрыгнула самка. Случилось это приблизительно в одиннадцать. Из разговора выяснилось, что антилопа была убита в двенадцати милях к востоку от лагеря. А мы в то время ушли примерно на столько же миль к западу. На таком расстоянии услышать звук выстрела было невозможно.

И наконец, еще один случай, который произошел с Лоуренсом в годы первой мировой войны. Здесь майор также ссылается на точные факты. Когда отряды африканских королевских стрелков и добровольцев из Ньясаленда двинулись на север, к Каронга, Лоуренс остался в Зомба обучать новобранцев. Вечером 9 сентября 1914 года африканские женщины подняли плач в казармах королевских стрелков в Зомба. Когда майор Лоуренс стал выяснять причину, ему сообщили, что женщины оплакивают своих погибших мужей. Ему сказали также, что во время боевых действий было убито и несколько белых офицеров.

На следующий день в Зомба пришла телеграмма из Каронга. В ней сообщалось о двух столкновениях с немцами, и был приложен список убитых и раненых. Среди убитых оказалось несколько офицеров-европейцев. Многие аскеры были убиты и ранены.

Все эти случаи, рассказанные майором Лоуренсом, могут показаться незначительными в сравнении с теми событиями исторической важности, слухи о которых, говорят, распространились телепатическим путем. И все-таки я считаю, что они представляют не меньшую ценность, чем некоторые полные драматизма события. Они могут дать истинное представление о жизни африканского буша. Не думаю, что эти случаи из жизни армейского офицера в Ньясаленде можно объяснить случайным совпадением. Это самая настоящая телепатия.

Еще один человек, словам которого я полностью доверяю, был покойный судья Фрэнк Браунли из знаменитой семьи миссионеров в Кинг-Вильямс-Тауне. Фрэнк Браунли и его предки гораздо лучше многих своих современников понимали душу африканца. Я всегда гордился тем, что такой человек, как Фрэнк Браунли, из года в год читал мои книги и присылал мне длинные письма, в которых он высказывал свои критические замечания о моих трудах. Эти письма я очень ценю: в них я нашел необычайно много интересных для меня сведений.

Случай, который убедил Браунли в том, что некоторые африканцы могут быть ясновидцами, произошел во время охоты в пустыне Калахари. На стоянке к костру Браунли подошел старый бушмен. Он разровнял песок и сел. Через некоторое время бушмен произнес:

---~Через два дня ты отправишься на север. Там ты пробудешь некоторое время, и все у тебя будет в порядке. Но потом ты вдруг повернешь и быстро поедешь на юг, и не на своих ослах и фургоне, а на машине.

Браунли действительно собирался через два дня на север, но никто из слуг об этом не знал. И он отправился в дорогу, добрался до источника Намкауб, раскинул там лагерь, отослал ослов и фургон обратно и приказал слугам вернуться за ним через месяц. Пока он был в Намкаубе, гонец-бушмен доставил ему письмо. Требовалось срочно ответить телеграммой. До ближайшего телеграфа было двести миль, и Браунли ничего не оставалось, как ждать прибытия фургона.

Через месяц фургон вернулся в лагерь, и Браунли отправился на юг, стараясь что есть силы погонять ослов. Но он не успел далеко уехать, его тут же обогнал автомобиль. Старый бушмен верно все предсказал.

Однажды судье Браунли пришлось столкнуться с удивительнейшим случаем. Владелец сельской гостиницы собирался провести вместе с женой свой воскресный отдых на морском побережье и подготовил для этого двадцать пять фунтов. Деньги пропали. Полиция не смогла найти вора. Тогда обратились к местному колдуну.

Внимательно выслушав всю историю, тот заявил, что деньги (за исключением одного фунта) спрятаны у большой скалы в верховьях одной речки. Он указал эту речку и назвал вора, слугу из гостиницы. Все подтвердилось, вор во всем сознался. Браунли уверен, что колдун совсем не знал о пропаже, пока к нему не обратились.

О телепатии и других загадочных вещах я часто беседовал с доктором Б.~Д.~Ф.~Лаубшером, который одно время был психиатром в лечебнице для душевнобольных в Кейптауне. Лаубшер специально занимался прорицателями среди жителей Транскея и написал научную работу под названием <<Пол, обычай и психопатология~--- исследование о язычниках Южной Африки>>. Он убедился, что наряду с многочисленными случаями шарлатанства там можно найти и настоящих ясновидцев. И он рассказал мне о ясновидце, который помог найти украденное стадо скота, угнанное за шестьдесят миль, и назвал вора. Однако доктор Лаубшер не мог дать объяснения этому случаю.

Научные исследования показывают, что телепатия и ясновидение имеют одни и те же общие законы у людей различных национальностей, будь то профессор или неграмотный бушмен, англичанин или африканец. Существует так называемая спонтанная телепатия, когда речь идет о болезни или смерти отсутствующих друзей и родственников и реже о радостных событиях, например рождении ребенка.

Некоторые случаи, о которых я тут рассказал, не прошли строгой научной проверки, чего должен требовать каждый исследователь. Я не очень полагаюсь на примеры и только в том случае поверю, что люди могут говорить или писать на языках им неизвестных, когда лично сумею убедиться, что это не трюкачество и не самообман. Равным образом я сомневаюсь и в способности людей к ясновидению. Предчувствия могут основываться на самом обычном здравом смысле. У меня и у самого бывают предчувствия. Д.~У.~Данн, автор книги <<Эксперимент со временем>>, заметил: <<Если предвидение действительно существует, то это полностью разрушает все наши прежние представления о вселенной>>. Данн верил в предвидение и был убежден, что доказал это с помощью математики. И все же я сомневаюсь в этом.

Однако телепатия кажется мне явлением совсем иного порядка. Возможно, оно слишком волнующее, но уж никак не сверхъестественное. Могу лишь сказать, что мы все еще стоим на пороге знаний о человеческом разуме. Человеческая мысль~--- это загадка. И когда нам смогут объяснить, как мы мыслим, чтение мыслей перестанет быть загадкой.

У неграмотных африканцев развито шестое чувство, которое помогает им не только ориентироваться в незнакомой местности, но и чувствовать присутствие людей, которых им очень хотелось бы увидеть. Тот, кто жил в наиболее глухих уголках Африки, знает об этом даре африканцев. Одни называют эту способность инстинктом, другие считают это психическим явлением. О самом замечательном случае, совершенно достоверном, рассказал мне недавно Г.~Ф.~Вариан, известный строитель железных дорог, автор проекта железнодорожной линии от Лобито до Конго.

В 1907 году Вариан закончил работы по строительству дороги в Родезии и уехал в отпуск в Англию. Там ему предложили на выбор работу в Судане, Перу, Аргентине и Анголе. Вариан выбрал Анголу, сел на пароход и отплыл в Лобито. Он уже навсегда распрощался с Родезией, и там никто не знал о его новом назначении.

Несколько месяцев спустя, когда он раскинул лагерь на реке Кубал, в удаленном от моря районе, к нему подошли два изможденных и оборванных африканца. Они заговорили с ним на <<испорченном кафрском>>, которым он обычно пользовался в Родезии.

---~Вы не узнаете меня?~--- спросил один из пришельцев.

Вариан с трудом узнал в одном из оборванцев Антонио~--- своего слугу в Родезии. А его спутник оказался поваренком, который работал когда-то у Вариана.

Оба они прошли почти всю Африку с твердым намерением снова встретиться с Варианом. Они покинули низовья Замбези приблизительно в то время, когда Вариан решил уехать из Лондона в Лобито. За время своего долгого путешествия они натерпелись всяких мук и почти умирали от голода, но ничто не могло поколебать их решимости. И вот наконец они нашли своего прежнего хозяина в том месте, где он, по их убеждению, должен был быть.

---~Я без конца расспрашивал Антонио, стараясь разрешить эту загадку,~--- сказал мне Вариан.~--- Я хотел узнать, почему он пришел ко мне и как он меня отыскал. Но я и до сих пор не знаю этого. На все мои вопросы Антонио отвечал: <<Так подсказало мне сердце>>.

\chapter{Королева дождя}

Миллионы африканцев верят, что с помощью тайных обрядов можно влиять на погоду, особенно на выпадение дождя. Бог дождя требует жертв. Иногда даже человеческих жертв. Да, старые жестокие обряды Африки все еще живы, хотя они и сохраняются в глубочайшей тайне.

Там, где дождь~--- источник жизни, а засуха несет голод и даже смерть, заклинатели дождя процветают. Без сомнения, в давние времена предки многих нынешних вождей были заклинателями. В каждом племени заклинатель действует по-своему. Он может и не верить в своего идола. Несомненно лишь одно: заклинатель дождя~--- тонкий знаток природы. Он не станет пускать в ход свою магию, прежде чем не заметит передвижение муравьев, кваканье лягушек и другие благоприятные признаки.

Джон Гантер писал, как в Южной Родезии колдуны осудили человека, который якобы обольстил богиню дождя. Его сожгли заживо. И тогда начались дожди. Случайное совпадение? Возможно. Я знал об этом деле не только из официальных сообщений и отчетов судьи, но и по рассказам полицейских и другим источникам. Эта история имела необычные последствия. Я имею в виду не дожди, которые начались вслед за гибелью африканца.

Был 1922 год. В отдаленном районе на севере, близ границы с Португальской Восточной Африкой, на землях мтувера, одного из племен банту, жители все еще не расстались с жестокими языческими обрядами. Полицейский участок был расположен там в Маунт-Дарвине. Капрал, начальник участка, понятия не имел о том, что происходит в горах Мавурадонна, всего лишь в нескольких милях от Маунт-Дарвина. И только засушливым летом 1922 года полиция наконец узнала о страшном ритуале.

Засуха эта была самым серьезным бедствием со времен голодного 1890 года, как раз накануне прихода в страну первых европейских поселенцев. Уже были принесены все необходимые в таком случае жертвы~--- пиво, которое разбрызгивали по земле мужчины с бритыми головами, чернопегие коровы, традиционная голубая одежда. Но дождевые тучи не появлялись.

Люди мтувара чтили дух предка Мвари, у которого на земле оставалась жена. Этой богиней дождя обычно была молодая девушка, девственница. И когда нужен был дождь, она просила об этом своего мужа. Ее могли сместить и заменить другой девушкой. В ту пору богиней дождя была худенькая девочка, по имени Неческва, что на языке мтувара означает <<та, которая обладает силой вызывать дождь>>. Неческва жила в священной роще Мити Мчена, что значит <<место, где растут белые деревья>>. Ее обслуживали женщины, которые обрабатывали землю и готовили ей еду. И только один мужчина имел право находиться в роще. Это был ее телохранитель Чиганго~--- жрец и вождь племени мтувара.

Кто-то пустил слух, что Неческву обольстили. Бог дождя разгневан, и, пока не найдут обольстителя и не принесут его в жертву, дождя не будет. Чиганго не возражал против человеческих жертвоприношений. Позднее выяснилось, что еще и до этого он был причастен к смерти трех человек, которых сожгли на костре, привязав к столбу. Но на этот раз все единодушно признали, что соблазнителем был Мандуза, сын Чиганго.

Чиганго не мог нарушить традицию и спасти своего сына. С одобрения верховного вождя Чизвити он собрал <<хондо>>~--- отряд из семидесяти копьеносцев, поставил во главе него человека, по имени Чиризери, и приказал ему арестовать Мандузу. От дальнейших дел Чиганго отстранился. Ведь не мог же он собственноручно зажечь погребальный костер своему сыну.

И вот однажды рано утром Мандузу вызвали из хижины. Он, должно быть, понял, в чем дело, поджег свою хижину и попытался уйти под завесой огня и дыма. Жене Мандузы удалось скрыться, но самого Мандузу поймали и связали. Его прикрутили к шесту и так несли некоторое время. Потом Мандузу отвязали, и четыре человека подвели его к месту, которое называлось Ньяма Кунгва~--- <<мясо для ворон>>. Пока один из конвоиров держал пленника, остальные разложили костер. Мандузу снова привязали веревкой из луба к шесту и положили на погребальный костер, добавив туда сухой травы. Обычай требовал, чтобы костер разжигали по старой традиции, получая искру трением. Когда костер был разожжен, мужчины ушли. Вслед им неслись крики обезумевшего от боли Мандузы. А ночью пошел проливной дождь. Это убедило всех жителей в справедливости наказания.

Через несколько дней мимо этого места проезжал полицейский. Он увидел обуглившийся человеческий череп, ребра, кости и кучи золы. Началось расследование. Допросили жену Мандузы. Было арестовано семь человек. После предварительного допроса в Маунт-Дарвине их отправили в Солсбери, где их судил главный судья сэр Кларксон Тредголд и суд присяжных. В числе обвиняемых был верховный вождь Чизвити, жрец Чиганго, глава <<хондо>> Чиризери и те четверо, которые сожгли Мандузу.

Всем присутствовавшим на суде было ясно, что подсудимые не понимают, в чем их вина. Ведь в жертву был принесен человек виновный, и сразу же после этого, как они и думали, пошел дождь, так что никто из них не считал это каким-то особым преступлением.

Защита указывала на то, что у этих людей была иллюстрированная Библия, изданная в Машоне. На одной картинке там был изображен Авраам, который готовился принести в жертву своего сына Исаака. Таким образом, простое знакомство этих людей с религией белого человека лишь укрепляло их веру в необходимость подобных жертвоприношений.

На суде давал показания один человек, хорошо знавший жизнь африканцев. Кларксон Тредголд спросил у него, проводилось ли систематическое обучение местных жителей обычному праву (в понимании этого права белыми). <<Нет, не проводилось>>,~--- последовал ответ. <<Они узнают наши законы только тогда, когда их нарушают>>,~--- заключил Кларксон Тредголд.

Один свидетель-африканец заявил: <<Если человек соблазнил богиню дождя, его надо сжечь заживо. Только тогда пойдет дождь, и так было уже много веков>>. Другой свидетель сказал, что приказ дает верховный вождь и никто не смеет его ослушаться.

Когда судья спросил у Чиганго, хочет ли он что-нибудь сказать, тот ответил: <<Я просто следовал обычаю племени>>.

Защита просила принять формулировку <<виновны, но невменяемы>>, учитывая образ мышления и суеверия обвиняемых.

В среду 23 мая 1923 года в четыре часа дня суд присяжных удалился на совещание. В это время в Маунт-Дарвине, за сто пятьдесят миль от Солсбери, в полицейском участке собралось несколько европейцев, чтобы узнать о приговоре по телефону. И тут капрал Трент записал беседу, вероятно самую странную из всех, когда-либо записанных полицией. На участок пришел Кусеквенья, второй сын Чиганги, здоровый африканец шести футов ростом. Он положил свои копья и поздоровался:

---~Мамбо!

Трент спросил, зачем он пришел.

---~Я хочу знать, что с моим отцом и со всеми другими,~--- заявил Кусеквенья.

---~Пока неизвестно,~--- объяснил Трент.~--- Приговор скажут по телефону, и мы все тут ждем.

В тот момент на часах было десять минут пятого.

---~Могу сказать,~--- произнес Кусеквенья, слегка улыбаясь,~--- что моего отца не повесят и других тоже. Мой отец вернется в свой крааль до наступления дождей.

В это время зазвонил телефон. Капрал снял трубку и стал слушать. Все видели, как он изменился в лице. Кусеквенья окинул всех торжествующим взглядом.

---~Чизивити оправдан,~--- повторил Трент.~--- Остальные приговорены к смертной казни, но суду очень рекомендуют проявить милосердие, и судья заявил, что приговор будет изменен.

На земли мтувара снова пришло лето. Пыльный октябрь сменился таким знойным ноябрем, что европейцы не могли спать. Декабрь тоже был сухой, и полиция напомнила жителям, что не допустит жертвоприношений. Незадолго до рождества пришло известие: Чиганго освобожден по состоянию здоровья и возвращается домой. В Маунт-Дарвине ему отвели на ночь хижину, а на следующий день туда прибыли носильщики и на носилках доставили Чиганго в его крааль.

В ту же ночь послышались раскаты грома. Вдалеке засверкала молния, и наконец на землю упали первые крупные капли дождя. Тридцать шесть часов подряд иссушенную землю заливали потоки дождя. Посевы были спасены. Через три дня Чиганго умер. Все эти сведения взяты из протоколов британской южноафриканской полиции. Для тех, кто верит в телепатию, интересно узнать об этом удивительном случае в далеком африканском буше. Кусеквенья пришел в полицейский участок в Маунт-Дарвине как раз в тот момент, когда присяжные выносили свой приговор. Предсказание Кусеквенья поразительно, если только не считать его случайным совпадением. А что касается дождя, то это, несомненно, простое совпадение, хотя Джон Гантер и верит в колдовство.

Интересна последняя запись в личной записной книжке покойного Кларксона Тредголда. Ученый судья не верил, что богиню дождя Неческву действительно соблазнили. Это было придумано нарочно, чтобы найти необходимую жертву. Мандуза, погибший в пламени костра, был невиновен.

Последней великой королевой дождя в Трансваале была знаменитая Муджаджи. Эту высохшую старушку знавал Райдер Хаггард. Он изобразил ее в одном из своих романов. Много лет спустя генерал Смэтс писал: <<Эта женщина поразила меня силой своего характера и непостижимой властью~--- она была настоящей королевой>>.

Среди своих соплеменников небольшого племени ловеду Муджаджи называлась Преображающей Облака. Когда-то ловеду жили к северу от реки Лимпопо, но около 1500 года они перебрались к подножию гор Заут-пансберг. С ними переселилась и их королева дождя.

Издавна уже повелось, что в старости королева дождя должна была передавать секреты ремесла своей дочери или другой молодой женщине, а затем совершить ритуальное самоубийство, приняв яд. Миссионеры убедили Муджаджи нарушить этот жестокий обычай, и она умерла собственной смертью.

Муджаджи стала королевой дождя в начале нашего столетия. Год от года росла ее слава и ее влияние. Она унаследовала глиняные <<котелки дождя>>, в которых хранились снадобья, якобы способные разверзать небеса. Она устраивала танцы дождя, которые сопровождались барабанным боем. Два этнографа, присутствовавшие на одной из таких церемоний, рассказывали, что чистые, серебряные тона свирелей создавали впечатление колокольного перезвона. Муджаджи, конечно, была самой удачливой королевой дождя. Бывали времена, когда земля страдала от избытка дождей, и тогда к ней стрепетом приходили вожди и просили послать сухую погоду.

Несколько лет назад мой друг Т.~К.~Робертсон разговаривал с Муджаджи и пытался выведать у нее секреты. Робертсон согласен со мной, что всему можно найти разумное объяснение вне сферы магии. И на мой взгляд, его суждения об искусстве Муджаджи верны.

Крааль Муджаджи, как заметил Робертсон, был построен на гребне горы, один склон которой обращен на юго-запад, другой~--- на северо-восток. Северо-восточный склон был покрыт лесом из саговника~--- древнего и необычного растения. Эти заросли саговника, вероятно, самые крупные в Южной Африке. Ботаники отмечают, что для такого леса нужны особые климатические условия, чтобы он мог уцелеть на северо-восточном склоне. Саговник обычно растет на склонах, открытых влажным ветрам с Индийского океана, и очень чувствителен к переменам погоды.

---~Муджаджи и ее предшественницы, очевидно, очень внимательно изучали поведение саговника,~--- сказал мне Робертсон.~--- И им часто приходилось говорить людям, что еще не пришло время вызывать дождь. Но вот в один прекрасный день они замечали перемены в листве саговника и другие признаки приближения дождя и разрешали приступить к церемонии.

Сам очень тонкий знаток и большой любитель природы, Робертсон привел еще один пример большой чувствительности к перемене погоды, о котором ему сообщил доктор Т.~Г.~Нел, биолог Национального парка Крюгера. Весной стада антилоп импала обычно распадаются, самцы в это время ревут и дерутся между собой. Это признак, что приближается период гона. В одну из весен африканцы сказали Нелу: <<В этом году дожди запоздают: антилопы еще не начали реветь>>.

И действительно, весна выдалась очень засушливая. Однако период гона антилоп наступил позднее, чем обычно, и, когда появились детеныши, пошли дожди, так что кормящие самки были вдоволь обеспечены свежей травой.

У народа свази время посева маиса называется месяцем импала. Рев этих антилоп служит сигналом к началу сева. Робертсон считает, что антилопы импала, как и ряд других животных, чувствуют перемену погоды за несколько недель и даже месяцев. Цапля голиаф, обитающая на реке Вааль, в те годы, когда бывают наводнения, вьет гнездо выше, чем обычно. Все это объясняется не психикой животных, а их инстинктами, которые вырабатывались на протяжении многих тысячелетий в условиях, где существование зависит от сезонного распределения дождей и травяного покрова.

Что же касается африканских заклинателей дождя и богинь дождя, то здесь нет никакого колдовства. Сами они не могут чувствовать погоду, зато умеют отлично наблюдать за животными и растениями.

Бум\ldots тэп\ldots бум! Африка, как огромный резонатор, разносит этот не тронутый веками древний призыв. Настойчивый, монотонный, временами раздражающий. И когда уже все остальные звуки душных джунглей постепенно изглаживаются из памяти, этот барабанный бой все еще отдается в ушах. Барабаны Африки забыть нельзя. Они вносят ритм в бесконечную драму Черного континента. Барабаны~--- одно из чудес Африки, их звуки несутся к кромке горизонта, как мощный человеческий голос.

Я слушал барабаны на западном побережье, от Сьерра-Леоне до Бомы. Слушал их в Конго во время бессонных ночей под противомоскитной сеткой, когда их низкие звуки то вздымались, то затихали, дрожа и трепеща в лесной чаще. И снова я слушал их в Восточной Африке, припоминая пословицу на языке суахили: <<Когда бьют в барабан на Занзибаре, танцует вся Африка вплоть до Великих озер>>.

Ни одно событие в тропической Африке~--- будь то рождение или смерть, охота или война~--- не обходится без барабанов, которые разносят новости из одной деревни в другую. Европейцы называют их <<телеграфом буша>>~--- очень образное название этого способа передачи новостей на большие расстояния там, где никогда не было телеграфных проводов.

<<Сначала бог создал Барабанщика, Охотника и Кузнеца>>,~--- гласит предание одного из крупнейших племен Западной Африки. Несомненно, в Западной Африке самые искусные барабанщики. Их барабаны буквально разговаривают. <<Телеграф буша>>, о котором говорится в тысячах сказок,~--- совсем не миф. Но лишь совсем недавно европейским исследователям удалось наконец выяснить, как же именно барабаны передают информацию.

Барабанщик в Западной Африке~--- лицо важное, и во многих племенах у него нет больше никаких обязанностей. У барабанщиков есть свой бог, и не кто иной, как Лунный Человек. Когда наступает полнолуние, можно увидеть, как этот бог держит палочки над барабаном. Если палочки опускаются, значит, где-то умер барабанщик. О важности барабанщика можно судить хотя бы по тому, что некоторые народы Западной Африки верят, что барабанщик может передавать вести своим предкам, населяющим мир духов.

\begin{figure}[ht!]
\centering
\includegraphics[width=90mm]{000006.jpg}
\caption{Барабаны~--- одно из чудес Африки}
\label{overflow}
\end{figure}


<<Рам, рам, рам! Бум, тэп, бум!>> Вслушайтесь в дикую музыку барабанов, а ваш слуга-африканец разъяснит вам ее смысл. Без барабанов не обходятся ни пиршество, ни похороны, ни переговоры, ни танцы. Верно говорят, что барабан заменяет африканцу граммофон, оркестр, радио, телефон и телеграф.

В далекие суровые времена новый большой барабан окропляли кровью человеческой жертвы. Считалось, что барабан не может <<говорить>> должным образом, пока не услышит предсмертного человеческого вопля. Один вождь на Нигере так гордился сделанным по его заказу исполинским барабаном, что принес в жертву его мастера, чтобы тот не мог сделать лучшего барабана для другого племени. Новичку опасно играть на таком барабане, потому что палки слишком сильно отскакивают от натянутой кожи и могут вывихнуть плечо барабанщика.

На какое расстояние разносится бой барабана? Не так давно в районе водопада Стенли на реке Конго существовал барабан, звуки которого в ночное время тренированное ухо могло уловить и понять в Ятоке, в двадцати милях от водопада вниз по течению. Я думаю, что это расстояние рекордное для Африки. Конечно, здесь сыграла свою роль река. При другом рельефе это было бы невозможно. Средняя слышимость барабанных сигналов лежит, вероятно, в пределах пяти миль днем и семи миль ночью. Потоки горячего воздуха ухудшают слышимость, и, если барабанщику нужно передать новости на значительное расстояние, он должен делать это ночью или на рассвете.

Так как передача вестей происходит по эстафете, барабанный разговор ограничен не только расстоянием, но и языковыми барьерами. Известия о знаменитом путешествии Стенли по реке Конго в 1877 году обгоняли самого путешественника на тысячу миль. Это один из тех случаев, когда с достоверностью был определен радиус действия <<телеграфа буша>>.

Другой замечательный случай передачи вестей на большое расстояние отмечен в Бельгийском Конго в период первой мировой войны, когда губернатор получал из Восточной Африки сведения о бельгийской армии. Барабаны передавали сообщения о ходе сражений и о потерях бельгийцев с большой аккуратностью и намного опережали официальные сообщения.

Чтобы получить более полное представление о барабанах, нужно обратиться в прошлое. За последние тридцать-сорок лет радио внесло путаницу в это дело. Теперь поступление новостей в отдаленные уголки можно объяснить наличием радиосвязи.

В те времена, когда в Америке Джефрис и Джонсон дрались на ринге за мировое первенство, известный охотник Арчер Рассел был в одной деревушке у истоков Конго, в четырехстах милях от ближайшего телеграфного пункта. И как он заявляет, он узнал о победе негритянского боксера через четырнадцать часов после нокаута. Эта новость, подрывающая престиж белых, была передана барабанами~--- другого способа тогда не существовало~--- и распространилась на обширной территории.

Более ранним событием, которое произвело сильное впечатление на умы африканцев, была смерть Великой Белой Королевы. Сообщение о смерти королевы Виктории сразу передали по телеграфу в Западную Африку. Но в глубь материка, где телеграфа тогда не было, эту весть передали искусные барабанщики. Многие должностные лица узнавали об этом событии от своих слуг за несколько дней и даже недель до того, как поступило официальное сообщение.

Одна любопытная легенда рассказывает, что о падении Хартума население Сьерра-Леоне узнало в тот же день. У меня нет причин сомневаться в этом, так как жители Западной Африки, должно быть, хорошо знали о войне в Судане и пристально следили за событиями. Не надо забывать, что из Судана в Западную Африку издавна пролегал караванный путь через Сахару и город Хартум был известен многим племенам Западной Африки.

Самый крупный знаток барабанов племени ашанти, капитан Р.~С.~Рэтрей, обучался игре на барабане. Вероятно, это был первый европеец, который узнал, что барабанный бой вовсе не африканская азбука Морзе. Барабан воспроизводит гласные и согласные звуки, ударения и паузы. Это по сути дела музыкальный язык. Обычные фразы превращаются в музыкальные такты. По сравнению с языком барабанов азбука Морзе совсем примитивна.

По свидетельству Рэтрея, ашанти при помощи барабанов передавали известия на расстояние двухсот миль с быстротой телеграфа. В случае объявления войны они могли бы за несколько часов собрать в одном месте все свои военные силы.

Предшественники Рэтрея думали, что с помощью барабанов можно передавать лишь самые простые известия о рождении, свадьбе, смерти, пожаре, налете саранчи, охоте, вечеринке, приближении белого человека. Конечно, такие известия передать и принять нетрудно. Их обычно понимают даже те африканцы, у которых нет особого навыка. Но Рэтрей узнал, что африканец может выстукать на барабане даже всю историю своего племени. Это делается во время некоторых празднеств, когда барабанщики перечисляют имена почивших вождей и описывают значительные события из жизни племени.

Своих барабанщиков ашанти называют небесными барабанщиками. По-видимому, они самые искусные во всей Африке. Барабанщики занимают высокое положение при дворе вождя ашанти, они обязаны следить, чтобы хижины жен вождя были в полном порядке. На землях ашанти женщины не имеют права прикасаться к барабану, а барабанщик не смеет переносить свой барабан с места на место. Считается, что при этом он может сойти с ума. Некоторые слова нельзя выстукивать на барабане, они~--- табу. Нельзя, например, упоминать слов <<кровь>> и <<череп>>. В давние времена барабанщику, если он допускал серьезную ошибку, передавая послание вождя, могли отрубить руки. Теперь такого обычая нет, и только в самых отдаленных уголках барабанщик и до сих пор может за небрежность лишиться уха.

Некоторые племена обожествляют барабан и приносят ему в жертву пальмовое вино и дичь. Когда барабанщик умирает, его душа переселяется в барабан. В Западной Африке любовь к барабану зародилась очень давно, почти одновременно с появлением там человека. В начале семнадцатого века английский путешественник Джоб-сон писал: <<Африканцы ни на один день не оставляют в покое свои барабаны. У них вошло в обычай собираться каждый вечер в определенном месте, после того как они набьют свои животы. Они разводят костры и без конца бьют в барабаны, кричат, поют и шумят. Это длится обычно до рассвета>>.

Очень внушительный барабанный бой слышала Западная Африка в те времена, когда султан Сокото строил в своих нигерийских владениях новую дорогу, для того чтобы губернатор сэр Фредерик Лугард мог навестить его. Султан выставил на строительство десять тысяч человек, и в каждой партии был свой барабанщик. Однажды у высохшего русла реки собрались все рабочие, чтобы засыпать русло землей. И все пятьсот барабанщиков тоже оказались вместе. По сигналу одного африканца они начали отбивать очень четкую дробь, и вся армия строителей закончила свою работу в рекордный срок.

Европейцы стараются приспособить барабаны к собственным нуждам. Миссионеры созывают барабанами свою паству. Типичный случай рассказал мне один католический священник, который начал разбивать участок для фермы и хотел вызвать жителей дальних деревень, чтобы они выжгли траву на берегу реки. Все явились в назначенное время и взяли с собой именно то, что было нужно: большие пальмовые ветки, которыми они гасили пламя, когда необходимый участок был расчищен.

Один молодой канадец, который за пять месяцев пересек на машине Африку от Каира до Кейптауна (это было в период между двумя мировыми войнами), рассказал мне еще об одном остроумном применении барабанов. Проезжая по Бельгийскому Конго, он встретил на своем пути участок протяженностью в десять миль, где строилась новая дорога. На этом участке могла проехать только одна машина, и рабочие-африканцы поставили вдоль трассы барабанщиков, которые регулировали движение.

Торговцы используют барабаны для связи с отдаленными торговыми пунктами. Одному торговцу, моему знакомому, удалось сообщить своему коллеге, что того срочно вызывают телеграммой в Лондон и он должен -отплыть с первым же кораблем. Сообщение, конечно, пришлось перефразировать. Океанский пароход на языке барабанов звучал как <<большая, очень большая лодка>>, а Лондон~--- как <<большая деревня белого человека за большой водой>>.

У автомобилистов, проникающих в глухие районы Африки, тоже есть основания быть признательными барабанщикам. Несколько лет назад два брата француза основали транспортную контору в районе Стэнливиля. Как-то у одного из них в ста милях от города лопнула шина, и он не смог ее починить. На следующий день приехал его брат с <<новыми колесами>>, как говорилось в сообщении барабанщиков.

Я знаю и более серьезный случай, когда <<телеграф буша>> сумел передать очень сложное сообщение. Два охотника на слонов поссорились с одним вождем. Вождь был с ними груб, и они беспокоились за судьбу своих ружей и слоновой кости, оставленных в лагере на берегу реки. Барабаны донесли эти вести до африканцев, дружески настроенных к охотникам, и те успели спрятать их вещи.

С помощью барабанов нельзя передавать незнакомые африканцам понятия и имена. Нельзя попросить барабанщика вызвать, скажем, господина Симпсона, если у него нет местного прозвища. Возможно, барабанщик и преодолеет эту трудность, выстукав <<Шимишоно>>~--- так африканцы произносят фамилию Симпсон. Задача барабанщика намного облегчается, если Симпсон носит очки или прихрамывает. Можете быть уверены, о таком человеке знают все жители на сто миль в округе.

Капитаны судов на реке Конго ежедневно прибегают к помощи барабанов. Пароходы жгут там дровяное топливо, и барабаны заранее посылают сообщения на заправочные станции о том, когда придет пароход и сколько дров ему понадобится. Первое мое знакомство с барабанами состоялось во время моего путешествия по верхнему Конго и напоминало театральное представление. К концу дня мы остановились у торговой фактории. Над палубой стелился серый едкий дым от дюжины костров, пока наши пассажиры-африканцы варили на берегу свою сушеную рыбу.

---~Здесь мы и заночуем,~--- спокойно объявил капитан-бельгиец, когда мы сидели под двойным тентом, попивая ледяное пиво.

Вскоре жаркий вечерний ветерок донес к нам по золотистой глади воды едва слышные звуки <<тэп-бум-тэп>>.

---~Говорящие барабаны, -лениво произнес капитан.

Минуту спустя перед нами вырос черный матрос и быстро сказал капитану что-то по-французски. Вялость капитана как рукой сняло.

---~Барабаны разговаривали с нами,~--- сказал, обращаясь ко мне, капитан.~--- Нужно плыть дальше. Там человек с женой и ребенком. Все трое больны и спешат в больницу в Альбертвиль. Молите бога, чтобы в темноте мы не наскочили на мель. Нам придется проплыть двадцать миль.

Гудок сирены, и вот мы уже плывем зигзагами вниз по течению, отбрасывая ил гребным колесом. Несколько часов спустя мы уже подплывали к берегу, где в темноте вырисовывалось здание миссии. К нам на борт поднялся бородатый католический пастор в белом облачении.

---~Хорошо, что вы приехали,~--- воскликнул он.~--- Управляющий рудником и его семья уже вышли. Скоро они будут здесь.

Из темноты пальмового леса показалась цепочка людей, на них упал свет палубных огней нашего парохода. Впереди неверным шагом шел высокий мужчина в изодранной одежде цвета хаки. Его белое лицо было воспалено. За ним величаво следовал неутомимый оруженосец. Следом появилась манила~--- носилки с брезентовым пологом. Полог был откинут, и мне удалось разглядеть изможденную женщину и хрупкую маленькую девочку. (И зачем, подумал я, мужчины берут свои семьи в эту суровую страну?) Процессию замыкали носильщики, которые несли на голове металлические ящики, лагерное снаряжение, тюки с провизией, корзину с детскими игрушками. У парохода некоторые из них в изнеможении опустились на землю. Это была действительно борьба за жизнь, как об этом правильно сказали барабаны, борьба, исход которой решала выносливость этих преданных носильщиков, безжалостное солнце и тропические дебри.

До самого устья Конго я слышал бой барабанов, который напоминал мне об этой несчастной семье. На протяжении всего пути, все эти две тысячи миль, барабаны говорили, радовались, предостерегали, скорбели.

<<Бум\ldots бум\ldots бум!>> Теперь эти звуки становятся громче, потому что мы приближаемся к деревушке, где играет барабанщик. Вот он под крышей из пальмовых листьев бьет в большой барабан~--- огромное, выдолбленное внутри бревно футов двенадцати в длину, с причудливой резьбой. Длинная щель и <<губы>> регулируют тональность звука. Это древнее изумительное искусство требует такого же умения, как и отливка колоколов. <<Губы>> позволяют барабану говорить мужским и женским голосом. Когда вырезают барабан, последний неверный штрих может свести на нет труды многих месяцев.

Музыковеды называют такие деревянные барабаны гонгами. Эти барабаны сделаны по тому же принципу, что и долбленые лодки. Одно племя их так и называет- <<говорящие лодки>>. Барабаны племени ашанти, нтумпане, всегда парные и носят название мужского и женского. На все барабаны натягивают кожу с уха слона. Существует определенная церемония освящения барабана, во время которой мастеру приносят в дар дичь, ром и золотой песок. У мужского барабана звук низкий. Небольшая металлическая пластинка акаса, лежащая на перепонке, придает ему резкость. Женский барабан издает звуки высокого тона.

В Нигерии у народности огбони составляют комплект из пяти барабанов (так называемая <<семья>>), самый большой из них именуется <<быком>>. Звуки этих барабанов напоминают кудахтанье кур, визг испуганного щенка, рычание леопарда и злобный рев слона-отшельника.

Для танцев часто делают барабаны из тыквы, они издают мужской звук. Существуют еще барабаны танге, сделанные из бедренной кости умершего вождя, на концах которой натянута полоска кожи. Играют на танге бамбуковой колотушкой. Барабаны из целой шкуры козла или антилопы, натянутой на плетенную из прутьев раму, называются ндембо. Я уверен, что где-нибудь в Западной Африке наверняка найдется барабан, обтянутый кожей вероломного европейца, который продал в рабство одного из своих слуг и был убит в отместку за это.

Понаблюдайте за выражением лица барабанщика, и вы заметите, как оно меняется при каждом ударе. Мне никогда не удавалось уловить тонкую связь между выражением его лица и содержанием передаваемого сообщения, хотя я почти уверен, что такая связь существует.

Обычно барабаны передают известия на дальние расстояния, но их часто используют также и для местной связи. Есть и другие способы передать вести в соседние районы. В горах на границе Нигерии и Камеруна народ сунквалла использует для этого рога крупных антилоп, буйволов, а иногда и бивни слона. Эти инструменты опять-таки имеют две тональности. Передавать новости могут одновременно два или даже три человека, не мешая друг другу. У каждого рога свой тон, своя длина волны. Поэтому жители деревни, расположенной по другую сторону долины, могут без труда разобраться в этом сплетении звуков.

Один охотник, много лет пробывший во Французском Конго, рассказал мне такой случай. Он шел по району, где свирепствовала сонная болезнь. Все население покинуло эти места. Вдруг он услышал слабую дробь. Кто-то бил палкой по пустому стволу дерева. Обернувшись к своему оруженосцу, он спросил:

---~Ты, кажется, говорил мне, что здесь нет людей? Африканец улыбнулся и ответил:

---~Сокомату.

Они пошли на звук, и вскоре охотник увидел <<сокомату>>~--- обезьяну шимпанзе, так похожую на человека. Она с увлечением барабанила по бревну.

<<Том\ldots том\ldots бум\ldots та-ра-рат\ldots бум!>> Нет ничего удивительного, что, когда европеец пробирается по африканскому бушу, известие о нем намного обгоняет его самого. Где-то там, в ночной тишине, африканцы отбивают этот древний-древний ритм. Доносится слабый ответ, настолько слабый, что отдельные его места восполняют, вероятно, так же, как мы восполняем пробелы в едва слышной, но знакомой мелодии.

Белый человек слышит звуки, только и всего. <<Бум\ldots та\ldots ра\ldots рат\ldots бум!>> Африка слышит их --- и понимает.

\chapter{Самый старый человек в Африке}

Наверное, многие из нас задумывались над тайной долголетия. Когда я был корреспондентом, я разговаривал со многими стариками. Некоторым из них перевалило за сто. И, как правило, это были подвижные и умные люди. Качества эти, по-видимому, играют тут определенную роль.

Самого старого человека я встретил случайно. И мне кажется, что он был самым старым во всей Африке, а может быть, и вообще на земле. В 1935 году я направлялся в Северную Родезию. Наш поезд сделал остановку в Бечуаналенде, на одной из тех пустынных станций, где нет ничего, кроме бака с водой, вывески с названием станции да забора, увешанного кароссами, которые иногда покупают пассажиры. А на этой станции мы еще увидели седовласого и седобородого старца, который, очевидно, был слеп. Его незрячие глаза были закрыты, и тем не менее он казался умиротворенным.

---~Это Рамонотване, самый старый человек из всех, кого вам только доведется увидеть,~--- заметил один пассажир с довольно колоритной внешностью. На нем были бриджи для верховой езды и полотняная рубашка. Я принял его за торговца, который приехал на станцию, чтобы забрать свой товар и хотя бы несколько минут побыть в обществе людей из внешнего мира. Его лицо светилось юмором, й мне захотелось, чтобы поезд немного задержался на этом полустанке.

Улыбнувшись на замечание торговца, я спросил недоверчиво:

---~Сколько же лет старику?

---~Говорят, сто двадцать,~--- ответил он.~--- Во всяком случае я больше, чем он, не протяну: брэнди меня доконает.

Но старику нельзя было дать столько. Может быть, он был чуточку опален пустыней или, я бы сказал, <<умиротворен песками>>.

В пустынях здоровый климат, и Рамонотване прожил еще десять лет, после того как я его встретил. Я всегда сожалел, что под рукой не оказалось переводчика и мне не удалось познакомиться поближе с этим старцем. Я просто увез с собой впечатление о человеке, который всю свою долгую жизнь грелся в лучах южноафриканского солнца и находил в этом удовольствие.

Большеголовый, но не очень высокий, старец относился именно к тому типу людей, которые живут долго. Худощавые, среднего роста, очень подвижные люди скорее доживают до ста лет, чем грузные гиганты. К тому же мне помнится, что у Рамонотване сохранились почти все зубы. Очень старые люди наверняка вам скажут, что они дожили до такого возраста благодаря своим, пусть даже стершимся, зубам, которыми они могут пережевывать любую пищу. И лучшие вставные челюсти не идут ни в какое сравнение с настоящими зубами.

Ученые Южной Африки с большой осторожностью относятся к сообщениям о столетних африканцах, но рядовая публика воспринимает это не так критически. Я не придал никакого значения словам того пассажира, определившего возраст старика в сто двадцать лет. Однако, к моему удивлению, в том же году о Рамонотване написали в газетах, так что он вполне может отказаться самым старым человеком на Африканском континенте.

Слава, однако, пришла к нему не сразу. Соплеменники Рамонотване на протяжении многих лет видели в нем почтенного и милого старика, умного человека, обладающего к тому же и чувством юмора. Миссионеры и чиновники также отлично знали о Рамонотване и ценили его как свидетеля истории Бечуаналенда. Что же касается людей из внешнего мира, то для них рассказ о человеке, который прожил сто двадцать лет, показался необычным и вызвал большой интерес.

Как можно установить возраст африканца, у которого нет свидетельства о рождении? Нужно быть историком, хорошо знающим жизнь племени и даты тех событий, которые чаще всего остаются в памяти африканцев. Войны и сражения, даты рождения и смерти великих вождей, основания городов, кометы и эпидемии чумы, необычайные наводнения и снегопады, засухи, бури и землетрясения~--- все эти события остаются в памяти людей и служат вехами века. Когда в 1951 году проводилась перепись населения Южной Африки, переписчиков снабдили списком наиболее значительных дат начиная со смерти короля Чака (1828). В этом списке были такие события, как основание Ловдейла (1842), массовый падеж скота (1857), поход Адама Кока (1862), первое появление на полях маиса (1865), взрыв порохового склада в Кокстаде (1878).

Если опрашиваемый говорил, что он родился в прошлом веке во время великого голода, его рождение датировалось 1885 годом. Многие африканские старики помнят пятидесятую годовщину царствования королевы Виктории (1887), приезд сэра Генри Лоха (1891), налет саранчи, который случился годом позже. Из событий нашего века в памяти многих тысяч африканцев остался неурожай 1904 года, комета Галлея (1910), первая перепись населения ЮАС (1911), гибель в 1917 году транспорта <<Менди>>, на борту которого были законтрактованные африканцы.

Впервые написал о Рамонотване редактор иоганнес-бургской газеты <<Стар>> Ф.~Р.~Пейвер, человек с большим талантом исследователя. Сведения о его находках публиковались газетами Южной Африки и по телеграфу передавались в Европу. Пейвер описывал Рамонотване в 1935 году в деревне Каламаре, среди гор Шошонг, в двадцати милях к востоку от железной дороги. Пейвер приехал туда вместе с подполковником Жюлем Эллен-бергером, бывшим комиссаром протектората Бечуана-ленд. Подполковник тридцать семь лет прожил в Бечуа-наленде и отлично говорил на нескольких местных языках.

Рамонотване заявил, что он знал Сешеле, вождя баквена, о котором упоминает Ливингстон.

---~Я родился на несколько лет позже Сешеле,~--- сказал Рамонотване.

Это был первый ключ к разгадке возраста Рамонотване. В Бечуаналенде Сешеле считают знаменитостью. Он жил в прекрасном доме с зеркалами, часами, серебряным чайником и прочими благами. По мнению охотника Селуса, Сешеле был самый цивилизованный африканец из всех, кого он когда-либо встречал. Родился Сешеле в 1811 году.

Рамонотване сказал также, что, когда он еще мальчишкой пас коз, был убит Макаба. Известно, что в 1825 году Макабу, вождя бангвакетсе, навестил миссионер Моффрат. А в 1826 году, когда путешественник и торговец Эндрю Геддес Бейн приехал в Бечуаналенд,. ему показали место, где был убит Макаба.

Бечуанские юноши отбывали военную службу в возрасте от четырнадцати до семнадцати лет. Рамонотване рассказал о своих боевых подвигах, до того как он попал в плен к воинам матабеле, которые под предводительством Мзиликази вторглись в Бечуаналенд. Произошло это приблизительно в 1832 году.

Вы, вероятно, помните, что Мзиликази (когда я учился в школе, это имя писалось как Маселекатсе) был зулусским индуной. После одной битвы он не прислал своему королю Шака военной добычи. Шака бросил против него огромную армию, но Мзиликази скрылся на территории нынешнего Трансвааля. Он и его сторонники стали известны там как матабеле. Десять тысяч жестоких завоевателей, убивавших ради того, чтобы убивать. Молодых девушек они щадили и иногда принимали в свои ряды подающих надежды молодых людей наподобие Рамонотване.

Рамонотване часто пел на зулусском языке хвалебные песни исибонго в честь своего бывшего вождя Мзиликази. Он рассказывал о походе на север, в страну, которая сейчас называется Родезия, и о покорении племен машона. Мзиликази доверял Рамонотване и даже назначил его стражем в своем гареме.

Однажды Рамонотване принимал участие в нападении на европейцев.

---~Я шел в бой с поднятым щитом, как вдруг почувствовал, что у меня отнялась левая рука,~--- рассказывал Рамонотване.~--- Я посмотрел и вскрикнул. На руке не хватало двух пальцев.

Пейвер без труда установил дату этого события, так как оно произошло в Матопо-Хилзе, где единственными европейцами в то время были воортреккеры во главе с Хендриком Потгитером. В 1847 году эти люди переправились через реку Лимпопо на обратном пути в Трансвааль, вскоре после того как они отбили атаку воинов Мзиликази. В 1869 году Мзиликази умер\footnote{Младший сын Мзиликази Маквеламбела умер в 1943 году в Эмпандени (Южная Родезия) в возрасте около ПО лет.~--- Прим. авт.}. Незадолго до смерти он вознаградил Рамонотване за хорошую службу, отправив его на родину.

Сопоставив рассказ Рамонотване с другими данными, Пейвер пришел к выводу, что старик говорит правду. Должно быть, Рамонотване родился в 1815 году или около того, то есть во времена битвы при Ватерлоо.

Естественно, всегда найдутся скептики. В 1938 году о Рамонотване заговорили в Лондоне. Газета <<Таймс>> поместила портрет Рамонотване и предоставила свои страницы для очень интересной полемики. Покойный сэр Джон Гаррис, глава Общества по борьбе с рабством и защите аборигенов, ездил в Бечуаналенд в начале того же года. Он выяснял на месте возраст Рамонотване. Вожди и миссионеры, должностные лица и друзья Рамонотване сообщили, что старик родился в том же году, что и Секгома, отец Камы. Гаррис подсчитал, что Рамонотване сто сорок лет, и посоветовал направить в Бечуаналенд компетентного ученого, чтобы он проверил этот факт.

В спор вмешался полковник Г. Маршалл Хоул, автор книги <<Основание Родезии>>, и убедительно доказал, что некоторые даты, приведенные Джоном Гаррисом, неточны. Но даже и в этом случае Рамонотване все же очень стар.

В полемику сразу же вступил доктор Морис Эрнест~--- биолог и основатель лондонского Клуба столетних. Этот клуб занимался изучением условий и средств, которые способствуют сохранению здоровья и силы и помогают прожить свыше ста лет. Эрнест исследовал много случаев долголетия и заявил, что самому старому человеку, которого он встречал, было сто тринадцать лет. Этого возраста достиг канадец французского происхождения Пьер Жубер, умерший в начале нашего столетия. Эрнест предложил Джону Гаррису двести пятьдесят фунтов стерлингов, чтобы Рамонотване прислали в Лондон для обследования.

Неразумное предложение, потому что наука не знает способов для определения возраста человека. Кроме того, как правильно заметил Джон Гаррис, перевозка старика из здорового климата пустыни в Лондон не пошла бы ему на пользу.

Полемика эта, как и следовало ожидать, ничего не дала. Однако в августе 1938 года крааль Рамонотване навестил один человек, выводы которого представляют для нас интерес. Это был покойный доктор Роберт Брум, этнограф, член королевского общества, сам доживший почти до ста лет. Брум заручился поддержкой Тшекеди Камы, регента Бамангвато, который дал ему в переводчики своего писца.

Когда Брум пришел к Рамонотване, тот, как всегда, грелся на солнце. Он был в шерстяной шапке, рубашке и брюках цвета хаки, старой летной куртке и сандалиях. Рамонотване показался Бруму упитанным, крепким и не очень морщинистым. Сердце и пульс у него были в норме. Слепота объяснялась катарактой. На лбу виднелся шрам от удара дубинкой. Брум подчеркивает, что медицинское обследование не принесло результатов. Возраст старика установить не удалось. Ученый, однако, принимает утверждение, что Рамонотване появился на свет на несколько месяцев раньше Секгомы, отца Камы, который родился примерно в 1797 году. Значит, старику было сто сорок лет. А так как он умер семь лет спустя (в 1945 году), то, если верить Бруму, всего Рамонотване прожил почти полтораста лет.

Я не могу согласиться с этим заявлением, хотя оно и принадлежит такому видному ученому, как доктор Брум. Очевидно, дата рождения Секгомы была ошибочна, а может быть, ошибался Рамонотване, думая, что он родился в том же году, что и Секгома.

Пейвер считал, что Рамонотване родился где-то между 1815 и 1817 годом, скорее всего в 1815 году. Он расспрашивал семидесятилетнего сына Рамонотване, и тот сказал, что всегда помнил своего отца седым. Однажды кто-то спросил Рамонотване, сколько у него детей. Старик ответил:

---~Если считать их до захода солнца, то и тогда всех не перечесть.

Рамонотване помнил миссионера Роберта Моффата, который впервые приехал в Бечуаналенд в 1820 году, и Давида Ливингстона, проходившего по этим местам в середине прошлого века.

На память об этой интересной встрече доктор Брум подарил Рамонотване новое теплое одеяло. <<Секрет долголетия старика наполовину заключается в его зубах,~--- заявил доктор Брум,~--- наполовину в его бодром духе. Он весь проникнут энтузиазмом, юмором и пафосом>>.

Я согласен с Мечниковым, что организм людей, которые постоянно едят простоквашу (например, русские крестьяне), меньше отравляется кишечными отходами, и поэтому они доживают до глубокой старости. Простокваша входила в простой рацион Рамонотване, который состоял в основном из каши и небольшого количества мяса. Он старался каждый день принимать горячую ванну и считал, что такая роскошь в Калахари необходима. Одно время Рамонотване курил даггу, но вождь запретил ему это. Курение, очевидно, не принесло ему вреда. Я не собираюсь уточнять возраст Рамонотване, но готов поверить, что в свое время он был самым старым человеком в Африке.

\chapter{Цыгане Нила}

Как-то летней ночью в Каире я сидел со своей подругой Шахразадой в плавучем кабаре. Цыганка исполняла танец живота. Мужской аудитории это никогда не надоедает. Цыганские девушки танцевали в Египте с незапамятных времен. Девушка, которая появилась в тот вечер, была лишь в одной юбке, и ее браслеты позвякивали в такт музыке.

Оркестр играл восточную мелодию, и девушка двигалась перед ним в странном ритме танца, эротического и пленительного. Как змея, подумал я, как змея, танцующая под флейту заклинателя. Своими едва заметными движениями она захватила зрителей. Никакой работы ног, ничего, кроме ритмичного покачивания, кроме изумительного подчинения тела тактам трепещущей музыки.

Шахразада объясняла мне многое в египетской жизни, и она всегда говорила правду, если только у нее не было личной заинтересованности. Я был рад, что в тот вечер Шахразада оказалась в кабаре. Она рассказала мне о цыганах.

---~Вы любите тайны~--- так вот вам тайна,~--- заметила она.~--- Кто такие цыгане? Взгляните на лицо этой девушки. Смелые, темные, раскосые глаза, высокие скулы. Она~--- неотделимая часть Египта, и все же она похожа на остальных цыган, разбросанных по всему свету. Откуда пришли цыгане? Они считают, что здесь, на берегах Нила, их родина. Но интересно, как они могут видеть будущее?

\begin{figure}[ht!]
\centering
\includegraphics[width=90mm]{000007.jpg}
\caption{Плавучее кабаре в Каире}
\label{overflow}
\end{figure}


Шахразада верила гадалкам, я же отношусь к ним скептически. Но в тот вечер она рассказала мне историю, в подлинности которой я позднее убедился сам. Если вы не объясните все совпадением, значит, как Шахразада, вы верите гадалкам.

В Моски, огромном квартале базаров в Каире, в восьмидесятых годах прошлого века жила старая мудрая цыганка, которую называли Предостерегающая Мать. За свои предсказания она брала большие деньги, и многие солидные туристы обращались к ней за советом.

Однажды в 1882 году левантийский переводчик Пранцини привел к старухе сэра Уильяма Гордона Камминга и Чарльза Инмана Барнарда. <<Один из вас троих лишится головы,~--- предсказала цыганка.~--- Другой станет жертвой несправедливости. А третий проживет долго>>.

Барнард, известный парижский корреспондент газеты <<Нью-Йорк трибюн>>, вспомнил это предсказание через пять лет, когда в Париже увидел Пранцини на гильотине, казненного за то, что он зарезал двух женщин и девочку. Затем выплыло наружу скандальное дело афериста Трэнби Крофта, из-за которого Гордон Камминг вынужден был оставить армию. Сбылось и третье предсказание цыганки, так как Барнард прожил до девяноста лет.

<<Цыган>> означает просто египтянин, и все другие названия, такие, как гитано, фараон и фараон-нефка, тоже связаны с этой страной на Ниле. И ни этнографы, ни историки, ни лингвисты все еще не могут раскрыть этой тайны. Сервантес назвал цыган <<королями природы>>. Другие довольно метко называли их избалованными детьми природы.

Надо сказать, что между египетскими цыганами и миллионами цыган в других странах существует одно различие. Цыганки в Европе и в остальных странах строго держатся единого морального кодекса: они сходятся и выходят замуж лишь за цыган. Но в Египте не носящие паранджу цыганские девушки~--- гавази, или гагар, как называют их феллахи, ведут себя вольно. Цыганки торгуют собой в Ваг-аль-Бирке. Мужья продают своих жен, тогда как в Европе за попытку сблизиться с цыганкой вас могут в кровь исполосовать кнутом.

Египетские цыгане, вероятно, понимают распространенный во всем мире цыганский язык, но они предпочитают говорить на арабском жаргоне, которого не понимает ни один араб. В их жаргоне встречаются слова из хиндустани и фарси, но из всего этого ученые могли лишь сделать вывод, что цыгане бродят по всему свету.

Высказывалось предположение, что цыгане вышли из северо-западной Индии, и живущие там сейчас джаты имеют с ними много общего. Но в очень старых рукописях говорится о цыганских заклинателях змей и гадалках из Египта и Священной земли. Возможно, за тысячи лет до нашей эры часть цыган переселилась в Индию и осталась там надолго, так что санскрит стал их родным языком. Некоторые распространенные цыганские слова~--- <<пани>> (вода), <<мачи>> (рыба) и <<бакра>> (суп), несомненно, индийского происхождения.

В одной легенде говорится, что цыгане~--- потомки Самера, изгнанника, сотворившего для израильтян в пустыне Золотого Тельца. Цыгане всегда занимались обработкой металла. Другая легенда утверждает, что языческие предки цыган были прокляты за то, что отказали в глотке воды богородице и ее сыну, спасавшимся бегством от гнева Ирода. Существует также предание, что основатель цыганского племени выковал гвозди, которыми был распят Христос.

Впервые появившись в Европе, цыгане заявили, что они пришли из <<Малого Египта>> и что они должны идти в Рим, чтобы искупить грехи своих предков. <<Я рассею египтян среди народов и разгоню их по всем странам>>,~--- говорится в Библии. Возможно, эти слова относились к цыганам. Исследователи Библии отмечают также, что цыгане могли составлять часть той <<смешанной толпы>>, которая <<ушла вместе с евреями из Египта>>. Может быть, тогда они и ушли на восток, в Индию, и потом снова вернулись в Египет, в то время как остальные разбрелись по Сирии и Европе.

Джордж Борроу, который в свое время знал о цыганах больше всякого другого англичанина, верил, что они выходцы из Египта. Так считают и сами цыгане. За давностью времени теперь уже трудно докопаться до истины. Фокусники, дававшие представления перед древними фараонами, были цыгане. Да, трудно сказать, откуда взялись эти странные люди, почему они унаследовали ту непоседливость, которая все гонит и гонит их вперед, и почему они радуются, когда смотрят на закат солнца.

Один старый испанский историк заявил, что в восьмом веке цыгане пришли в Испанию с побережий Северной Африки. Однако в Германии они появились лишь в начале пятнадцатого века, когда по селам и городам промчались орды оборванцев во главе с всадниками. Они пронеслись быстро, как хищные птицы, вызывая удивление и гнев. Позднее их стали преследовать, подвергать пыткам, вешать. Говорят, что они были каннибалами, но это неверно. Прочитав старинные записи о цыганах, я увидел, что за многие века они совсем не изменились: <<У них нет ни дома, ни родины, и всюду они чувствуют себя одинаково>>.

Цыгане редко бывают высокого роста. Обычно это смуглые люди с копной густых черных волос, в которых седина появляется лишь в преклонном возрасте. Египетских цыган можно узнать с первого взгляда. Они резко выделяются среди бедуинов и феллахов, особенно женщины, с их броской красотой. Они носят ожерелья из красных бус или золотых монет и медные серьги. Часто у женщин можно увидеть синюю татуировку, и некоторые из них умеют ее делать очень искусно. Татуировка считается одним из видов колдовства, она предохраняет от дурного глаза. При накалывании цыгане применяют какое-то средство, обеззараживающее ранки.

Английский поэт Ньюболт сказал, что самой изумительной женщиной из всех, кого он видел на Востоке, была цыганка, которая плясала на канате перед дворцом одного вельможи в Каире. У некоторых цыганок красивые глаза, и у всех острый взгляд. Когда они говорят с вами, кажется, что они смотрят куда-то вдаль, и <<ничто в мире не может сравниться с этим необычайным взглядом>>,~--- как отметил Борроу. Сэр Ричард Бертон тоже писал о цыганских глазах, глазах, которые пронизывают вас насквозь и в то же время смотрят куда-то мимо вас. Друзья Бертона говорили, что и у него самого был почти такой же взгляд. Они подозревали, что в его жилах текла и цыганская кровь.

Когда мой переводчик привел меня на каирский верблюжий базар, я сразу же понял, что многие из тех жуликов в красных фесках, которые стараются всучить вам дряхлого верблюда, были цыгане. Эти известные во всем мире барышники проделывали с верблюдами те же трюки, что и с лошадьми. Они подновляли им зубы, заставляли блестеть глаза от мышьяка, и на время животные теряли свое упрямство и другие дурные привычки, их хромота исчезала и характерный качающийся шаг верблюда вновь становился молодым и упругим. Верблюды, лошади или ослы. Цыган их знает всех. Он лечит их своими тайными средствами, а потом продает с той изумительной ловкостью, которая отличает всех его собратьев. <<Правоверные, вот благородный скакун, который принесет славу своему господину!>>

Цыгане, обитающие в египетских пустынях, охотятся на газелей и зайцев, следуя своему старинному правилу, что <<дичь не имеет хозяина>>. Однако за дичь они принимают и всю домашнюю птицу.

Цыгане живут в палатках, переезжая от ярмарки к ярмарке и всегда занимаясь мелкой торговлей. Более бедная братия клянчит милостыню, и не без успеха. <<Молочка для ребенка>>,~--- кричит нищая цыганка. И кто же ей откажет в такой просьбе? Рассел-паша, начальник каирской полиции, однажды рассказывал, как молодая цыганка пыталась продать ему своего ребенка. Он ей сразу ответил на языке гавазских цыган, что у него и своих детей достаточно. Цыгане очень удивились, что Рассел-паша знает их язык. Рассел изучал работы немецкого ориенталиста фон Кремера, который в середине прошлого века составил словарь египетских цыган. Цыгане, с которыми встретился Рассел, были уверены, что их язык для всех тайна и его не может понять никто из посторонних.

Среди наиболее достойных занятий цыган~--- плетение корзин и лужение. А как мастера-медники они не имеют себе равных. Используя лишь небольшую переносную наковальню и маленький молот, они могут сделать почти любой котелок. Искусные мастера часто поражались технике цыган. И это еще один из их секретов. Кто, кроме этих бродяг, может выковать из пенсовой монетки чудесный миниатюрный котелок?

Мужчины у цыган часто занимаются физической тренировкой и выступают как атлеты, борцы и боксеры. Цыганские музыканты предпочитают скрипку, хотя в Египте их оркестры представляют собой очаровательную смесь восточных и западных ритмов и мелодий. Цимбалы, лютня, флейта и тамбурин или ксилофон~--- послушайте, как исполняет цыганский оркестр сарабанду или цыганочку, и вы поймете, где черпали вдохновение Лист и Брамс. Цыгане искусно исполняют баллады и народные песни. Даже детей матери учат бренчать на каманге~--- восточной цитре и бить в маленький барабан.

Хотя у цыганки много занятий, но ярче всего ее способности проявляются в гадании. Я уже рассказал один случай, однако объяснить его я не могу. Но в каирской городской полиции у меня был друг, сыщик, который, как и я, считал, что если подольше изучать предмет, то для большинства загадок можно найти разумное объяснение. По его словам, некоторые цыганки применяют при гадании гипноз. Клиента просят сосредоточиться на черном круге, нарисованном на ладони гадалки. Вскоре он теряет над собой контроль и выбалтывает все свои секреты.

Обычно цыганки рассчитывают на свою способность узнавать характер человека. В этих сверкающих глазах бездна интуиции. Богатство и любовь, заботы и опасности. Цыганки знают, как создавать тайны и выдумывать романтические истории, без которых жизнь была бы скучной. Египет~--- очень подходящее место для гадалок. Египтяне, прежде чем взяться за новое дело или выдать замуж дочь, очень хотят узнать, что скажут об этом карты. <<Геззане! --- кричит цыганка. --- Кто хочет узнать свое будущее? Мы покажем хорошее! Мы найдем потерянное! Идите и узнайте свою судьбу!>>

Возможно, что карты появились впервые у цыган. Цыганские карты~--- тарок~--- имеют особое значение, и цыгане хранят это в глубокой тайне. Но египетские гадалки используют также и морские ракушки, которые они прикладывают к уху и потом истолковывают звуки.

Всегда находятся доверчивые люди. Сначала цыганка говорит сдержанно, пока не увидит, что ее слова производят впечатление и что ей удалось проникнуть в скрытые мысли, тайные желания или важные события в жизни клиента. Это обман, но в нем достаточно истины, чтобы вызвать интерес у человека. Некоторые предсказания цыганок могут показаться ясновидением, если они имеют такой же исход, как прорицания Предостерегающей Матери.

Цыганки идут на всякие хитрости. Иногда они оставляют особые тайные знаки для других гадалок, которые придут после них. И вот очередная гадалка, еще не войдя в дом, уже знает, есть ли размолвка между мужем и женой, верны ли супруги друг другу, жаждет ли бездетная женщина ребенка.

По всему Египту встречаются кусты, увешанные исцарапанными костями или другими неприметными вещицами, которые свидетельствуют о том, что здесь проходил цыганский табор. Все события в жизни табора от рождения до смерти, каждая примечательная подробность об окрестностях могут быть описаны при помощи палочек и камешков на перекрестках дорог и на местах стоянок табора.

Кто-то определил религию цыган как веру в гадания. Я уже говорил, что у этого народа нет никакой религии, но многие цыгане заявляют о своей приверженности к церкви той страны, где они кочуют. Леланд, президент Общества, изучающего жизнь цыган, знавший их, как никто другой в его время, называл цыган скромными жрецами религии всех крестьян и бедняков~--- той <<старой веры>>, которая восходит к колдовству. В самом деле, цыгане больше всех способствовали распространению среди народа веры в гадания, магию, симпатические лекарства, амулеты и прочее мелкое колдовство. С незапамятных времен цыганки старались внушить людям веру в свои пророческие способности.

В склепе одной церкви во Франции находится гробница святой Сары Египетской, которую цыгане считают своей покровительницей. Но говорят, что эта церковь стоит на том месте, где происходили языческие обряды. Среди цыган и сейчас еще встречаются огнепоклонники.

По знанию свойств растений цыгане не уступят монахам-бенедиктинцам, хотя лично я предпочел бы лечиться у монахов.

Сами цыгане редко обращаются к врачу. Слабые дети могут и не выдержать кочевой жизни, но остальные растут крепкими. Редко можно встретить толстого цыгана, и это свидетельствует о хорошем здоровье народа. Хотя они к любят разбивать лагерь на берегу рек, но не изводят много воды на стирку. И все же не заметно, чтобы эта их нечистоплотность вредила здоровью. Пока цыган волен бродяжничать, его ничто не тревожит. Единственный для него настоящий отдых~--- движение.

Цыган~--- фаталист, он спокойно умрет, <<когда кончатся его дни>>. Но пока он жив, он считает себя выше обычных смертных. <<Тот, кто никогда не жил, как цыган, не знает, как по-барски наслаждаться жизнью>>,~--- гласит одна старинная пословица, явно цыганского происхождения. И в самом деле, цыгане~--- сыны вольных дорог, они не терпят никакого принуждения, любят уединенные уголки под звездным небом, наслаждаются вином и музыкой, днем находят удовлетворение в звуке подков, а ночью~--- в дружеской беседе у лагерного костра. Образ жизни цыган~--- еще одна их тайна. И всякая другая жизнь не годится для них. Они бедны, но полны поэзии. Возможно, что это самые счастливые люди на земле.

На мой взгляд, цыгане~--- дети Африки. По их собственным преданиям, Египет~--- их родина, где они и до сих пор чувствуют себя как дома. Это тайна для них самих и для всех, кто с ними встречается.

\chapter{Тайны заклинателей змей}

Заклинание змей~--- удивительная и опасная профессия. Почти все заклинатели, которых я знал, умирали от укусов своих змей. Эти бесстрашные люди никак не могли овладеть одним секретом~--- как остаться в живых.

Мне кажется, что искусство заклинания змей возникло в Египте, который был колыбелью многих искусств. Змеи~--- бич для египетской деревни. Может быть, именно поэтому там появились самые искусные охотники за змеями и заклинатели. На берегах Нила я видел представления куда более сложные, чем в Индии.

Кобры были символом царского величия. Тиары в виде кобр венчают головы египетских статуй. Клеопатра погибла от укуса кобры. Маги при дворе фараонов могли превращать змею в палку, повторяя чудо, некогда сотворенное пророком Моисеем. Видимо, они сдавливали змее шею так, что парализовался мозг и змея становилась твердой, как палка.

Африканские колдуны превосходно знают повадки змей. Европейцы в тропической Африке нередко обращаются к колдунам, если заподозрят присутствие змеи в своем доме. И почти никогда не бывает, чтобы мганга не обнаружил змеи и ушел бы без вознаграждения. А что значат пять или десять шиллингов, когда дом избавляется от мамбы?

Обычно колдун приносит с собой свирель и начинает наигрывать свою мелодию в разных частях помещения, ожидая, когда мамба выскользнет на открытое место. Гибкое, грациозное создание, но оно несет в своем зубе достаточно яда, чтобы убить слона. Колдун улучает момент, быстро захватывает змею раздвоенной на конце палкой и бросает в свою сумку. В наши дни это почти всегда мошенничество. Колдун обычно подбрасывает в дом прирученную змею, у которой вырваны ядовитые зубы, а затем силами <<чар>> вызывает ее из убежища.

Лучшим заклинателем своего времени был, вероятно, шейх Муса (арабское Моисей) из Луксора, известный многим тысячам туристов. Дед и отец Мусы тоже были заклинателями и погибли от змеиных укусов. Та же участь постигла младшего сына Мусы, когда он отправился в пустыню за змеями. Муса всегда считал, что и его ожидает такой же конец. И действительно, он умер в 1939 году, когда слишком настойчиво пытался извлечь кобру из ее гнезда.

Шейх Муса никогда не прибегал к обману. До начала представления он разрешал обыскать и даже раздеть себя. Змеи, которых он извлекал из нор под глинобитными хижинами, не были ручными. Он мог почуять скорпиона, затаившегося под камнем, или змею в ее убежище. По словам Мусы, запах змеи напоминает нашатырный спирт.

Монотонным пением Муса выманивал змей из их гнезд и подзывал к себе. Иногда кобра бросалась на него. Муса мягко отгонял ее своей палочкой. Потом кобра поднималась и пристально смотрела на заклинателя. Муса ждал этого момента. Продолжая напевать, он медленно приближался к змее. Затем опускал руку на землю, и кобра клала свою голову ему на ладонь.

Такие представления могли показывать и другие заклинатели, в том числе главный смотритель лондонского зоопарка, по имени Бадд. Номер этот был гвоздем программы очень способного заклинателя Хусейна Миа, который многие годы показывал его в Кейптауне. Но у старого Мусы были и другие поразительные номера, и их могли повторить лишь немногие заклинатели прошлого и настоящего.

Очертив палочкой круг на песке, Муса сажал туда только что пойманную кобру, и она оставалась в этом круге, как привязанная, до тех пор, пока Муса не отпускал ее. Не спорю, многие могут таким же образом заворожить курицу. Но попробуйте проделать это с коброй! В конце Муса сажал в такой же круг четыре или пять змей и всех завораживал. Зрители хорошо видели, что змеи пытались выбраться из круга, но ни одна не уползала далеко, пока Муса на нее смотрел.

Несомненно, своим пением Муса просто хотел воздействовать на публику, так как змеи почти ничего не слышат. Однако они воспринимают высокие звуки флейты. Существует мнение, что кожа змеи или кончики ее ребер реагируют на определенные колебания воздуха, например от шагов по земле. А звуки флейты скорее возбуждают кобру, чем усыпляют.

Понаблюдайте за заклинателем с его плоскими корзинками, и вы увидите, что он выманивает змей не звуками флейты. Заклинатель слегка постукивает но корзинке, и тогда появляется змея. В искусстве заклинателя змей нет ничего сверхъестественного. Но зрители редко понимают, что происходит на самом деле. Им кажется, что змея извивается и покачивается в такт музыке, а в действительности она следует за движениями руки человека. Присмотритесь внимательно к заклинателю, и вы увидите, что умелые движения его руки и тела направляют действия змеи. Он всегда извлекает змею медленно, боясь возбудить ее. Если же змея проявляет признаки раздражения, он кладет ее обратно в корзинку и выбирает для представления другую.

Еще один известный египетский заклинатель змей, Хадж Ахмед, друг Рассел-паши, утверждал, что может заворожить змею свистом. Он поставлял редких змей зоопаркам и производителям вакцины. Хадж Ахмед был членом <<Рифаи>>~--- тайного общества заклинателей змей. Общество это носило религиозный характер и имело строгий устав. Он сделал себе прививку, как и остальные члены общества. Однако полного иммунитета против укуса змей не существует. Его карьера была весьма успешной вплоть до того дня, когда он погиб от укуса кобры.

Рассел-паша держал в штате каирской городской полиции специального эксперта англичанина Бейна. И Рассел, и Бейн изучали технику заклинателей и пришли к одним и тем же выводам. Они считали, что секрет выманивания змей из их убежища нередко заключается в умении заклинателя подражать звукам змеи. Конечно, во время спячки змею ничем нельзя пробудить, но в брачный период заклинатель, имитируя специфическое шипение самки, заставляет самца выползать на звук.

Однако я слышал и другое объяснение, пока был в Египте. Мне сказали, что опытный заклинатель использует змеиные экскременты, запах которых привлекает других змей. По-моему, это объяснение имеет под собой научную основу. Говорят, что этот способ особенно эффективен при ловле гадюк.

Рассел-паша отмечал, что у заклинателя должен быть зоркий глаз и быстрые руки. Я бы добавил к этому способность в любом возрасте ни на миг не отвлекаться от танца змеи. Многие заклинатели погибли лишь потому, что во время представления думали о чем-то другом.

Когда я впервые познакомился с песками и странностями Египта (это было через пять лет после первой мировой войны), мне встретился особый тип молодых заклинателей змей, представления которых были настолько возбуждающими, что правительству пришлось ограничить их деятельность. В кафе на бульваре Порт-Саид или даже на священной веранде отеля <<Шеферд>> эти отчаюги подходили к вашему столику и предлагали посмотреть, как они будут заглатывать живую кобру. Находились любители острых ощущений, которые всегда были готовы заплатить за такое зрелище. Но даже сильные мужчины чувствовали себя при этом плохо, а женщины падали в обморок. В фешенебельных отелях подобные артисты больше не появляются.

Я помню одного молодого парня, который держал в своих длинных черных волосах скорпиона и носил на себе кобру. Некоторые заклинатели смазывали свое тело змеиным жиром, надеясь тем самым снискать милость у змеиного племени. Возможно, это им удавалось. Особенно долго я раздумывал над одним номером. Заклинатель хватал кобру за шею, сдавливал ее так, что раскрывалась страшная пасть, и плевал туда. Не очень эстетическое зрелище. Но реакция змеи была совершенно неожиданной. Мгновенно она деревенела, и ею можно было манипулировать, как тросточкой. Много лет спустя мне объяснили, что в слюне заклинателя был наркотик, который оказывал на змею мгновенное действие. Это как раз один из тех номеров, которые кажутся сверхъестественными.

Некоторые заклинатели, показывая две маленькие ранки на пальце, делают вид, что их укусила кобра. Можете быть уверены, что <<укус>> был там еще до начала представления. Они обычно прикладывают к ранке пористый <<змеиный камень>>~--- средство, которое они никогда бы не применили, если бы их действительно укусила змея.

Заклинатели всегда отдают предпочтение кобре. Несомненно, зловещий капюшон усиливает впечатление от зрелища. Надо сказать, что кобра раздувает свой капюшон лишь в возбужденном состоянии. Следовательно, двигаясь за дудочкой заклинателя, змея не находится под гипнозом и, конечно, она не танцует. Скорее всего, она следит за движениями заклинателя. Разумеется, и заклинатель тоже внимательно наблюдает за глазами змеи, чтобы знать, не собирается ли она вцепиться ему в руку.

В Африке водятся семь видов кобр, и их повсюду так много, что заклинателю ничего не стоит наловить, сколько ему надо. Так называемая египетская кобра, которая встречается от Средиземного моря до Южной Африки, не относится к плюющимся змеям, так же как и капская кобра. Но рингал и черногорлая змея нацеливаются прямо в глаза жертве и поражают ее на расстоянии семи футов. И вы не часто найдете в корзинке заклинателя такую змею. Представление с ней было бы равносильно самоубийству.

Египетские заклинатели часто демонстрируют очень ядовитую рогатую гадюку. Ловят они и опасную ковровую гадюку. Но это очень редкие виды.

Заклинатель Хусейн Миа (о его представлениях в Кейптауне я уже говорил) время от времени посылал за королевскими кобрами в Бирму. Это необычайно эффектная змея и самая крупная среди ядовитых змей. Во время представления она выглядит очень внушительно среди более мелких (но не менее смертоносных) собратьев. Крупнейшие королевские кобры достигают восемнадцати футов в длину. Это каннибалы, они поедают себе подобных. Поэтому заклинатель, имеющий королевскую кобру, может лишиться остальных змей, если будет неосторожен.

К сожалению, королевская кобра не может долга жить в Южной Африке. Хусейн Миа потерял одну за другой четырнадцать дорогих змей. Но когда у него бывали королевские кобры, представления оживлялись. Одни кобры имеют добродушный характер, другие~--- злобный. И все же каждый заклинатель жаждет той бури аплодисментов, которые ему может принести лишь огромная послушная королевская кобра. Эта змея используется в номере <<смертельный поцелуй>>. Иногда его демонстрируют заклинательницы. Чтобы поцеловать кобру в раскрытую пасть, действительно нужен какой-то гипнотизм.

Хусейн Миа очень любил Кейптаун и называл себя Чарли из Кейптауна. Он, как и подобает потомственному индийскому магу, окончил Пунский университет по магии, глотанию огня и заклинанию змей. В Южную Африку Хусейн Миа приехал в конце прошлого века, и вряд ли найдется хоть одна деревня в Северной и Южной Родезии и в Южно-Африканском Союзе, где бы не видели этого бородатого, улыбающегося артиста в тюрбане, с маленьким там-тамом и змеями. Он утверждал, что выступал в Букингемском дворце. (<<Я заставлял змей танцевать для короля Эдуарда и короля Георга>>,~--- хвастал он.) Безусловно, он давал представления в доме правительства в Кейптауне, но его обычное место было у пристани на Эдерли-стрит. Когда же пристань снесли, он стал давать свои представления на плацу.

Среди номеров Хусейна Миа была одна комическая сценка, которую я мог без устали смотреть десятки раз. Хусейн ставил на землю небольшую корзинку с крышкой. Затем выбирал в толпе подходящую жертву, обычно какого-нибудь зубоскала, который насмехался над представлением. Ему предлагалось тщательно осмотреть корзинку и показать всем присутствующим, что она пуста. Хусейн накрывал корзинку куском материи, играл на флейте несколько таинственных тактов, доставал корзинку из-под покрывала и просил вызванного человека опустить в нее руку и взять себе все, что там есть. Ему намекали, что корзинка таинственным образом наполнилась деньгами. В этом и заключался особый успех номера. В следующее мгновение перепуганная <<жертва>> обнаруживала в своей руке живую змею. Это была неядовитая змея, но выглядела она отнюдь не безобидно. Возможно, у меня примитивное представление о юморе, но в своей жизни я редко смеялся так от души.

Хусейн Миа мог давать представление по нескольку часов подряд, не повторяя ни одного номера. Когда его сын Ибрагим был маленьким, Хусейн Миа демонстрировал исключительно отработанный номер с плетеной корзинкой. Ибрагим забирался в корзинку, а отец пронзал ее плетеные бока кинжалом. Но прежде всего Хусейн был заклинателем змей. Он послал сына в Пуну, чтобы тот как следует отшлифовал свое искусство и продолжил дело отца.

Представления Хусейна Миа развлекали меня с самого детства. Когда он умер, я был уже зрелым человеком. Прожил Хусейн Миа до семидесяти пяти лет. Вероятно, это рекордный возраст для людей такой опасной профессии. Во время второй мировой войны на представлении около отеля <<Маунт Нельсон>> его укусила в большой палец правой руки капская кобра. Срочно вызвали его сына, который в тот момент давал представление в другом месте на расстоянии мили. Когда он прибыл, Хусейн был уже без сознания, в больницу его доставили слишком поздно.

Доктор Гамильтон Фэрлей, который интересовался этим опасным занятием, проследил судьбу двадцати одного заклинателя на протяжении пятнадцати лет. За этот срок девятнадцать из них умерли от змеиного яда. Я перебираю в уме всех заклинателей Южной Африки. Самым знаменитым из них был Берти Пирс, известный ученым всего мира. Его основным занятием была продажа змей музеям, а также <<выдаивание>> змеиного яда для сывороток.

Для Пирса с его слабым сердцем это было неподходящее занятие. Каждый укус заставлял его думать, сможет ли он выдержать лечение. Однажды его укусила в руку африканская гадюка. Вакцины у него не оказалось, и он выжег укушенное место. На руке остались ужасные шрамы. Как-то в Кейптауне в отсутствие своего заболевшего помощника Пирс для развлечения публики вошел в яму со змеями. Маленькая кобра укусила его в лодыжку~--- очень опасное место из-за множества расположенных там мелких кровеносных сосудов. Пирса лечили, но на этот раз лечение не помогло. Это был десятый и роковой укус.

Возможно, вы спросите, почему заклинатели не <<выдаивают>> змеиный яд, перед тем как брать змею в руки. Беда в том, что ядовитые мешочки очень быстро вновь заполняются ядом. А заставлять змею перед представлением без конца кусать тряпочку, пока не опустошится весь мешочек,~--- процедура утомительная и долгая. Конечно, заклинатель может вырвать у змеи зубы. Но тот, кто гордится своей профессией, редко идет на это. К тому же лишенные зубов змеи долго не живут.

Однажды доктор Десмонт Фитцсимонс, южноафриканский специалист по змеям и сын известного У. Фитц-симонса из террариума в Порт-Элизабете, увидел представление с гадюкой. Это было настолько необычно, что он стал внимательно приглядываться. Гадюка оказалась безобидной ковровой змеей. Но она была так искусно подкрашена, что издали почти не отличалась от африканской гадюки.

В Южной Родезии, в местечке Синоя, жил колдун, завоевавший славу тем, что бесстрашно брал в руки зеленых мамб. Во время одного из представлений он получил смертельный укус. Местный хирург послал одну из змей колдуна Фитцсимонсу для определения ее вида. Это оказалась светло-зеленая разновидность бум-сленга, или древесной змеи. У бумсленга ядовитые зубы расположены глубоко во рту, на заднем крае верхней челюсти, так что ему редко удается укусить кого-нибудь и пустить свой смертельный яд. Колдуну не повезло. Это и был как раз такой редкий случай. Но теперь, когда установили вид змеи, тайна колдуна раскрылась. Ни один заклинатель, каким бы искусным он ни был, не смог бы безнаказанно устраивать столько представлений с мамбой, подпуская ее к самой флейте.

Заклинание змей восходит, вероятно, к древнему культу поклонения змеям. Каждый храм имел своих змей. Знахари были одновременно и заклинателями, и до сих пор змея символизирует медицину. Поэтому не удивительно, что рифаи, наиболее искусные заклинатели змей Египта, религиозны. Они могут очистить ваш дом от змей. Но при одном непременном условии: змеи должны быть отвезены в пустыню и выпущены там на волю. Несомненно, у заклинателей змей до сих пор еще есть тайны, и они не открывают их никому из посторонних.

\chapter{Испытание огнем}

Всякий раз, когда я слышу звуки там-тамов и свирелей, передо мной из глубины лет всплывает видение огня. Я вспоминаю храм Умбило близ Дурбана и многотысячную толпу, которая устремляется к нему по дороге вдоль побережья. В толпе встречаются и европейцы, но в основном это черноволосые индийцы с гирляндами цветов на шее и в одежде малиновых, шафранных и других ярких тонов. Все они направляются на праздник испытания огнем.

До недавнего времени многие считали, что хождение по огню~--- таинство. Теперь же, как вы увидите, эта церемония отчасти потеряла свою загадочность. Я видел этот древний ритуал и на севере и на юге континента~--- в Каире, Дурбане и Кейптауне. Говорят, что на-учиться этому нетрудно, но сам я все же не отважился бы ступить босой ногой на раскаленные угли.

Звуки там-тамов и свирелей, запах ладана и жасмина\ldots И вот вокруг огненной ямы уже собралась толпа. В этом пламени сгорело шесть тонн дров. Они были разложены в яме четырнадцати футов длиной и десяти футов шириной. Этот свирепый жар обжигает вам лицо и глаза, когда вы приближаетесь к веревочной преграде в тридцати футах от ямы.

Благочестивые индусы следуют за своими жрецами. Они несут странные изображения. Вот верховное божество Брахма и его воплощения Вишну и Шива. Здесь и более мелкие божества: Аннама, Чандасверама, Майесверама, богиня холеры Марама в желтом одеянии, богиня кашля Коккалама, богиня оспы и кори Сукхад-жама. Следует недолгая пуджа~--- чтение молитв жрецами.

\begin{figure}[ht!]
\centering
\includegraphics[width=90mm]{000008.jpg}
\caption{У некоторых молодых мужчин язык пронзен спицами, а в грудь воткнуты десятки серебряных крючков. На некоторых крючках висят небольшие лимоны}
\label{overflow}
\end{figure}



Барабаны и свирели ускоряют темп, толпа ревет. Приближаются сутри~--- индусские огнеходцы. Среди них пожилой седовласый мужчина с белой бородой. У некоторых молодых мужчин язык пронзен спицами, а в грудь воткнуты десятки серебряных крючков. На некоторых крючках висят небольшие лимоны. Это похоже на пытку, и тем не менее большинство сутри не проявляют ни признаков боли, ни страха. Один человек стонет. Ногти на его ногах пропущены через подошвы деревянных сандалий и загнуты кверху. Для него хождение по огню будет облегчением этой добровольной пытки.

Звуки музыки то вздымаются, то затихают, когда ог-неходцы окружают яму. Самый старший из них первым ступает на огненный ковер, сверкающий на ветру. Остальные без колебаний следуют за ним. Некоторые совершают это путешествие дважды, один пересекает четырнадцатифутовую яму три раза. Затем через яму идет женщина с белоснежными волосами.

На церемонии присутствуют шеф дурбанской полиции и два европейских врача. Они внимательно осматривают ноги сутри. На этот раз все обходится без ожогов и волдырей. И ни капли крови в тех местах, где в тело вонзились спицы и крючки. Но не всегда все оканчивается благополучно. Бывает, что некоторые из этих фанатиков умирают или получают сильные ожоги, тогда как другие остаются невредимыми.

В Египте хождение по огню, должно быть, почти так же древне, как и поклонение солнцу. Мой переводчик Ахмед, хороший переводчик, знавший мою любовь ко всему необычному, странному и непостижимому, повел меня однажды в район Моски в Каире, чтобы показать представление, которое давал там марабут\footnote{Мусульманский отшельник.}.

Было время года, которое египтяне называют <<запахом ветра>>. Душное время. В странах с более благодатным климатом оно соответствует весне. Весна, сулящая обжигающий ветер хамсин из южной пустыни и песчаные бури, которые приносят в Каир массу мелкого песка. Весна с ее возрастающей жарой и безветренными ночами. Весна, несущая с собой сухость, от которой пощипывает в носу и раздражается кожа. Я не мог себе представить более неподходящего времени для действа с огнем. И все-таки я пошел туда.

Представление проходило прямо на улице. Марабут оказался пожилым великаном в чалме и длинном га-лабие из голубого полотна. Перед ним была большая жаровня, куда мальчик подбрасывал ветки и раздувал мехами огонь. С одной стороны сидели музыканты с флейтами и барабанами. <<Аллах! Аллах! Аллах!>>~--- монотонно тянул марабут, дирижируя оркестром и настраивая себя для представления.

Наконец он сбросил галабие, и я увидел на его теле шрамы~--- следы истязаний, которым он сам себя подвергал всю жизнь. Взор его стал неподвижным, когда с тяжелым вздохом он шагнул к жаровне. Внезапно марабут схватил обеими руками горящие ветви и стал осыпать ударами свое искалеченное тело с головы до пояса.

---~Это святой человек, много раз побывавший в Мекке,~--- пояснил мне переводчик.~--- Такие люди не чувствуют боли.

Мальчик добела раскалил на огне металлический прут и поднес его к лицу марабута. И тот лизнул раскаленный прут. В заключение мальчик перевернул жаровню, и марабут босиком прошелся по горящим веткам.

---~Он молодец,~--- заключил переводчик Ахмед, когда я дал несколько пиастров помощнику марабута.~--- Такого марабута не каждый день увидишь. Вряд ли есть другой такой святой от Суэца и до Сивы.

Ахмед рассказал мне, что мусульмане обоих направлений, сунниты и шииты, рассматривают хождение по огню как одну из форм покаяния за убийство халифами своих соперников в борьбе за трон. Одним из пострадавших был Хусейн, потомок пророка Мухаммеда, и его смерть оплакивают и сунниты, и шииты. Религиозные обряды хождения по огню всегда совершаются в разгар мухаррама. Мухаррам~--- это время траура, первые десять дней мусульманского года.

Хождение по огню в Африке распространено не только среди индусов и мусульман. Этот церемониал встречается и у африканских народов, в частности у племени вакимбу в Танганьике. Люди этого племени проходят через костер, осыпают себя раскаленными углями, проводят по лицу горящими ветками, выхватывают ртом головни.

Один из колдунов вакимбу пошел еще дальше. Он засунул свою голову в небольшую ямку, обложенную раскаленными камнями, и продержал ее там двадцать минут. Присутствовавшие на этом странном представлении должностные лица засекли время. Перед представлениями подобного рода вакимбу собирают листья определенных деревьев и кустарников, тщательно пережевывают их и смазывают свое тело. Они утверждают, что это магическое лекарство, или дава, защищает их от ожогов. Не знаю, так ли это.

В Кейптауне я видел, как по огню ходил европеец. Он называл себя карасом~--- белым йогом. Это было удивительное зрелище. Карас разложил раскаленные угли на бетонной площадке в саду гостиницы <<Си-Пойнт>>. Вокруг сидели зрители и пили пиво. Однако из-за отсутствия соответствующей атмосферы, которая обычно сопровождает такого рода зрелища, это представление, кажется, не производило на зрителей никакого впечатления. Возбужденная восточная толпа придает этой церемонии гораздо больше таинственности.

Все же этот человек совершил прогулку по горящим углям, не впадая при этом в состояние транса. Когда осмотрели его ноги, оказалось, что они не были ничем смазаны, и тем не менее он их не обжег. Поздно вечером появился еще один доброволец, а может быть, и сообщник караса. Карас стал делать пассы, создавая впечатление, что гипнотизирует добровольца. Затем новичок прошелся по огненной тропинке, не получив никаких ожогов. Благовоспитанная публика наградила его сдержанными аплодисментами, а я сожалел, что не было там-тамов и свирелей.

Как я уже говорил, иногда подобные представления кончались смертью или очень сильными ожогами. Ряд таких случаев со смертельным исходом зафиксировал подполковник Р.~Г.~Элиот, хирург, изучавший различные виды магии.

За несколько лет до начала второй мировой войны в дурбанских газетах появились сообщения о серьезном несчастном случае во время одной из таких церемоний на страстной пятнице. Молодой индиец из числа <<приносящих покаяние>>, дойдя до середины двадцатифутовой ямы, наполненной раскаленными углями, зашатался и упал на четвереньки. До этого уже четверо мужчин благополучно пересекли яму. Казалось, что никто не сможет ему помочь. Юноша подполз к краю ямы, ему протянули руки и вытащили из огня. Он был весь в ожогах. Сначала его положили в неглубокий ров с водой, вырытый около ямы, а затем доставили в больницу. Другие огнеходцы заявили журналистам, что этот человек вышел из состояния транса, поэтому он и упал.

Отчего же во время этой странной древней церемонии одни остаются целыми и невредимыми, а другие получают серьезные ожоги или даже умирают? Несомненно, существуют определенные химические составы, которые в значительной степени защищают от ожогов. Однако врачи и ученые, бесчисленное количество раз исследовавшие подошвы ходящих по огню людей, не нашли никаких признаков химических веществ. Так что это объяснение отпадает. Возможно, ожоги получают те, у кого понежнее кожа. Но все-таки ответ на эту загадку нужно искать в чем-то другом.

Не доказывает ли все это еще раз превосходства духа над плотью. Только до некоторой степени. Люди с плохими нервами, которые имеют глупость так играть с огнем, почти всегда получают ожоги. Гипноз, религиозный экстаз, полная уверенность~--- вот что помогает человеку пройти по углям без всяких колебаний, а это имеет первостепенное значение.

Индусские огнеходцы из Наталя могут рассказать вам, что за десять дней до хождения по огню они переходят на строгую вегетарианскую диету, читают молитвы, принимают ванны и занимаются специальной гимнастикой. Утром в день испытания в храме устраивается особая церемония. Она завершается всеобщим омовением в ближайшей реке. Испытание огнем представляет собой либо покаяние за дурной поступок, либо исполнение клятвы, данной во время болезни или душевного кризиса. Одна индийская девушка из Питер-марицбурга страдала от боли в желудке. Кто-то сказал ей, что она может вылечиться, если трижды подвергнет свою веру испытанию огнем. Девушка так и поступила. Она провела неделю в храме, где питалась только фруктами и молоком, и затем бесстрашно прошла по огненной дорожке, не получив при этом никаких ожогов. Девушка утверждала потом, что боли у нее в желудке прекратились совершенно.

Вера~--- это главное. Некоторые утверждают, что, прежде чем ступить на пылающую тропинку, сутри ждет, пока на поверхности не образуется слой золы.

Это неверно. Несколько лет назад кашмирец Куда Букс ходил по огню перед группой английских врачей и ученых. Перед началом представления он специально смел золу, убедительно доказав, что предпочитает ходить по раскаленным углям.

Куда Букс проходил по огню довольно быстро. Огненная дорожка была двадцать пять футов в длину, три фута в ширину и толщиной двенадцать дюймов. Такие размеры обычно приняты в Индии. Врачи установили, что подошвы ног у Куды Букса нормальны, кожа на них довольно толстая. Их вытерли тампоном и тампон взяли на исследование. Медицинская комиссия с удовлетворением отметила, что Букс не пользовался никакими химическими препаратами, которые могли бы предохранить его от ожогов. После заключительной прогулки на его ногах не было и намека на волдыри.

Любопытно, что во время представления Куда Букс был одет в черный хлопчатобумажный сюртук и брюки, доходившие до щиколоток, и тем не менее его одежда не опалилась. Букс проходил огненную дорожку за шестнадцать секунд.

Покойный доктор Т.~У.~Б.~Осборн, специалист-медик и член парламента Южной Африки, занимался огнеходцами из Наталя, но так и не смог найти соответствующего физиологического объяснения. Он отметил, что в этой церемонии мы имеем дело с жаром, который не вызывает ожогов. Гипнотизер может вызвать ожог без жара. Он дает своему пациенту холодную монетку и говорит, что она раскалена докрасна. Таким образом огнеходец, видимо, вполне владеет собой. Возможно, что из-за сильного возбуждения его симпатическая нервная система напряжена до предела. А в такие моменты кровяные сосуды суживаются, кровь свертывается быстрее, и поэтому опасность повреждения сведена до минимума.

Доктор Осборн проводил параллель с индийскими йогами, которые в течение длительного времени могут переносить сильный холод. Эти люди несколько часов подряд сидят обнаженные на льду замерзшей реки. Некоторые из них живут в пещерах среди снегов Гималайских гор. У них нет огня, и ходят они только в набедренных повязках. Человек с нормальной физиологией не способен выдержать подобных испытаний. На востоке же эти испытания считаются естественным образом жизни верующих.

Профессор Университета святого Эндрюса Дэвид Уотерстон видел хождение по огню на островах Фиджи, устроенное специально для членов Британского медицинского общества. Жители Фиджи ходили босиком по камням, настолько раскаленным, что от прикосновения к ним вспыхивала бумага. Уотерстон высказал предположение, что исполнители долгое время приучали свои подошвы к высокой температуре и добивались того, что не чувствовали сильной боли при хождении по раскаленным камням, а для нетренированных людей такая процедура оказывалась невыполнимой. Таким же образом дети могут научиться ходить по колкому гравию. Граница выносливости повышается.

Сэр Первес-Стюарт, еще один медик, присутствовавший на представлении, усомнился в выводах профессора Уотерстона. Он предположил, что невосприимчивость исполнителей к высокой температуре объясняется либо самовнушением, либо внушением, которое было сделано вождем или жрецом. Религиозный экстаз, заявил Первес-Стюарт, способен на время снять болевые ощущения. Оба ученых-медика с удовлетворением отметили, что исполнители не принимали наркотиков, не пользовались никакими химическими препаратами и что кожа на их ступнях не была чересчур грубой.

Многие из этих объяснений содержат в себе крупицу истины, однако полностью истина еще не раскрыта. На мой взгляд, тайну хождения по огню сумел объяснить доктор философии из Флориды Майн Рид Коу. Просматривая как-то книгу по физике, опубликованную в конце прошлого века, он совершенно случайно наткнулся на описание одного забытого открытия, известного под названием <<эффект Лейденфроста>>.

Лейденфрост первым заметил, что при выливании жидкости на раскаленную металлическую поверхность наблюдается интересное явление. Если накалить докрасна сравнительно толстое серебряное или платиновое блюдо, а затем капать на него водой, предварительно подогретой, то она не растекается по блюду, как это происходит при обычной температуре, а принимает форму сплюснутого шарика.

Шарик быстро вращается по дну блюда, но не закипает. Правда, он испаряется, но в пятьдесят раз медленнее, чем при кипении. Жидкость, принявшая сфероидальную форму, фактически не касается раскаленной докрасна поверхности, по которой катается. Между ней и металлом возникает изолирующая подушка из пара.

Коу проверил <<эффект Лейденфроста>> в своей лаборатории и подтвердил его научную обоснованность. Он сам отважился пройти босиком по раскаленным углям и металлу. Уверенный в своих научных выводах, Коу решился даже на то, что приложил язык к раскаленному докрасна стальному бруску.

Вот истинная тайна хождения по огню. Никаких наркотиков. Никакого гипноза, хотя исполнитель и должен иметь твердую уверенность, что он благополучно выдержит это испытание. Никакой анестезии и никакого обмана. Просто ноги выделяют пот, и образующиеся шарики жидкости предохраняют их от ожогов. Точно так же слюна предохраняет рот глотателя огня.

Теперь понятно, почему Куда Букс сметал золу и ходил по раскаленным углям. Если угли недостаточно горячи, изоляция пропадает, и человек получает ожоги.

Вероятно, немногие огнеходцы догадываются о том, что защищает их от ожогов. Однако они хорошо знают, что долгие прогулки опасны. Можно ходить на двадцать максимум двадцать пять футов. Дело в том, что при каждом шаге подошва находится в соприкосновении с углями менее полсекунды, а каждая ее часть и того меньше~--- какую-то долю секунды. В этом и заключается спасение ходящих по огню. Тому, кто потеряет хладнокровие и на секунду задержится или повернет назад, приходится плохо.

Худини помещал в печь человека и клал туда кусок сырого мяса. Мясо запекалось, так как оно не может выделять пота. А человек выходил оттуда невредимым. Его спасало охлаждающее действие испарения с поверхности его кожи.

Понаблюдайте за женщиной, которая пробует утюг, смочив слюной палец. Это та же магия, что и магия хождения по огню. Водяные пары между утюгом и кожей предохраняют от ожога. Еще большую смелость проявляет паяльщик, когда смоченной слюной ладонью направляет поток расплавленного олова.

Итак, почти всякий может научиться ходить по огню. Это не шарлатанство. Способность ходить по раскаленным углям не есть достояние одних факиров. Смочите руку мокрой тряпкой, а затем окуните ее в расплавленное олово~--- с нею ничего не произойдет до тех пор, пока температура расплавленного металла значительно превышает точку его плавления.

В Натале и до сих пор проводят испытание огнем, правда, теперь уже не в старом храме Умбило. Храм превратился в развалины, заросшие травой и кишащие змеями. Но этот храм видел в свое время великие зрелища, когда из года в год там собирались отважные ог-неходцы и устраивали представления.

В один памятный день три молодые европейские девушки сбросили свои туфли и вслед за индийцами пошли по горящим углям. Когда их увидели на огненной дорожке, в толпе раздались крики. Невероятно, чтобы белые девушки с такой верой и хладнокровием могли отважиться пройти по огню. Свирели и там-тамы играли в тот день для них странную мелодию. И все три девушки прошли по огненной дорожке, не получив при том никаких ожогов. Их защищала собственная смелость и тот таинственный закон физики, который ускользал от многих невнимательных ученых.

\chapter{Великаны и карлики}

Много лет тому назад в редакцию одной кейптаунской газеты, где я тогда работал, вошел великан. Он лизнул почтовую марку и прилепил ее под потолком, как память о своем посещении.

Это был самый высокий белый человек, какого я когда-либо встречал в Африке. Но я как-то совершенно забыл о нем. Через несколько лет он появился снова, показал на почтовую марку, отлепил ее и вручил мне. Звали его Динни Даффи. Жаль, что я не помню, какой у него был рост. Видимо, почти девять футов. Демонстрируя свою персону, он легко зарабатывал на жизнь. По достоверным данным, самым высоким человеком был один русский. Его рост~--- девять футов и три дюйма. Правда, в последние годы я слышал о человеке девяти футов и десяти дюймов.

Некоторые ученые утверждают, что на заре человечества существовали еще более внушительные гиганты. Об этом говорят так часто цитируемые строки из Книги Бытия: <<В те дни на земле жили великаны. И когда сыны бога пришли к дочерям людей и народили им детей, те стали такими же великанами\ldots>>

Профессор американского Музея естественной истории Франц Вейденрейх, один из главных сторонников гипотезы о существовании в древности великанов, считает, что в <<те далекие времена>>~--- полмиллиона или более лет до нашей эры~--- в Африке обитали гиганты. Другие антропологи до сих пор думают, что первые люди были карликами. Они ссылаются на эволюцию мелких млекопитающих (например, подотряд лошадиных), которые постепенно становились крупнее. Вейденрейх исследовал различные черепа и челюсти, в том числе кости массивного Paranlhropus robustus, найденного Брумом в Южной Африке. Большое впечатление на него произвели огромные зубы, найденные в Китае. Он не смог составить таблицу для определения роста по размеру зубов, но утверждал, что большие зубы и мощные челюсти должны соответствовать крупному телу.

Вейденрейх определил, что предки человека были в два раза больше самца гориллы. А это животное часто достигает шести футов в высоту и весит больше четырехсот фунтов. Удвойте эти цифры, и вы получите настоящего гиганта. <<Я считаю, что, чем дальше в глубь веков, тем крупнее были люди,~--- с уверенностью заявляет Вейденрейх.~--- Прямыми предками человека были, вероятно, великаны>>.

Какое же подтверждение можно найти в Африке для его интересной гипотезы? Еще один сторонник теории гигантов, профессор Дени Сора, считает, что громадные египетские статуи могут быть одним из ее доказательств. Он ссылается также на огромные каменные орудия, не так давно найденные в Марокко. Такими орудиями могли владеть лишь великаны ростом в двенадцать

футов. Сора весьма остроумно объясняет существование в древности великанов. Он считает, что притяжение луны, которая раньше была гораздо ближе к земле, чем сейчас, уменьшало вес всех живых существ и способствовало их вытягиванию вверх. Живые существа достигали высокого роста, появлялись великаны, которые жили, как об этом говорится в Библии, до глубокой старости.

Критики Сора заявляют, что понятия о времени и росте в библейские времена были другими. Моисей принимал лунный месяц за год, что позволяло считать Мафусаила почтенным семидесятипятилетним старцем. Вероятно, рост Голиафа был равен девяти футам и девяти дюймам.

Великаны, несомненно, поражают воображение людей. Огромные изображения в Карнаке и других местах говорят о том, что в древности люди верили в великанов, хотя статуи и нельзя считать доказательством их существования. Многие легенды о великанах возникали, видимо, тогда, когда человек встречался с людьми выше себя. Из поколения в поколение рассказы эти приукрашивались, пока шестифутовый человек не превратился в двенадцатифутового.

Другим источником легенд о великанах были скелеты и зубы, найденные в разных местах и ошибочно принятые за человеческие. Огромные зубы и кости принимались за останки великанов, тогда как современные анатомы без труда определили их как части мамонтов, мастодонтов и других предков слона или длинношерстых носорогов. Даже череп Циклопа, одноглазого великана Гомера, оказался всего-навсего черепом слона без бивней. Доктор Абель из Венского университета продемонстрировал сходство между черепом слона и человека и показал, как легко было принять носовое отверстие в черепе слона за единственную глазницу во лбу.

И все же в Африке есть великаны, целые племена великанов, в которых мужчины намного превосходят средний рост. По берегам Нила я видел стройных обнаженных черных рыбаков племени динка, метавших копья в мутную воду. Рост некоторых из них достигал восьми футов. Длинные ноги позволяют им проходить за день семьдесят четыре мили. Там же живут и шиллуки со своими камышовыми лодками и нуэры, у которых по племенному обычаю весь лоб изборожден шрамами.

Люди племени ватусси~--- самые высокие в Африке и даже в мире. Средний рост мужчин этого племени~--- шесть футов и шесть дюймов, и нередко там можно встретить человека ростом до девяти футов. Ватусси населяют провинцию Руанда в Конго. Как говорят их предания (а этнографы не имеют причин в них сомневаться), некогда они жили в Египте. Когда ватусси покинули эту страну, они увели с собой замечательный длиннорогий скот иньямбо. Размах рогов у этого скота достигает восьми, девяти и даже десяти футов. Поистине, скот народа-великана. Это тот самый скот, который можно увидеть на изумительных рисунках на стенах гробниц в долине царей в Луксоре. Больше нигде в мире нет таких замечательных лирообразных рогов.

Еще одним доказательством в пользу египетского происхождения ватусси служит тот факт, что они хамиты, а не негроиды. Ватусси осели в Руанде лет четыреста назад, подчинили себе многие низкорослые племена, в том числе пигмеев. Профиль некоторых вождей ватусси сильно напоминает лица древних египетских царей.

Откуда же появляются великаны и карлики? Врачи указывают на питание. Однако ни один из них не может объяснить, почему ватусси, эти гордые африканские аристократы, достигли такого необычного высокого роста. Очевидно, из поколения в поколение они сохраняли чистоту своей крови. Но внешний облик этих великанов представляет неразрешимую загадку.

Танцы, бег, метание копья и стрельба из лука составляют основные развлечения ватусси. Молодые воины могут метнуть копье с такой силой, что его древко расщепляется в воздухе. Если королю Руанды нужно развлечь белых гостей, он созывает своих молодых великанов и велит им прыгать через головы собравшихся. Выстроившись в ряд футах в сорока от гостей, юноши один за другим подбегают к ним и легко прыгают через их головы. Меня уверяли, что ватусси совершают прыжки выше восьми футов. Другими словами, лучшие прыгуны ватусси могут больше чем на фут побить мировой рекорд по прыжкам в высоту!

В один прекрасный день атлет ватусси появится на Олимпийских играх, и тогда уж никто не будет сомневаться в существовании этих прыгающих великанов. Между прочим, ватусси надевают свободную шелковую одежду, когда прыгают через головы своих гостей. В Руанде было снято множество фотографий и фильмов, так что рассказы об этих прыжках больше не считаются баснями. Один прыгун, по имени Бутера (его рост семь футов и пять дюймов), прославился настолько, что его портрет поместили на местных банкнотах.

Как ни странно, но женщины ватусси редко бывают выше среднего роста. Многие из них не превышают пяти футов. Знатные женщины очень редко появляются на людях, а если и появляются, то обычно их несут слуги в закрытых носилках.

Среди сомалийцев, живущих между рекой Джуба и озером Рудольфа, до сих пор сохранилась одна из наиболее интересных в Африке легенд о великанах. Это почти безводный район, где кочевники сомалийцы ведут жизнь скорее верблюда, чем человека. Однако они разводят крупный рогатый скот и коз, а некоторые торгуют слоновой костью. В разных местах там встречаются заброшенные, высохшие колодцы. Задолго до того как сюда прибыли сомалийцы, эти колодцы были построены неким древним народом, владевшим такими стадами скота, которые теперь здесь нельзя прокормить.

Как утверждают сомалийцы, эти колодцы построены великанами. Только племя великанов, говорят они, могло пробить скалы таким необычным способом. Да и найденные скелеты вполне подтверждают, что люди, пробившие колодцы, были огромного роста. Где еще можно встретить бедренную кость длиной в пять футов?

Ясно, что такие колодцы мог придумать, построить и использовать лишь развитый народ, обладающий к тому же необычайной физической силой. Сотни таких колодцев можно встретить в районе Ваджер Дима, а искусственные хранилища для воды, сделанные тем же исчезнувшим народом, разбросаны по всей стране. Вверху эти колодцы начинаются круглым отверстием диаметром три фута и идут отвесно вниз на восемь футов. Дальше они постепенно расширяются до значительных размеров и в целом похожи на гигантский графин с коротким горлышком.

В сезон дождей колодцы при помощи наземных каналов наполнялись водой. По бокам колодца были выбиты ступеньки и перила, так что великан, поивший свой скот, мог спускаться в него и набирать воду в бурдюки из козьей шкуры. Глубина многих колодцев достигает шестидесяти футов, поэтому попытка поднять с такой глубины сосуд с водой потребовала бы действительно гигантских усилий. Сомалийцы и не пытаются этого делать. Из некоторых колодцев выросли деревья, которым уже лет по восемьдесят.

Сомалийцы называют строителей колодцев маантенле. Некоторые говорят, что маантенле погибли от эпидемии, другие ссылаются на летающих драконов, которые пожрали весь скот великанов, так что тем пришлось покинуть страну. И конечно же, духи маантенле все еще посещают древние колодцы. Духи десятифутовых людей.

Археологи считают, что сомалийская легенда имеет под собой определенную основу. Возможно, что одно из высокорослых нилотских племен в прошлом населяло эту страну, а потом ушло, оставив только таинственные колодцы и надгробные пирамиды, которые еще ни разу не были исследованы. Среди племен в Кении есть пастушье племя туркана. Это высокие длинноголовые люди. Рост мужчин очень часто достигает семи футов. Возможно, что туркана и есть те самые маантенле.

Но, размышляя о всем народе, жившем вдоль берегов Нила, сталкиваешься с одной из самых старых тайн Африки. Вспоминая тех голых людей на болотистом берегу Нила, я ясно вижу, как они стоят друг за другом в типичной для них позе, поджав одну ногу. У каждого в руках длинное копье. Они связаны с Нилом с незапамятных времен. Среди тех, кто когда-то покинул Египет и пошел по Нилу к его истокам, были и легендарные великаны.

На холмистом северном берегу реки Габон во Французской Экваториальной Африке среди манговых деревьев расположен Либревиль\footnote{В настоящее время Либревиль~--- столица Республики Габон.}, очаровательный город пальмовых аллей с порхающими среди цветов золотистыми бабочками. Он лежит в двенадцати милях к северу от географического экватора, но климатические особенности там такие же, как в южном полушарии. Мне было интересно также узнать, что Либревиль~--- единственное место в Западной Африке, где на окраине города, на побережье, убивали слонов. Было это в двадцатых годах нашего века, и я думаю, что с тех пор слоны ушли в глубь страны.

Проезжая через Либревиль, я думал не о слонах. Мое воображение занимали пигмеи. В то время французские власти поселили часть пигмеев на небольших участках в окрестностях города, и мне не терпелось сравнить этот малорослый народ с бушменами южноафриканских пустынь.

О пигмеях знали еще в древности. Гомер писал об этом крошечном народе в <<Илиаде>>, а Свифт использовал легенду о них в <<Путешествиях Гулливера>>. Несомненно, пленные пигмеи развлекали царей Древнего Египта, так как фигурки пигмеев высечены на гробницах в Саккаре. Но только после того, как в шестидесятые годы девятнадцатого века Поль Дю Шайю совершил свою вторую крупную экспедицию в габонские леса, мир узнал, что пигмеи не просто легенда, а существующий народ.

Когда Дю Шайю разыскивал в лесах горилл, он слышал о карликовых племенах, но, как сам признавался, не верил в это. Он снова вернулся в Западную Африку, <<больно уязвленный несправедливой и жестокой критикой>>, и решил реабилитировать себя. В один прекрасный день во внутренних районах Габона, примерно в ста пятидесяти милях от побережья, он случайно наткнулся на несколько маленьких хижин. Его проводники сообщили ему, что это хижины карликов. Однако маленькие люди разбежались. После этого Дю Шайю везде старался искать пигмеев, и на землях ашанго ему наконец повезло. Проводники ашанго привели его в деревню, людей которых они называли обонго.

"Чтобы не спугнуть обитателей, мы приближались с величайшей осторожностью. Мои проводники ашанго дружелюбно протягивали связки бус,~--- писал Дю Шайю.~--- Но наша предосторожность оказалась напрасной, потому что, когда мы подошли, мужчин по крайней мере уже не было. Мы поспешили к хижинам, и нам удалось найти трех старых женщин и одного юношу, который не успел убежать, а в одной из хижин спряталось несколько детей.

Казалось, страх лишил их способности двигаться. Я протянул им бусы. Одна из старух осмелела и принялась высмеивать мужчин за то, что они от нас убежали. Она сравнила их с пугливыми нченде (белками), которые пищат <<ке-ке-ке>>, и стала так забавно передразнивать писк и движения белки, виляя своим маленьким телом, что мы все рассмеялись".

Это описание Дю Шайю показывает одну особенность пигмеев, которую все исследователи находят очаровательной,~--- дар подражания. (Между прочим, эта же особенность свойственна и бушменам Южной Африки.)

Дю Шайю говорит, что кожа у пигмеев грязно-желтого цвета, намного светлее, чем у окружающих их африканцев. В глазах у них <<неукротимое буйство>>. Ладони~--- белые. Ноги по сравнению с туловищем~--- короткие. У многих мужчин необычайно волосатые грудь и ноги.

Как отмечал Дю Шайю, в стране ашанго к пигмеям относились очень доброжелательно из-за их искусства ловить диких зверей и рыбу. Избыток добычи они обменивали у соседей на бананы, кухонную посуду и кувшины для воды. В лесах вокруг деревень пигмеев было так много ловушек и западней, что ходить там было опасно.

Преследуя дичь, пигмеи все время передвигались с места на место, но никогда не выходили за пределы земель ашанго. <<Пигмеи похожи на европейских цыган. Они так же отличаются от народа, среди которого живут, хотя уже давно не покидают пределов этой страны,~--- писал Дю Шайю.~--- Они ничего не выращивают. Питаются кореньями, ягодами и орехами, но особенно любят мясо>>.

Через несколько лет немецкий исследователь Швайн-фурт открыл пигмеев акка в лесах Итури в Конго, вдали от побережья. Однако лишь в последние годы этот интересный народ раскрыл некоторые свои тайны. Но во многом пигмеи все еще загадочны.

Каково происхождение пигмеев? Ни один ученый не осмеливается высказать твердую точку зрения на этот счет. В том, что они близки к первобытным людям, не может быть ни малейшего сомнения, поскольку они так и остались примитивными охотниками. Человеческие расы распространились по всей земле и создали различные цивилизации. Но в лесах, джунглях и пустынях сохранились разрозненные поселения немногочисленного народа, которого не коснулся прогресс. И вне Африки еще сохранились другие охотничьи племена~--- негроиды Андаманских островов и Малайского полуострова. Некоторые ученые считают, что когда-то пигмеи населяли экваториальные районы чуть ли не на всем земном шаре, а затем опустившийся древний индо-африкан-ский континент отделил африканских пигмеев от их восточных родственников.

Возможно, что когда-то пигмеи были обычного роста, но из-за своего образа жизни постепенно мельчали. А может быть, они всегда были карликами, составляющими совершенно особую группу среди всех народов мира. Сейчас это самый малорослый народ в мире. Крошечные бородатые мужчины четырех футов и четырех дюймов и крошечные женщины всего лишь четырех футов.

У пигмеев, которых встречал Дю Шайю, кожа была желтоватая. На окраинах экваториальных лесов есть и черные пигмеи, но они, видимо, смешались с негроидами. Наиболее типична для пигмеев коричневая кожа. Лица у них некрасивые, только маленькие дети производят хорошее впечатление. Руки длинные, но кисти рук и ступни хорошей формы, а плечи широкие.

Доктор Пауль Шебеста, многие годы проживший среди пигмеев Итури, говорит, что они всегда вынуждены бороться против порабощения и гордятся своей свободой. Пигмеи делали набеги на банановые плантации других племен, стреляя из-за укрытий отравленными стрелами, и скрывались в лесах со своей добычей.

При охоте на слонов пигмеи используют копья, ножи и топоры. Раздевшись донага и намазав тело пометом слона, чтобы заглушить человеческий запах, пигмеи подкрадываются к слону и окружают его, прежде чем он обнаружит опасность. С ножом в руке предводитель охоты подбегает к слону и перерезает у него сухожилия на задней ноге. Потом охотники разделяются на две группы и преследуют искалеченного слона, стараясь его раздразнить. Выбрав подходящий момент, предводитель перерезает слону сухожилие на другой ноге. Когда огромный противник оседает на задние ноги, пигмеи топорами разрубают ему брюхо, выпуская громадные внутренности. Предводитель берет копье с загнутым зубцом, к которому привязана веревка, закрепленная другим концом на дереве, и вонзает его в разрубленное брюхо. Обезумевший от боли слон бросается в сторону, и за ним тянутся выпадающие внутренности. Это самый безжалостный и жестокий способ охоты жителей старой Африки, хотя и эффективный. Иногда пигмеи перерезают не только сухожилия ног, но и хобот. Животное умирает от сильного кровотечения. Порой охота кончается трагически. Слон поддевает охотников своими бивнями и превращает их в лепешку.

Однажды южноафриканский охотник Джон Молтено преследовал слонов. Дорогу ему преградило болото, такое глубокое, что пройти через него было невозможно. Слоны обычно пользовались этим обстоятельством и уходили через болото от преследования. Молтено договорился с вождем, чтобы ему притащили лодку, на которой можно было бы пересечь болото. Так он попал в лес, где до этого, вероятно, не ступала нога белого человека. И вот там состоялась волнующая встреча с маленьким народцем.

Молтено увидел следы ног пигмеев и пошел по ним, но заблудился. Вскоре он услышал голоса и, подкравшись к поляне, увидел толпу пигмеев, собравшихся вокруг костра. Копья их были сложены в одну кучу и направлены острием вниз. Несколько пигмеев разделывали дикого кабана, остальные доставали из дупла мед.

Молтено с двумя проводниками бесшумно двинулся вперед и захватил копья. Пигмеи бросились бежать, но проводники дружески окликнули их, и вождь пигмеев остановился. Когда Молтено пообещал им мяса, все оживились. Тогда он вышел из укрытия, сел на поваленное дерево, и вокруг него стали робко собираться пигмеи. Их вождь знал язык одного из проводников, так что Молтено смог вести разговор. Он узнал, что некоторые пигмеи лишь один раз издали видели белого торговца, остальные же никогда не сталкивались с белым человеком.

"Они угостили меня свининой и медом,~--- рассказывал мне Молтено.~--- Пока я ел, пигмеи суетились вокруг, срезали ветки и куски коры, сооружая надо мной навес. Они видели, что надвигается тропический ливень, и он действительно разразился, едва они закончили постройку. Позднее пигмеи отправились со мною на охоту. В лесу они передвигались легко и бесшумно. Мне было трудно поспевать за ними, хотя я и был на добрых три фута выше большинства пигмеев. Там, где пигмей мог пройти под веткой во весь рост, мне часто приходилось сгибаться в три погибели.

Пигмеи, видимо, считали меня неуязвимым, и поэтому я часто оказывался в довольно трудном положении. Так один раз я неожиданно очутился среди стада слонов. Слониха подняла тревогу. Пигмеи разбежались. Мне ничего не оставалось, как довериться своему ружью. У пигмеев были с собой барабаны, и, когда я убил слона, они передали об этом в свою деревню, и все племя моментально сбежалось на пир. Пигмеи питаются в основном мясом и, только когда не хватает дичи, выкапывают коренья. Свои хижины они делают наподобие ульев, к которым пристраивают крытый ход, где разводится дымный костер, чтобы отпугнуть мошкару. Через две-три недели они покидают эти хижины и уходят на новые места в поисках дичи.

Я держал у себя в лагере одного пигмея в качестве заложника. В один прекрасный день все племя вдруг исчезло. Я спросил у своего пленника, в чем дело. <<Мы хорошо знаем, что вы пришли сюда не для охоты на слонов. Вы пришли, чтобы съесть нас>>,~--- просто ответил пигмей. Сотни лет каннибалы совершали набеги на пигмеев, и этот человек даже не мог представить себе иной причины моего появления здесь".

Пигмеи частенько используют на охоте сети. Они выбирают в лесу подходящий участок, вокруг которого собираются все мужчины, женщины и дети, и начинают гнать мелкую дичь, размахивая дубинками и факелами, пока не загонят ее в расставленные сети.

Пигмеи, которых я видел в окрестностях Либревиля, казались забитыми созданиями. Ведь их вырвали из родной лесной среды и принудили к работе, которая не может им дать такого удовлетворения, какое дает охота. На мой взгляд, пигмеи сильно отличались от бушменов юга. Оба народа~--- охотники и собиратели плодов, оба пользуются луком и стрелами, оба веселые и расточительные, оба обладают врожденным даром подражания и любят танцы. Но, помимо роста, я не мог найти другого физического сходства между пигмеями и бушменами. Лица у них совершенно разные. Бушмен~--- это морщинистая благожелательность и добродушие, тогда как лицо пигмея лишено привлекательности.

Я был почти уверен, что встречу в Либревиле старых знакомых. Но на этой лесной поляне во Французском Конго я почувствовал, что она и ее жители не имеют ничего общего с жителями Калахари. Тогда я сделал эти выводы на основании чисто внешних впечатлений, а потом они были подтверждены исследованиями ученых, которые применяли современные методы определения групп крови. И бушмены, и пигмеи~--- охотники. Но между ними нет даже отдаленного родства.

\chapter{Тайны перелетов птиц}

На юге~--- весна. И миллионы перелетных птиц устремляются в Южную Африку. Полосатогрудые ласточки из неизвестных гнездовий где-то в Африке и голубовато-серые европейские ласточки из года в год направляются к одним и тем же гнездам на старых верандах. Это тоже загадка Африки. И каждый раз с наступлением лета все это повторяется снова и снова.

Вот с громким свистом появляются золотистые иболги, отыскивая высокие деревья, где можно спрятаться от человеческого глаза. К болотам направляются луни. С севера летят огромные стаи белых аистов и поселяются на деревьях и на склонах холмов. Пугливые черные аисты прилетают к знакомым пещерам, а иногда и к пересыхающим озеркам Капской провинции. Городские ласточки и каменные стрижи, козодои и кукушки, песчаные ржанки, серые кулики из России и другие болотные птицы~--- все они летят в солнечный край, на юг. Особенно сильное впечатление произвел на меня самый замечательный из всех пернатых путешественников~--- большой буревестник. Этого морского бродягу с бурой спиной можно увидеть во многих районах земного шара, в том числе и на мысе Доброй Надежды. До начала нашего века никто не знал, где находятся гнездовья этого замечательного буревестника. Но потом тайна раскрылась, когда в южноафриканский музей попала шкурка с островов Тристан-да-Кунья. Я был на этих островах, видел там птиц и пробовал их яйца. Каждый год островитяне убивали огромное количество буревестников из-за их мяса и жира. Но теперь там существует некоторая охрана этих птиц.

Большие буревестники никогда не залетают на сушу, и только острова Тристан-да-Кунья и соседний с ними остров Гау представляют исключение. Здесь эти птицы устраивают свои гнезда в подземных норах. Буревестников видели в Арктике и Антарктиде, у берегов Северной и Южной Америки, Европы и Африки. И все же они безошибочно находят путь к своему дому~--- небольшой группе островов в самом сердце океана, удаленных от ближайшего материка на тысячу пятьсот миль.

На антарктическом побережье гнездится буревестник Вильсона, который совершает перелеты на север почти до Северного полюса. А полярные крачки летят из Арктики к ледникам Антарктиды, и только у Западной Африки они пролетают вблизи берегов. Окольцованная в Гренландии молодая полярная крачка за шестнадцать недель преодолела одиннадцать тысяч миль. Ее обнаружили в Дурбане в провинции Наталь. По сравнению с этими перелетами знаменитые почтовые голуби выглядят домоседами. Между прочим, люди приручали длиннокрылого фрегата и использовали его для почтовой связи. Но все же самый удивительный путешественник~--- большой буревестник. Разыскать затерянный в океане островок и выбрать на нем место для посадки значительно труднее, чем долететь до ледяного барьера или континента. Да и летают эти птицы с немалой скоростью. Один окольцованный буревестник за двенадцать дней преодолел три тысячи миль.

Некоторые ученые высказывали предположение, что у птиц, совершающих перелеты через океан, сохранилась наследственная память о некогда существовавших участках суши, которые соединяли материки. Это звучит фантастично. Птицы пересекают морские просторы и там, где в давно минувшие геологические эпохи не было никакой суши.

Как же это происходит? Как все эти птицы, большие и малые, находят путь через материки и океаны? Почему одни птицы преодолевают многие тысячи миль, а другие, менее отважные, остаются на месте? Почему перелетные птицы возвращаются обратно? Все это загадки, которые многие века привлекают внимание ученых орнитологов и простых наблюдателей. Однако сейчас мы знаем о перелетах птиц уже гораздо больше, чем знали в прошлом веке. И все же окончательная разгадка этой тайны, видимо, отодвигается все больше по мере расширения человеческих знаний.

Возможно, первые перелеты на юг птицы стали совершать в ледниковый период миллионы лет назад, когда сплошные льды покрывали огромные пространства Северной Америки и Европы. До сих пор большинство перелетных птиц летом гнездится на севере, а с наступлением зимы улетает на юг. Но были отмечены и иные пути. Есть птицы, которые совершают перелеты с Мадагаскара на Африканский континент или из Трансвааля в Анголу. Некоторые птицы южных морей облетают вокруг всего земного шара. Но основные, классические пути перелетов пролегают с севера на юг, по тем маршрутам, которых придерживались птицы, вынужденные покинуть свою родину, спасаясь от надвигавшихся ледников. Такое объяснение кажется мне правдоподобным. И все же не верится, что <<наследственная память>> о их древней северной родине оказалась настолько крепкой, что птицы стали возвращаться назад, когда ледники исчезли. Следует искать другие причины.

Уже много веков перелеты птиц ставят в тупик самых мудрых философов. Должно быть, первобытный человек знал, что отлет европейских ласточек предвещает наступление зимы. Однако еще в семнадцатом веке люди считали, что ласточки зарываются в грязь на дне озера и впадают в зимнюю спячку, подобно летучим мышам. Многие верили, что аисты улетают на луну и что большие птицы, например журавли, переносят на своих спинах мелких собратьев.

Когда видишь птиц вдали от их родины, невольно возникает мысль, что они запомнили дорогу при прежних перелетах. Это верно только в отношении некоторых видов птиц, но далеко не всех. Даже при первом знакомстве с перелетами убеждаешься, что существует не так-то уж много неизменных маршрутов. Необходимо изучить перелеты многих видов птиц, чтобы убедиться в несостоятельности своих прежних взглядов. Зяблики, например, совершают перелеты большими стаями во главе со старыми птицами, за которыми летит молодежь. Установлено, что они действительно из года в год летят и возвращаются на старые места одним и тем же путем.

Многие птицы, видимо, предпочитают лететь вдоль побережий и рек. Они придерживаются долин, а не переваливают через высокие гребни. Наблюдения с маяков и кольцевание подтвердили, что ласточки, летящие из Европы в Южную Африку, всегда следуют определенными путями. Они облюбовали долину Нила. А пестро-носые крачки пересекают Бискайский залив, пролетают над островом Мадейра, а затем вдоль побережья Сенегала и направляются в Капскую провинцию как раз тем путем, каким следуют почтовые пароходы. Сорокопуты, каменные стрижи и козодои летят на юг через Францию, Испанию и Сахару.

Долгое время считали, что перелетные птицы находят путь по определенным ориентирам местности. С птичьего полета открывается такая обширная панорама, что заблудиться почти невозможно. Словом, никакого инстинкта, ничего таинственного. Природа дала птицам необычайно острое зрение, и при перелетах они легко находят дорогу.

Отчасти это верно, но лишь отчасти. Есть птицы, которые летят очень низко над землей, а некоторые морские птицы~--- даже над самыми волнами. Многочисленные опыты и наблюдения дают возможность смело утверждать, что гипотеза о <<фотографической памяти>> не объясняет тайн перелетов. Да, зрение играет определенную роль, и действительно при приближении к дому птица отыскивает свое гнездо по ориентирам. Но при перелетах по ранее неизвестным путям, над незнакомой местностью у птиц должны быть и другие способности. Многие молодые птицы совершают свой первый перелет без родителей или старших товарищей.

Однажды несколько сотен ласточек было увезено в разных направлениях от своих гнезд. Затем их выпускали и в солнечную погоду, и в туман. Большинство из них вернулось домой. Одна ласточка преодолела за двадцать шесть часов двести пятьдесят миль.

Я думаю, что такие опыты проводились чаще всего с почтовыми голубями. Дикий голубь, родоначальник всех породистых голубей, не был перелетной птицей. Это говорит о том, что во всех птицах от природы заложены определенные возможности, которые можно значительно развить соответствующей тренировкой. Многие дикие птицы, безусловно, умеют находить дорогу гораздо лучше, чем почтовые голуби.

В опытах с почтовыми голубями экспериментаторы пытались всяческими способами сбить их с толку. Голубей увозили даже под наркозом, но они благополучно возвращались обратно. Уже давно было высказано предположение, что птицы отыскивают направление по магнитным полюсам. Однако голуби возвращались домой даже в том случае, когда к их крыльям прикреплялись магниты, что не давало им возможности чувствовать земной магнетизм.

Во время второй мировой войны я служил в военно-воздушном флоте Южной Африки, где нас специально обучали обращению с почтовыми голубями. Мы брали на борт каждого самолета нашей эскадрильи по четыре голубя и летали над океаном. Инструктор, который в мирное время занимался голубями просто как любитель, показал нам, как надо завертывать голубя в бумагу и как выбрасывать его из самолета, чтобы он не попадал в воздушные завихрения. Наш учитель не разделял мнения, что голуби не могут летать над морем, хотя он заметил, что все они спешат вернуться к материку кратчайшим путем.

Офицеры, занимавшиеся тренировкой голубей, вероятно, установили в то время мировой рекорд. Они отправили из Кейптауна в Преторию двух голубей месячного возраста, продержали их два с половиной месяца в Претории, а затем выпустили. Один из них благополучно вернулся в голубятню в Кейптауне. Как правило, первые короткие перелеты голубь совершает в возрасте четырех месяцев и лишь к трем годам добивается наивысших успехов.

В пасмурную погоду мы поднимали голубей на высоту десять тысяч футов и выпускали их. Все благополучно возвращались из этого <<слепого полета>>. Для перелетных птиц облака всегда большая помеха. Летчикам, у которых есть голуби, советуют в случае вынужденной посадки в пустыне выпустить голубя и по компасу определить направление его полета. Это всегда покажет путь к дому. Почтовые голуби имеют удивительную способность ориентироваться. Их выпускали в незнакомой местности и засекали время. Через десять секунд они уже брали нужное направление к дому.

Несостоятельность гипотезы о <<фотографической памяти>> может подтвердить такой опыт. Если взять яйца почтовых голубей хорошей породы, отвезти их за сто и даже больше миль от дома и вывести там птенцов, молодые голуби могут вернуться в ту голубятню, где были снесены яйца. Этот случай опровергает все подобные гипотезы, за исключением версии о <<групповой памяти>>. Однако, когда вы говорите о групповой, или наследственной, памяти, вы только даете тайне название, но она по-прежнему остается тайной.

Из всех птиц, которые совершают перелеты в Южную Африку, больше всего обращает на себя внимание любимый всеми белый аист. Эта крупная птица, ставшая символом любви и верности, настолько привыкла к человеку, что в период выкармливания птенцов ищет у него защиты. Аисты~--- почетные гости печных труб Дании и Голландии, и в Южной Африке они повсюду находятся под охраной закона. Фермеры знают, что эти птицы очищают вельд от саранчи и прочих вредных насекомых.

Над аистами тоже производили научные опыты. В период между двумя мировыми войнами сотрудники известной орнитологической станции Росситтен в Восточной Пруссии, а также сотрудники других станций окольцевали многих аистов, чтобы проследить маршрут их перелета в Африку. Оказалось, что аисты из страны Ханса Андерсена и большинство аистов из Западной Европы летят на юг через Францию и Испанию и зимуют в Марокко. Другие летят на юго-восток через Сахару. Путь этот точно не установлен, но, видимо, птицы летят над озером Чад, рекой Конго и затем в Восточной Африке присоединяются к другим зимующим на юге аистам. Однако аисты из Восточной Пруссии и Венгрии следуют восточным путем через Балканы и враждебную им Турцию (где невежественный народ истребляет их), а затем спускаются по Нильской долине к югу и из Восточной Африки направляются в Центральную Африку или в Южную Африку. Места зимовок аистов и других птиц очень разнообразны. Некоторые птицы летят к самой южной оконечности Африки, другие оседают на зимние квартиры вдали от Капской провинции.

Сотрудники станции Росситтен отвезли на запад, в Эссен, сто пятьдесят аистов, родившихся в Восточной Европе. Их окольцевали и выпустили, когда уже прошла пора отлета, так что не было старых аистов, которые могли бы сопровождать их. Сообщения из различных пунктов на пути перелета окольцованных аистов показали, что большинство из них не полетели в Африку западным путем (как ожидали некоторые ученые), а избрали восточный путь. Два аиста оказались во Франции, куда они летели своим собственным маршрутом. Эти опыты подтверждают, что унаследованный от родителей инстинкт оказался сильнее, чем все другие факторы, в том числе ветер и погода. Известный орнитолог доктор Хейнрот заявил, что аисты вопреки мнению отдельных специалистов никогда не придерживаются строго одного и того же маршрута. Он высказал предположение, что во время перелета на юг некоторые перелетные птицы следуют за другими видами.

Ученые не раз наблюдали за перелетом аистов. При этом новые открытия породили новые тайны. Наивысшая скорость полета аиста (она была установлена с самолета, летевшего следом)~--- около пятидесяти миль в час. Совершая ежегодные перелеты на юг, аисты проводят в пути два месяца, а возвращаются за месяц. Чем это объясняется? Тем ли, что выращивание потомства на севере так истощает этих крупных птиц, что они не могут за более короткий срок преодолеть шесть тысяч миль? Или тем, что лягушки и насекомые Южной Африки более питательны, нежели добыча скудных северных стран? Трансваальский зоолог доктор Р. Бигальке утверждает, что аисты спешат вернуться в Европу для выведения потомства.

Кажется, в Европе аистов становится все меньше, а в Африке их число возрастает. Но когда в ноябре 1940 года доктор Остин Робертс сообщил, что видел белого аиста, который крыльями прикрывал от солнца трех птенцов в своем громоздком гнезде на акации у дороги между Аудсхорном и Калицдорпом, это было сенсационным открытием в мире птиц. Один фермер, который даже не представлял, как необычно его сообщение, сказал доктору Робертсу, что на его ферме в течение семи лет подряд гнездилась пара аистов. С наступлением зимы они не улетали, а оставались на старом месте. Это было первое свидетельство, что белые аисты размножаются в Южной Африке, и никто не придавал ему особого значения, пока Робертс, автор книги <<Птицы Южной Африки>>, не обратил на это внимания. Затем другая пара аистов свила гнездо в Оранжевой республике. Никто не мог объяснить, почему эти птицы нарушили древние законы перелетов и отвергли свои наследственные инстинкты.

В перелетах птиц для нас еще много загадочного. Когда говорят, что с наступлением зимы перелетные птицы лишаются корма, а поэтому и улетают, это звучит правдоподобно. Но если как следует вникнуть в дело, выясняется, что птицы улетают задолго до того, как исчезает их корм. На юге в это время теплее, но откуда, черт возьми, об этом знают молодые птицы? Во всяком случае массовый перелет начинается задолго до наступления холодов. Почему перелетные птицы так боятся зимы, если на протяжении бесчисленных поколений они ни разу не испытывали на себе всей ее суровости?

Некоторые ученые пытались объяснить причины перелетов птиц инстинктом размножения и разработали особую теорию. Но их противники доказали, что и незрелые в половом отношении птицы и даже кастрированные тоже участвуют в перелетах. Это, конечно, еще не доказательство, чтобы опровергнуть теорию, но разгадка все же так и не найдена.

Посадите перелетную птицу с самого рождения в клетку, и вы увидите, что с приближением поры перелета она будет все больше и больше нервничать. Поместите клетку в комнату без окон и проследите, в какую сторону все время рвется птица. Окажется, что это путь ее перелета. Можно выводить перелетных птиц в неволе, но вам никогда не удастся подавить их стремления пуститься в путь, как только наступает пора.

Южноафриканский орнитолог доктор Г.~Я.~Брукхюзен считает, что кое-какие сдвиги в выяснении причин перелетов все же наметились. По его мнению, стимулом к перелету являются прежде всего изменения в физиологии птиц, а климатические условия играют подчиненную роль. Этот авторитетный специалист допускает также, что у птиц есть <<чувство магнетизма>>.

Кажется, ученые все чаще и чаще приходят к выводу, что птицы обладают особыми органами чувств, о которых человеку ничего не известно. Знаменитый немецкий психолог доктор Давид Кац заявил: <<Многие теории о перелетах птиц можно принять лишь в том случае, если допустить существование какого-то неизвестного нам фактора>>.

Зоолог Кэмбриджского университета доктор Г.~В.~Т.~Мэтьюз считает, что птицы ориентируются по солнцу, как это веками делали люди. Нет никакой нужды отыскивать какое-то шестое чувство, если все дело заключается в чувстве времени и в ориентировании по солнцу. Доказано, что и люди, и птицы обладают чувством времени, причем точным, как хронометр. Птицы знают, по какому пути движется солнце у них на родине и какое положение оно занимает на этом пути в разное время. Когда птица отправляется в полет, она сознательно или бессознательно определяет свое положение, подобно тому как это делает моряк по секстанту. Избрав на закате направление полета, морская птица и ночью может придерживаться его, ориентируясь по звездам и луне. Или же она отдыхает на воде до восхода.

Я охотно принимаю эту гипотезу, поскольку она соответствует ряду установленных фактов и не прибегает к раздражающему слову <<инстинкт>>, которое ничего не объясняет. Факты свидетельствуют, что без солнца~--- при сильной облачности или в густой туман~--- птицы могут сбиться с пути. Вполне возможно, что птица знает, с какой стороны от места своего назначения она находится~--- с востока или с запада, даже если она летит на затерянный в океане остров. Я уже говорил об острове Тристан-да-Кунья, который представляет собой конус вулканического происхождения высотой больше семи тысяч футов. Возможно, что на поверхности океана существуют какие-то приметы, свидетельствующие о близости суши, и острое зрение птицы позволяет ей их обнаружить. Поэтому-то большому буревестнику при возвращении на родной остров и не приходится понапрасну терять времени на его поиски.

Перелет~--- это тепло и обильный корм в конце пути, но путешествие это опасное. Множество птиц гибнет в дороге. Ведь у них нет метеорологической службы, которая могла бы их предупредить о песчаных бурях в Сахаре. В Европе во время одной снежной бури погибла огромная масса птиц, на каждую милю приходилось по пяти тысяч их трупов. Этот подсчет был сделан на берегах одного озера, куда водой прибило погибших птиц. В холодную погоду ласточки гибнут сотнями тысяч.

Многие внимательные наблюдатели отмечали, что улетающие из Европы птицы не устремляются на юг сломя голову. Их, правда, охватывает беспокойство, как и птиц в клетке, о которых я говорил, но к отлету они готовятся не спеша, собираясь в стаи. И как только солнце переходит в южную половину неба, птицы улетают на юг.

С окончанием лета в южном полушарии птицы улетают обратно в Европу. Наблюдения показали, что возвращение это происходит в спешном порядке. Высказывалось предположение, что птицы стремятся как можно быстрее восполнить недостаток некоторых витаминов, которыми богата весенняя растительность северного полушария. Анализы подтверждают это предположение.

На африканском побережье наступает март, и морские птицы со штормовых островов великого южного океана на зиму устремляются к континенту. Буревестник и чайка, величественный альбатрос и черная крачка~--- всем им придется приложить все свое навигационное искусство, чтобы потом опять отыскать туманную родину, где так мало солнца. Когда эти морские путешественники приближаются к Южной Африке, ласточки уже готовы отлететь на север. Вновь начинается древний загадочный цикл, какая-то непостижимая древняя сила вновь гонит миллионы птиц в дальние края. И человечество еще не в силах раскрыть эту тайну.

\chapter{Властелины пустыни}

Юго-Западная Африка~--- последнее пристанище бушменов, небольшого народа, который когда-то скитался по всему Африканскому континенту. А теперь, может быть, только в Юго-Западной Африке у них есть еще надежда выжить.

До недавнего времени ученые считали, что бушмены обречены на вымирание. На границе пустыни Калахари с Капской провинцией живут бушмены комани и ауни. Их осталось всего лишь несколько человек, так что судьба их совершенно ясна. На огромной территории протектората Бечуаналенд осталось меньше тысячи бушменов. В Анголе~--- три тысячи. Цифра эта кажется внушительной, но на самом деле это не так. Ведь изолированное сообщество людей быстро вымирает или, смешавшись с соседними народами, теряет свое лицо. Многие годы численность бушменов в Юго-Западной Африке определялась в пять тысяч человек~--- цифра малоутешительная. К счастью, последняя перепись показала, что бушменов здесь гораздо больше~--- от десяти до двенадцати тысяч. Если этим древним охотникам предоставить свободу, они смогут вновь обрести силы, и численность их возрастет.

Бушмены всегда были для меня самым очаровательным, самым романтическим народом древней Африки. И я уверен, что со мной согласятся почти все, кому пришлось с ними соприкоснуться. Лишь фермеры, у которых пропадает скот, да полицейские, которым знаком звук пролетающей над головой отравленной стрелы, не смогут разделить моей любви к этим <<властелинам пустыни>>.

Около полусотни лет назад, когда Южно-Африканский Союз оккупировал Юго-Западную Африку, от рук бушменов действительно погиб один судья, несколько полицейских и ряд африканцев. (Я встретил одного сержанта полиции, который был ранен в руку отравленной стрелой. Следопыт-бушмен спас ему жизнь, высосав яд из раны сразу после ранения.) Столкновения бушменов с гереро или овамбо происходят и до сих пор, и с обеих сторон бывают потери. И все же нельзя сказать, чтобы бушмены в целом представляли серьезную угрозу общественному порядку. В памяти людей еще сохранились воспоминания об охоте на бушменов, поставленных вне закона. Их преследовали, как диких животных, и убивали. Теперь уже многие бушмены перестали панически бояться белых людей. Некоторые чиновники и фермеры даже завоевали наконец их доверие.

Бушмены относятся к самым низкорослым народам на земле. Однако это не карлики. Бушмены очень пропорционально сложены и обладают значительной силой. У них превосходная форма рук и ног. Измерения черепа показывают, что у бушменов самый маленький в мире по объему мозг. На этом основании их и поместили на самую низшую ступень человеческой лестницы. Но среди ученых все больше крепнет убеждение, что бушмены в конечном счете не такие уж простаки. Как бы первобытны они ни были, образ их жизни далеко не всегда можно объяснить простой необходимостью. Если бушмену предоставить возможность работать на ферме, он все же предпочтет жизнь в буше. Там он чувствует себя более счастливым, и в этом есть здравый смысл. Величина мозга бушменов пропорциональна их росту. И в этом крошечном мозгу заложена масса всяких знаний, к которым ученые начинают теперь проникаться все большим уважением.

Никто в Африке не может сравниться с бушменами своим знанием природы. Бушмены непревзойденные охотники и художники, знатоки змей, насекомых и растений и наследники богатого фольклора. Они лучшие танцовщики и сказители во всей Африке и наделены удивительной способностью к подражанию. Одного старика бушмена из племени кунг как-то спросили, сколько ему лет. И вот как переводчик (наполовину бушмен, наполовину готтентот) перевел ответ старика: <<Я молод, как самое прекрасное желание моей души, и стар, как все несбывшиеся мечты моей жизни>>.

Пословицы бушменов убеждают нас в том, что маленький мозг этих людей обладает богатым воображением. <<Пища вкусна только тогда, когда ты добудешь ее своим умом и ловкостью>>,~--- гласит одна из них. И никто не станет отрицать художественную силу такой пословицы: <<Новости как ветер. Они летят через горы и долины и вместе с ароматом цветов несут и неприятные запахи>>.

В Юго-Западной Африке бушмены населяют в основном территорию к югу от реки Окаванго. На этих землях нет хороших пастбищ и нет проточной воды. Фермеры не претендуют на этот засушливый район, поэтому бушменов здесь никто не беспокоит. Этнограф П. Шуман, известный как <<покровитель бушменов>>, сообщил мне, что, по его подсчетам, в этом районе живет восемь тысяч бушменов кунг, что, видимо, соответствует действительности.

В настоящее время бушмены кунг являются, пожалуй, самыми чистокровными представителями этого народа. Кожа у них имеет желтый оттенок, а рост очень маленький. (У бушменов Анголы, которых я уже упоминал, есть сходство с бушменами кунг, и говорят они на одном и том же языке. Но называют они себя лесными людьми и до некоторой степени смешались с народами банту.) К югу от территории бушменов кунг, в районе Гобабис, на границе с пустыней Калахари, проживает значительное количество бушменов нарон. Они славятся своим искусством изготовлять ритуальную одежду из бус, сделанных из скорлупы страусиных яиц. По своему внешнему виду и обычаям бушмены нарон мало чем отличаются от бушменов кунг.

Однако к северу встречаются бушмены высокого роста. Они не похожи на истинных бушменов ни по внешнему виду, ни по другим особенностям. Это хейкум. Их часто видят люди, приезжающие в район Этоша. Там доктор Шуман провел почти точную перепись и насчитал 1200 мужчин, женщин и детей хейкум. Он относит их к бушменам на том основании, что они ведут сходный с ними образ жизни. Видимо, предками хейкум было какое-то бушменское племя, завоеванное потом готтентотами, с которыми оно смешалось и растеряло многие свои обычаи. Родным их языком стал один из диалектов языка победителей~--- нама. Покойная Доротея Блик, большой знаток бушменских языков, обнаружила, что в некоторых отдаленных районах хейкум до сих пор говорят на языке своих предков-бушменов, родственном языку кунг. Это открытие сыграло огромную роль в выяснении происхождения единственного в Юго-Западной Африке племени высокорослых бушменов.

Ученый-анатом Раймонд Дарт обнаружил, что в полунаклонной осанке, позвоночном столбе и ногах бушменов есть сходство со скелетами неандертальцев, населявших южную Европу двадцать пять тысяч лет тому назад. Бушмены~--- это живые ископаемые. Однако, если обратить внимание на азиатское веко бушменов, проблема происхождения этого народа окажется гораздо более сложной. Ослепительное солнце за многие столетия сделало их глаза узкими и собрало их в забавные складочки.

Для настоящих бушменов характерны сильно курчавые волосы, расположенные пучками. На теле волос почти нет, хотя у некоторых мужчин растут реденькие усы и бородка. Пикантное личико бушмена выглядит по-детски даже в пожилом возрасте. Цвет кожи~--- от желтого до шоколадного. Не так давно доктор Шуман открыл в западной части Каприви Ципфель бушменское племя, совершенно неизвестное ученым. Это были маленькие чернокожие бушмены с длинными волосами, говорившие на своем щелкающем языке.

Немецкий ученый доктор Зигфрид Пассарг, занимавшийся в конце прошлого века бушменами, встретил старика бушмена, который утверждал, что понимает язык бабуинов. Понятно, язык бушменов ничего общего не имеет с <<языком>> бабуинов, но все же это язык первобытный, и его невозможно отнести ни к какой языковой группе. Европейцам, которые пытаются изучить язык бушменов, не удается осилить щелкающие звуки. Лишь немногие сумели с этим справиться.

Когда-то племена бушменов были разбросаны по все- му побережью пустыни Намиб в Юго-Западной Африке, от берегов реки Кунене до реки Оранжевой. На юге бушмены, вероятно, исчезли полностью. Однако в 1931 году сержант полиции Юго-Западной Африки И.~В.~ван Зил наткнулся в горах Аурус, в запретном районе добычи алмазов, южнее Людерица, на бушменское поселение. В этом отдаленном районе у колодца, которого не было на карте, жило семнадцать мужчин, женщин и детей. Они питались дичью, яйцами страусов, медом и кореньями. Эти бушмены прожили там всю жизнь, и никто из европейцев не знал об их существовании. Поэтому нельзя утверждать, что все прибрежные бушмены исчезли.

В естественных условиях бушмены самые крепкие люди, с которыми когда-либо приходилось сталкиваться врачам. Хотя точно установить возраст бушмена и трудно, но все же можно с уверенностью сказать, что столетние бушмены составляют значительный процент, если учесть те опасности, которым постоянно подвергаются жители пустыни. Я знал одного бушмена, которому было за сто, и у него сохранились зубы. А физическая сила бушменов несоразмерно велика по сравнению с их ростом. Один молодой бушмен, работающий на ферме в районе Гобабис, научился ездить верхом и на лошади преследовал антилоп бейза. Нагнав свою жертву, он на полном скаку спрыгивал с лошади и душил антилопу веревкой из воловьей кожи.

Бушменки так легко рожают детей, что, если, например, роды застают женщину в то время, когда все поселение перекочевывает на другое место, она просто исчезает часа на два, а затем с родившимся ребенком спешит вслед за остальными. Ребенка не отнимают от груди до следующих родов, а это может произойти через три-четыре года. Мяса, ягод и кореньев бушменам не хватает, а коров и коз они не держат.

Бушмены не придают значения даже серьезным ранениям. После столкновения с полицией группа раненых бушменов была доставлена в Гротфонтейн. Один молодой бушмен во время стычки лишился руки. Но он продолжал выпускать свои стрелы, зубами натягивая тетиву. Полицейские думали, что он умрет от шока и потери крови, но бушмен выздоровел. Когда хирург обрабатывал пулевые раны, бушмены весело переговаривались.

Сесиль ван дер Спуй, семь лет проработавший судьей в Юго-Западной Африке, записал такой случай. Один бушмен попал под девятитонный фургон. Колесо проехало ему по голове, а он потом только жаловался на головную боль. Однажды я встретил бушмена, которому бейза пропорола рогами нижнюю челюсть и нёбо. Он не испытывал никаких неудобств. Лишь во время разговора ему приходилось затыкать отверстие тряпочкой, чтобы можно было говорить быстро. Профессор Э.~Г.~Л.~Шварц рассказал ужасную историю о мальчике, который попал ногой в стальной капкан. Не сумев разжать капкан, он отрезал себе ногу и таким образом высвободился. Мальчик знал, что, если этого не сделать, он станет добычей леопарда.

\begin{figure}[ht!]
\centering
\includegraphics[width=90mm]{000009.jpg}
\caption{Столетний бушмен Абрахам}
\label{overflow}
\end{figure}



Бушмены, живущие в отдаленных районах (кунг и другие), видимо, не страдают от эпидемических болезней, распространенных среди европейцев. Их, например, почти не затронула эпидемия испанки в 1918 году. Самый страшный враг бушменов~--- малярия. В последние годы некоторых бушменов уговорили сделать прививку от черной оспы. У бушменов есть, конечно, свои собственные лекарственные травы. Путешественник Чампен не раз избавлялся от головной боли, используя средство бушменов~--- какой-то корень, который разогревают на огне и прикладывают ко лбу.

Шуман отмечал высокую смертность среди детей бушменов кунг. На обширной территории ему удалось встретить лишь три семьи, где родители смогли вырастить пятерых детей. Но семей с двумя-тремя детьми не так уж мало. Выжившие дети редко потом болеют и отлично переносят голодное время.

Когда у бушмена есть мясо, у него на удивление хороший аппетит. За один присест он может уничтожить курдючного барана весом тридцать фунтов. Правда, этот обед длится, пожалуй, подольше, чем наш банкет. Если бы бушмены не обладали такой способностью, они всегда бы рисковали умереть с голоду, так как дичь, которую они убивают, скоро портится и неизвестно, когда еще антилопа подпустит их на выстрел из лука. В этой жизни пиршеств и голода на помощь бушмену приходит природа. У взрослых желудок имеет способность растягиваться, как гармошка, если пищи много. У женщин несоразмерно развиты ягодицы и бедра~--- явление, известное ученым как стеатопигия. Из всех видов пищи бушмены по-настоящему любят только мясо. Однако им нередко приходится утолять голод такой скромной пищей, как каша из саранчи, <<бушменский рис>> (разновидность термитов), змеи и лягушки. Ни один народ не смог бы жить в тех пустынях, где живут бушмены. Даже дети, которых матери еще носят на спине, могут пить воду, как верблюды. Они тоже знают, что от одного источника воды до другого большое расстояние"

В Юго-Западной Африке сохранились самые прекрасные и самые древние рисунки бушменов на всем Африканском континенте. Они поразительно напоминают рисунки ледникового периода, обнаруженные на стенах пещер в Испании. Авторы этих реалистических произведений обращали внимание в основном не на цвет, а на композицию.

Лишь недавно благодаря инициативе доктора Эрнста Шерца из Виндхука многие замечательные произведения бушменской наскальной живописи и гравировки стали достоянием мировой общественности. Доктор Шерц и его супруга сфотографировали и описали шестьсот произведений искусства бушменов, найденных в различных районах Юго-Западной Африки. Кроме того, они показали многим ученым, художникам и любителям искусства пещеры и обиталища бушменов, в которых сохранились наиболее примечательные образцы искусства этого народа.

Потребовалось много времени, чтобы обнаружить эти шедевры наскального искусства, так как самые значительные из них расположены в отдаленных районах. Побывавшие здесь раньше путешественники и охотники или вообще их не замечали, или не придавали им никакого значения. Мне кажется, что первое сообщение о живописи бушменов Юго-Западной Африки сделал в 1870 году торговец (не миссионер) доктор Теофил Хан. Он обнаружил образцы цветной живописи, выполненные черной, желтой, красной и белой краской на скалах у реки Хойчаб. И что самое важное, доктор встретил старика бушмена, который все еще занимался живописью.

Есть люди, которые хотели бы лишить маленьких бушменов права называться авторами этих произведений первобытного искусства. В настоящее время бушмены не занимаются живописью и почти ничего или даже совсем ничего не могут сказать о рисунках, которые оставили их предки. На этом основании высказывались предположения, что наскальные рисунки были творением какого-то другого народа. Однако есть достаточные доказательства, что бушмены еще не так давно занимались живописью. Профессор Соллас, большой знаток жизни этих древних охотников, сообщил, что после столкновения фермеров с бушменами у одного из убитых бушменов нашли кожаный пояс, к которому было прикреплено двенадцать небольших рогов, наполненных различными красками.

Пастор К.~Г.~Бютнер, заключивший в восьмидесятых годах прошлого века от имени германского правительства ряд соглашений с местными племенами, открыл знаменитые амеибские рисунки. Амеиб~--- название фермы, расположенной в горах Эронго к северу от Карибиба. Там в одной из пещер (которая сейчас носит название <<пещера Филипса>>~--- по имени владельца фермы) миссионер Бютнер наткнулся на изумительные рисунки животных. Недавно их исследовал французский археолог аббат Брейль. Он заявил, что это самые выразительные рисунки из всех, какие он видел. Изображение слона на стене пещеры, по мнению Брейля, было сделано около пяти тысяч лет тому назад. С тех пор природные условия этого района значительно изменились. Там, где теперь пустыня Намиб, раньше текли полноводные реки, и на одном из рисунков изображен крокодил. В этой огромной пещере были найдены также изображения людей~--- мужчин и женщин. Эта таинственная символическая композиция представляет загадку, которую не так-то просто разрешить. При изучении некоторых рисунков Юго-Западной Африки чувствуется необходимость во втором <<розеттском камне>>, с помощью которого были в свое время расшифрованы египетские иероглифы. А пока все теории современных исследователей сводятся в основном к догадкам, иногда до смешного нелепым.

В районе Эронго найдены и другие замечательные произведения живописи. В пяти милях от фермы Отья-мапауэ, на гранитной стене пещеры Этемба высечена жирафа, несколько охотников и разные мифические фигуры. Ближе к берегу, над тенистой равниной, возвышаются вершины Большого и Малого Шпицкопа. В пещере Носорога на Большом Шпицкопе изображены великолепный красный носорог, охотники и снова загадочные фигуры с головой и телом человека и ногами животных. Неподалеку, среди отвесных скал, расположена Райская пещера. Несколько десятилетий назад, во времена германского господства, группа охотников забрела в эту сокровищницу искусства бушменов. Потехи ради охотники открыли стрельбу по некоторым из рисунков. Позднее сюда пришли люди, которые знали толк в искусстве, но они натворили не меньше бед. Пытаясь сдвинуть каменную глыбу, они повредили рисунки. В этой пещере найдено самое реалистическое изображение буйволов, которое только известно ученым. В наши дни в пустыне Намиб буйволы не водятся. На Малом Шпицкопе, в пещере Привидений, изображены таинственные фигурки из мифологии бушменов, а в пещере Жираф~--- красные и белые жирафы,~--- это вершина творчества древних художников.

В восьмидесяти милях от Шпицкопа поднимается гора, которую справедливо называют подлинным музеем искусства бушменов. Это гранитная гора Брандберг, расположенная на базальтовом основании. С южной стороны гора обрывается очень круто, но с севера и с востока до нее можно добраться по лабиринту диких горных ущелий. В 1907 году в эту неизвестную страну чудес, не нанесенную еще тогда на карту, попал офицер немецкой полиции лейтенант Йохман, который должен был произвести съемку. С трудом пробираясь по ущелью Цисаб (ущелье Леопарда), он обнаружил пещеру, на стенах которой были изображены ушастые змеи и другие сказочные животные, а также фигурки танцующих людей. Теперь эта пещера носит название <<пещера Йохмана>>. Древние бушмены оставили повсюду следы своего пребывания, но Йохман не был археологом. Он ограничился лишь тем, что нанес Брандберг на карту.

Затем здесь появился профессионал. Рейнхард Маак в первую мировую войну попал в плен, бежал оттуда и очутился в пустыне, где скрывался несколько месяцев. Маак был способным художником. От нечего делать он стал срисовывать изображения бушменов. Когда военные действия в Юго-Западной Африке прекратились, он объявился и был освобожден. К тому времени пустыня окончательно его очаровала. Вместе со своим товарищем Альфредом Гофманом он решил заняться исследованием горы Брандберг. Они провели в пустыне три месяца и обнаружили несколько стоянок бушменов, что было самым значительным из всех открытий того времени.

В горах Брандберг они не встретили ни одного человеческого существа. Лишь юго-западный ветер гулял по ущельям, и ему вторило глухое эхо, нарушая мертвую тишину. На песке, однако, они заметили человеческие следы, и, когда у них кончилась вода, путешественники пришли по этим следам к источнику. Он был прикрыт травяными циновками и сверху засыпан песком. Ни души вокруг. Даже животные, видимо, попрятались в своих укрытиях, хотя обычно около Брандберга бродили целые стада зебр, газелей и страусов.

Вот в какой мрачной обстановке Маак и Гофман вступили в пещеру, на которую в свое время не обратил внимания Йохман. На стенах было множество ясных рисунков. Огромные черные фигуры людей, газели и птицы, изящно выписанный носорог, антилопа гну, страус, гепард и другие животные. Один из наиболее древних рисунков изображал, наверное, тюленя~--- редкостный случай. Но что особенно очаровало Маака и Гофмана, а вслед за ними и всех других, так это вереница животных, людей и обитателей мира духов. Изображения не были сделаны рукой одного художника. На протяжении многих веков бушмены добавляли все новые и новые детали к этой странной панораме. Здесь всем нашлось место~--- и антилопе канне, и бейзе, и южноафриканской газели. Были там и антилопы с человеческими конечностями, и человек с мордой крокодила, а также получеловек-полуобезьяна. Музыканты, охотники и, наконец, странная фигура, которая вызвала бурные споры среди ученых, занимающихся искусством бушменов.

Маак срисовал эту фигуру и многие другие рисунки и направил их в Германию признанному авторитету в области искусства бушменов Гуго Обермайеру. Вместе с художником Гербертом Кюном Обермайер выпустил по материалам Маака объемистый труд <<Искусство бушменов. Наскальная живопись Юго-Западной Африки>>. Примечательно, что рисунок, вызвавший споры, Кюн описал как <<раскрашенный мужчина с луком, стрелой и цветком>>.

Несколько лет спустя доктор Шерц сфотографировал эти рисунки. Генерал Сиэтс, просмотрев их, понял, насколько важно сделанное открытие, и направил на место аббата Брейля. В <<раскрашенном мужчине>> аббат признал белую женщину, которая держала в правой руке не то чашку, не то белый цветок, а в левой~--- лук и стрелы. По его описаниям, тело женщины <<от талии до ног бело-розового цвета>>, а лицо белое, с тонкими чертами, ничего не имеющее общего с африканскими лицами. Аббат даже высказал мнение, что это женщина-воин с острова Крит и напоминает критских участниц боя быков. Он предположил, что корабль из Средиземного моря потерпел крушение в районе Капского Креста, в семидесяти милях от Брандберга, и что потерпевшие кораблекрушение нашли пристанище в горах. Они-то, по мнению аббата, и оставили этот рисунок.

Толкование наскальных рисунков~--- дело специалистов. С этим вопросом я обратился к покойной теперь Доротее Блик, одному из крупнейших в Южной Африке ученых своего времени. Блик без колебаний заявила, что на рисунке изображена бушменка, лицо и тело которой покрыты розовой глиной. Вот и все, что можно сказать о всемирно известной <<Белой Даме Брандберга>>. Однако в этой пещере есть на что посмотреть и помимо изображений грузных слонов и быстрых антилоп, которых достаточно и в других пещерах. Власти благоразумно ограничили посещение пещеры Белой Дамы, закрыв туда доступ случайным людям.

В том же ущелье расположена пещера Дождя. На ее стенах изображен красный дождь, льющийся из облака. Вполне возможно, что автор этого рисунка пытался запечатлеть истекающее кровью раненое животное. Здесь же неподалеку находится и открытая Йохманом пещера Скелетов, известная символическими изображениями смерти. На одном рисунке изображен человек, который держит в руках части человеческого тела. Рядом нарисован скелет. В одной из близлежащих пещер есть весьма необычный рисунок, изображающий кокербом~--- дерево, из коры которого бушмены изготовляют колчаны для стрел. Бушмены почему-то никогда не рисовали пейзажей и очень редко отдельные детали пейзажа. Растения на их рисунках встречаются очень редко. Лужу они могут обозначить кружком, но это, пожалуй, и все, на что способна фантазия бушменов в отношении ландшафта.

На одной стороне ущелья, напротив пещеры Белой Дамы, найден рисунок, который аббат Брейль назвал <<Школа девушек>>. На этом рисунке изображена процессия девушек-бушменок с бусами в волосах. Шествие возглавляет женщина, по виду настоящая старая ведьма. Эта работа производит впечатление карикатуры. Нос у старухи по форме отличается от носа девушек.

Из росписей в ущелье Цисаб Кюн выделяет два самых замечательных рисунка. На них изображены антилопы. Одна антилопа стоит боком, голова ее повернута к зрителю, другая антилопа изображена спереди. Эти рисунки проникнуты тонким чувством ритма.

Другие ущелья Брандберга пока еще не были обследованы учеными. Во время своей экспедиции аббат Брейль большую часть времени провел в пещере Белой Дамы. Десять дней он ночевал <<у ее ног>>. Еду ему доставляли прямо в пещеру. Прекрасные рисунки встречаются и в других местах этого района, особенно в ущелье Эмиса с западной стороны.

Несколько лет назад доктор Шерп, обнаружил ряд рисунков, высеченных на скале в сотне миль к северу от Брандберга. На одном из них изображена шестифутовая змея с головой зебры. В двадцати милях к северо-западу от Францфонтейна, в районе источников Ауб, есть наскальные изображения слона, жирафы и носорога. Шерц работал в основном на севере Юго-Западной Африки. Некоторые ученые высказали предположение, что, чем дальше к югу, тем реже встречаются рисунки бушменов. Я же полагаю, что тщательные обследования стоянок бушменов в горах, примыкающих к реке Оранжевой, приведут к новым ценным открытиям. Один сержант полиции рассказал мне, что в пятнадцати милях к востоку от Сэнделингс Дрифт, на северном берегу реки Оранжевой, он наткнулся на пещеру, на стенах которой изображены человеческие фигурки. До сих пор это изображение видели, вероятно, лишь полицейские да геологи. Я не встречал сообщений об этой пещере ни в одной научной работе. И наверняка найдется еще много таких пещер.

Перед маленьким бушменским художником была лишь только стена пещеры, и ему часто не хватало места для рисунков. Поэтому, когда со временем древние изображения стирались, он наносил <<современные рисунки>> на шедевры своих предков. Но, по мнению профессора Солласа, пока великий художник был еще жив или пока жива была память о нем, его творения сохранялись. Даже пещерные жители чтили красоту линий и красок этих гениальных творений.

Если вы и теперь сомневаетесь в интеллекте бушмена, обратитесь к его ядам, противоядиям и лекарствам. В этой области бушмен и по сей день удивляет современную науку. Как отравитель он намного превосходит Борджиа. Химики считают, что только после бесконечных поисков, опытов и ошибок бушмены открыли наконец те сложные яды, которые они употребляют сейчас.

Яды, которыми бушмены смачивают свои стрелы, представляют ценные лекарства, и в малах дозах они используются медициной для лечения сердечных заболеваний. Вполне возможно, что фармакопея бушменов содержит и ряд других средств первостепенной важности, и, если этот народ исчезнет, вместе с ним исчезнут и его секреты.

Доктор Ганс Шинц, медик и крупный ботаник, описывает свой эксперимент с одним бушменом, который утверждал, что не боится яда змей и скорпионов. Шинц взял двенадцать скорпионов и посадил их на обнаженное тело бушмена, выбрав наиболее чувствительные места. Скорпионы задрали свои хвосты и стали жалить бушмена, иногда по нескольку раз. Бушмен, казалось, совсем не чувствовал боли. Шинц осмотрел места укусов и не обнаружил никакой опухоли. Бушмен объяснил, что он глотает небольшими дозами яд скорпиона, и поэтому ему не страшны укусы. Таким же образом он может защитить себя от укуса змеи. И Шинц считает, что бушмен говорил правду.

Путешественник Чапмен узнал, что от укусов змей бушмены применяют какое-то ползучее растение, которое они называют эокам. К сожалению, ему не удалось определить, что это за растение. Сначала на месте укуса делается надрез. Затем знахарь пережевывает корень эокама и, оставляя во рту получившуюся кашицу, высасывает кровь из ранки. Вслед за этим больному дают рвотное из семян эокама. <<Бушмены, у которых на шее есть кусочек этого корня, смеются над змеями>>,~--- писал Чапмен.

Змеиный яд на острие стрелы~--- страшная штука. Редко кому удается выжить, если этот яд попадает в кровь. Каждое племя бушменов имеет свои ядовитые смеси в зависимости от того, что они могут найти на своих землях. Яд африканской гадюки и кобры имеет необычайную силу. Но и яд паука-землекопа почти так же страшен. Бушмены кунг умеют приготовить из гусеницы такой яд, что лев, раненный отравленной стрелой, начинает грызть землю в предсмертных судорогах. На стенах пещер было обнаружено черное вещество, в котором, возможно, есть мышьяк. Яды, как правило, смешиваются с соком растений, которые сами по себе могут быть и ядовитыми, и неядовитыми. Растительные яды не всегда годятся, так как действуют медленно, но они служат в качестве фиксирующего средства.

Ветками Euphorbia candelabra бушмены отравляют приходящих на водопой антилоп. Для этого основной источник воды отгораживается забором, чтобы животные не могли туда проникнуть. Затем в лужу набрасываются ветки эвфорбии. На поверхности воды образуется ядовитая пена, от которой погибают даже зебры. Мясо отравленных животных неядовито. Интересно отметить, что бушмены далеко не всегда вырезают мясо вокруг раны животного, убитого отравленной стрелой. Некоторые утверждают, что этот кусок~--- самый приятный на вкус.

Свои стрелы, как и свои яды, бушмены различных племен изготовляют по-разному. В Юго-Западной Африке стрелы чаще всего делают из тростника, а наконечники из кости предпочитают металлическим. По мнению охотников, они дольше сохраняют яд. Наконечники на стрелах делают съемные и носят в отдельной сумке. Ведь бушмены слишком хорошо знают, как велика опасность случайного отравления. Кроме того, цельная стрела может выпасть из раны антилопы, когда она убегает сквозь лесную чащу. А охотник должен быть уверен, что смертельный наконечник прочно засел в теле животного.

Бушмены Юго-Западной Африки знают, что такое стрелы с оперением, но почти не пользуются ими. Они полагаются на свое умение подкрадываться к жертве и стреляют с такого близкого расстояния, что в оперении нет необходимости. Стрелы бушменов~--- это настоящее произведение искусства, луки же сделаны грубо, хотя и отвечают всем требованиям. Нужно обладать большой силой и ловкостью, чтобы до предела натянуть тетиву, и, когда бушмен выпускает стрелу, она сдирает ему кожу на большом и указательном пальцах. При охоте на антилоп сильно натягивать тетиву нет необходимости. Но нельзя забывать, что бушмены убивают стрелами слонов и других крупных животных. Тут нужно, чтобы стрела глубоко вонзилась в тело. Бушмены могут пробить стрелой доску толщиной в дюйм. В свое время они пробивали насквозь стенки фургонов первых европейских переселенцев, и тем приходилось отрезать отравленные наконечники с внутренней стороны фургона. Однажды стрела бушмена~--- легкий тростник с каменным наконечником~--- пробила навылет тело лошади.

Мне доводилось видеть крошечные луки, величиной с человеческий палец. Сделаны они были из рога антилопы. В маленьком колчане хранились стрелы, пропитанные ядом. Бушмен не очень-то охотно расстанется с таким луком. Это не детская игрушка, а грозное оружие~--- так называемый <<любовный лук>> бушмена, или его <<револьвер>>. Это оружие он пускает в ход в тех случаях, когда поссорится с кем-нибудь из-за женщины. Обычно он направляет свою смертоносную стрелу в ухо спящего соперника.

Немцы не сразу поняли нрав бушменов. Губернатор Лейтвейн решил, что из них выйдут хорошие рабочие. В 1895 году, впервые приехав в район, заселенный бушменами кунг, Лейтвейн назначил образованного африканца Йоханнеса Крюгера, который жил в Гротфон-тейне, <<вождем бушменов>>. Крюгер неохотно принял назначение, так как знал, что управлять бушменами будет трудно. Но соглашение было составлено, и он его подписал. <<Я говорил губернатору, что бушмены не будут мне подчиняться, тем более что сам я не бушмен,~--- заявил Крюгер спустя двадцать лет представителям правительства Южно-Африканского Союза.~--- Он же мне ответил, что, раз я знаю бушменов и их язык, мне нетрудно будет оказывать на них влияние>>.

Крюгер получал пять фунтов в месяц и должен был поставлять рабочую силу. Он прислал более двухсот бушменов хейкум, но из-за недостатка вельдкоса (диких плодов, кореньев, ягод и трав) они вскоре возвратились обратно в родные места. Когда в этом районе поселились немецкие фермеры, начались настоящие неурядицы. По этому поводу Крюгер сделал заявление южноафриканским властям:

<<Бушмен, как правило, имеет одну жену. Он ее очень любит и хорошо к ней относится. Немцы же уводили у бушменов их жен и превращали их в своих наложниц. В районе появилась масса детей смешанной крови. И тогда впервые бушмены начали воровать у немецких фермеров скот. Один бушмен убил немца, который увел у него жену. Полиция и немецкие фермеры стали беспощадно истреблять бушменов>>.

Незадолго до начала первой мировой войны положение так обострилось, что фон Застрову, начальнику управления округа Гротфонтейн, предложили написать рапорт о возможности истребления или выселения из его округа всех бушменов. Фон Застров, мягкий человек (которого очень не любили фермеры), ответил, что такое предложение не заслуживает внимания. По его мнению, половина всех фермеров округа использует труд бушменов и не сможет вести хозяйство, если бушмены исчезнут. <<Надо понять, что люди, кочевавшие всю свою жизнь по вельду и никогда не занимавшиеся тяжелым физическим трудом, не могут сразу расстаться со своими привычками и стать умелыми работниками,~--- писал фон Застров.~--- Однако бушмены удивительно быстро научились пахать, управлять волами и выращивать табак. Многие подолгу работают на фермах, и без них не обойтись. Воровством занимаются только те бушмены, которые работали на фермах, а затем вновь вернулись домой. Иногда подобные преступления совершаются ими из мести. Я полагаю, что для бушменов нужно высвободить определенные территории~--- резервации. При хорошем обращении эти люди могут отказаться от кочевого образа жизни, осядут и будут приносить пользу>>.

В 1912 году бушмены убили в, округе Гротфонтейн сержанта немецкой полиции Гельфриха. Это привело к дальнейшему кровопролитию. Несмотря на распоряжения фон Застрова, многие фермеры устраивали самочинные расправы, охотясь на бушменов, как на диких зверей. Повторилась та же история, что и в Капской колонии в прошлом веке. Фермеры решили расширить свои владения за счет земель бушменов и начали уничтожать этих первобытных охотников. Бушменские племена были истреблены там почти полностью. Бушменов Юго-Западной Африки от подобной участи спасла, вероятно, первая мировая война. Впоследствии политика в отношении бушменов стала более человечной.

Майор Фрэнк Браунли, чиновник окружного управления в Гротфоктейне, в своем сообщении отмечал, что бушмены перестают испытывать страх перед белыми. Они приходят к источникам и просят табаку у проезжих. Если Браунли выражал желание увидеть вождя бушменов, тот исполнял его просьбу.

Однажды Браунли должен был выслать полицейский патруль, чтобы захватить бушменов, которые убили несколько человек из племени окаванго. Эти люди нанимались работать на фермах и рудниках и теперь возвращались домой, купив на заработанные деньги одежду и бусы. Полиция на верблюдах окружила убийц и доставила их в Гротфонтейн. Это оказались бушмены кунг. Браунли увидел, как отлично они сложены, и послал телеграмму Перингею, директору Кейптаунского музея. Опытный таксидермист, Джеймс Друри сделал гипсовые слепки, которые были затем выставлены в музее. Тысячи людей подолгу рассматривали эти изумительные фигуры, но вряд ли кто-нибудь из них мог предположить, что натурщиками Друри были убийцы.

Одним из актов милосердия по отношению к бушменам со времени образования Южно-Африканского Союза было принятие оговорки, которая давала тюремным властям право освобождать бушменов, если они в очень плохом состоянии. В тюрьме маленькие выносливые охотники нередко чахнут и умирают. Для многих бушменов шестимесячное заключение равнозначно смертному приговору.

Работники юстиции отмечали, что по своей природе бушмены очень правдивы. По законам пустыни мать-бушменка обычно убивает новорожденного, если он появляется вскоре после предыдущего. Такие преступники и до сих пор предстают перед судом, но судьи относятся к ним снисходительно. Приведу небольшой диалог, взятый из судебного протокола.

--- Судья. Вы убили своего ребенка, чтобы спасти другого, который у вас на спине.

--- Бушменка. Да.

--- Судья. Можете ли вы когда-нибудь сделать то же самое и с этим ребенком?

--- Бушменка. Нет, я люблю его.

Бушмену непонятна судебная процедура. Как-то одного бушмена судили за убийство, и судья, который, как обычно, был в парике и регалиях, приговорил его к смертной казни. Когда у бушмена спросили, не желает ли он что-нибудь сказать, тот через переводчика задал вопрос: <<Какое отношение ко всему этому делу имеет вон та пожилая женщина в красном платье?>>

Бушмены долго помнят обиды. Недавно сенатор Феддер сделал сообщение о наследственной вражде между двумя бушменскими племенами. Вражда началась с убийства, которое произошло сто лет назад, но и до сих пор эти старые враги посылают друг в друга свои отравленные стрелы. Феддер обратился к сенату Южно-Африканского Союза с просьбой создать резервации для бушменов, причем не одну, а несколько, иначе бушмены могут истребить друг друга.

Сенатор Феддер указал также, что в северных районах есть фермеры, которые охотно используют труд бушменов, но есть и такие, кто почти не рискует выходить из дома, опасаясь ядовитых бушменских стрел. Фермерам, которые хотят жить с бушменами в добрых отношениях, достаточно лишь сказать им: <<Живите там, где жили до сих пор. Пользуйтесь источниками, которыми вы пользовались. Собирайте в вельде коренья и стреляйте антилоп. А если ваши молодые люди пожелают у меня работать, я дам им табаку>>.

Вот и весь секрет. Бушмены смотрят на земли, занятые фермерами, как на свои владения, какими они были на протяжении многих столетий. Они не возражают против того, чтобы там поселились белые, но они ревниво оберегают свои старинные права. Если у бушменов не будет воды и кореньев и если им запретят охотиться, то они погибнут, Фермер, который отнимает у них средства к существованию, вполне может ожидать, что его настигнет отравленная стрела.

После окончания второй мировой войны многие жители Гобабиса в той или иной мере использовали на своих фермах труд бушменов нарон. Одним из них был К.~Р.~Пайпер. За неимением рабочих рук ему пришлось расстаться со своими предубеждениями и изменить свое отношение к бушменам. Молодой бушмен Конеллан, работавший у него на ферме, научился водить трактор, еще несколько человек работало на других машинах. Остальные доили коров, ухаживали за садом и выполняли работы по дому.

\begin{figure}[ht!]
\centering
\includegraphics[width=90mm]{000010.jpg}
\caption{Танец бушменок}
\label{overflow}
\end{figure}


Пайпер совсем не удивился, когда увидел, что у бушменов нет точного представления о времени. Если бушмен уходил на неделю, он мог вернуться только через месяц. А в зимние месяцы на ферме не оставалось ни одной бушменки: они собирали в это время дикие орехи, и удержать их было невозможно. Но Пайпер нашел выход. Он стал давать им обезьяньи орехи, которые они предпочитали своему вельдкосу. За корзину орехов бушмены могут исполнить танец ветра над травой. Женщины становятся в полукруг, поют и хлопают в ладоши, а мужчины танцуют и подражают завыванию ветра.

Бушмен не знает, что такое деньги. Он отдает монеты детям вместо игрушек или женщинам на украшения. Недельного запаса кофе и сахара ему хватает на два дня. Но наступает момент, когда даже эта роскошь больше не может удержать бушмена на ферме, и он уходит на широкие просторы вельда.

Бушмены не заглядывают в будущее. Когда у них есть пища, они играют и веселятся, покоряя сердца всех окружающих. В Африке нет более счастливых людей, чем бушмены, у которых достаточно пищи. Бушмены очень любят своих детей и выражают эту любовь разными способами. У них есть, например, танец антилопы бейзы, и в этом танце отцы преподают детям такой урок, который в будущем помогает им сохранить свою жизнь во время охоты. Хитрый старый охотник исполняет роль бейзы, приставив к голове острые палочки вместо рогов. Дети выступают в роли собак. Вся эта сцена охоты настолько реалистична, что люди как бы исчезают. Старый бушмен превращается в рогатую антилопу, а нападающие на него дети, которые лают и щелкают зубами,~--- в настоящих собак. Антилопа яростно отбивается рогами, бросаясь из стороны в сторону, затем постепенно устает, начинает задыхаться, продолжая отчаянно бороться за свою жизнь. И тогда охотники с криком наносят ей удары копьями. Наконец обессилев, антилопа опускается на землю. Охотники и собаки окружают ее, и на этом танец заканчивается. Пока юноша не убьет бейзу или другого опасного зверя, его не считают мужчиной. Поэтому родители очень заботятся, чтобы их сын был хорошо подготовлен к такому испытанию.

Из домашних животных у бушменов есть только собаки. Они никогда не держали скота, и поэтому (не в пример более зажиточным готтентотам) имеют весьма ограниченное понятие о счете. (<<Раз, два, три, много>>,~--- считает бушмен.) Собаки служат им уже много веков. Вместе они скитались по Африке, вместе одерживали небольшие победы. Самая лучшая порода бушменских охотничьих собак~--- светло-коричневая, с темным ремнем вдоль спины дворняжка, чем-то напоминающая борзую. Сейчас она на грани вымирания. Это, вне всякого сомнения, лучшая в мире охотничья собака, поджарая и злая, которая может не подпускать раненого леопарда до тех пор, пока ее хозяин не улучит момента, чтобы метнуть копье. Я видел таких собак. Самая типичная из них имела четырнадцать дюймов в холке, а длина ее тела, казалось, не соответствовала такой высоте. В ней не было ничего красивого. Широкий лоб, острая морда, стоячие уши, длинный опущенный хвост. Зато это, может быть, самая древняя и, конечно, самая умная собака в мире.

Такой же скиталец, как и ее хозяин, бушменская собака постигла искусство молчания. Некоторые утверждают, что она вообще никогда не лает. Во всяком случае она никогда не выдаст беспричинным лаем убежище хозяина. Собака всегда тихонько следует за своим хозяином, стараясь по возможности укрыться в тени, и бережет свои силы, пока ей не прикажут отвлечь на себя внимание животного. Почуяв опасность, она лишь слегка тявкает в знак предупреждения. Бушмен и его собака верят и понимают друг друга, Многие путешественники пытались выменять у бушменов охотничью собаку на табак, но я еще не слышал, чтобы кому-нибудь удалось это сделать.

Охотясь на антилоп и страусов, бушмен непременно воспользуется соответствующей маскировкой и своим даром подражания. Надев на себя шкурку и перья страуса и поддерживая шею птицы длинной палкой, он может войти в середину страусового стада, прихорашиваясь на ходу, как это делают обычно страусы. Чтобы подкрасться к антилопе, бушмен использует небольшой куст. Во время охоты терпению бушмена нет предела. Чапмен вспоминает свою встречу с бушменом, который целых пятьдесят миль преследовал раненную им жирафу. Многие бушмены могут три дня подряд идти без отдыха по следу раненого животного, но ни за что не расстанутся со своей добычей.

Не удивительно, что этот первобытный народ знает толк в таком древнем искусстве, как устный рассказ. Однажды мне самому довелось увидеть и услышать, как один бушмен, который только что вернулся домой издалека, присел на корточки у костра и заворожил небольшую аудиторию своим рассказом,

---~Он говорит обо всем, что видел во время своего путешествия,~--- объяснил мне переводчик.~--- О животных, птицах, деревьях, всех живых существах, населяющих вельд, об отдаленных источниках, у которых он утолял свою жажду. Для этих людей любая подробность имеет значение. Он будет говорить час, может быть, и два, а окружающие будут слушать его, не проронив ни слова. А потом, когда он закончит, они будут задавать ему вопросы.

В одном из ранних сообщений о бушменах, составленном голландским путешественником семнадцатого века, говорится, что это <<совершенно дикий народ, у которого нет ни жилищ, ни скота, но который хорошо вооружен луками и стрелами>>. Такими они остались и сейчас, в чем я убедился собственными глазами. В поисках пищи и воды они постоянно переходят с места на место, строят примитивные укрытия из травы, которые защищают их от ветра., Там они разводят огонь, чтобы отпугивать львов. У каждого пожилого бушмена найдешь на морщинистом животе следы ожогов, которые он получил, когда, скорчившись, проводил холодные ночи у костра. Бушмены живут небольшими группами, не больше двадцати человек. Слишком тяжелы для них поиски пищи, и, чтобы выжить, они вынуждены разбиваться на такие маленькие группы.

Воду бушмены хранят в скорлупе страусиных яиц, и у каждой группы есть свои тайные источники воды~--- маленькие колодцы, заложенные камнями и засыпанные песком так, чтобы ни малейший признак не выдал местонахождения драгоценного хранилища. Если во время засухи источники высыхают, а запасы воды в скорлупе страусиных яиц кончаются, бушмены всегда могут найти коренья, луковицы и удивительное растение тсам-ма~--- дикую дыню и тем самым поддержать свое существование. Дикая дыня содержит много влаги, но в ней почти нет питательных веществ, и я видел детей, у которых от злоупотребления ею были страшно вздуты животы. Однако, не будь этих дынь, пустыня стала бы необитаема. Они всегда растут на верхушках дюн, бушмены собирают их и зарывают в песок, где они могут храниться в течение нескольких недель. Из размолотых семян дыни готовят напиток~--- жалкий кофе бушменов. Этот зеленый пятнистый плод употребляют в пищу все животные, начиная от слона и льва и кончая мышью. Бушмены разогревают дыню на костре, затем охлаждают ее и утоляют свою жажду. Поджаренная тсамма и мясо шакала~--- любимое лакомство бушменов.

Вероятно, больше ни один народ не выдержал бы такого существования и погиб, если бы ему пришлось жить только на соке дыни, кореньях и ягодах. А бушмены вынуждены были так жить, и все же, несмотря на все лишения, эти выносливые и отважные охотники дожили до наших дней. Как бы примитивны ни были бушмены, однако они могут выжить в таких условиях, которые для цивилизованного человека означали бы неминуемую смерть. Оставьте бушмена-охотника одного в пустыне, голого, с пустыми руками, и он раздобудет себе пищу, одежду, высечет огонь и будет жить обычной жизнью. Когда вы видите бушменов в их родной обстановке, вы видите своих предков. Вот почему я назвал бушменов романтическим реликтом, настоящими властелинами пустынь.



\chapter{Рай бушменов}

В Юго-Западной Африке все слышали легенду о <<рае бушменов>>~--- затерянном оазисе, где дети играют алмазами. Но никому еще не удалось найти его.

В 1929 году мне довелось лететь из Кейптауна на самолете, зафрахтованном одной компанией для поисков <<рая бушменов>>. Самолет вел капитан Р.~Р.~Бентли, служивший когда-то в британском, а потом в южноафриканском военно-воздушном флоте. Он считался отличным летчиком, который может справиться с любым заданием. Нередко Бентли приходилось развозить документальные фильмы в самые отдаленные уголки Южной Африки, и он всегда поспевал вовремя.

Дик Бентли, естественно, ничего не говорил о своем задании. Он высадил меня в пустынной местности к северу от реки Оранжевой, приземлившись на дне огромной впадины, которая послужила ему отличным аэродромом. Затем Бентли снова поднялся в воздух и взял курс на Юго-Западную Африку. Там ему надлежало забрать представителя компании, который должен был вести поиски.

У меня было достаточно времени, чтобы обдумать как следует всю эту загадочную историю с сокровищами. Вполне достаточно времени, чтобы порассуждать, добьется ли Бентли успеха или нет. В этой легенде о <<рае бушменов>> меня всегда удивляло одно странное обстоятельство: легенда появилась в Юго-Западной Африке задолго до того, как там было открыто месторождение алмазов в 1908 году. Мне подробно рассказали об одной экспедиции 1871 года, которая отправилась на поиски <<рая бушменов>>. А с тех пор было еще много экспедиций.

В 1871 году организатором поисков был английский охотник и изыскатель, работавший на только что открытых алмазных копях в Кимберли. Среди людей, охваченных алмазной лихорадкой, он мог бы выбрать для себя сколько угодно компаньонов. Но ему хотелось отправиться в эти пустынные районы одному. Бушмен, служивший у него погонщиком волов, сказал ему однажды, что он знает место, где такие алмазы, как в Кимберли, можно собирать пригоршнями. Предприимчивый англичанин приготовился к длительному путешествию. Он нагрузил фургон провиантом и направился к реке Оранжевой.

Погонщик был, по-видимому, одним из тех бушменов, которые угнали однажды скот с берегов реки Оранжевой. Отряд европейцев разыскал их и почти всех уничтожил. Этот бушмен остался в живых, но ему пришлось покинуть горную цитадель, где он жил со своими соплеменниками, -<<рай бушменов>>~--- и уйти в Капскую колонию, чтобы устроиться там на работу.

Англичанин переправился со своей повозкой на северный берег реки Оранжевой чуть выше одинокого водопада Ауграбис и там остановился на четыре дня для отдыха. Кроме бушмена-погонщика с ним было еще несколько слуг. Англичанин взял с собой двадцать два вола, так что вышедших из строя животных было чем заменить.

Я хорошо знаю этот пустынный уголок и жалею, что не видел его в те дни, когда немцы еще не захватили Юго-Западной Африки. Эта местность и теперь безлюдна, но в те времена в реке водились бегомоты, а дичи было наверняка значительно больше. Но я могу себе представить, как этот английский искатель приключений отдыхает где-нибудь у воды, в тени благоухающей мимозы, как он стреляет себе на обед антилоп-серн и цесарок, ловит в ручье усачей, любуется фламинго и обезьянами.

Стоит только отойти от берега, как перед вами открывается унылая пустыня. Кроме того, в тех местах берега реки Оранжевой окружены горами, и редко где путешественнику удается найти доступ к воде. Проводник-бушмен привел англичанина в выжженный солнцем безводный район к западу от водопада Ауграбис, и дальше они поехали по горячему песку. Для волов эти дни оказались тяжелым испытанием. Силы их были уже на исходе, когда бушмен вывел наконец экспедицию по узкому ущелью к реке.

Они ехали, пока позволяла дорога, а затем распрягли обессилевших волов и повели их по звериной тропе. Эту тропу издавна проложили слоны, носороги и другие дикие животные. Она вела к прекрасному озеру среди густого кустарника. Озеро это питал ключ. Тут, видимо, и был <<рай бушменов>>. Ведь если у бушмена есть мясо и вода, он доволен. А здесь и того и другого было вдоволь.

Но в ту же ночь случилось первое несчастье. Леопард подкрался к фургону и задрал собаку. Затем, отведав крови, вернулся снова и напал на одного из слуг. Бушмен, схватив ассегай, смело вступил в бой с леопардом. На помощь подоспел англичанин с ружьем, но слишком поздно. Растерзанный леопардом человек был уже мертв.

На следующий день англичанин и бушмен пешком отправились на поиски алмазных россыпей, захватив с собой вяленого мяса, сухого печенья и бутылки с водой. Углубившись в горы, они наткнулись на ограду из колючих растений, явно созданную рукой человека. И там, к великому удивлению, бушмен встретил своего деда. Этот морщинистый старик, по-видимому, сумел убежать от отряда европейцев и теперь жил одиноко в этом <<раю>>. Как и все бушмены, он питался ящерицами, насекомыми, кореньями и дикими плодами. Иногда к этой пище добавлялось мясо антилопы, которую он подстреливал из лука. Опасаясь леопардов, старик соорудил себе убежище из колючего кустарника.

Последний этап пути к алмазным россыпям начинался с пещеры, которая вскоре сузилась и перешла в туннель. Здесь было душно, и путники, пробираясь в темноте ползком, совершенно задыхались. Бушмен знал, что тут они могут наткнуться на змей. Англичанин же боялся, что туннель обвалится и они попадут в западню. Только мысль об алмазах заставляла его идти вперед.

Наконец они благополучно добрались до открытой площадки, окруженной крутыми стенами. Это напоминало кратер вулкана, и англичанин сообразил, что выбраться отсюда можно только тем же самым путем~--- через туннель. Под ногами у них был алмазоносный песок, из-за него-то англичанин и пустился в такой долгий путь. За час они нашли двадцать пять алмазов. В кармане у англичанина теперь лежало целое состояние, но на душе у него было неспокойно. Его угнетала мысль о смерти слуги и, кроме того, страшил обратный путь через туннель. Он решил сразу же отправиться обратно, чтобы этот опасный участок поскорее остался позади.

Предчувствие не обмануло англичанина. На обратном пути бушмена укусила змея. Проводник дополз до выхода из пещеры, и тут силы оставили его. Англичанин бросился за помощью к старику. У бушменов действительно есть противоядия против укусов змей и отравленных стрел, неизвестные еще европейцам. Но на этот раз помощь пришла слишком поздно, и проводник умер.

Англичанин уже был сыт по горло <<раем бушменов>>. Он возвратился в Кимберли и продал алмазы. Вырученная сумма была значительной, хотя теперь называют разные цифры. Потом он уехал в Англию, но перед отъездом успел рассказать о своем открытии.

В разное время на поиски этого кратера, затерянного в горах к северу от реки Оранжевой, было снаряжено восемь экспедиций. Отправлялись туда на верблюдах, на лошадях, в запряженных волами повозках. Были обследованы многие отдаленные горные ущелья. Но англичанин не оставил карту, а проводник-бушмен был мертв, так что ни одна из этих экспедиций не напала на след <<рая>>.

И вот теперь была снаряжена девятая экспедиция, на самолете. Я с нетерпением ждал возвращения Дика Бентли. Когда наконец маленький <<мотылек>> приземлился, я сразу понял по выражению лица Бентли, что местонахождение <<рая бушменов>> по-прежнему остается тайной. У меня и сейчас хранится блокнот, в котором записан рассказ летчика.

<<Расставшись с вами,~--- рассказывал он мне,~--- я полетел над унылой рыжей равниной, на которой лишь кое-где были разбросаны дома фермеров, примерно милях в двадцати пяти друг от друга. Потом я поднялся выше и полетел вдоль реки над острыми пиками гор. Ферм там уже не было. Лететь в таких местах очень опасно. С воздуха эти голубые зубчатые горы очаровательны, но в случае вынужденной посадки там негде приземлиться.

В одном месте среди красных вершин выросла темно-синяя громада. Затем еще один горный страж, как бы согнутый рукой игравшего с ним великана и до сих пор не затвердевший. Сквозь нагромождение всех этих скал несла свои мутные воды река. К северу простиралась мертвая пустыня, изборожденная руслами высохших рек.

Я приземлился в назначенном месте, гораздо севернее реки, и встретил там пожилого человека, которому компания поручила вести поиски. Из разговора с ним я выяснил, что ни у кого не было ясного представления о местонахождении ``рая бушменов''. Мне предстояло обследовать весь горный массив между морем и водопадом Ауграбис.

Я был не рад, что связался с этим рискованным делом, да еще на таком маленьком одномоторном самолете. На обратном пути я пытался разыскать кратер и озеро, но подо мной были лишь неприветливые, голые горы. Над ними можно лететь месяц и ничего не найдешь. Этим алмазам долго еще предстоит лежать в ``раю бушменов''>>.

Существует и другая легенда, которой многие верят. Она основана на немецких военных документах, найденных в Виндхуке во время войны 1914-1918 годов, когда южноафриканские войска вторглись в Юго-Западную Африку.

В ней говорится, что задолго до того, как в пустыне Намиб были открыты залежи алмазов, там появился немецкий патруль и во время песчаной бури один из солдат отстал от отряда. Следы его замело, и все думали, что он потерялся, короче говоря, погиб. Однако несколько недель спустя пропавший солдат приковылял на сторожевую заставу и сообщил странную историю. Он сказал, что его подобрали бушмены и привели в оазис среди дюн. И там он увидел, как дети бушменов играют алмазами.

Ему никто не поверил. Солдат же, получив увольнение, сел на верблюда и один отправился на поиски оазиса.. Через некоторое время еще какой-то патруль наткнулся на его труп. Вероятно, солдат побывал в оазисе, так как на этот раз составил план маршрута и в его кармане нашли четыре алмаза. Из спины у него торчала одна из тех маленьких смертоносных стрел, которыми стреляют бушмены.

Все прежние старатели пустыни Намиб твердо верили в эту легенду. Покойный Фред Корнелл, самый известный до первой мировой войны искатель сокровищ, отправился разыскивать <<рай бушменов>> на маленьком катере. Он высаживался в разных районах побережья и затем углублялся в пустыню, насколько ему позволяли запасы питьевой воды. Много раз он был на краю гибели, но смерть свою нашел в Лондоне, в автомобильной катастрофе. Корнелл думал, что скорее всего он отыщет путь к <<раю>>, если высадится в районе островка Хол-лэмз Берд (в ста пятидесяти милях к югу от залива Уэлвис) и затем пойдет прямо на восток.

Я знал человека, который утверждал, что побывал в <<раю бушменов>>. Звали его Г.~Л.~Гринфилд. В 1931 году он руководил разработками алмазных месторождений на побережье Юго-Западной Африки. К нему явился какой-то готтентот и предложил провести его к этому месту. Гринфилд решил, что лучше всего добираться туда с запада, так как с той стороны можно проехать на автомобиле до самых дюн.

Они отправились в путь, добрались на машине до дюн и дальше поехали верхом. Готтентот привел его в расположенную среди дюн долину с явными признаками алмазов, но никакого намека на легендарный оазис там не было. Вскоре Гринфилд вынужден был вернуться. Причиной, как всегда, оказался недостаток воды. Гринфилд собирался снарядить туда еще одну экспедицию, но правительство решило ограничить такие розыски, и ему не удалось получить разрешения.

Мне не раз говорили, что <<рай бушменов>> лежит в районе Богенфелса. Богенфелс~--- это огромный естественный туннель, вымытый в известняковых породах на побережье. Настоящее геологическое чудо. Многие находили тут сокровища. Но в этих местах такой сильный прибой, что пристать к берегу почти невозможно. Здесь утонуло немало народу. Недалеко от известнякового прохода похоронен искатель сокровищ, который предпочел пулю смерти от жажды. В песках Богенфелса был найден меч и другие предметы средневековья. Некоторых это натолкнуло на мысль, что у берегов потерпел крушение испанский галеон. В этих краях погибло столько людей, что стали говорить о <<проклятии Богенфелса>>. Один человек нажил здесь миллионное состояние на алмазных участках, но это богатство не пошло ему впрок, и впоследствии он застрелился.

И наконец, еще одно свидетельство, подтверждающее реальность легенды о <<рае бушменов>>. Это мне сообщил один известный южноафриканский летчик, принимавший участие в войне и теперь обслуживающий пассажирские линии. Имени его я называть не буду.

Когда-то, еще до войны, этому летчику пришлось лететь из Виндхука к бухте Людериц. В те годы авиация в Юго-Западной Африке делала свои первые шаги. Полет был вызван крайней необходимостью, и летчик выбрал такой маршрут, который сейчас никто не одобрил бы. Он летел над такой пустынной местностью, что, если бы ему пришлось сделать вынужденную посадку, его никогда бы не смогли найти.

Пролетая над дюнами пустыни Намиб, летчик заметил участок, покрытый деревьями и травой. На его карте этот оазис не значился. Пилотам дают хорошие карты, но, как я уже сказал, самолет летел над неисследованным районом пустыни. Чтобы лучше рассмотреть оазис, он спустился пониже. Кругом было множество животных, но никаких следов человека. Возможно, этот оазис, который пилот разглядывал несколько минут, и был <<раем бушменов>>.

Мне кажется, что Дик Бентли в общем прав. Слишком много людей погибло в поисках <<рая бушменов>>, а когда люди ищут сокровища, они способны обследовать самые отдаленные уголки и идти на большой риск. Может быть, <<рай бушменов>> и будет найден, но я чувствую, что пустыня потребует еще много человеческих жертв, прежде чем кому-нибудь из путешественников посчастливится наконец открыть этот оазис и разыскать алмазы.

\chapter{Замок в пустыне}

В Юго-Западной Африке есть замок, история которого так же удивительна, как и все другие африканские истории. Это замок Дувизиб, расположенный на окраине пустыни Намиб, к востоку от Мальтахёэ. Дувизиб производит необычайное впечатление. Он возникает перед вами совсем неожиданно. Вы взбираетесь по крутым каменистым тропам и вдруг в кольце голых, иссушенных гор видите замок. Когда вы попадаете в прохладный баронский зал, у вас опять захватывает дух от изумления.

Барон Гансгенрих фон Вольф, построивший этот замок, был офицером германской артиллерии и относился к числу тех аристократов, которые отличаются странными манерами и склонностью к спиртным напиткам. Я не собираюсь делать из него героя, но в его трагической судьбе произошли события, которые говорят в его пользу. Он, несомненно, не мог бы ужиться с нацистами. Барон жил как ему хотелось, и в этом отдаленном районе о нем до сих пор хорошо отзываются.

Жена фон Вольфа, миниатюрная блондинка Джейта, была внучкой доктора Фредерика Хамфриса, нью-йоркского предпринимателя, производящего гомеопатические лекарства. Она родилась в городке Саммите, штат Нью-Джерси, в 1881 году. Когда отец Джейты умер, мать ее вышла вторично замуж за адвоката ирландско-американского происхождения Гэффни, который был другом кайзера Вильгельма Второго. Гэффни был назначен американским генеральным консулом в Дрездене, и Джейта жила там вместе с ним до 1907 года, пока не встретила Гансгенриха фон Вольфа и не вышла за него замуж.

Фон Вольф был в опале. Он принимал участие в войне с готтентотами, защищая отдаленный аванпост в районе Мальтахёэ. В столкновении с превосходящими силами готтентотов он потерял полевые орудия. Более опытный офицер на его месте отбил бы атаку врага.

Но семья фон Вольфов не отличалась военными талантами. Во время франко-прусской войны 1870 года отец барона потерял целую батарею. Теперь сам барон оставил готтентотам свои орудия и запасы продовольствия. Вместе с остатками гарнизона он спасся бегством и добрался до деревни Мальтахёэ.

Я неспроста вспомнил этот небольшой эпизод теперь уже забытой войны, потому что он поможет вам понять историю замка Дувизиб, историю, которая иначе осталась бы для вас непонятной. Барону фон Вольфу разрешили уйти в отставку. Всякий другой, оказавшись в его положении, пал бы духом.

Но барон фон Вольф через несколько лет вернулся в Юго-Западную Африку~--- туда, где он потерпел поражение. Он приехал вместе с женой, поселился в Лю-дерице и немедленно взялся за дело, поражавшее всех, кто о нем слышал. В пятидесяти милях от Мальтахёэ барон купил у правительства ферму в пятьдесят шесть тысяч гектаров, уплатив по три пенса за гектар. Другими словами, за семьсот фунтов он приобрел участок приблизительно в сто тридцать тысяч акров. Сейчас он, вероятно, оценивается в пятьдесят тысяч фунтов вместе с замком, строительство которого обошлось барону в двадцать пять тысяч фунтов. Рабочая сила в те времена стоила очень дешево.

Каждый пароход, прибывавший из Германии, привозил барону старинную мебель, строительные материалы, стальные балки. Все это предназначалось для замка. Африкандер Адриан Эстеруизен на двадцати воловьих упряжках вез этот груз через пустыню Намиб. Каждый раз он проезжал туда и обратно четыреста миль. Два года понадобилось на эту перевозку. И вот наконец последний груз был доставлен на уединенную ферму. К тому времени сюда уже прибыли мастера~--- итальянские каменщики и плотник-швед. Целая армия рабочих возила камни из карьера.

Пока строился замок, барон и его жена жили в домике неподалеку. Нужно побывать в Дувизибе летом, чтобы понять, с какими трудностями им пришлось столкнуться. Я приехал туда в конце октября, когда земля изнывала от самой сильной засухи, которая только случалась в этих местах. Даже у неприхотливых южноафриканских газелей от истощения торчали ребра. Подгоняемые голодом куду совершали по ночам набеги на сады фермеров. Бабуины висели на деревьях, как меховые мешки, и лишь слегка шевелились при появлении автомобиля. Сдохшие овцы были сложены в кучу для сожжения.

Это земля, где деревья почти не дают тени, земля, где даже гадюки и скорпионы вынуждены искать защиты от солнца. На языке готтентотов Дувизиб означает <<место белого мела, где нет воды>>. Обычно в этих краях меловые обнажения показывают, что недалеко от поверхности есть вода. Однако барону пришлось бурить ручным буром скважину глубиной двести футов, пока он добрался до грунтовых вод. Этим источником на ферме пользуются и сейчас. До сих пор безотказно работает установленная бароном ветряная мельница с маркой дрезденской фирмы.

Во время строительства замка барон с женой ездил в Соединенные Штаты, чтобы раздобыть там денег для завершения своего грандиозного строительства. От старика бухгалтера, покойного теперь Герберта Хассенштейна, я узнал, что военная пенсия барона составляла пятнадцать фунтов в месяц, а ежегодный доход его жены равнялся пятнадцати тысячам фунтов.

К концу 1909 года строительство замка было завершено. А теперь пройдемся по замку Дувизиб и попытаемся воссоздать картину жизни барона и его маленькой белокурой американки. Целый год они прожили в домике из двух небольших комнат и теперь наконец могли в полной мере насладиться роскошью дворца. У барона и баронессы детей не было, и все же, по общему признанию, это была счастливая пара. Даже безрассудное поведение барона не омрачало их счастья.

Внешние стены замка имеют в толщину два фута. В торцовой части здания проделаны бойницы, на фасаде~--- забранные металлическими решетками окна. В центре, над порталом, возвышается массивная башня, а по углам~--- башенки поменьше. Как только вы откроете входную дверь, вам бросятся в глаза редкие старые цветные гравюры, которые свидетельствуют, что хозяин замка был неравнодушен к лошадям. Тут же вы увидите резной ларь, сделанный в 1700 году. Цена его~--- пятьсот фунтов. В огромном холле с каменным полом по стенам развешаны пистолеты, мечи и сабли и почти полный набор рисунков <<испанской школы верховой езды>>. По узкой лестнице вы попадаете на галерею. Сверху вы можете полюбоваться стеклянными канделябрами и каминами, стульями, фамильным гербом фон Вольфов и старинными столами. Отсюда виден также двор с фонтаном и клумбами и пальма, которую посадил барон.

В 1909 году вы могли бы увидеть с этой галереи барона и баронессу, угощающих своих гостей шампанским. Среди гостей немецкие офицеры, приехавшие покупать у барона лошадей для армии, бородатые возчики-африкандеры, вдохнувшие в замок жизнь, а также управляющий округа Мальтахёэ. Барон фон Вольф был аристократом с демократическими принципами, и тех, кому не нравились его друзья, он не задерживал в своем доме.

Под залом помещался винный погреб, доверху наполненный бутылками писпортера, рислинга, бернкаст-лера, либфраумилха, нирштейнера и цельтингена, бочонками пива и ящиками с шотландским виски.

В замке семь спален для гостей, в каждой из них камин и медная кровать. Во дворе помещение для слуг. Большие комнаты обшиты дубом. Сам барон и баронесса занимают роскошные апартаменты в одной из угловых башен. Баронский зал выложен каменными плитами, в других комнатах полы паркетные.

В замке сразу же был установлен современный водопровод. Однако план водопровода затерялся, и нынешний управляющий говорит мне, что в случае неполадок с отстойником он вынужден производить большие земляные работы, чтобы устранить повреждение.

Одна комната в замке особенно меня притягивает. Роспись на ее потолке, должно быть, имела какое-то значение, но теперь ее смысл не ясен. Эта маленькая комната расположена в башне над главным входом. Странная комната со старинным зеркалом. Мне сказали, что это была дамская туалетная комната. Картина изображает летящий над северным полюсом цеппелин. А за окном мерцает бурая земля в потоках горячего воздуха.

Барон~--- блестящий пианист и прекрасный певец. В тот вечер, когда в замке справляется новоселье, он в отличной форме. Гости в изумлении бродят по длинным залам. Они осматривают <<комнату Наполеона>>, где на гравюрах изображен Наполеон в разные периоды жизни (не удивительно, что неудавшийся артиллерийский офицер был страстным поклонником Наполеона). Они трогают руками дубовый гардероб 1735 года, украшенный мозаикой, пытаясь найти, как и я теперь, потайной замок. Затем почтительно останавливаются перед портретом крон-принца. Этот портрет кронпринц подарил барону лично. Гости восхищаются золотой гравировкой на мече, его эфесом в виде волчьей головы с рубиновыми глазами. Это первый прием гостей. Таких приемов будет еще немало в этом уединенном замке. Это начало легенды.

В числе гостей из Германии, которые останавливались в замке, была сестра Гансгенриха Эллен, фрейлина принцессы Термины, второй жены кайзера Вильгельма Второго. Когда Эллен отказалась от обязанностей фрейлины, ее родители сочли это глупым капризом и в наказание отправили ее в Африку. Эллен провела год в Дувизибе. Там она обучала готтентоток вязанию. Все это было так непохоже на ее прежнюю жизнь. Говорили, что после второй мировой войны она поселилась в деревне.

В замке Дувизиб вас окружает старина. Больше всего мне нравится старый буфет в столовой, украшенный резными кистями винограда, и старинные рюмки. В этой комнате висят портреты родителей барона.

Раньше здесь были серебряные кубки, преподнесенные барону, когда его восточнопрусские тракенские лошади взяли приз на выставке и на скачках. Были у барона и награды за ценного ирландского жеребца Крэ-керджека, сдохшего потом от старости на ферме, и за его австралийскую лошадь, которая появилась на свет в океане, на пути из Мельбурна в Кейптаун, и поэтому получила от барона кличку Нептун. Многочисленное потомство всех этих лошадей и до сих пор живет в Дувизибе, но оно постепенно дичает.

Барон всегда предпочитал лучшие породы. Ему доставляли верблюдов из Египта и Аравии, крупный рогатый скот из Херефорда, овец-мериносов из Австралии. В 1910 году у барона, одного из первых в Юго-Западной Африке, появились отары горных каракулевых овец, на которых впоследствии обогатилась эта страна. В районе Мальтахёэ говорят, что, если бы не начавшаяся в 1914 году война, барон сколотил бы себе огромное состояние. Конечно, и тратил он немало.

Барон играл в карты и устраивал попойки, которые длились по нескольку дней. Каждый месяц он отправлялся в Мальтахёэ в карете, запряженной шестеркой лошадей. Карету сопровождал фургон, нагруженный бутылками. Я видел остроумно сконструированный шкаф, в котором барон охлаждал напитки. Шкаф этот обшит металлом, внутри у него гнезда для бутылок всевозможных размеров и отделения для льда. В Мальтахёэ тех дней не было ничего, кроме административного управления, полицейского участка, почты, магазинов и гостиницы. Дорога от замка к Мальтахёэ была ужасна. Барон как-то заметил: <<Если я попаду в ад, он окажется не хуже этой дороги>>.

Свой приезд в Мальтахёэ барон всегда отмечал одинаково. Он входил в бар гостиницы, вытаскивал свой пистолет и разбивал пять бутылок на полках. Последний выстрел предназначался для лампы. Владелец гостиницы быстро подсчитывал убытки и представлял счет барону. Если счет был верен, барон охотно его оплачивал. Он был готов заплатить любую сумму, если она ни на пфенниг не превышала действительного убытка. Это была одна из его странностей. Человек добродушный, он, однако, выходил из себя, когда его обманывали или когда кто-нибудь выпивал его пиво.

В картах барону, видимо, не везло. Старый бухгалтер уверял меня, что однажды видел подписанный бароном чек на шестьдесят тысяч марок (три тысячи фунтов стерлингов)~--- проигрыш всего одной ночи. Владелец одного отеля в Виндхуке до сих пор слишком хорошо помнит, какую слабость питал барон к картам. Однажды ночью, в 1914 году, когда барон играл в отеле в карты со своими друзьями, нагрянула полиция. Владелец потерял право держать гостиницу. <<Я вынужден был снова стать официантом,~--- рассказывал он мне,~--- но я знаю, что если бы не война, барон возместил бы мне убытки. Он был прекрасным человеком, этот барон>>.

Поселенцы округа Мальтахёэ выбрали барона своим представителем в законодательную ассамблею в Виндхуке (ассамблея была лишь слегка демократичной). Фон Вольф завоевал там популярность. Причина, я думаю, заключалась в том, что он никогда не разыгрывал из себя барона. Он слишком откровенно высказывался по поводу немецких властей, и губернатор Зейц был им недоволен. Но барон не унимался.

Среди всех его безрассудств были и поиски алмазов во время бума 1908 года. Расстояние от Дувизиба до алмазного побережья больше ста миль. Барон отправился туда вместе с друзьями. Первые шестьдесят миль они проехали на верблюдах. Затем начались сыпучие подвижные дюны, где нельзя было найти прохода для верблюдов. Поэтому они отправили верблюдов обратно и добрались пешком до Меоба, расположенного на побережье. В Меобе оказалась только солоноватая вода, и, пополнив ею свои запасы, они отправились на юг в район Сильвиа Хилл. Там они застолбили участок, где нашли несколько небольших алмазов, а затем побрели в Людериц. Это был трудный переход по пескам почти в сто пятьдесят миль. Можете себе представить, какая жажда мучила барона. Он осушил бутылку шампанского и сел за карты. Игра продолжалась всю ночь.

В эти дни барон решил устроить морскую прогулку. Он зафрахтовал парусную яхту <<Рана>>. Мотора на ней не было. Барон хотел побывать на Ичабое, птичьем острове в тридцати милях к северу от Людерица. Вместе с ним в прогулке приняли участие шеф полиции и другие чиновники. Они захватили ящики с шампанским, пивом и ромом. На выпивку были приглашены все обитатели острова. Пока <<Рана>> стояла у острова, подул сильный юго-западный ветер, и это задержало яхту на несколько дней. Гости барона забеспокоились, так как в Людериц должен был прибыть почтовый пароход и им нужно было выполнять свои служебные обязанности. Самого барона это нисколько не беспокоило. И только когда было выпито все шампанское, он попросил шкипера доставить их на материк. Чиновники вернулись в Людериц с бутылками, пива в карманах.

В замке есть портрет барона фон Вольфа. Это высокий, чисто выбритый, темноволосый мужчина с волевым ртом. Портрета жены не сохранилось, но меня уверяли, что она была привлекательной женщиной. Джейта фон Вольф всегда оставалась в тени, и если многие жители Мальтахёэ и всей округи помнят подвиги самого барона, то почти ничего не могут сказать о его жене. Видимо, она плохо говорила по-немецки, но была способной женщиной. Однажды, когда барон и его рабочие попивали пиво, вместо того чтобы обжигать кирпичи, Джейта подошла к печи и принялась сама за эту работу. Когда был выстроен замок, у нее появилась горничная и швея. Кроме того, у них были шеф-повар, плотник, кузнец, жокей, конюх, мясник и бухгалтер, о котором я уже упоминал. На ферме работало много африканцев, в том числе пастух-готтентот, который умер при эпидемии тифа всего лишь за несколько недель до моего приезда.

Здания, окружающие замок, построены в средневековом стиле. Вы можете искупаться в огромном круглом бассейне двенадцати футов глубиной, выложенном серым дувизибским камнем. Под крышей дома управляющего расположилась кузница со старомодными мехами. Неподалеку печь для копчения мяса. Впервые в жизни я видел собачьи конуры из камня и свинарники с башенками. Массивные стены манежа слегка наклонены. Виноградники и тутовые деревья резко выделяются среди выжженного солнцем ландшафта.

Этот замок Дувизиб представляет собой замкнутый мир, и это вполне понятно. В среднем через каждые одиннадцать лет высохшие русла рек Юго-Западной Африки заполняются бешеными потоками воды и преграждают все пути к замку. По нескольку недель Дувизиб бывает отрезан от внешнего мира.

Наступает август 1914 года, и барон фон Вольф отправляется вместе с женой в Германию на лайнере <<Гертруда Вурман>>. Проницательный барон почувствовал, что над Европой сгущаются предгрозовые облака, а он вовсе не желает пережить еще одно поражение в Юго-Западной Африке. Он направляется в Германию, чтобы снова вернуться в армию.

<<Гертруда Вурман>> нашла прибежище в Рио-де-Жанейро, а это барона не устраивает. Его жена заказывает билет на голландский пароход, идущий из Рио-де-Жанейро в Роттердам. На борт парохода она входит с огромным чемоданом. Багаж приносят в ее каюту. Джейту провожает ее <<компаньонка>>~--- переодетый барон! Перед самым отплытием Джейта заявляет, что ее подруга сошла на берег. Тем временем барон прячется в чемодане. Во время плавания барон выходит из каюты жены только по ночам. Официанты, судача в буфетной, удивляются огромному количеству пищи, которую поглощает маленькая баронесса. Она постоянно заказывает в свою каюту сэндвичи и фрукты. Мало того, в день она выпивает по бутылке виски. И тем не менее никому не приходит в голову, что в ее каюте живет барон.

В Фалмуте корабль обыскивают офицеры британского флота. Они стучатся в каюту и застают баронессу раздетой. Дама из Америки полна негодования, и британские офицеры удаляются с извинениями. Корабль продолжает свой путь в Роттердам, и наконец торжествующий барон фон Вольф сходит на берег. Когда он приезжает в Германию, никто не вспоминает о случае с готтентотами, и майора барона фон Вольфа восстанавливают в артиллерии.

В сентябре 1916 года майор фон Вольф был убит на поле боя во Франции. Французский офицер обыскал труп и нашел письма любящей жены. Через Красный Крест письма и другие личные вещи барона были пересланы Джейте. Каким бы пьяницей, мотом и безрассудным картежником ни был барон, последний трагический эпизод из его жизни заслуживает, на мой взгляд, восхищения.

Когда в 1914 году барон уехал из Дувизиба, он оставил вместо себя своего друга герцога Макса фон Лют-тихау. Вскоре после войны было объявлено о банкротстве, и имение со всеми своими сокровищами продано за 7050 фунтов стерлингов.

Новыми владельцами оказалась преуспевающая шведская чета Мурманов. Когда сын Мурманов подрос, он научился водить самолет. У Мурманов был собственный самолет, который они держали в котловине неподалеку от замка. С грустью должен сказать, что Мурман неожиданно скончался, а его сын, летчик южноафриканских воздушных сил, погиб во время второй мировой войны. Замок и ферма были снова проданы. На этот раз их купила какая-то компания за двадцать пять тысяч фунтов. Когда Мурман жил в Дувизибе, он продал некоторые картины, и это почти возместило ему сумму, которую он заплатил при покупке замка.

В период между двумя войнами Джейта фон Вольф снова вышла замуж. Ее вторым мужем стал Эрих Шлеммер, сиамский генеральный консул в Мюнхене. Однако перед началом второй мировой войны Джейта вернулась к себе на родину, в городок Саммит в штате Нью-Джерси.

Во время войны 1914 года в замке Дувизиб пропала часть ценностей, в их числе персидский ковер стоимостью десять тысяч фунтов. После войны баронесса предъявила свои права на старинное серебро, но в замке нельзя было отыскать и ложки. К счастью, унести мебель грабители не смогли, а о ценности картин они не имели представления. К чести компании, которая теперь владеет замком Дувизиб, нужно сказать, что она сохраняет его почти в том же виде, каким он был во времена барона.

Из окна столовой замка видна возвышающаяся вдали горная вершина Вольфсберг. Так что барон оставил свое имя на карте.

Теперь я собираюсь провести ночь в этом замке из камня, где барон кутил с друзьями. В замке нет потайных ходов, нет привидений, но у него своя тайна.

Почему барон вернулся с женой в эти края, где он пережил позор? Он мог бы устроить себе более роскошную жизнь в таком приятном городе, как Дрезден. Ведь барон, несомненно, понимал толк в современном комфорте. Но вместо этого он надолго поселился на краю света, где нет ничего, кроме песчаных бурь да палящего летнего зноя. Такая жизнь не годится для женщины, но эта женщина вложила все свои средства в такое фантастическое предприятие и много лет прожила с мужем в добровольной ссылке. Почему они построили этот замок?

Спросите об этом фермеров Мальтахёэ, которые знали барона, и они вам ответят: <<О, у его жены было много денег>>. Но это вовсе не ответ. И если бы не старый друг барона фон Вольфа, который многое мне объяснил, я бы всю ночь мучился в поисках ответа.

Еще до замужества Джейта Хамфрис была последователем Зигмунда Фрейда. Когда фон Вольф вернулся в Германию после поражения, она поняла, что это его надломило. Погибла его военная карьера, и он был в отчаянии. Она отнеслась к нему с состраданием и проявила редкое чутье. В конце концов она придумала выход.

---~Мы должны вернуться на место твоего поражения,~--- сказала она ему.~--- Только там ты поймешь, как мало оно значит для всей твоей жизни. Мы вместе будем смело смотреть людям в глаза\ldots построим замок и будем жить на широкую ногу, так что они сочтут за честь бывать в нашем обществе. Замок в пустыне, Ганс-генрих фон Вольф\ldots

И вот серый замок Дувизиб, странный памятник вдохновения любящей женщины, до сих пор возвышается в этом глухом уголке Юго-Западной Африки.

\chapter{Легенда о дереве-людоеде}

Не пугайтесь. Никакого дерева-людоеда не существует, как нет и <<недостающего звена>> между растительным и животным миром. И все же в неумирающей легенде о зловещем дереве может заключаться крупица правды. Легенду рассказывают по-разному. Я пытался докопаться до причины возникновения этих выдумок, и у меня сложилось собственное мнение.

Некоторые исследователи считают, что источником возникновения легенды о дереве-людоеде были насекомоядные растения. Безусловно, эти примечательные растения вдохновляли сочинителей небылиц. Поэтому, прежде чем перейти к основному вопросу и к более убедительной теории о происхождении легендарного дерева, пожирающего людей, я немного остановлюсь на этих растениях.

В Африке много растений-хищников. Покрытые чувствительными волосками, листья крошечной болотной росянки втягивают в себя насекомых. Около Кейптауна, в горах Седарберг растут большие кусты Roridula, которые ловят и поглощают небольших животных вплоть до лягушки. Ясно, что такое растение могло бы питаться и мелкими млекопитающими вроде мышей. В реках у самых берегов растет пузырчатка, створки-ловушки которой всегда готовы захватить мелкую рыбешку, икру и насекомых.

Эти и другие растения, как, например, еще более эффективный непентес, ловят и поглощают свои жертвы. Возможно, что насекомоядные растения и породили некоторые легенды, например легенду о смертельном цветке, который своим необычным ароматом завлек путешественника в пещеру и одурманил его. Затем лепестки цветка обволокли его тело и выпустили сок, который все растворял. Вскоре от человека остались одни кости. <<Итак, умирая в чудесных сновидениях, путешественник отдает себя на съедение растению>>, -заканчивается это трагическое повествование.

Слышал я и о хищной лиане, которая где-то в тропическом лесу схватила собаку белого охотника. Обрубая лиану, чтобы освободить собаку, охотник, к своему ужасу, увидел, что своими живыми, гибкими щупальцами лиана стала обвивать его руку. Охотник вырвался, но на его теле остались красные пятна и волдыри.

Еще одна легенда рассказывает о дереве-змее, хватающем каждую птицу, которая садится на его липкие ветви. Земля под этим деревом усеяна костями и перьями. Один исследователь кормил якобы это дерево цыплятами, чтобы иметь возможность наблюдать процесс поглощения. Он заметил, что ветки дерева имеют присоски, как у осьминога. Ими-то дерево и высасывает кровь из своих жертв. Другое дерево~--- <<обезьянья ловушка>>- специализируется, как говорят, на обезьянах. Стоит только обезьяне взобраться на дерево, как ее обхватывают листья. От обезьяны остаются лишь одни кости, которые через несколько дней падают на землю.

Теперь мы перейдем к настоящему чудовищу~--- дереву-людоеду, слухи о котором впервые появились на Мадагаскаре, а несколько позднее в Португальской Восточной Африке. Разыскивая самое первое описание этого дерева, я некоторое время провел в лондонской библиотеке Британского музея. И кажется, нашел его. Оно принадлежит немецкому путешественнику доктору Карлу Лихе. Статья Лихе была опубликована в номере журнала <<Антананариво эннуэл энд Мадагаскар мэгэзин>> за 1881 год. Его издает Лондонское миссионерское общество. Печатается журнал в столице Мадагаскара Антана-нариве (Тананариве). Миссионеры не ручались за достоверность сообщения, но и не отрицали ее. Оно было преподнесено, как говорят газеты, <<за что купил, за то и продаю>>~--- просто интересный рассказ о стране, богатой необычными растениями и животными. Свой рассказ Лихе отослал также и в Карлсруэ ботанику доктору Омелиусу Фредловскому. Фредловский потребовал у Лихе более подробных сведений и опубликовал их со своими комментариями.

<<Я отправился на Мадагаскар, в эту страну лемуров и деревьев-людоедов, с визитом к королеве Ранавалоне Второй,~--- писал Лихе. -Мадагаскарский проводник Хендрик знал, что помимо высокой ежедневной платы я обычно щедро вознаграждал всех, кто показывал мне что-нибудь удивительное и необычное. Он уговорил меня побывать в юго-восточной части острова, где среди холмов, покрытых девственными лесами, живут мкодо. Мкодо~--- первобытный народ, у которого нет никакой другой религии, кроме поклонения священному дереву>>.

Лихе сообщает, что мкодо живут в пещерах, вымытых в известняковых породах. Это совсем небольшие люди, среди которых лишь немногие достигают пяти футов.

В конце долины проводник Хендрик показал Лихе глубокое озеро. От южного берега тропинка шла в густой лес. Их сопровождала толпа мкодо~--- мужчины, женщины и дети. Внезапно все закричали: <<Тепе! Тепе!>> Хендрик сразу же остановился. Здесь на поляне стояло толстое конусообразное, похожее на ананас дерево высотой восемь футов. Ствол его был темно-коричневым и казался твердым, как железо. От верхушки к земле свисало восемь листьев, словно дверцы, откинутые на петлях. Каждый лист на лицевой стороне был покрыт твердыми изогнутыми колючками и заканчивался острым шипом. Когда Лихе подошел к дереву, все листья своеобразного зеленого цвета были неподвижны и свисали вяло и безжизненно. Но чувствовалось, что они обладают огромной силой.

У основания этих чудовищных листьев выделяется светлый приторный сок. Хендрик сказал, что этот сок пьянит, и кто его выпьет, вскоре засыпает. Когда приносят жертвы, одну из женщин мкодо заставляют взобраться на дерево и пить этот сок. Если злой дух внутри дерева в хорошем настроении, женщина благополучно спускается на землю.

Тем временем мкодо начали петь, чтобы умилостивить священное дерево. Их крики становились все более громкими, и. наконец мужчины копьями выгнали вперед женщину. Она медленно поднялась по стволу и достигла вершины конуса. Дерево выпустило щупальца или усики, которые извивались над женщиной. <<Тсик! Тсик! (Пей! Пей!)>>,~--- кричали мужчины. Наклонившись, она выпила священную жидкость. Тогда все дерево ожило, щупальца, как змеи, обвились вокруг головы девушки.

Лихе продолжал: <<Теперь огромные листья медленно расправлялись. Как стрела подъемного крана, они поднялись в воздух и сомкнулись вокруг жертвы с тихой силой гидравлического пресса и безжалостностью тисков. Еще немного, и, пока я наблюдал, как все плотнее сжимались огромные листья, вниз по стволу потекла сладкая жидкость, смешиваясь с кровью жертвы. Толпа, стоявшая вокруг меня, отчаянно взвыла и окружила дерево со всех сторон. Кто из чаш, кто из листьев, кто из ладоней, а кто и прямо ртом пили эту жидкость, пьянели и приходили в неистовство. Затем началась нелепая, отвратительная оргия, которая перешла в бесчувственное исступление. Наконец Хендрик поспешно увел меня прочь, спрятав в укромном месте в лесу от опасных дикарей. Не дай бог увидеть такое еще раз!>>

Лихе добавляет, что листья дерева оставались поднятыми десять дней. Затем он увидел, что они опустились, а у основания дерева нашел череп.

Вероятно, до старого Лихе дошли слухи о дереве-людоеде, и он по каким-то соображениям написал свою страшную сказку. Может быть, это была просто мистификация. Но все же я думаю, что ему хотелось завоевать репутацию исследователя. Несомненно, он не видел того, о чем писал, и переусердствовал, рассказывая о жертвоприношении. Возможно, ученые и поверили бы в рассказ о странном дереве, вокруг которого разбросаны кости. Но жертвоприношение выдало Лихе.

К сказке Лихе было сделано много добавлений, и по всему свету ее рассказывали по-разному. Передо мной вырезка из одной лондонской газеты за февраль 1924 года. В ней говорится о тяжелом испытании, выпавшем на долю двух ботаников~--- Жозефа Вилларо и Жоржа Гастрона, собиравших растения среди болот в сорока милях от Нью-Орлеана. Проплутав целую неделю по болоту, они добрались до небольшого островка, где у самой воды росло таинственное дерево, вид которого они определить не смогли. Дерево было похоже на серую пальму. Неподалеку Вилларо заметил желтые душистые цветы и уже хотел сорвать их, как вдруг несколько огромных листьев этого дерева схватили его и стали прижимать к стволу. И тут же лианообразные отростки скрутили его так крепко, что он не мог пошевелиться.

Схватив топор, Гастрон бросился на помощь и освободил Вилларо. Тут они заметили, что дерево уже скрутило и умертвило несколько мелких животных~--- белок и кроликов. Гастрон потом рассказывал, что, когда его топор врезался в дерево, оно корчилось как от боли, а из разрезов сочился красный, похожий на кровь сок.

Лондонская газета <<Дэйли Кроникл>>, опубликовавшая этот фантастический рассказ с разрешения информационного агентства (спешу добавить, не <<Рейтер>>), снабдила его следующим торжественным примечанием: <<Хищные растения отнюдь не редки. Такие растения есть повсюду в тропиках, обычно в болотах и топях>>.

Это сообщение заставило вспомнить о журналисте Артуре Р. Амори. Как-то в дождливый полдень этот журналист написал рассказ о дереве-людоеде, сюжет которого он считал оригинальным (о Карле Лихе он, очевидно, совсем не знал). Было это в Индии. Амори и его друзья рассуждали о непентесе. Они видели, как это растение ловит насекомых. Амори стал фантазировать. А что если бы этот непентес вырос до гигантских размеров?

И он написал рассказ об экспедиции, которая собирала орхидеи. Один из ее участников потерял собаку и отправился ее разыскивать. После долгих поисков он наконец увидел своего терьера в плену у огромного растения, которое схватило собаку клейкими усиками. Он вынул нож и стал один за другим обрубать эти отростки, но растение тут же выбрасывало новые. Они обвились и вокруг него, и в конце концов этот охотник за орхидеями был безжалостно задушен. Через неделю остальные члены экспедиции нашли его скелет и рядом е ним скелет терьера.

Это была чистая выдумка. Но когда этот красочный рассказ был опубликован в одной из бомбейских газет, по всей Индии его восприняли как действительный случай. Некоторые газеты добавляли от себя всякие подробности, чтобы вся история выглядела более правдоподобно. Вскоре этот рассказ попал в Австралию, в дальневосточные страны Азии, в Соединенные Штаты и Канаду. Пересекая океаны и границы, он становился все более длинным и обстоятельным. Наконец рассказ дошел до Англии. Амори был ошеломлен, когда прочитал придуманную им же историю, где назывались имена всех участников экспедиции и сообщались подробности их биографий.

<<Не говорите мне о чудовище, которое создал Франкенштейн,~--- заявил Амори.~--- Франкенштейн был просто дилетантом>>.

В 1924 году американский путешественник и член Мальгашской академии наук Сэлмон Чейс Осборн отправился в леса Мадагаскара, чтобы проверить, есть ли там на самом деле дерево-людоед. Вот что он писал об этом:

<<Не знаю, существует ли в действительности это кровожадное дерево, или же все страшные рассказы о нем~--- чистейшая выдумка. Но почему бы ему и не быть? У всех племен, с которыми я встречался,~--- хова, сакалава, сиханака, бетсилео~--- есть легенды и предания об этом дереве. Я исходил весь остров вдоль и поперек. Миссионеры говорят, что такого дерева нет. Однако кое-кто из них придерживается иного мнения. Некоторые миссионеры говорили мне, что вряд ли все племена так упорно верили бы в существование такого дерева, если бы для этого не было никаких оснований>>.

Бывший офицер индийской армии капитан Л.~Р.~Херст, много путешествовавший по Мадагаскару, в 1932 году сообщал в лондонской газете, что он собирает экспедицию на западное побережье Мадагаскара для поисков дерева-людоеда. <<Смею заявить,~--- писал Херст,~--- что это дерево действительно поедает людей. Местные жители держат его в большом секрете и не очень-то стремятся показать, где оно растет. Как мне сказали вожди, дереву этому приносят жертвы, и я надеюсь создать кинофильм о таком ритуале. Но мые не хотелось бы много об этом говорить, так как меня могут принять за второго де Ружемона>>.

Заявление капитана Херста вызвало у меня интерес, и я с нетерпением ожидал результатов его экспедиции. К сожалению, больше о ней я ничего не слышал.

Существует еще одно объяснение легенд о дереве-людоеде. Я упоминаю здесь о нем исключительно из-за его романтичности. Некоторые считают, что дерево-людоед было выдумано старыми пиратами Индийского океана Киддом и другими, когда они объявили Мадагаскар республикой. Пираты не хотели, чтобы было раскрыто их убежище и найдены спрятанные там сокровища. Возможно, дерево и предназначалось для того, чтобы удерживать нежелательных пришельцев. Слишком уж остроумное предположение, на мой взгляд.

Теперь я несколько отойду в сторону, чтобы по-другому взглянуть на эту легенду, так давно заинтересовавшую меня. Триста лет назад губернатором Мадагаскара был Этьен де Флакур, который написал книгу о гигантской птице, несшей огромные яйца. Она была похожа на сказочную птицу рух из <<Тысячи и одной ночи>>, которая забросала камнями корабль Синдбада и потопила его. О такой птице Марко Поло говорил, что она способна поднять слона. Де Флакур не заявлял, что он видел птицу. Это просто местная легенда, подобная легенде о дереве-людоеде. Даже в те времена, времена простодушного семнадцатого века, многие не поверили словам де Флакура.

Но в начале прошлого века исследователь Мадагаскара Сганзин нашел куски огромного яйца, сделал зарисовки и отправил их французскому натуралисту Жю-лю Верро, который был тогда в Кейптауне. Теперь, конечно, мы знаем, что на Мадагаскаре существовала гигантская бескрылая птица~--- эпиорнис ростом двенадцать футов. Птица эта дожила до двенадцатого века. Она несла самые крупные на земле яйца, яйца трех футов в окружности. Конечно, эпиорнис не пожирал слонов, но это была настоящая, а не мифическая птица.

Есть ли на Мадагаскаре дерево, обладающее некоторыми смертоносными свойствами, которые могли бы послужить причиной появления легенды о дереве-людоеде? Я думаю, что такое дерево вполне может быть. Вспомните необычную, но правдивую историю об анчаре на острове Ява. Сок этого дерева смертелен. Оно было описано монахом Одерихом еще в четырнадцатом веке. Постепенно приукрашиваясь, история эта стала утверждать, что анчар умертвляет каждого, кто уснет даже в нескольких милях от него. Легенда оказалась такой живучей, что в 1837 году член Королевского ученого общества подполковник У. X. Сайкс отправился на поиски этого дерева. Проводники привели его в долину, заваленную скелетами, среди них были и человеческие черепа. В долине рос анчар.

В поисках иного источника смерти Сайкс осмотрел все вокруг и вскоре нашел причину. Ява~--- страна вулканов, которые выделяют углекислый газ. Оставив на ночь в этой ядовитой долине собак и кур, Сайке наутро нашел их мертвыми. В безветренную погоду газ обычно скапливается в долинах. Но островитяне считали причиной смерти ядовитый анчар.

Я думаю, что и на Мадагаскаре может расти ядовитое дерево, особенно в некоторых нездоровых районах, где люди часто умирают преждевременно. Разбросанные вокруг таких деревьев кости человека и животных вполне могли породить легенду, которая, несмотря на все насмешки, все еще бродит по земному шару.

\chapter{Никто не знает Сахары}

<<Люди думают, что знают Сахару,~--- прошептал, умирая от жажды, французский генерал Лапперэн.~--- Никто ее не знает. Я пересек ее десять раз, а на одиннадцатый она меня одолела>>.

Никто не знает Сахары. Вот почему о ней распространено столько легенд. Это настоящая пустыня, самая большая пустыня в мире. Больше трех миллионов квадратных миль сухого песчаного океана. В два часа пополудни здесь можно испечь яйцо в песке, а в два часа ночи заморозить его. Здесь находятся самые жаркие в мире районы, где человек без воды умирает через девятнадцать часов.

Я знал людей, умерших от жажды в этой пустыне. Это очень грустная история. При мысли о ней мое сердце и сейчас сжимается от боли. Но я укажу лишь на один факт, который не перестает удивлять меня с того самого дня, когда я узнал подробности этой трагедии. Пропало двенадцать южноафриканцев. Когда их нашли, одиннадцать уже были мертвы. Они умерли очень быстро. Спасся только один. Целую неделю он бродил по пустыне без воды. <<Оставшийся в живых серьезно болен>>,~--- запомнил я фразу. Было начало лета. Мне казалось чудом, что один все же остался в живых.

Летом Сахара~--- самое ужасное место в мире. Жара обжигает глаза. Если вы летите на самолете, безбрежная голая пустыня пугает вас больше, чем любой океан". Но с высоты пустыня лишь мимолетное виденье другого мира. В самолете вы находитесь во власти белых облаков и голубого неба. И серый песок внизу~--- просто мрачная картина, мелькнувшая за окном. Когда видишь там людей, они кажутся существами с другой планеты.

Даже если самолет летит низко, все выглядит нереальным. Деревушки вдоль Нила похожи на форты, окруженные земляными валами. Вы замечаете работающих судостроителей, старинные лодки~--- и все тут же исчезает. На мгновенье появляются плантации сахарного тростника и красные всплески цветущего мака. Они так же нереальны, как и древние города, гробницы и храмы, некрополь или памятник. И все время внизу вьется лента, зеленая лента, окаймляющая берега великой реки. Но даже и Нил с высоты десяти тысяч футов кажется ничтожным ручейком.

Реальность начинается только на земле. Жара дрожит и ослепляет. Она причиняет такую боль, что не может быть нереальной. Как люди живут в этом пекле? Я достаточно пробыл в пустыне, чтобы узнать это.

Да, Сахара~--- страшная пустыня. Но вот уже две тысячи лет ее пересекают караваны верблюдов. Здесь пролегают древнейшие торговые караванные пути. Нагруженные солью, мешками соли, шли верблюды по этой земле невыносимой жажды! С этим товаром, заменявшим деньги, шли они через дюны от оазиса к оазису. Соль, слоновая кость, рабы. Вот те богатства, которые заставляли людей рисковать жизнью в пустыне. Когда я думаю о скелетах многих тысяч рабов и несчастных евнухов, которых заставляли проделывать это путешествие, я не жалею, что так много работорговцев погибло от жажды.

Должно быть, тысячи торговцев и многие тысячи воинов оставили свои кости в этой пустыне. За пять веков до нашей эры здесь погибла целая армия, армия, посланная царем Камбизом в оазис Сива. В начале прошлого века из Тимбукту к Средиземноморскому побережью вышел караван, в котором было две тысячи человек и примерно столько же верблюдов. И ни один человек, ни один верблюд не дошел до места.

В древности караваны вели слепые проводники. Тропы в пустыне пропитаны запахом верблюдов. Песчаные бури могли засыпать следы, но запах оставался~--- слабый, но все же достаточный для чуткого носа слепого проводника. Через каждую милю он брал горсть песку и нюхал его. На земле, где нет никаких ориентиров, слепец был полезнее зрячего.

За несколько веков до того, как в Сахаре появился радиотелеграф, в крупных торговых центрах~--- Тимбукту и Кано, Каире и Хартуме~--- были люди, предсказывавшие прибытие караванов. Даже теперь в отдаленных оазисах можно найти старца, который может назвать день и час, когда под пальмами появятся новые люди. Интересно, достаточно ли часто оправдываются эти предсказания, чтоб не считать их простым совпадением?

Некогда исследователями и хозяевами Сахары были туареги. Эти мусульмане были когда-то христианами, и до сих пор их седла украшает крест. Они рыли колодцы и брали за воду плату. Но с одиннадцатого века их стали вытеснять арабы. Арабские караваны были огромны. Караван в пятнадцать тысяч верблюдов мог доставить в Тимбукту полторы тысячи тонн риса, проса и горьких орехов кола. Назад он возвращался с золотом и солью. Ежегодно из Каира специальный караван в двенадцать тысяч верблюдов направлялся к горнорудному центру Таккеда за слитками меди.

В четырнадцатом веке Манса Муса, король малий-ского народа мандигон, возглавил величественную кавалькаду, которая пересекла всю Сахару, направляясь из Западной Африки в Каир и потом в Мекку. Властитель ехал на лошади, а пятьсот его рабов несли слитки золота, ценой в миллионы фунтов. Манса Муса благополучно совершил путешествие туда и обратно, но многие погибли в пути. Вот как появляются бесчисленные легенды о сокровищах Сахары.

В старинных <<книгах о сокровищах>> (египетские астрологи с радостью вам их продадут) говорится о богатствах царя Камбиза. Я уже упоминал о его армии, которая была послана в Сиву, чтобы уничтожить храм Юпитера Аммона, и трагически погибла от жажды. Несмотря на это, Камбиз все же завоевал Египет. У него были медные рудники и изумрудные копи, а золотые самородки с его сахарских рудников достигали, говорят, размеров дыни. Однако, несмотря на помощь астрологов, эти богатства так до сих пор и не разыскали.

В знаменитом арабском манускрипте~--- <<Книге о спрятанном жемчуге>> неизвестный автор пятнадцатого века дает подробное описание четырехсот мест в пустыне, где якобы находятся сокровища. И больше полувека археологи проклинали автора этой работы. Ее французский перевод появился в Каире в 1907 году. Почти все места, указанные в манускрипте, были вблизи различных древних памятников. И в поисках сокровищ вандалы нанесли им непоправимый ущерб.

Очень часто искатели сокровищ попадали в беду. В 1922 году три человека~--- Хамер, Русек и Фоклер- отправились с какой-то тайной целью в Ливийскую пустыню и были захвачены племенем сенусси. Женщины племени подвергли Фоклера и Хамера пытке, а затем умертвили их. Полумертвый Русек, заклейменный каленым железом, бежал в оазис Дендера.

Еще одну злополучную экспедицию организовал немец Эрих Баумгартнер, воевавший в армии Роммеля. После второй мировой войны он вернулся в Египет. Проработав несколько лет в пароходной компании, он скопил денег, купил детектор, динамит и автомашины. В 1952 году он отправился по пути, который в 1874 году проделал его соотечественник Рольфс, так как верил, что тот нашел копи царя Камбиза.

По словам его рабочих, Баумгартнер кое-что обнаружил. Но рабочие не стали помогать ему в раскопках. Они считали, что он наткнулся на старинный храм, охранявшийся джинном или злым духом. Тогда Баумгартнер заложил динамит. Но заряд оказался слишком большим. Огромная дюна провалилась и погребла его под собой.

Сомнительно, чтобы в Сахаре можно еще было найти хоть один <<потерянный оазис>>, который представлял бы какой-то интерес, хотя последний из них был обнаружен и нанесен на карту лишь в период между двумя мировыми войнами. Легенды об оазисах возникают по-разному. Внимательные наблюдатели прослеживали пути полета пальмовых голубей и ворон и заключали, что они прилетают от какого-то неизвестного источника воды.

Так, один человек проследил, откуда в его оазис прилетали голуби, и засек по компасу направление. Потом подстрелил несколько голубей и вскрыл их. У всех в желудках оказались маслины. Тогда этот изобретательный человек поймал несколько местных голубей и стал их кормить маслинами. Каждый час он убивал одного голубя и исследовал его желудок. Наконец он увидел, что маслины в желудке только что убитого голубя переварились до такой же степени, как и маслины в желудках подстреленных им чужих голубей. Скорость полета голубя равна двадцати пяти милям в час, а он определил, что прилетевшие .к нему из пустыни голуби съели маслины за девять часов до того, как он их подстрелил. Тогда он отправился из оазиса в направлении, засеченном по компасу, и проехал на верблюде около двухсот двадцати пяти миль. Он был вознагражден, оказавшись в неизвестном оазисе с оливковыми деревьями.

Оазис Куфра, где белый человек впервые побывал лишь в 1921 году, был найден одним наблюдательным бедуином из оазиса Обейяд, расположенного гораздо севернее Куфры. Бедуин наблюдал за вороной, которая регулярно улетала на юг, но через определенное время вновь возвращалась в Обейяд. Руководствуясь только этим, бедуин отважно пустился в путь и в конце концов добрался до финиковых пальм и воды. Так был найден оазис Куфра. Открытие это было чрезвычайно ценно. Оно обозначало пищу и воду в самом сердце восточной Сахары. Оазис мог стать не только прибежищем для караванов, но и райским садом, где можно создать крупное поселение. Новый оазис действительно оказался сокровищем.

До открытия этого оазиса в районе Куфры погибло много караванов. В мире дюн вы передвигаетесь среди высоких песчаных гряд. Если оазис расположен на противоположном склоне дюны, его нельзя обнаружить, даже если он лежит всего в нескольких сотнях ярдов. Я знал людей, умерших от жажды в шестидесяти милях от оазиса. И совсем неподалеку был колодец. Всего в миле от них прошла партия изыскателей, но дюны скрыли от них умирающих людей.

Легенда о Зерзуре (<<оазис птичек>>)~--- самое известное из всех преданий о Ливийских оазисах. На поиски этого места отправлялось много экспедиций. Члены привилегированного клуба <<Зерзура>> в Лондоне (куда принимали только тех, кто участвовал в поисках) за обедом беседовали об этом неуловимом оазисе. Из года в год журнал Королевского географического общества включал оазис Зерзура в список географических названий. И наконец эта тайна была раскрыта.

Романтическое название <<Зерзура>> впервые появилось в арабской рукописи семьсот лет назад. В <<Книге о спрятанном жемчуге>>, которая причинила столько бед, тоже описывается Зерзура в самых заманчивых выражениях. Приведу один отрывок:

<<От этого последнего вади начинается дорога, которая приведет вас к городу Зерзура, к его закрытым вратам. Город этот белый, как голубь, а на его вратах вырезана птица. Возьмите ключ из клюва птицы и откройте врата города. Войдите, и там найдете огромные богатства, а также царя и царицу, спящих в своем замке. Не подходите к ним, а возьмите сокровища>>.

Зерзура, о которой в начале прошлого века писал сэр Гардинер Уилкинсон, менее сказочна. Он узнал о существовании оазиса Черных, названном так потому, что оазис этот, расположенный к западу от Нила, был захвачен чернокожими людьми, пришедшими неизвестно откуда. Они похитили несколько человек и увели их в пустыню. Уилкинсон, автор, заслуживающий доверия, предположил, что оазис Черных мог быть Зерзурой.

В те времена, когда из Французской Экваториальной Африки шли торговые караваны в Египет через Куфру, упорно ходили слухи об арабах, которые заблудились в пустыне и неожиданно наткнулись на чудесный оазис. Там среди пальм возвышался золотой минарет и в солнечных лучах сверкало озеро. Но видимо, пустыня обманула их своими миражами. Обратно эти арабы так и не вернулись.

В начале нашего века исследователь Гардинг Кинг отправился на запад от оазиса Дахла, где он слышал много рассказов о Зерзуре. Говорили, что в Дахлу вновь прибыли из песков таинственные черные люди. Кинг встретил двух бедуинов, которые сказали ему, что видели небольшой оазис с пальмами и развалинами в том месте, где на карте было белое пятно.

Много лет люди думали, что сыпучие пески, надвигающиеся на Дахлу, поглотили Зерзуру. Дюны могут вновь переместиться и открыть потерянный оазис, но кто знает, когда это случится? Летом 1932 года молодой исследователь сэр Роберт Клейтон-Ист-Клейтон с автомобилями и небольшим самолетом отправился на поиски Зерзуры. С самолета летчик увидел и сфотографировал широкую долину с акациями. Из-за страшной жары и отсутствия воды приземляться было слишком рискованно, и дальнейшее исследование было отложено до зимы. Сэр Роберт Клейтон-Ист-Клейтон чем-то заразился в пустыне и умер. На этом дело и кончилось.

И все же через некоторое время два участника авиационной экспедиции добрались до этой долины, не отмеченной на карте. Один из них, П.~А.~Клейтон, обнаружил к востоку от нее еще одну долину, а другой~--- граф Алмаси\footnote{Ладислас Алмаси во время второй мировой войны служил в войсках Роммеля и совершил несколько вылазок за линию фронта в тыл англичан. Однажды он освободил около Асиута двух немецких шпионов. Алмаси благополучно вернулся, но шпионы были схвачены в Каире. Алмаси умер в 1951 году.~--- Прим. авт.}, член Египетского королевского географического общества,~--- нашел с западной стороны третью долину. Во время своего путешествия Алмаси встретил старого араба, многие годы жившего в Куфре и, как никто, знавшего пустыню. Араб уверял графа, что эти долины, которые видели с воздуха, а потом исследовали, известны населению Куфры как вади Зерзура. <<В долине водятся горные бараны, лисы, но особенно много птиц. Поэтому-то долина и была названа вади Зерзура>>,~--- сказал старик.

Один из наиболее опытных и смелых исследователей Ливии, президент клуба <<Зерзура>> майор Р.~А.~Бэгнолд, решил тогда, что эти вади (сейчас на картах они названы Гилф Кебир) и являются легендарным оазисом Зер-зура. Но он отмечал, что происхождение названия <<Зер-зура>> все еще неясно, потому что оно появилось в арабских рукописях за много веков до того, как арабы открыли это место. <<Я по-прежнему думаю, что Зерзу-ра~--- одно из многих названий, которое давали многим легендарным городам, созданным на протяжении веков загадочной великой Североафриканской пустыней в воображении тех, для кого эта пустыня была недоступна>>,~--- заключает Бэгнолд.

Все, кто летал над африканскими пустынями на небольшой высоте, должны были увидеть там много следов от колес. Кажется, что некоторые из них уходят в бесконечность, и это больше всего волнует. Двадцать лет назад я сам оставил следы в пустыне. И хотя найти их невозможно, хочется думать, что они все еще сохранились. В этом безводном районе следы могут оставаться больше ста лет. Картер Вильсон, занимавший высокий пост в Египте в начале нашего века, обнаружил следы от обозов наполеоновской армии, оставленные в 1798 году во время перехода из Сальхуджи в Кантару. В 1909 году Рассел-паша нашел следы от колес орудий, применявшихся в бою при Телль-аль-Кебире в 1882 году. Во время первой мировой войны в битве с сенусситами в Западной пустыне впервые применялись автомобили, и в отдаленных районах до сих пор находят следы их узких шин. Специалист может указать на следы тягачей принца Кемаль эд-Дина и на следы шестиколесных автомашин принца Омара Туссуна~--- этих исследователей пустыни двадцатых годов нашего столетия. Следопыт-бедуин может указать на следы почти каждой остановки, каждого сражения с песками, каждого бивуака. Найдя свежий след верблюда, он, безусловно, покажет, где спали люди, где они совершали утреннюю молитву, где их верблюды шли тихим шагом, а где пускались вскачь.

Старые автомобильные следы я видел не только в Сахаре, но и в Юго-Западной Африке~--- в прибрежной пустыне Намиб и в красной пустыне Калахари. Но я считаю, что пустыни Северной Африки намного дольше сохраняют следы вторжения человека.

Путешественники оставляют и другие знаки своего продвижения через безжизненные просторы пустынь. Вдоль многих путей в Сахаре летчики видят поблескивающие на солнце бутылки. А автомобильная дорога южнее Танжера (от Реггана к заброшенному Бидон-Сэнку и дальше) так усеяна бутылками, выброшенными из автобусов, что там не нужны никакие дорожные знаки. Но самой интересной бутылкой в Сахаре была, на мой взгляд, бутылка, оставленная Рольфсом в груде камней где-то южнее Сивы. В 1922 году эта бутылка была найдена принцем Кемаль эд-Дином, который прочитал находившееся в ней послание: <<Ступит ли здесь еще когда-нибудь нога человека?>>

Геологи до сих пор не могут объяснить происхождение песчаного океана Сахары. Одно время считали, что раньше на этом месте было море. Но потом гипотезу эту отвергли. Когда-то в Сахаре был рай. В исчезнувших теперь городах была вода, и в них кипела жизнь. Даже слонам хватало воды. Пять тысяч лет назад Сахара стала высыхать, но там все еще сохранялись открытые травянистые пространства, кустарники, сохранялась жизнь.

Реки превратились в лужи. Животные собирались вокруг оазисов. Теперь они не могли их покидать. Антилопа и газель, шакал и лиса, страус и перепел, утка и фламинго. Когда-то они бродили, где хотели. А потом только птицы могли летать над безводной пустыней.

Экспедиции находят многочисленные доказательства существования в прошлом воды в Сахаре. Вдали от оазисов, в местах, где теперь нет ни единого человека, ученые нашли и скопировали наскальные красные и белые рисунки, изображающие скот. Были вскрыты каменные склепы и исследованы скелеты, на которых нашли ожерелья из бирюзы и из скорлупы страусовых яиц. Исследователи откопали шлифованные каменные топоры, ступы, жернова. Ясно, что когда-то здесь находились большие поселения. По жерновам можно судить, что там, где сейчас господствуют одни лишь пески, раньше выращивали зерно. Что же за люди охотились в сахарском раю? Кто эти неизвестные авторы наскальных рисунков? Это тоже тайна. Однако существует предположение, что в давние времена Сахару населяла негроидная раса, еще задолго до прихода туда берберов и арабов.

Так что и в Сахаре есть свои контрасты. Когда-то здесь обитали крокодилы. В источниках оазисов и до сих пор еще водится рыба, а в горах~--- берберийские овцы. Все это реликты совершенно различных климатических эпох. И почти везде песок, непостижимый песок, поглотивший древний рай. На путника в Ливийской пустыне пески производят неотразимое впечатление, потому что это самые огромные песчаные пространства в мире, и песок здесь уходит на очень большую глубину. Чтобы добраться до гробницы Тутанхамона, Говарду Картеру пришлось вынуть четверть миллиона тонн песку, но это лишь капля в песчаном океане Ливийской пустыни.

Песок может быть страшен. Параллельными рядами, гряда за грядой уходят вдаль огромные желтые дюны. Эти гряды, отстоящие друг от друга на милю или две, достигают двухсот футов в высоту и тянутся миль на тридцать. Дюны в Калахари, по которым я проезжал на машине, просто карлики в сравнении с этими великанами Ливийской пустыни.

Только сенусситы чувствуют себя здесь как дома. Этот бедуинский религиозный орден был основан более ста лет назад одним из потомков пророка Мухаммеда. Сенусситы издавна занимались работорговлей. В Сахаре в первую мировую войну они сражались против английских и южноафриканских войск, а во второй мировой войне поддерживали экспедиционный корпус дальнего действия.

Эти суровые люди не знают, что такое алкоголь, табак или кофе, но любят чай. Я помню сенусситов главным образом потому, что частенько обменивал у них свой армейский чай на свежие яйца. Но у меня есть и другое воспоминание о сенусситах. Оно связано с одним случаем, который до сих пор остается для меня загадкой.

Рядом с нашим лагерем под Тобруком находился жалкий клочок земли с посевами сенусситов. Трудно ожидать, чтобы в пустыне хорошо росли зерновые. Но все же сенусситы как-то ухитрялись выращивать на этом крохотном участке не то пшеницу, не то ячмень. В один прекрасный день, когда урожай еще не созрел, сенусситы появились на поле с длинными ножами и срезали растения. Потом, как арабы, свернули свои палатки и молча уехали. Они добрались до Аламейна, прежде чем я узнал, что что-то случилось. Кажется, сенусситы могут чувствовать перелом в битве не хуже, если не лучше, чем генералы воюющих сторон.

И это вполне соответствует их репутации. Все другие бедуины приписывают им колдовскую силу. Человек, которого проклял сенуссит, живет в постоянном страхе, что его разобьет паралич или настигнет смерть. Говорят, что свое колдовство сенусситы распространяют и на животных, так что отара овец, забредшая на ячменное поле какого-нибудь старика, обладающего колдовской силой, может и не вернуться. Конечно, овец можно убить и множеством других способов помимо проклятья. Сенусситские гипнотизеры отличаются одной удивительной способностью. Они могут заставить человека увидеть события, происшедшие в его родном городе или оазисе, от которого в данный момент его отделяют сотни миль.

Сеид аль-Махди~--- вероятно, самый знаменитый сенусситский пророк нашего века~--- был добросердечным чародеем. Он утверждал, что может почувствовать беду на огромном расстоянии. Рассказывают множество историй, как по указаниям Сеида отправляли спасательные отряды и те находили гибнущие караваны.

Есть ли у сенусситов шестое чувство или нет, но их проводники в пустыне обладают необыкновенной способностью ориентироваться на местности. Изменение направления не сбивает их с толку, поскольку у каждого проводника в голове есть свой компас. И даже в беззвездную ночь, находясь в незнакомой местности, сенуссит уверенно идет в нужном направлении. Никто не знает Сахары, но проводник-сенуссит почти никогда в ней не заблудится. Возможно, этим он обязан инстинкту своего верблюда. Говорят, что верблюд всегда найдет обратный путь в оазис, где он когда-то пасся. Много раз караваны бывали спасены именно потому, что верблюд вдруг брал на себя роль проводника, словно по наитию.

Никто не знает Сахары, и повсюду вокруг оазисов в песке погребены тайны. Вот оазис Сива. Его дома, построенные из земли и соли, лепятся друг над другом на огромной скале. Этот человеческий муравейник возвышается над святилищем оракула, к которому приходил за советами Александр Македонский. Неподалеку от Сивы есть другие древние города, все еще ожидающие заступа археолога. Где-то около этого оазиса находятся и затерянные изумрудные копи древности. И вот уже больше тридцати веков пальмовые рощи Сивы дают самые лучшие в мире финики.

Интересно, доведется ли мне когда-нибудь вновь пересечь магическую границу у Мены, этот узкий канал между пустыней и посевами? Как радостно мне было выйти сюда из слепящей желтой пустыни и, миновав пирамиды, очутиться среди зелени этих изумительных окрестностей. И как был бы я рад сейчас снова отправиться из Каира в эту загадочную пустыню, в пустыню, которой никто не знает.

\chapter{Затерянный город пустыни Калахари}

Где-то в песках Калахари есть <<затерянный город>> Зимбабве. Я принимал участие в экспедиции, которая тщетно пыталась найти его. Однако город существует. Это совсем не легенда. И хотя он исчез больше семидесяти лет тому назад, все же эти древние развалины будут когда-нибудь найдены.

Передо мной лежит пыльный блокнот, где я делал записи во время нашей экспедиции. 8 июля 1936 года автомобиль с прицепом покинул источник Гейнаб и направился по высохшему руслу реки Носсоб. Экспедиция должна была проникнуть в неизведанную пустыню и разыскать затерянные развалины. В восемнадцати милях к югу от Гейнаба в русло Носсоба впадает высохший приток, хотя он и не обозначен на карте. В Апингтоне нам сказали, что он называется Гротбрак~--- название, ничем не отличающееся от многих других. Я почти уверен, что с 8 июля 1936 года никто больше не бывал в этих местах.

Мы взяли запас горючего на двести сорок миль пути. Заполнили водой канистры, а также бутыли~--- на случай, если придется возвращаться пешком,~--- и запаслись пищей на неделю. Часть людей осталась в лагере. Они подписали обязательство (в моем блокноте), что в случае, если наша группа не возвратится через шесть дней, они отправятся на поиски.

В своем дневнике я отмечал весь наш путь и все ориентиры: вытянутую впадину с выходами известняков, цепь песчаных холмов, излучину высохшей реки, не помеченной на карте, проходы в дюнах, участки высокой пожелтевшей травы. Мы ночевали в неприютных местах, где еще не ступала нога человека. Иногда мы принимали следы антилоп за дорогу, но все они вели к ямам, вырытым животными в поисках соли.

Один случай ясно запечатлелся в моей памяти. Чтобы припомнить его, мне нет необходимости рыться в своем дневнике. В самом начале путешествия мы наткнулись на колодец, вырытый на дне впадины среди известняков. Колодец казался высохшим, но, появись здесь бушмен, он, несомненно, сумел бы добыть из него влагу. Через несколько дней мы возвращались тем же путем обратно в Носсоб и снова набрели на этот колодец. Поверх следов от наших ботинок мы различили отпечатки маленькой ступни. Не было смысла искать вокруг. Если бушмен не хочет, чтобы его увидели, то, уж будьте уверены, вы ничего не найдете, кроме следов его ног.

Мы решили, что поиски <<затерянного города>> нужно производить с самолета. Я не забыл предостережения одного археолога, который побывал в Носсобе до нас. <<Когда вы увидите эту пустыню,~--- писал он,~--- вы поймете, что можно месяцами бродить среди песчаных дюн и даже близко не подойти к тем местам, где расположен ``затерянный город''>>. Он был прав.

Человек, который открыл <<затерянный город>>, был колоритной фигурой, как, впрочем, всякий, кто в те дни путешествовал по Калахари. Это американец Г.~А.~Фарини, владелец ранчо. В Калахари он приехал за алмазами. Как ни странно, в Америке Фарини повстречал охотника из Калахари, по имени Герт Лоу. Герт, в котором текла и бушменская кровь, был привезен в Нью-Йорк балаганщиком, показывавшим этого уродца на выставке в Кони-Айленд. Когда Фарини встретил Герта, тот сильно тосковал по родине. Охотник рассказал Фарини легенду об алмазах Калахари. Возможно, что он ее и выдумал, но для Фарини этого оказалось достаточно, чтобы снарядить туда экспедицию.

В Фарини самом было что-то от балаганщика. Когда на пути в Кейптаун они остановились в Лондоне, Фарини представил Герта Лоу королеве Виктории. Герт, по-видимому, был первым цветным охотником из Калахари, который пожал руку королеве. Герт дожил до ста лет, но никогда не забывал о своих заморских впечатлениях. Вспоминая об этих днях, он обычно говорил: <<В одном доме в Лондоне можно разместить весь мой народ. Люди там словно саранча~--- так их много>>.

Итак, 30 января 1885 года Фарини, его сын (фотограф) и Герт Лоу сошли в Кейптауне с корабля <<Рос-лин Касл>> на берег. Через три дня они отправились в Кимберли. Там Фарини приобрел фургон на рессорах и мулов. На реке Оранжевой, в районе Апингтона, он обменял мулов на волов и поспешил в пустыню.

На реке Молопо, к северу от Кея, Фарини встретил немецкого торговца Фрица Ландвера, который погибал от дизентерии и голода. Вскоре Ландвер оправился и присоединился к экспедиции.

Герт Лоу посоветовал нанять слугу Яна, африканца смешанной крови. В экспедиции было еще два таких же африканца, имен которых никто не знал. Добавлю, что Лоу умер примерно в 1915 году, а Ян был жив еще в ноябре 1933 года.

Перед отъездом из Кейптауна Фарини получил от некоего Д.~Д.~Причарда схематическую карту Калахари. Причард был инженером и, кажется, по заданию Се-силя Родса ездил на озеро Нгами, чтобы составить карту. С помощью этой карты и указаний Герта Лоу Фарини вышел в район озера Нгами. Год выдался на редкость влажным, и Калахари напоминала цветущий сад. Эта холмистая земля, покрытая золотистыми зреющими травами, напоминала один из хлебных районов Англии. Все дюны были сплошь покрыты дынями тсам-ма, и каждый день можно было подстрелить антилопу канну или гну.

Фарини не пошел по известным маршрутам. В своем дневнике он сделал запись: <<Мы не испытывали недостатка в пище~--- ее хватало нам самим и нашим животным, и, кроме того, я верил в удачу, которая меня никогда не оставляла>>.

В районе южнее Нгами Фарини не нашел обещанных алмазов. Он добрался до местечка Керсиз, где познакомился с одним англичанином, женатым на африканке. <<Это был образованный человек из хорошей семьи, и, слушая его, я удивлялся, почему он поселился в этом заброшенном уголке,~--- писал Фарини.~--- По его личной просьбе я не упоминаю здесь его имени>>.

Фарини побывал в селении Миер (ныне Ритфон-тейн), а затем снова двинулся на восток. По высохшему руслу реки Носсоб он добрался до места слияния ее с таким же высохшим притоком Ауб и отправился дальше на север. Через три дня Фарини достиг гор Ки-Ки. Если вы собираетесь отправиться на поиски <<затерянного города>>, вам очень пригодятся эти ориентиры.

В районе Ки-Ки Фарини свернул в сторону от Носсоба и пошел на восток через пески. Спустя четыре дня он оказался у лесного массива Кгунг. Там он занялся охотой, а также ловлей бабочек и других насекомых. Жизнь в пустыне, несомненно, пришлась по душе Фарини и его сыну. Только когда у них кончились запасы риса, они двинулись на юг в Апингтон. На другой день впереди показалась высокая горная вершина. Проводник Ян сказал, что это Ки-Ки. Но когда они к ней подошли, оказалось, что никто этой горы никогда не видел и ничего о ней не слыхал.

И тут произошло волнующее открытие. "Мы раскинули лагерь у подножия горы,~--- писал Фарини,~--- у каменистой гряды, по своему виду напоминавшей китайскую стену после землетрясения. Это оказались развалины огромного строения, местами занесенного песком. Мы тщательно осмотрели эти развалины протяженностью почти в милю. Они представляли собой груду огромных тесаных камней, и кое-где между ними были ясно видны следы цемента. Камни верхнего ряда сильно выветрились, некоторые из них были похожи на стол на одной короткой ножке.

\begin{figure}[ht!]
\centering
\includegraphics[width=90mm]{000011.jpg}
\caption{Развалины в Зимбабве}
\label{overflow}
\end{figure}


В общем стена имела форму полукруга, внутри которого на расстоянии приблизительно сорок футов друг от друга располагались груды каменной кладки в форме овала или тупого эллипса высотой полтора фута. Основание у них было плоское, но по бокам примерно на фут от края шла выемка. Некоторые из этих сооружений были выбиты из цельного камня, другие состояли из нескольких камней, тщательно подогнанных друг к другу. Поскольку все они в той или иной мере были занесены песком, мы приказали всем своим людям раскопать лопатами самое большое из них (эта работа явно пришлась им не по вкусу) и обнаружили, что песок предохранил места стыка от разрушения. Раскопки отняли почти целый день, что вызвало немалое возмущение у Яна. Он не мог понять, зачем понадобилось откапывать старые камни. Для него это занятие представлялось пустой тратой времени. Я объяснил ему, что это остатки города, или места поклонения, или же кладбища великого народа, жившего здесь, может быть, много тысяч лет тому назад.

Мы стали раскапывать песок в средней части полукруга и обнаружили мостовую футов двадцать шириной, выложенную крупными камнями. Верхний слой был из продолговатых камней, поставленных под прямым углом к нижнему слою. Эту мостовую пересекала ругая такая же мостовая, образуя мальтийский крест, идимо, в центре его был когда-то какой-нибудь алтарь, колонна или памятник, о чем свидетельствовало охранившееся основание~--- полуразрушенная каменная :ладка. Мой сын попытался отыскать какие-нибудь иеро-лифы или надписи, но ничего не нашел. Тогда он сделал несколько фотоснимков и набросков. Пусть более сведущие люди, чем я, судят по ним о том, когда и кем был построен этот город".

Покинув развалины, Фарини через три дня снова оказался в районе гор Ки-Ки.

Это было единственное подробное описание <<затерянного города>>. Оно появилось в книге Фарини <<Через пустыню Калахари>>, которая вышла в Лондоне в 1886 году. В том же году Фарини сделал сообщение в Королевском географическом обществе в Лондоне. Его никак нельзя было заподозрить в шарлатанстве. Шутливый тон Фарини может удивить современного читателя, но что касается самого рассказа, то он подвергся тщательной проверке и полностью ее выдержал. Сын Фарини сделал ряд зарисовок и фотоснимков известных мест и образцов живописи бушменов. Подлинность их не вызывает сомнений.

В заключение можно сказать, что у Фарини не было оснований выдумывать <<затерянный город>>, а у его сына~--- делать наброски развалин воображаемого города. Книга и без того интересна. Фарини не ставит <<затерянный город>> в центре описания, как это сделал бы беллетрист. Наоборот, он пишет о нем вскользь, не выделяя его среди других эпизодов путешествия. Я убежден, что Фарини писал лишь о том, что видел. И он действительно видел <<китайскую стену>>, а также каменную кладку и другие останки <<затерянного города>>.

В своем сообщении в Королевском географическом обществе Фарини заявил, что <<затерянный город>> расположен на 23,5$^{\circ}$ южной широты и 21,5$^{\circ}$ восточной долготы. Путешественник, конечно, не сомневался в верности этих координат. Теперь доказано, что карта, которой он пользовался, страдала погрешностями. Не удивительно, что при сравнении ее с последними картами этого малоизвестного района было найдено много расхождений и пропусков. Тщательные подсчеты показали, что город может находиться на семьдесят миль севернее или южнее и на сорок миль западнее или восточнее точки, указанной Фарини.

Фарини, однако, оставил и другие данные о местоположении <<затерянного города>>. Он говорит, что покрывал за день расстояние от двадцати до тридцати миль, так что его путь можно восстановить по упомянутым им ориентирам, подобно тому как опытный моряк может (без современных средств) прокладывать путь в тумане с помощью счисления. Фарини отметил также, что развалины находились в тридцати или тридцати пяти милях от устья довольно-таки длинного притока реки Носсоб, и добавил к тому же, что приток этот направляется почти точно с севера на юг. А это очень ценная информация, если кто-нибудь станет изучать данный район по карте аэрофотосъемки. Однако в войну 1939-1945 годов район реки Носсоб был далеко в стороне от военных действий, поэтому никаких аэросъемок не производилось.

Какова же дальнейшая судьба удивительного открытия Фарини? Были ли с тех пор еще какие-нибудь слухи об этих развалинах или открытие забыто?

Слухи, конечно, были. В уединенном селении Ганзи, в Бечуаналенде, я встречал фермеров, которые слышали от африканцев о существовании развалин~--- грудах камней, где в давние времена жили люди. Шотландец Смит, известный грабитель из Калахари, частенько говорил, что видел эти руины.

Вероятно, самые ценные сведения сообщил недавно молодой фермер Николас Кютзее из Гордонии. В 1933 году он сказал доктору У. Минту Борчердсу из Апингтона, что несколько лет назад, охотясь в районе к востоку от Носсоба, он увидел каменное строение, такое же, как описал Фарини. Тогда Кютзее очень торопился. Он не был археологом и не стал задерживаться, чтобы получше осмотреть развалины. Место он запомнил лишь приблизительно. Все же нет сомнений, что Кютзее (у которого не было причин для выдумок) действительно видел что-то интересное.

Наиболее солидную попытку отыскать <<затерянный город>> предпринял в 1933 году Ф.~Р.~Пейвер\footnote{Тот самый Пеивер, который занимался выяснением возраста Рамонотване 
(см.~главу седьмую).~--- \textit{Прим. авт.}}), заслуживающий доверия археолог-любитель из Иоганнесбурга. Тщательно изучив книгу Фарини и его сообщение Королевскому географическому обществу, всесторонне проверив приведенные им факты и взвесив все прочие доказательства, о которых я уже говорил выше, Пейвер выработал план предварительной экспедиции. Она должна была выяснить достоверность географических данных Фарини.

Пейвер выехал из Апингтона с доктором Борчерд-сом, неутомимым путешественником пустыни, который хорошо знал Шотландца Смита и других знаменитостей Калахари. Полиция оказала путешественникам содействие и доставила их к Яну Абрахамсу, охотнику, который был проводником у самого Фарини. Старый Ян хорошо помнил дорогу, но его рассказ о развалинах звучал неубедительно. Фарини, как вы, может быть, помните, говорил, что Яна не интересовали развалины. И теперь, почти полвека спустя, он был к ним также равнодушен. Старик вспоминал охоту, но не груды камней.

В распоряжении экспедиции были легковой автомобиль и грузовик. По дороге на север Пейвер прихватил Николаса Кютзее, а затем местного торговца Йосте. В составе экспедиции был также проводник-готтентот, который не в первый раз сопровождал путешественников в район к востоку от реки Носсоб. С хорошим снаряжением и тщательно проверенной информацией Пейвер смело направился в страну Фарини.

Он решил обследовать средний из трех притоков реки Носсоб. На немецкой карте это высохшее русло называлось Молентсване. Но ни в лондонском атласе <<Таймса>>, ни даже на официальной карте Бечуанален-да 1933 года масштаба восемь миль в дюйме Пейверу этого притока обнаружить не удалось. И все же они отыскали приток и стали пробираться по его высохшему дну. Машины утопали в глубоком песке, расходуя по галлону бензина на каждые семь миль пути.

За весь первый день они проехали всего лишь тридцать миль. Я вполне могу им посочувствовать, потому что однажды тоже потратил целый день (с девяти утра до одиннадцати вечера) и продвинулся только на пятьдесят миль. Это был один из тех дней, о котором стоит подумать заранее, когда еще готовишься к экспедиции по Калахари. Один из тех дней, когда начинаешь задумываться, почему ты не захватил с собой инструменты вместо безделушек.

В тот же день Пейвер и его спутники потеряли следы притока. А на следующий день проводник-готтентот признался, что он никогда еще не заходил дальше этих мест. Они миновали единственный отмеченный на карте ориентир на этой площади в две тысячи квадратных миль. Это была впадина Димпо, оказавшаяся на самом деле группой впадин. Наконец они подошли к стране дюн и окинули взглядом расстилавшуюся перед ними равнину. К реке Носсоб экспедиция возвращалась другим путем, изучая встречавшиеся выходы известняков. Все были уверены, что такой тонкий наблюдатель, как Фарини, не мог принять естественные скалы за развалины города.

Прежде чем самому отправиться на поиски, я написал Пейверу. И вот что он мне ответил: <<Из отчета Фарини я понял, что указанное место находится примерно в шестидесяти милях от реки Носсоб, вероятно, на двадцать пятом градусе южной широты>>. Затем следовали слова, которые я уже цитировал: <<Все это очень туманно. Когда вы увидите эту пустыню, вы поймете, что можно месяцами бродить среди песчаных дюн и даже близко не подойти к тем местам, где расположен ``затерянный город''>>.

В июне 1947 года я снова был в Апингтоне и долго беседовал с доктором Борчердсом о <<затерянном городе>>. Я увидел, что немолодого уже доктора по-прежнему притягивают эти развалины.

---~Недавно мне встретились два человека,~--- сказал Борчердс,~--- которые заявили, что они побывали в <<затерянном городе>>. Я не могу назвать вам их имен. Это фермеры, которые незаконно охотились в Бечуаналендс. Именно поэтому они и не заявили о своей находке. Но я подробно расспросил их и могу сказать, что их описание полностью соответствует данным Фарини.

\begin{figure}[ht!]
\centering
\includegraphics[width=90mm]{000012.jpg}
\caption{Лед в пустыне Калахари}
\label{overflow}
\end{figure}


Кроме того, доктор Борчердс сообщил мне кое-какие сведения, которые совершенно по-новому освещали тайну <<затерянного города>>. Однажды сержант полиции рассказал ему, что много лет назад во время объезда он наткнулся на древнюю каменоломню. Там он увидел несколько обтесанных камней. Каменоломня эта была как раз в районе <<затерянного города>>. Сержант откопал в песке остов лодки длиной четырнадцать футов.

---~Теперь я слишком стар для путешествий по пус-тыне,~--- с грустью заметил доктор Борчердс.~--- Но я уверен, что много веков назад в Калахари действительно существовало поселение, которое описал Фарини. Нам известно, что из озера Нгами вытекали реки, которые направлялись через пустыню на юг и впадали в реку Оранжевую. Значит, у жителей этого селения была вода, а теперь мы узнали, что у них были и средства передвижения. Эта лодка кажется мне убедительным свидетельством. Думаю, что я и сам очень близко подходил к этому месту. Несомненно, в скором времени дюны раскроют свою тайну.

Но пока загадка не разгадана. Многочисленные попытки добраться до <<затерянного города>> на джипах, на самолетах, пешком кончались неудачей. Ни одной экспедиции не удалось отыскать развалин Фарини. Видно, какая-нибудь сильная буря занесла песком древние стены. И только еще более яростная буря может развеять этот песок.

\chapter{Тайны водопада Виктория}

<<Моси-оа-Тунья!>> Нужно самому услышать грохот водопада, чтобы как следует понять, почему люди ма-шона назвали это место <<гремящий дым>>. Но у меня закружилась голова, когда я слишком близко подошел к краю пропасти, и пришлось отодвинуться, чтобы прийти в себя.

Углубившись в тропический лес, я невольно вспомнил карту, которую видел когда-то, карту д'Анвилля, изданную почти два столетия назад Исааком Тирьоном в Амстердаме. На ней отмечен Великий водопад в самом центре Южной Африки и Зимбабве на землях мономотапа.

Некоторые историки утверждают, что европейцы видели ревущие воды водопада Виктория задолго до того, как их открыл Давид Ливингстон. Я уже давно понял, что Киплинг был прав, когда писал об <<измученных путниках, попавших туда до первых исследователей>>. У меня есть веские доказательства, что такие искатели приключений побывали во многих глухих уголках Африки. Легенды о водопаде Виктория стоят того, чтобы ими заняться.

Я знаю, с каким сарказмом и возмущением относится современное правительство Родезии к любой попытке поставить под сомнение приоритет путешественника, памятник которому возвышается у Чертова водопада. Однако слава Давида Ливингстона бесспорна. Он открыл миру водопад Виктория, и значение подвигов этого замечательного человека выходит далеко за рамки простых географических открытий. А теперь позвольте мне добавить, что водопад Виктория впервые открыл не Ливингстон.

Раньше всех туда могли проникнуть португальцы. В библиотеке Ватикана хранятся португальские карты семнадцатого века, на которых показан Великий водопад на реке Куаме, как тогда ее называли португальцы. Очевидно, это Замбези. (О Великом водопаде я буду говорить позднее.) Я обсуждал этот вопрос с Эдвардом К. Рэшлеем, автором превосходной книги о величайших водопадах мира. У него есть данные, что в начале восемнадцатого века у водопада побывал португальский священник Силбьера.

Капитан Д.~Д.~Рейнард, бывший смотритель водопада Виктория, вместе с преподобным Э. Кингом проделал большую исследовательскую работу в этом направлении. Обоих исследователей поразили сведения о португальцах. Эти путешественники из Лиссабона совершали настоящие подвиги. Историк Баррос упоминает озеро Ньяса еще в начале шестнадцатого века, хотя считается, что его открыл Ливингстон в 1859 году. В 1578 году Лопес опубликовал книгу путешествий с картой, на которой было обозначено не только озеро Ньяса, но также Виктория-Ньянца и Танганьика. Португальцы, бесспорно, уже несколько столетий назад знали о

Зимбабве. Это название (по-португальски Симбаоэ) появилось на их картах в середине шестнадцатого века, а вскоре после этого португальские рыцари в доспехах проникли в поисках золота на территорию нынешней Родезии. Они побывали в Зимбабве и, вполне возможно, добрались и до водопада Виктория. Однако Рей-нард и Кинг сделали печальное открытие, что во время разразившегося в 1775 году в Лиссабоне землетрясения и пожара погибли записки о путешествии к Замбези.

Теперь я расскажу о Великом водопаде на старинных картах, который многих вводил в заблуждение. На реке Замбези, чуть выше Тете, в шестистах милях вниз по течению от водопада Виктория, есть величественное ущелье с водопадами Кебра-баса. По своей грандиозности они лишь немногим уступают водопаду Виктория, и мимо них едва ли может пройти хоть один картограф. Водопады расположены в нескольких сотнях миль от морского побережья, а этого вполне достаточно, чтобы сбить с толку неопытных исследователей, изучающих старинные карты, где отмечены эти водопады. В библиотеке парламента в Кейптауне есть карта Боултона 1794 года. Естественно, что Великий водопад показан на ней не на том месте, где находится водопад Виктория. Но любители считают, что это простительная ошибка.

На самом же деле составители карт того времени знали, что делали. Они наносили на карту Великий водопад, который видели их соотечественники, а вовсе не водопад Виктория. Так что первенство португальцев все еще не доказано. Прошло немало времени, прежде чем на сцене появился еще один возможный первооткрыватель водопада Виктория. Это был Карел Три-хардт, старший сын грозного Луи. Оба эти воортрекке-ра относятся к числу первых исследователей Африки, и теперь любой школьник в Южной Африке знает об их путешествиях.

В 1838 году Карел Трихардт на португальской шхуне плавал вдоль побережья Восточной Африки, выискивая место, где могли бы обосноваться поселенцы, которых он оставил в заливе Делагоа. Он добрался до берегов Абиссинии и видел там, как в Берберу прибыл караван слонов с товарами под охраной вооруженных всадников. В некоторых местах побережья Трихардт задерживался по нескольку недель и даже месяцев и совершал смелые вылазки в глубь неизведанной страны. Наняв носильщиков, он прошел от Сафалы до Зимбабве, а из Келимане направился вверх по течению Замбези. Многие писатели предполагают, что именно во время этого путешествия Трихардт открыл водопад Виктория. А некоторые в этом абсолютно уверены. В одной географической работе, одобренной министерством просвещения Трансвааля, об этом говорится как о несомненном факте.

Здесь та же самая ошибка. Трихардт, конечно, был у водопадов Кебра-баса. У него не было времени, чтобы добраться до водопада Виктория, да он этого никогда и не утверждал. (Сотрудник архива Претории Д.~У.~Крюгер в своей работе, написанной несколько лет назад, доказал это совершенно ясно.) Трихардт умер в 1901 году. Незадолго до смерти, в возрасте девяноста лет, он рассказал наиболее памятные случаи из своей жизни Г.~А.~Оде, историку Южно-Африканской Республики. В записях Оде сказано, что Трихардт видел <<водопады где-то южнее Сандии>>. Несомненно, это водопады Кебра-баса. Никому не удалось установить, что Карел Трихардт когда-нибудь мог побывать у водопада Виктория. А ведь это был человек, который ни перед чем не останавливался. Если бы Трихардт оказался поблизости от водопада, он непременно бы там побывал. Историк сэр Джордж Кори был твердо убежден, что Трихардт видел водопад Виктория. Но на мой взгляд, Кори стал такой же жертвой Кебра-баса, как и все другие.

Следующий претендент~--- Генри Хартли. Этот косолапый мужчина с серо-голубыми глазами и львиной гривой в течение многих лет бродил по диким местам нынешней Родезии и пустыни Калахари. Его потомки уверены, что он открыл водопад Виктория за шесть лет до Ливингстона. И на мой взгляд, они представили довольно убедительные доказательства.

Хартли был из семьи поселенцев 1820 года. Когда первые воортреккеры отправились в путь, Хартли обуяла жажда приключений. Вскоре он сам переехал в Трансвааль, где основал табачную фабрику. Эта фабрика в Магалисбурге процветает и до сих пор. Впервые реку

Лимпопо Хартли пересек в 1846 году. С ним было несколько слуг, в том числе погонщик волов готтентот Оресьян.

Следующий переход Хартли совершил в то время, когда его старшему сыну Фреду исполнилось три года. Это было в 1849 году. На этот раз они забрались далеко на север, в районы, где до этого никто из них не был. Однажды они услышали в отдалении непрерывный громоподобный гул. Определив, с какой стороны доносились эти звуки, Хартли и Оресьян отправились туда и вышли к водопаду Виктория.

Племянник Хартли, капитан Р. Хартли Тэккерей, записал подробности этого путешествия со слов родственников и друзей Хартли. Некоторые припомнили, как описывал водопад Оресьян. Готтентот с удивлением рассказывал о радуге, раскинувшейся над водопадом, и непрерывном дожде, который падал с безоблачного неба.

В 1948 году был еще жив младший сын Хартли. Ему тогда исполнилось восемьдесят восемь лет, и он жил з Иоганнесбурге. Генри Хартли-младший помнил рассказы своего отца об этом открытии и сообщил один любопытный факт. Хартли был охотником и обычно продавал слоновую кость, рога и шкуры Форсману, владельцу магазина в Почефстроме. Он рассказал Форсману о водопаде. И вот однажды, это было в 1852 году, Фор-сман познакомил Хартли с одним путешественником, который хотел знать поподробнее, как добраться до водопада. Хартли ему объяснил. Этим путешественником был Ливингстон.

Бывший ректор Витватерсрандского университета X. Р. Райкес считает, что его дед, У.~К.~Осуэлл, видел водопад раньше Ливингстона. Этот стройный обаятельный человек, опытный охотник на слонов, был замечательным путешественником. (Он получил золотую медаль Парижского географического общества за открытие озера Нгами, а Ливингстон был награжден золотой медалью Лондонского географического общества.) Нет никаких сомнений, что самую раннюю точную карту с указанием местоположения водопада Виктория составил Осуэлл, после того как в 1851 году он вместе с Ли-вингстоном совершил путешествие к Замбези. Я только не могу понять, почему Ливингстон и Осуэлл не дошли тогда до водопада (может быть, конечно, Осуэлл все же был там)~--- ведь на карте Осуэлла есть пометка: <<Водопад. Брызги видны за десять миль>>.

Осуэлл никогда не писал о своих путешествиях. Он был скромным человеком и предпочитал, чтобы честь их совместных открытий принадлежала его другу Ли-вингстону. У нас нет точных сведений о маршруте путешествия 1851 года. Возможно, что Осуэлл все же видел великий водопад. Отсюда и семейное предание. Если бы Осуэлл не ленился писать, то вполне возможно, что история открытия водопада Виктория выглядела бы совсем по-другому.

В 1855 году Джеймс Чапмен впервые совершил путешествие из Дурбана к заливу Уэлвис. Некоторые авторы считают, что по пути он мог видеть водопад Виктория. Я не разделяю этого мнения, хотя очень внимательно просматривал записки Чапмена в архиве Кейптауна. Видимо, эта мысль возникла при изучении маршрута Чапмена. В одном месте он пролегал всего в семидесяти милях от водопада.

Чапмен, однако, рассказывает одну интересную историю. В 1852 гбду, возвращаясь из экспедиции по реке Дека, он встретился с человеком, по имени Д. Симпстон, у которого случилось несчастье. Он занимался торговлей и охотой в районе Чобе, зараженном мухой цеце, и у него погибли все быки. Симпсон сказал, что, когда он возвращался из Линьянти по реке Замбези, ему встретился огромный водопад. Вскоре после разговора с Чапменом Симпсон уехал в Австралию, где в это время начиналась <<золотая лихорадка>>. Больше он никому не рассказывал о своем открытии. Мне иногда хочется, чтобы все путешественники поддавались всеобщему искушению, перед которым не мог устоять даже великий Ливингстон, и вырезали на деревьях свои имена и даты. Тогда многие легенды стали бы реальностью.

На честь открытия водопада Виктория упорно претендуют потомки Яна Вильена, одного из первых бурских охотников. Этого отчаянного смельчака пытались заполучить англичане за его участие в битве при Бомплатсе. Одно время Вильен был связан с Чапменом, потом стал самостоятельно снаряжать экспедиции в страну Мзиликази. Вместе с проводником и пятьюдесятью воинами, которых ему прислал Мзиликази, Вильен направился к водопаду. С ним были его сыновья Георг и Петрус и соотечественники Якоб Эразмус, Пит Якобс и Герман Энгельбрехт.

Эта экспедиция, по семейному преданию Вильенов, побывала у водопада раньше Ливингстона трижды~--- в 1851, 1853 и 1854 годах. Сохранилось множество подробностей, которые не оставляют сомнений в том, что Вильен и его товарищи были у водопада. Однако мысль описать эти экспедиции пришла в голову их участникам лишь тогда, когда они уже стали стариками и не помнили точных дат. Доктор X. К. де Вет, занимавшийся изучением всех сохранившихся сведений, выяснил, что первым отправился в Булавайо миссионер Моффат. Мзиликази встревожился, увидев фургон Моффата, который показался ему очень странным сооружением. Это было в 1855 году. А Вильен, как выясняется, впервые посетил Мзиликази в 1859 году. В 1860 году Ливингстон побывал у водопада Виктория второй раз. В то время бурские охотники не знали о его предшествующем посещении, поэтому-то они и считали, что первые открыли водопад. Однако легенда о Вильене все еще живет.

В семейном архиве Преториусов из Мэридейля (Капская провинция) хранится одно прелестное повествование. Историю эту рассказал много лет назад Биллем Хендрик Преториус из Ритпорта (Трансвааль), а записал внук его 3. К. Преториус из Мэридейля.

В. X. Преториус родился в Граф-Рейнете в 1821 году и дожил до ста лет. В 1855 году Преториус и его молодой друг Стофель Снимай в фургонах, запряженных волами, выехали из Трансвааля, чтобы поохотиться на крупную дичь к северу от реки Лимпопо.

Добравшись до крааля Мзиликази, они оставили там своих волов и фургоны, наняли двести носильщиков и направились через зараженный мухой цеце район. Проводники привели их к огромному водопаду, где они устроили стоянку. На девятый день друзья заметили дым костра. Значит, кто-то еще разбил там лагерь. Это оказался Давид Ливингстон, больной и голодный. Его проводникам-африканцам приходилось растирать и поджаривать сыромятную кожу, которой были обтянуты их щиты. Преториус и Снимай дали Ливингстону еду и лекарства и оставались с ним до тех пор, пока он не поправился.

Снимай умер в 1920 году. Он тоже очень любил рассказывать историю открытия водопада. Но почему же сам Ливингстон не упоминает о своих спасителях? Это непостижимая загадка. Как-то не верится, что Преториус и Снимай просто все выдумали, потому что рассказ этот звучит правдоподобно. Единственное возможное здесь объяснение могло бы оказаться несправедливым по отношению к великому миссионеру-путешественнику. Так что примите это объяснение лишь как попытку пролить хоть какой-то свет на глубокую тайну прошлого.

Главной задачей Ливингстона было не исследование Африки, а борьба против рабства. Он редко завязывал дружбу с первыми бурскими охотниками, так как считал некоторых из них врагами дела, дорогого его сердцу. Кроме того, Ливингстон восстал против бессмысленного уничтожения животных, считая это безумием. Тот, кто изучал его путешествия, должен был заметить, как резко отзывается Ливингстон о тех белых, с которыми ему доводилось встретиться, если они заслуживали его презрения. Иногда он просто не упоминает о них. И тем не менее трудно представить, чтобы такой добрый христианин, как Ливингстон, ничего не сказал бы о той помощи, какую ему оказали Преториус и Снимай. Трудно к тому же увязать всю эту историю со словами Ливингстона о своем великом открытии в книге <<Путешествия миссионера>>: <<Никогда еще его не видели глаза европейца>>. Вполне возможно, однако, что европейцами Ливингстон считал только тех, кто, подобно ему самому, родился в Европе.

Когда первые европейские путешественники пришли к водопаду Виктория, да и долгое время после этого, там не было никаких поселений на шестьдесят миль вокруг. Африканцы боялись злого духа, который, по их поверью, обитает у водопада. Остров Катаракт, расположенный у края водопада, носил некогда название Чертова острова. Миссионер Койяр писал о нем так: "Туземцы верят, что на нем обитает злобное и жестокое божество, и, чтобы снискать его расположение, приносят ему в жертву ожерелья, браслеты и прочие предметы.

Они бросают эти вещи в пропасть и произносят при этом мрачные заклинания, что вполне соответствует их страху и ужасу".

В существование <<чудовища>>, обитающего у водопада, верят и многие европейцы. Капитан Рейнар, о котором я уже упоминал, сообщил мне, что три человека, в чьих словах он не сомневается, видели это чудовище.

У Ливингстона есть упоминание о змее, живущем в этих водах. Это змей из фольклора народа баротсе. Африканцы уверяли Ливингстона, что змей настолько велик, что может остановить лодку и никакие усилия гребцов не смогут сдвинуть ее с места. Говорят, что этот змей достигает тридцати футов в длину, у него небольшая серо-голубая голова и толстое черное туловище.

В. Пэр, заведовавший много лет пароходством на реке Замбези, спустился в 1925 году на дно ущелья у водопада Виктория. Уровень воды в реке был в тот год самый низкий за последние десятилетия. И тут он впервые увидел чудовище. Это было змееподобное существо. Заметив Пэра, оно поднялось и исчезло в глубокой пещере. Через несколько лет Пэр заявил, что видел его опять у Чертова острова.

Африканцы называют этого монстра Чипик и утверждают, что он приплыл к водопаду из океана, проделав путь в тысячу миль. Местные рыбаки так боятся его, что не рискуют выходить по ночам. <<В ночные часы Чипик~--- хозяин реки>>,~--- говорят рыбаки.

Д.~У.~Соупер, который поймал и застрелил в районе водопада множество крокодилов, слышал от африканцев о существовании очень крупных особей. Но едва ли Пэр мог принять крокодила за монстра. Возможно, это был крупный питон, подобный легендарной гигантской змее реки Оранжевой.

Ходили слухи, что один отчаянный пилот пролетел однажды на небольшом самолете под мостом водопада Виктория. Когда я был журналистом, я попытался выяснить, насколько достоверны эти слухи. В то время ко мне в редакцию пришел Д.~Д.~Джекобс из Иоганнесбурга и сообщил нужные мне сведения.

В 1931 году Джекобс летал вместе с летчиком Пэтом Холлиндрейком, служившим в авиакомпании Ньяса-ленда и Родезии. Однажды эти отчаюги прикинули, возможно ли пролететь под мостом, и решили рискнуть. На следующий день они чуть не погибли на глазах у огромной толпы, собравшейся здесь по случаю пасхи.

<<Когда мы пролетали над главным водопадом, мы думали, что разобьемся,~--- вспоминал Джекобс- Нас засасывало. Водопад ``притягивал'' нас к себе с такой силой, что, казалось, не было никакой возможности выровнять самолет. Холлиндрейк потянул ручку управления на себя и дал полный газ. Мы пролетели в нескольких футах от моста. Пошли слухи, что нам удалось пролететь под мостом. Об этом говорят и по сей день>>.

В течение четверти века власти запрещали пролетать под мостом водопада Виктория. Сомневаюсь, однако, что пилот, рискнувший нарушить запрет, смог бы заплатить штраф. Едва ли бы он остался в живых. До сих пор в районе водопада воздушные катастрофы случались очень редко. Но все же над одной из могил на кладбище в Ливингстоне возвышается огромный металлический пропеллер. Это могила летчика, который впервые увидел водопад Виктория незадолго до начала второй мировой войны. <<Очень не хочется улетать отсюда>>,~--- сказал пилот, забираясь в кабину самолета. День выдался абсолютно безветренный, и взлетная дорожка была слишком коротка. Через несколько секунд летчик погиб.

Во время строительства моста через водопад Виктория подрядчикам было дано официальное распоряжение натянуть под ним предохранительную сетку, какими пользуются во время работы на трапеции цирковые артисты. Сетка была натянута. Но тут забастовали рабочие-африканцы. Они решили, что им прикажут прыгать на эту сетку, а они не переносили одного ее вида. И только когда сетку убрали, они снова приступили к работе.

При строительстве использовались ракеты, чтобы перебросить тросы с одного берега на другой. Строительство велось одновременно с двух сторон и было закончено точно в срок - 1 апреля 1905 года. Пока еще не были установлены фермы, люди и материалы переправлялись через ущелье в брезентовых мешках. Главный инженер строительства М. Джордж Имболт сам бесстрашно переправился в этом мешке, потому что среди строителей не нашлось добровольцев сделать это даже за вознаграждение.

Когда настало время убрать висящий под мостом стальной канат и блоки, Имболт выполнил и эту рискованную операцию. Ни один рабочий не взялся за это, хотя снова было предложено вознаграждение. Имболта спустили с моста на небольшой доске, и он, ни за что не держась, обеими руками освобождал блоки.

Однажды на строительстве соскользнула балка. Она убила механика-европейца и сшибла одного африканца, который нашел свою смерть в Кипящем Котле. Еще один рабочий свалился с моста и, пролетев семьдесят футов, упал на пологий откос. Он остался жив и возбудил судебное дело, но проиграл его. С тех пор на строительстве было еще много опасных работ, однако все обошлось без человеческих жертв.

Бывали случаи, когда люди падали в водоворот у основания водопада и оставались живы. Я слышал историю, возможно и вымышленную, которая относится еще ко времени строительства моста. Говорят, что служащий северородезийской полиции Рамсей плыл однажды во время наводнения на лодке по реке Замбези, в семи милях от водопада Виктория. Случайно он выронил весло, и безжалостный поток понес его лодку прямо к водопаду. У самого края его выбросило из лодки, и он упал в Кипящий Котел с высоты четыреста футов. При этом захватывающем дух происшествии присутствовало несколько человек, в их числе один полицейский. Они тут же бросились к водопаду. Один из них крепко обвязал веревку вокруг пояса, прыгнул в Кипящий Котел и схватил Рамсея. Течение вынесло их из опасной зоны. Оба оказались живы и невредимы. Так благополучно завершилась эта история об изумительной смелости и воле случая. Может быть, все это так и было на самом деле. По крайней мере так рассказывают люди. Мне бы хотелось найти какого-нибудь <<старого бродячего торговца>>, который подтвердил бы этот рассказ.

\begin{figure}[ht!]
\centering
\includegraphics[width=90mm]{000013.jpg}
\caption{Водопад Виктория}
\label{overflow}
\end{figure}


Один из трагических случаев водопада Виктория произошел из-за того, что бегемот перевернул лодку, в которой плыли люди. Двое белых мужчин, две женщины с ребенком и команда гребцов-африканцев очутились в быстром потоке, который понес их к водопаду. Оба европейца утонули. Один из гребцов спас ребенка и вернул его матери, за что ему была назначена пожизненная пенсия.

Много лет назад в районе водопада Виктория было найдено тело одного юноши. Неподалеку от водопада есть тропинка, ведущая к уступу, удачно названному Острие Ножа. По этой тропинке могут ходить только люди с крепкими нервами. Пониже тропинки, на выступе скалы, и было найдено это тело. Юноша находился в сидячем положении. Видимо, он свалился и, падая через заросли кустарника, повредил себе позвоночник. В карманах его нашли пятнадцать монет по полсоверена и железнодорожный билет до Элизабетвиля (Бельгийское Конго). Но личность этого человека так и не была установлена.

Среди тех, кто благодаря счастливой случайности остался жив после падения с обрыва, стоит упомянуть одного любителя раутов, пожилого мужчину, который однажды темной ночью хотел пройти по тропинке, ведущей от моста на дорогу Ливингстон-Роуд. Он заблудился и упал вниз. По счастью, он не разбился, так как повис на дереве. На рассвете кто-то услышал его крики и пришел на помощь. Спасенный горько сетовал на то, что потерял вставную челюсть и бутылку виски.

Менее удачливым оказался один военный моряк из африканской эскадры. Он проводил свой отпуск в районе водопада Виктория. В сухой сезон водопад обычно истощается, превращаясь из гигантского потока в ряд небольших ручейков. А тот год, когда моряку было суждено погибнуть, выдался на редкость засушливым. Моряк решил переправиться по кромке водопада на остров Ли-вингстона. Но он недооценил силу этого мощного потока. Со стороны водопад кажется просто струйками, однако попадать в них рискованно. Моряк ступил на шаткий камень и тут же полетел вниз. Тело его было найдено у самой поверхности воды. Оно застряло между двумя скалами, всего в футе от большого уступа водопада.

Тысячи туристов из Южной Африки с сожалением узнали о кончине известного <<старого бродячего торговца>> Перси М. Кларка, который продавал фотографии и всякие изделия африканцев в живописных хижинах на берегу Замбези. Перси Кларк умер в 1937 году. Это был один из тех людей, о смерти которых сообщают ошибочно. Еще в 1904 году после одной отчаянной проделки в Булавайо пришла весть о его смерти.

Вместе с инженером Фоксом Кларк решил исследовать ущелье водопада Виктория в то время, когда там строился мост. До них еще никто не пробовал спуститься на дно с южного берега. Там они разошлись. Измученный подъемом, Кларк остался ночевать в ущелье.

Фокс стал один взбираться наверх и сорвался. Он пролетел сотню футов, зацепившись по пути за дерево, что ослабило силу падения, и свалился на выступ скалы. Спасательная партия подняла его краном наверх. Тут же стали распространяться слухи о смерти Кларка. А Кларк тем временем благополучно выбрался из ущелья и увидел, что его друзья уже собрались распить бутылку виски, чтобы почтить его память.

У водопада вы можете услышать много правдивых историй о смелых людях. Однажды два африканца были выброшены вместе со своей лодкой на островок с восточной стороны водопада. Река в это время сильно разлилась. Казалось, они были обречены на этом островке на голодную смерть. Однако Пэр и полицейский Геральд Мартин тут же разделись, взяли большую лодку и вместе с пятью африканскими гребцами направились к острову. Шансов было очень мало, но Пэр так умело вел лодку, что им удалось подойти к острову, забрать потерпевших крушение и благополучно вернуться обратно. За этот подвиг Пэр и Мартин были награждены медалями Британской империи, а на долю пяти африканских гребцов досталась порядочная сумма денег, собранная жителями Ливингстона.

А вот какой случай произошел во время второй мировой войны. Отважные, но очень самоуверенные юноши, проходившие летную подготовку в Родезии, задумали слишком рискованное дело. Курсант летного училища Стэнтон решил взобраться на обрыв, возвышающийся на четыреста пятьдесят футов над Кипящим Котлом. Забравшись на высоту триста футов, он увидел, что попал в ловушку, из которой нельзя выбраться. Целый час пробыл он там, пока его не заметили с моста.

На помощь пришли сержанты полиции Пайвелл и Вордсворт. Вордсворт в специальных спасательных штанах попытался добраться до Стэнтона с помощью веревочной лестницы, которая оказалась слишком короткой. Тогда Пайвелл взял лестницу подлиннее. Но она зацепилась за куст, и он должен был остановиться и долго распутывать ее. Когда Пайвелл спустился до конца лестницы, он увидел, что их со Стэнтоном все еще разделяет расстояние в тридцать футов. Тогда сержант стал раскачивать лестницу, как цирковой артист трапецию, но из этого ничего не вышло.

На мосту собралась толпа любопытных зрителей. Они видели, как Стэнтон стоял, уцепившись за скалу, чтобы не свалиться в пропасть, и как висящий на высоте трехсот футов над Кипящим Котлом Пайвелл пытается дотянуться до Стэнтона. Это было очень волнующее зрелище. Внизу, как зловещее музыкальное сопровождение, ревел водопад Виктория. Во все стороны летели брызги, словно это был специально задуманный сценический эффект, а фоном служила зеленая пропасть.

Вдруг толпа вздрогнула. Зрители судорожно ухватились за перила, а многие из них зашептали молитвы, когда Пайвелл выбрался из спасательных штанов на веревочную лестницу. Сержант полиции с отчаянной смелостью принялся раскачиваться, как маятник, вместе с лестницей. Наконец он схватил Стэнтона, помог ему забраться в спасательные штаны и стал наблюдать, как его поднимали наверх. Только после этого Пайвелл выбрался сам. Своим поступком он, как никто другой, заслужил врученную ему бронзовую медаль Королевского общества спасения утопающих.

Февральским вечером 1955 года Алан Перри в компании друзей вышел из отеля, чтобы полюбоваться лунной радугой над Восточным водопадом. В ту ночь ему пришлось пережить больше страданий и мучений, чем любой другой жертве водопада Виктория.

Он разговаривал со своими друзьями у края обрыва, как вдруг почувствовал, что падает. И по сей день Перри не знает, как это случилось. То ли он поскользнулся, то ли потерял равновесие. Пролетев сто пятьдесят футов, он упал на дерево или на куст, и это смягчило его падение. И тем не менее он сломал с одного бока все ребра.

Бывший солдат и гонщик, Перри даже в таком страшном положении не потерял самообладания. Взглянув вниз, он увидел там, в сотнях футов, реку и сообразил, что нужно найти более безопасное место. Несмотря на сильную боль и потрясение, ему удалось проползти футов двенадцать к узкому выступу. Перри понимал, что может потерять сознание. Он собрал все силы и привязался шарфом к пню. По обрыву во все стороны метались лучи автомобильных фар и факелов, но только на рассвете Перри смог известить своих спасателей, что он не свалился в пропасть.

В спасении принимали участие Р.~Э.~Данн и лесничий Д.~В.~Теббитт. С веревками и носилками они бесстрашно спустились в пропасть по веревочной лестнице. Это было очень рискованное дело. Камни градом сыпались вокруг Перри. Один крупный камень, задетый носилками, попал ему в голову. Когда спасающие добрались до Перри, Данн вспрыснул ему морфий, а потом привязал к носилкам. В течение десяти часов, пока его медленно поднимали на вершину обрыва, Перри был в очень опасном состоянии. За эти десять часов он постарел на десять лет.

<<Моси-оа-Тунья!>> Какие удивительные истории рассказываешь ты своим могучим голосом! Замолкнет ли когда-нибудь этот грохот? Африканцы говорят, что триста лет назад водопад был в другом месте. Аэрофотосъемки показывают, что уступ Западного водопада подвергся эрозии в двух направлениях. Можно думать, что в будущем водопад переместится снова. Пройдет пятьдесят, сто лет, и Южная Африка, может быть, уже не будет привлекать туристов, приезжающих сейчас туда из разных уголков земного шара, чтобы пережить тот восторг, который в свое время пережил Ливингстон, когда как зачарованный смотрел на водопад, названный им в честь английской королевы водопадом Виктория.

\chapter{Самое замечательное зрелище Африки}

Грандиозные миграции южноафриканских газелей\footnote{Небольшая антилопа (\textit{Antidorcas marsupialis}), близкая к настоящим газелям, но отличающаяся от них строением зубов и характером волосяного покрова. В русской литературе называется также антилопа-скакун и горный скакун.}, опустошавшие вплоть до конца прошлого века обширные земли Карру, представляют собой, пожалуй, одну из наиболее волнующих картин в мире млекопитающих.

Шум, подобный сильному ветру перед грозой, будит жителей ферм и деревень, будит людей, спящих в фургонах посреди открытого вельда. Облака пыли на горизонте говорят о приближении стада антилоп. Вот уже слышится грохот копыт, мычание, свист и храп~--- и все вокруг становится океаном бегущих животных, океаном белых и бурых тонов, волнами светло-бурых спин, темно-бурых полос, белых животов, веерообразных белых волос на бедрах, потоком живой плоти. Один бур рассказал мне, какой страшный исход может иметь встреча человека с таким потоком. Обладая большим природным даром рассказчика, бур сумел очень ярко передать свои личные переживания и впечатления.

В семидесятые годы прошлого века, еще десятилетним мальчишкой, этот бур уехал вместе с семьей из Трансвааля. Это были участники первого <<перехода через сухую страну>>. Истинные или воображаемые обиды и мятущийся дух этих людей гнали их на поиски новых земель. Многие нашли тогда свою смерть по пути через пустыню. Некоторые добрались до Анголы. Что же касается семьи Ван дер Мерве, то она отделилась от остальных переселенцев и направилась на юг. Они все время перебирались с места на место со своими быками и овцами в поисках пастбищ. Когда старики умерли, их сын Герт продолжал вести кочевой образ жизни, к которому он привык. Он бродил по Бечуаналенду, Калахари и часто появлялся на северо-западе Капской провинции. В двадцать один год у него уже была жена и трое детей. Он держал двух чернокожих пастухов и погонщика волов, бушмена, который умел находить путь от одного источника или озера к другому.

Однажды утром фургон Герта Ван дер Мерве тащился по сухому, затвердевшему руслу реки Молопо. По этой реке как раз проходит южная граница протектората Бечуаналенд. Герт заметил, что бушмен чем-то обеспокоен. Через некоторое время погонщик вдруг спрыгнул с фургона и бросился в заросли на высоком северном берегу реки. В полдень Герт, как обычно, остановился, чтобы поесть и дать волам отдохнуть. Жена только начала готовить еду, как на стоянку примчался бушмен и стал уговаривать всех запрягать волов и сейчас же уезжать. <<Приближаются антилопы,~--- заявил он,~--- если вы останетесь здесь, вы погибли>>.

Герт стал укладывать вещи, не очень-то веря в опасность, но он не забывал, что с ним была семья. Бушмен вывел фургон из русла на северный берег и направился к холму. Ван дер Мерве гнал волов вверх по склону, пока это им было под силу, а потом все люди вышли из фургона и пешком добрались до вершины холма. Бушмен показал на горизонт.

Сначала Герт не заметил ничего особенного, но через некоторое время увидел едва заметное облако пыли. Оно было очень далеко, и Герт подумал, что не такая уж это опасность. Но бушмен все же уговорил его нарезать колючего кустарника и обнести им фургон и скот. Он сказал, что если животные пойдут не в обход, а прямо через холм, то они растопчут все на своем пути. Однако он надеялся, что колючий кустарник заставит их свернуть.

Защитив таким образом фургон и волов, Герт снова взобрался на вершину холма. Теперь облако было уже всего лишь в нескольких милях. Пыль поднималась высоко в небо и закрывала весь горизонт. Холм, кажется, был как раз на пути приближающихся животных. И тут в первый раз Герт почувствовал некоторую тревогу, так как понял, что всякое может случиться, если эта лавина обрушится на лагерь. Он приказал жене и детям забраться в фургон и крепко привязал собак. Вместе с бушменом и пастухами он набрал сухих дров и сложил их в кучу перед фургоном. Сверху они набросали зеленых веток и травы, чтобы костер сильнее дымился. Герт надеялся, что дым отпугнет газелей и заставит их обогнуть холм.

Он стоял на вершине холма и ждал. Газелей все еще не было видно за облаком пыли. Но мимо холма, не замечая людей, уже проносились зайцы, шакалы и другие мелкие животные. Из своих убежищ выползли змеи и спешили спрятаться под скалами холма. Герт и его спутники бросали в них камнями, если они подползали слишком близко. Но змеи, казалось, были охвачены каким-то более сильным страхом. Появилось также множество мангуст и полевок.

Наконец послышался отдаленный топот. Несомненно, бушмен слышал его уже несколько часов назад, приложив ухо к земле. Герт же уловил его только теперь. Пыль стала еще гуще, и в этом громадном облаке можно уже было различить первые ряды газелей, мчавшихся быстрее, чем скачущие галопом лошади. Животных было столько, что Герт ужаснулся. Он видел лишь первые ряды, растянувшиеся на три мили по фронту, но совершенно не представлял, сколько их еще там сзади. Впереди стада неслись вожаки, словно полководцы, ведущие в бой свои полки.

Когда газели были уже в миле от холма, бушмен побежал к фургону и забрался внутрь, несмотря на рычанье собак. Герт и пастухи стали поджигать сложенные в кучи дрова. Они не отходили от волов, которые, чуя опасность, метались во все стороны и в страхе припадали к земле. Жена уговаривала Герта влезть в фургон, но он был потрясен этим грандиозным зрелищем и взобрался на крышу, чтобы лучше видеть.

Первые крупные отряды газелей пронеслись мимо холма, обогнув его с обеих сторон. Затем животные потекли сплошным потоком, направляясь к реке и дальше на равнину. Напор все возрастал, поток становился более плотным. Животные уже не в состоянии были свернуть в сторону, когда приближались к кострам и фургону. Герт мог теперь стегнуть их своим кнутом с крыши фургона. Газели врезались в фургон и, израненные, падали под колеса. Их топтали следующие за ними животные, и вскоре у фургона образовалась целая гора раздавленных и умирающих газелей. Герт видел перед собой столько билтонга\footnote{Билтонг~--- вяленое мясо, нечто вроде пемикана американских индейцев.~--- Прим. авт.}, сколько ему не приходилось добывать и за год охоты. Но вот барьер из колючего кустарника был снесен, и газели смешались с волами. Охваченный паническим страхом, скот с ревом бросился к реке, исчезая в пыли. Герт их не удерживал. Всякий, кто попытался бы пуститься вслед за волами, тут же бы погиб под копытами и рогами газелей.

Все вокруг грохотало. Под ударами бесчисленных копыт земля превращалась в тончайшую пыль. Стало трудно дышать. Жена Герта, со страхом и интересом наблюдавшая за стремительным потоком животных, должна была закрыть себя и своих детей одеялами. Они чуть не задохнулись от пыли. Бледно-желтая пыль на целый дюйм покрыла все внутри фургона. Чернокожие африканцы тоже стали желтыми.

Целый час двигалась основная лавина, но и это еще не был конец. Уже давно зашло солнце, а сотни отставших животных все шли и шли вслед за основным стадом. Обессиленные, искалеченные, истекающие кровью\ldots Герт подумал, что же сталось с теми зайцами, шакалами и змеями, которые не смогли вовремя скрыться. На следующий день он это увидел.

Отдельные газели пробегали мимо фургона всю ночь. Воздух теперь очистился, но пыль поднималась всякий раз, как только в лагере начиналось какое-нибудь движение. На рассвете Герт взобрался на вершину холма, чтобы поискать своих волов. Еды у него было достаточно, неподалеку, в высохшем русле реки, был источник, но без волов он был бы беспомощен.

Утренний воздух был так чист, а солнце светило так ярко, что на какой-то момент Герту показалось, будто все события предыдущего дня были лишь ночным кошмаром. Но тут он заметил, что высокие деревья, листья которых служили хорошей пищей для его волов, превратились в жалкие обрубки с голыми ветвями. Газели вытоптали всю траву на своем пути, сломали молодые деревья, так что им никогда уже теперь не оправиться.

Герту показалось, что вдали он видит нескольких волов. После завтрака он отправился за ними со своими людьми. Все овраги и лощинки были заполнены трупами газелей. Видимо, первые животные остановились у края обрыва, раздумывая, смогут ли они его перепрыгнуть. Но не успели они принять решения, как на них обрушился безжалостный поток других животных. Одна за другой газели валились в овраг, пока он не сровнялся с землей, а неумолимая лавина продолжала двигаться по телам подмятых собратьев.

Повсюду валялись трупы мелких животных, расплющенные черепахи, клочья заячьих шкурок. Дерево, наклоненное навстречу двигавшемуся потоку, стало смертоносной пикой, на которой были нанизаны две газели. Когда Герт увидел все это, он понял, какая страшная участь могла бы постигнуть его семью, если бы он не послушался бушмена.

В течение двух недель Герт продолжал жить на берегу Молопо, разыскивая пропавший скот. Ему удалось отыскать половину упряжки. Он так и не узнал, куда делись остальные волы. Может быть, они бежали в паническом страхе перед лавиной, пока она их не настигла и не растоптала. Но возможно, им все же удалось выбраться из живой западни и убежать. Герт был рад, что осталась хоть часть волов. Он запряг их в фургон, и вся семья покинула место катастрофы. Герт закончил свой рассказ. Было ясно, что это самый памятный случай в его жизни, которую он очень любил. <<Мы живем хорошо,~--- сказал Герт.~--- Нас такая жизнь устраивает>>.

Вот какие испытания выпадают невзначай на долю фермеров и их семей, в основном в отдаленных районах, хотя в наши дни уже трудно найти очевидца подобных массовых стихийных миграций. Об этом ходят легенды, рассказанные отцами и дедами. Но я всегда предпочитаю легенде рассказ очевидца. Поэтому стараюсь отыскать очевидцев среди людей семидесяти- или восьмидесятилетнего возраста. Мне удалось встретить двух стариков, которым было уже за девяносто. Эти люди многое повидали на своем веку, но они до сих пор говорили с волнением о миграциях антилоп.

Я знаю, как переселяются могучие слоны, иногда большими стадами. Удивительны переселения североамериканских бизонов и оленей карибу. Сотни лет люди изучают и обсуждают переселение маленьких норвежских леммингов, которые миллионами покидают свои убежища в горах и опустошают все вокруг. Газели тоже движутся стадами в миллионы голов, покрывая значительные расстояния. И тоже тонут тысячами, когда подходят к рекам или морю.

Однажды я познакомился с человеком, который в конце прошлого века держал лавку на берегу реки Оранжевой. Он видел на реке живой мост из газелей, когда те направлялись к Калахари в поисках лучших пастбищ, как он выразился. Там погибло множество газелей, а основная масса, не замочив копыт, пересекла реку по их спинам.

Потом я встретился с Кохраном, бывшим полицейским из Капской провинции. В 1897 году ему пришлось охранять обнесенную забором территорию на южном берегу реки Оранжевой. Ограда была предназначена для того, чтобы не допустить распространения на Капскую колонию чумы рогатого скота. На глазах Кохрана газели налетели на забор и снесли его на протяжении пятисот ярдов. Бегущие впереди животные падали и были смяты и растоптаны. От разлагающихся трупов стояло такое зловоние, что пришлось нанять готтентотов, которые копали рвы и зарывали погибших антилоп. <<Я подобрал у забора две пары огромных рогов,~--- сказал мне Кохран.~--- Рога были такие крупные, что каждый хотел купить их у меня. Некоторые молодые полицейские получили за свои сувениры в барах Апингтона несколько бутылок легкого немецкого пива. Принадлежавшие мне рога я продал за шесть футов. Мне нужно было бы привезти их в Англию и сдать в музей. Эти рога были необычайных размеров>>.

В том же году Кохрану довелось увидеть, как тысячи газелей пронеслись по деревне Кенхардт. Эта миграция, видимо, была самая опустошительная. В животных стреляли из всех домов. Полиция подняла тревогу и раздавала фермерам патроны за полцены. Ущерб, нанесенный жителям деревни, был огромен. Он был бы еще больше, если бы поток животных вдруг не остановился и не повернул обратно~--- к пустыне Калахари. Говорили, что там выпал дождь и северный ветер донес до них за несколько сот миль желанный запах влажной земли и молодой травы.

Один фермер из округа Кальвиния показал мне плато. Оно полого спускалось к равнине, а с другой стороны неожиданно обрывалось пропастью. Фермер рассказал, что как-то бушмены увидели огромное стадо газелей, пасущихся на плато. Бушмены ловко подогнали стадо к краю пропасти и ранили стрелой одно из животных. Как они и ожидали, раненое животное в паническом страхе бросилось в пропасть, а вслед за ним последовали и тысячи других. Таким образом бушмены получили возможность устроить величайшее пиршество. Они разослали приглашения всем племенам, объедались и танцевали. Много лет спустя все еще можно было увидеть кости животных на дне пропасти.

Опытные натуралисты, по-видимому, проглядели миграцию газелей. Поэтому картину миграций можно воссоздать лишь по рассказам или по беглым заметкам фермеров, охотников и путешественников. Сохранились рисунки Джона Миллэ, но очень редко эти огромные стада попадали з объектив фотоаппарата. Существует множество ярких описаний миграций южноафриканских газелей, однако никто не пытался объяснить их причину.

Южноафриканские газели распространены в основном на территории бывшей Капской колонии. Их можно было встретить в Оранжевом Свободном Государстве и в Трансваале. Но самые большие стада бродили в Калахари и в районе Карру. Ван Рибек и его соотечественники, основавшие поселение в районе Капского мыса, никогда не встречали там газелей. Первое описание южноафриканской газели дал садовод-англичанин Фрэнсис Мэссон из Кью. Это было около двух столетий назад. Вместе с доктором Танбергом Мэссон отправился в <<Куд-боке вельд>>~--- <<холодную страну антилоп>>, названную по имени одной из разновидностей этих животных. Мэссон писал: <<Когда на этих животных охотятся, они не бегут, а совершают удивительные скачки или прыжки>>.

Позднее Мэссон еще раз побывал в Холодном вельде. Он сообщал, что с того времени, как этот район стали заселять европейцы, газелей там значительно поубавилось. И все же раз в семь-восемь лет многочисленные стада, по нескольку сот тысяч голов в каждом, перекочевывали из внутренних районов и наводняли вельд, не оставляя ни травинки, ни кустика. Крестьяне вынуждены были денно и нощно стеречь свои поля, так как там, где проходили антилопы, они все сметали на своем пути.

Мэссон отметил, что за кочующими газелями всегда следуют львы. <<Замечено, что львы предпочитают открытые широкие пространства>>,~--- писал он. (Позднее другой наблюдатель утверждал, что один лев был задавлен лавиной газелей насмерть, хотя он и оставил достаточно доказательств своего гнева.) Сам Мэссон признавался, что не видел стада, в котором было бы больше, двадцати газелей. Но ему встретились голландцы, которые видели в северных районах огромнейшие стада.

И вот Мэссон выдвигает первую из многочисленных теорий. По его мнению, газели перекочевывают к югу из-за засухи. С началом дождей они вновь возвращаются в глубинные районы.

К тому же выводу пришел примерно полвека спустя поэт Томас Прингл, когда увидел в районе, прилегающем к реке Литтл Фиш, необозримые стада газелей. "Мы подсчитали, что иногда перед нами было не меньше двадцати тысяч этих прекрасных животных,~--- писал

Прингл.~--- И они, вероятно, составляли только часть той огромной кочующей массы, которая во время долгих засух на северных равнинах наводняет Капскую колонию".

В очень засушливый 1921 год Ландрост Стокенстром из Граф-Рейнета писал секретарю по делам колоний: <<Газели появились из выжженной пустыни в таком количестве, что любая цифра может показаться преувеличенной. Очевидцы отмечали, что фермеры покидают свои хозяйства, которые были приведены этими животными в плачевное состояние, и там уже невозможно содержать скот>>.

Стокенстром писал об этом Принглу: <<Человек, который любуется бродящими по равнине газелями, едва ли может себе представить, что это украшение пустыни нередко становится таким же бедствием, как и саранча. Когда во время длительных засух несметные стада этих животных направляются к югу, они приносят фермерам невообразимые беды>>.

Когда приближается стадо газелей (говорит Стокенстром), фермеры окружают свои поля кучами сухого навоза~--- топлива Снеэвбергена~--- и поджигают его, надеясь, что животные из-за дыма свернут в сторону. Однако это редко помогает. Часто мчащиеся газели увлекают за собой отары овец, и их владельцам так и не удается разыскать свою собственность.

Стокенстром долго бился над этой загадкой и наконец смело заявил, что он разрешил проблему миграции. Газели, сказал он, размножаются в пустынях к югу от реки Оранжевой. Там их стада не тревожит никто, за исключением редкого охотника-бушмена. Со временем антилоп становится очень много. Наступает период засухи, источники пересыхают, земля трескается. Жажда гонит газелей из пустыни. Обратно они возвращаются лишь тогда, когда в этой пустынной равнине начинаются дожди.

Майор Корнуоллис Гаррис, охотник, попавший на территорию Западного Грикваленда, видел равнину, <<буквально белую от газелей, их были там мириады>>. Гаррис писал: <<Во время засухи не остается даже луж, поэтому газели, подобно саранче, этому бичу Египта, устремляются со своих родных равнин во внутренние районы>>.

Джон Фрейзер, сын священника голландской реформистской церкви в Бофорт-Уэст, оставил интересные записи о вторжении в эту деревню антилоп в 1849 году. Однажды в деревне появился бродячий торговец. Он был взволнован и рассказал, что огромные стада антилоп покидают вельд и направляются к деревне. Его слова не были восприняты всерьез. Но через несколько дней жителей Бофорт-Уэста разбудил топот животных. Южноафриканские газели заполонили все улицы и сады. С ними смешались антилопы гну, канны, зебры квагга и белолобые антилопы. Целых трое суток поток животных непрерывно лился через деревню. Когда он прекратился, вельд выглядел как после пожара.

Некоторые наблюдатели отмечают, что незадолго до начала миграции газели начинают испытывать беспокойство и сбиваются в небольшие группы. Постепенно стадо растет и наконец превращается в неудержимый поток. Чутье подсказывает животным направление. Для телят устраивается своего рода <<детская>> с одной стороны движущегося стада. Время от времени матери навещают своих детенышей и кормят их молоком. Вдруг, чего-то испугавшись, антилопы выгибают спины и несутся вперед двадцатифутовыми прыжками. Животные мчатся быстрее, чем лошади, и даже грациознее их. Иногда они делают короткие остановки для кормежки и снова устремляются вперед, оставляя за собой лишь взрытую землю. Газели прорываются через любые проволочные ограды, которыми с конца прошлого века стали обносить фермы. Они бесстрашно прокатываются между домами и пристройками, а у плотин безжалостно топчут своих собратьев и проходят по их телам.

В 1875 году Ливингстон наблюдал миграцию небольших стад антилоп и сделал свои выводы. Он обнаружил, что нередко животные уходят из северных районов в то время, когда травы и воды там в изобилии. <<Антилопы избегают мест, где им нелегко заметить приближение врага. Видимо, это и есть причина миграций, -заключает Ливингстон,~--- Быков часто пугает высокая трава. А южноафриканским газелям такое чувство страха свойственно в высшей степени, и, когда в Калахари подрастает трава, они начинают сильно тревожиться и вскоре устремляются на юг, где растительность более бедная. По дороге стадо все увеличивается и уже не может прокормиться на скудных пастбищах, так что антилопы вынуждены переправляться через реку Оранжевую. На этих землях, где подходящих кормов еще меньше, антилопы становятся настоящим бедствием для владельцев овечьих ферм>>.

Я нашел подтверждение теории Ливингстона в более поздних наблюдениях естествоиспытателя Г.~У.~Пенри-са, который изучал жизнь южноафриканских газелей в прибрежных районах Анголы. <<В определенные периоды они собираются в одно огромное стадо и направляются к какому-нибудь другому вельду, где снова разбиваются на более мелкие группы,~--- писал Пенрио~--- Южноафриканские газели никогда не встречаются там, где растет высокая трава. Видимо, они предпочитают открытые места. Когда на побережье выдался особенно дождливый год и появилась очень высокая трава, все газели устремились к югу, на более сухой вельд>>.

В то время когда в районе Намакваленда в последний раз появились стада газелей, поэт и писатель Уильям Чарльз Скулли занимал административный пост в Спрингбокфонтейне. Он тоже выдвинул свою гипотезу. Скулли считает, что причина миграции проста и очевидна, хотя с незапамятных времен была загадкой для охотников и естествоиспытателей. В стране бушменов дожди выпадают в летнее время, зимой же их там не бывает. К западу от Бушменленда над песчанистой равниной возвышается гранитный массив. <<Летом здесь осадков не выпадает, но в начале зимы юго-западный ветер приносит сюда проливные дожди, и примыкающие к горам песчаные равнины на несколько недель покрываются пышной, сочной растительностью,~--- продолжает Скулли.~--- Это происходит как раз в то время, когда газели телятся и когда, естественно, самки нуждаются в зеленой траве. Поэтому-то газели и мигрируют на запад, и эта миграция, на мой взгляд, древнего происхождения>>.

Скулли описывает самую потрясающую миграцию в 1892 году, когда лавина докатилась до южного побережья Атлантического океана. <<Антилопы, как правило, обходятся без воды,~--- отмечает автор.~--- Но изредка, может быть раз в десятилетие, их обуревает страшная жажда, и они как безумные мчатся вперед, пока не встретят воду. Не так давно миллионы этих животных перевалили через горный хребет и добрались до моря. Они бросались в волны, пили соленую воду и гибли. На тридцать миль вдоль берега протянулись гряды их трупов. Буры, раскинувшие лагерь на побережье, вынуждены были из-за зловония уйти далеко в глубь материка>>.

Некоторые фермеры считают, что передвижение антилоп объясняется болезнями, такими, как чесотка или чума. Есть сведения, что во время чумы 1896~-- 1897 годов южноафриканские газели ее избежали, хотя чесотка явно была обнаружена у некоторых пристреленных животных. Однако теория эта все же сомнительна, и ей противоречит тот факт, что в одни годы животные во время миграций бывают изнуренными, тогда как в другие они выглядят явно здоровыми и упитанными.

Во время миграции 1896 года, самой последней крупной миграции, С.~К.~Кроунрайт-Шрейнер сделал решительную попытку разгадать тайну. Он ехал по пятам животных и повсюду видел фермы, увешанные гирляндами билтонга. В одном лишь районе Приски было тогда убито много сотен тысяч животных и почти столько же ранено. Осиротевшие детеныши умирали тысячами. А миграция продолжалась. Газели шли миллионами.

Кроунрайт-Шрейнер не мог сделать выводы. Он стал изучать работы Дарвина и Ллойда Моргана о миграциях животных, проанализировал все высказывания ученых Южной Африки и наконец заявил:

<<Очевидно, они не располагают достаточными данными для того, чтобы прийти к какому-нибудь определенному решению. Да, у нас недостаточно отобранных, хорошо изученных и строго проверенных фактов, чтобы можно было сделать ясный вывод о сущности этих миграций. Будут ли у нас когда-нибудь такие факты? >>

Никто не зафиксировал точных маршрутов миграции газелей, так что эти важные данные для нас навсегда потеряны. Известно, что животные никогда не возвращались назад тем же самым путем, а описывали огромный прямоугольник или овал. Никто не знает, сколько времени продолжалась каждая миграция, хотя есть сведения, что газели возвращались на прежнее место через полгода, через год. Скорость передвижения газелей различна. За день они проходят примерно сто миль, но могут преодолеть и гораздо большие расстояния.

В прошлом веке фермеры Карру были твердо уверены, что существуют две разновидности южноафриканской газели~--- тощий trekbokke и толстый houbok (фунтов на пятнадцать тяжелее), которые держатся в одном и том же районе. Такой опытный наблюдатель, как Скулли, писал, что в районе Рихтерсвельда был застрелен houbok, который был почти вдвое крупнее газелей, обитающих в пустыне. Взрослый самец южноафриканской газели весит от семидесяти до восьмидесяти фунтов, и лишь в редких случаях вес его превышает девяносто фунтов. В Южной Африке водится только один вид газели, известный ученым как Antidorcas marsupialis marsupialis. А различия в весе объясняются просто возрастом и условиями жизни. Газели же, обитающие в Юго-Западной Африке, более крупные и принадлежат к другому виду.

Хотя фермеров и бурских переселенцев никогда не радовало нашествие газелей, они по крайней мере старались извлекать из этих нашествий пользу, чтобы хоть как-то возместить нанесенный ущерб. Цепь фургонов преграждала животным дорогу, и в ход пускались шомпольные ружья. Нередко одним выстрелом убивали сразу двух газелей.

Это была грандиозная охота. Мир еще не видел такой бойни. Каждая партия охотников расставляла капские телеги и фургоны в форме подковы. Мужчины и мальчики скакали на лошадях вдоль движущегося потока и стреляли в животных. Женщины снимали шкуру с убитых газелей и нарезали мясо для билтонга.

В прошлом веке на протяжении целых десятилетий шкуры газелей продавались по шести пенсов за штуку. (Тонкая кожа использовалась в переплетном деле.) Фунт билтонга стоил три пенса, и тощей считалась та газель, из которой нельзя было сделать восемь фунтов вяленого билтонга. В 1839 году Бэкхаус отмечал, что на рынке в Крэдоке целая свежая туша газели продавалась за тринадцать пенсов. Бывали времена, когда в деревнях Карру жирную тушу газели можно было купить за один шиллинг и шесть пенсов.

Действительно ли эти кочующие стада насчитывали миллионы животных? Некоторые ученые сомневаются, что газели могли когда-нибудь водиться в таком огромном количестве, которое ошеломляло первых путешественников. Однако все очевидцы выражали единодушное мнение на этот счет. Одно из лучших описаний оставил нам охотник прошлого века Гордон Камминг, очень колоритная фигура, бывший воспитанник Итонского колледжа, кавалерийский офицер, рыжебородый шотландец в юбке. Пять лет он путешествовал в фургоне и беспощадно бил зверей. Это происходило в ту пору, когда Южная Африка была еще раем охотников и никто, вероятно, не предполагал, что в один прекрасный день некоторые животные будут истреблены. Его <<ягдташ>> был гораздо внушительнее, чем у более искусных охотников последующих лет, например у Селуса.

Однажды ночью, часа за два до рассвета, Гордон Камминг лежал в своем фургоне и прислушивался к фырканью газелей. Он сообразил, что неподалеку от лагеря пасется огромное стадо. Когда он выглянул из фургона, перед ним было не просто стадо, а сплошная живая река.

Этот бесконечный поток вливался в ущелье и исчезал за горной грядой. <<Почти два часа стоял я на передке фургона, зачарованный удивительным зрелищем,~--- писал Гордон Камминг.~--- С трудом я убедил себя, что это была реальность, а не фантастическая картина из снов охотника. А в это время легионы животных плотной сплошной лентой продолжали шествие через горный проход. Наконец я схватил ружье, вскочил на коня и, врезавшись в самую гущу, открыл огонь. За мной последовали и другие всадники. ``Хватит'',~--- воскликнул я, когда четырнадцать животных упало на землю. Потом мы повернули обратно, чтобы спасти от назойливых стервятников разбросанную на пути добычу>>.

Гордон Камминг признавался, что не может сказать, сколько видел газелей. Но он без колебаний заявил, что <<их было несколько тысяч>>. Один местный бур сказал Гордону Каммингу: <<Этим утром вы видели лишь одну равнину с газелями. Я же целый день ехал от равнины к равнине, и везде, насколько хватало глаз, были газели, сгрудившиеся, как овцы в загоне>>.

Скулли затруднялся ответить, сколько он видел газелей во время миграции 1892 года. <<Когда имеешь дело с мириадами животных, цифры теряют всякий смысл,~--- заявил он.~--- Определить число антилоп в живой волне, которая катится через пустыню и, словно пена, разбивается о западный гранитный хребет, все равно что пытаться определить массу песчаной дюны в милю длиной по количеству песчинок>>.

 Т.~Б.~Дэви из Приски, описал свои впечатления о четырех крупных миграциях между 1887 и 1896 годами, свидетелем которых он был. <<Казалось, что вся местность двигалась, но не поспешно и стремительно, а спокойно и медленно, так же как при нашествии бескрылой саранчи>>,~--- писал он. Дэви видел непрерывный поток южноафриканских газелей на всем расстоянии от Приски до Драгоэндера (сорок семь миль), причем животные спокойно продвигались вперед и лишь чуть-чуть сторонились, чтобы не попасть под колеса фургона.

Одна семья на ферме Витвлей должна была охранять колодец, пулями и камнями отгоняя животных. Газели уже забили плотину, и теперь колодец оставался единственным источником воды. Но в конце концов жаждую-щие животные сломили защиту, и вскоре колодец был завален телами погибших и раненых газелей.

В том же году газели запрудили всю центральную улицу в Приске. Местный судья, сидя на ступеньках здания суда, подстрелил несколько отличных экземпляров. Приска всегда лежала на пути миграций.

Во время миграции 1888 года Дэви и его друг доктор Гиббонс пытались подсчитать количество животных. Когда море антилоп захлестнуло район Приски, эти двое находились на ферме Нельса Портье. Они стояли у крааля, в котором, по словам фермера, было полторы тысячи овец.

<<Итак,~--- сказал доктор Гиббонс,~--- если в краале полторы тысячи овец, то на одном акре их может поместиться около десяти тысяч, а перед нами десять тысяч акров земли. Значит, здесь по крайней мере сто миллионов антилоп. Сколько же их в таком случае на всей территории, которая простирается на многие мили вокруг и которая, куда ни глянь, вся забита этими животными?>>

Их нельзя было сосчитать. Не удивительно, что люди обычно говорят о мириадах животных.

В 1896 году Кроунрайт-Шрейнер и еще два фермера, привыкшие иметь дело с небольшими стадами, наблюдали в бинокль, как через огромную открытую равнину проходят стада газелей, и попытались их подсчитать. Они старательно подсчитывали животных квадрат за квадратом и определили, что на этой равнине было пол миллиона газелей. А стада занимали площадь сто сорок миль на пятнадцать. <<Когда говорят, что антилоп были миллионы, то это чистая правда>>,~--- заявил Кроунрайт-Шрейнер.

В книге об охотнике Селусе Милле пишет, что массовое истребление животных началось в семидесятые годы девятнадцатого века, когда в Южной Африке появились ружья, заряжающиеся с казенной части. Он встретил торговца, который вел подсчет проданных им шкур. Лишь в 1878-1880 годах этот торговец продал почти два миллиона шкур, по преимуществу южноафриканских газелей.

Да, в этих миграционных потоках были миллионы газелей, миллионы животных, вслед за которыми шли львы и леопарды, гиены и шакалы, а также стервятники, выклевывавшие глаза погибшим. Если человек бывал застигнут бегущими антилопами в узком проходе, это означало для него смерть. Во времена переселения буров один фермер потерял трех сыновей и пастуха-готтентота, которые были растоптаны движущейся по вельду массой животных.

Около семидесяти лет назад в Калахари был торговец Альберт Джексон. Не так давно я познакомился с ним в Порт-Элизабете, и он поделился со мной своими впечатлениями о миграции газелей. Его слова помогли мне живо представить всю картину. <<Во время миграции 1896 года я жил в вельде,~--- вспоминает Джексон.~--- Нередко я прикладывал ухо к земле и чувствовал, что даже ночью, когда животные отдыхали, земля дрожала, как при землетрясении>>.

Сейчас уже не встретишь многомиллионных стад южноафриканских газелей. И тем не менее этому животному, которое стало национальной эмблемой Южной Африки, еще не грозит уничтожение. В мае 1954 года стадо газелей, насчитывающее не менее пятнадцати тысяч голов, хлынуло широким потоком, как это бывало в прошлом веке, из пустыни Калахари в район Гордонии. От фермеров посыпались отчаянные жалобы, что животные снесли ограды и уничтожили пастбища. Судья округа и полицейский офицер обследовали на самолете район вторжения антилоп и решили, что нет необходимости отменять запрет на охоту, действовавший в том районе на протяжении трех лет. Фермеры получили разрешение стрелять в воздух, чтобы отпугивать газелей, но это должно было происходить под надзором полиции.

Мясо газелей пользуется большим спросом. В тех районах страны, где охота разрешена, фермеры тщательно оберегают стада газелей. Отстреливают лишь самцов и старых самок. И гость, нарушивший правила охоты, едва ли получит приглашение снова посетить эти места.

Теперь вы редко увидите, чтобы на одной ферме за день убивали по пятьдесят газелей. А ведь в девяностые годы прошлого века партия охотников истребляла тысячу~--- тысячу двести животных в день. Сейчас нет ни миграций антилоп, ни их варварского уничтожения. Но тем не менее тайна остается тайной.

\chapter{Африканские собаки}

Один мой друг, французский врач, который поведал мне некоторые тайны африканской медицины, привлек также мое внимание к одной из самых удивительных собак на свете. А я равнодушно прошел мимо нее, приняв за дворняжку, в одном поселке Бельгийского Конго, где наш пароходик заправлялся горючим.

---~Взгляните на басенжи,~--- сказал мне врач.~--- Вот вам тайна: родственник эскимосской собаки в конго лезском лесу.

Я внимательно присмотрелся к собаке с шерстью цвета кофе и величиной с терьера. У нее был очень забавный, растерянный вид. Над морщинистым лбом большие лисьи уши, острая морда, карие глаза и загнутый крендельком короткий хвост.

---~Одна из старейших пород на земле,~--- заметил мой друг.~--- Очень старая в Африке и новая в Европе.

Не так-то просто найти хоть одну такую во всей Англии или Франции.

С тех пор прошло более тридцати лет, но и теперь басенжи~--- <<немые>> басенжи~--- очень редко встречаются вне районов тропической Африки. Последний раз мне довелось увидеть собаку этой породы накануне второй мировой войны, когда я охотился в Анголе на реке Окаванго. У одного из охотников племени куангари, который помогал мне выследить антилопу, был чудесный каштаново-белый басенжи. Я не ожидал увидеть эту собаку так далеко от экватора, потому что никогда прежде не встречал ее в Южной Африке.

В дремучих лесах Конго и Анголы местные жители охотятся с басенжи на дичь. Собаки загоняют мелких антилоп в сети или же выгоняют их из зарослей на открытое место. Басенжи также отважно вступают в бой со свирепыми камышовыми крысами весом до двадцати фунтов, которых местные жители употребляют в пищу. Басенжи чуют дичь за восемьдесят ярдов. Кроме того, хозяин может быть уверен, что собака не спугнет ее лаем. Нередко охотники даже надевают на басенжи деревянные колокольчики или погремушки, чтобы знать, где она находится.

Эти собаки встречаются не так уж часто, и африканцы ценят их очень высоко. Охотник Бельгийского Конго отдаст дюжину отличных копий за натасканную басенжи. Иногда такая собака стоит дороже выкупа за жену.

Слово <<басенжи>> означает <<дикая>>. Ее называют также собакой конголезских лесов. Африканцы еще называют ее М'бва м'кубва, М'бва мамвиту, что значит <<прыгающий вверх и вниз>>, так как эти собаки высоко подпрыгивают в слоновой траве, чтобы осмотреться.

Конечно, не все басенжи безмолвны. Никто не знает, почему домашние собаки стали лаять. Ведь дикие собаки не лают, а скулят или подвывают, как шакалы, или же воют, как волки. Лай~--- признак общения с человеком. Есть басенжи, которые лают. Несколько лет назад одна собака, стоившая двести фунтов стерлингов на выставке в Лондоне, потеряла почти всякую цену, когда разразилась вдруг громким лаем совсем не к месту. Ее запретили использовать как производителя. Но если басенжи издает своеобразный жалобный звук, напоминающий тирольский йодль, или же мелодичное рычание, это не считается пороком.

Но меня басенжи привлекла не своим голосом, а благодаря той тайне, о которой говорил французский врач. <<Родственник эскимосской собаки в конголезских лесах>>,~--- сказал он. Возможно ли это? Я стал докапываться до истины, и слова француза подтвердились.

Басенжи относятся к группе шпицев, или померанских собак, которые считаются северной породой. Может быть, вы знаете, что самые древние четыре породы собак были, вероятно, пария, шпиц, борзая и мастиф. Никто не может с абсолютной уверенностью сказать, откуда произошла собака. Специалисты расходятся во мнениях, а история происхождения собаки уходит в весьма отдаленное прошлое. Некоторые утверждают, что люди каменного века приручили на территории Египта шакала, от которого вывели шпица. Другие заявляют, что все собаки имеют одного предка~--- полярного волка. Брайан Весей-Фитцджеральд, крупный английский специалист в этой области, высказал предположение, что прародитель домашней собаки~--- дикая собака. Вы можете услышать от африканцев, что их охотничьи собаки произошли от лисы, но в это трудно поверить. Собаки легко скрещиваются с волками и шакалами. Возможность же скрещивания собаки с лисой еще требует подтверждения. Гиена тоже далека от собаки.

Но что бы ни случилось за много тысяч лет до наших дней, какое бы скрещивание ни произошло, шпиц определенно унаследовал признаки шакала. Особенно бросается в глаза это сходство тогда, когда шпиц скулит или подвывает. Анатомы обнаружили много общего в строении скелета шакала и шпица. Я считаю, что разрабатывать подобные гипотезы о событиях давно минувших дней слишком смело, а поэтому не собираюсь выдвигать своей. Однако совершенно очевидно, что собаки типа шпицев были завезены к варварам Европы из Египта мореплавателями каменного века. Самым крупным авторитетом для меня является немецкий археолог Обер-майер, исследовавший множество пещер с кухонными отбросами первобытного человека, а также скопировавший наскальные рисунки. Обермайер доказал, что в Европе существовала египетская шакалообразная собака, предок шпица (известного также как торфяная померанская собака) и терьера. К тому же семейству относится эскимосская лайка, сибирская лайка со светлой шерстью, скандинавская лайка, малютка шипперк с голландских судов и самое занятное существо~--- чау-чау с ее черной или рыжей шерстью и необыкновенным синеватым языком. Если внимательно присмотреться ко всем этим собакам, то можно сразу увидеть, что у них общий признак~--- загнутый на спину хвост.

Еще одно доказательство в пользу египетского происхождения шпица и его потомка басенжи было найдено в надгробных памятниках Древнего Египта. Египетские художники еще шесть тысяч лет назад рисовали собак, очень похожих на шпицев. В гробнице Тутанха-мона были найдены рисунки, где изображены басенжи в ошейниках с драгоценностями. Их ведут на поводке карлики. Итак, эти похожие на пуделя собаки были, вероятно, самыми первыми охотничьими собаками Древнего Египта, прирученными задолго до борзых и других пород. При раскопках пирамид пятой династии египетских фараонов была найдена забальзамированная, пропитанная благовониями басенжи, завернутая в тонкую льняную ткань. Видно, эта собака принадлежала самому фараону.

Вполне вероятно, что первые басенжи были приручены где-нибудь далеко на юге, в верховьях Нила, и привезены в дар фараонам. Они уже совершенно исчезли в Египте, но живут и здравствуют в тропиках. Теперь они чувствуют себя как дома только в дремучих экваториальных лесах. Здесь их ценят, как друзей (и как деликатес к столу), многие африканские племена, в том числе пигмеи.

Европа впервые узнала басенжи лишь в конце прошлого века, когда один английский путешественник привез двух собак в Англию. Они демонстрировались на выставке в Крафте в 1895 году, но вскоре погибли от чумы. Другая пара, которую несколько десятилетий спустя привезла из Хартума леди Элен Наттинг, тоже погибла.

Оливия Бернс приобрела в Бельгийском Конго знаменитого кобеля Бокото и двух сук и выставила их в Крафте в 1937 году, где они возбудили такой интерес, что пришлось вызывать специальный наряд полиции для наведения порядка. Любители собак приходили в восторг, когда басенжи мыли свои лапы, как кошки, или же, как котята, принимались гоняться за своей тенью. Если басенжи в игривом настроении, они очаровательно закрывают свой нос передними лапами. Вообще эти собаки хороши во всех отношениях. Они игривы, чистоплотны, послушны, добры и очень любят детей.

У африканских басенжи короткая, гладкая, шелковистая шерсть с красноватым оттенком, которая блестит, как медь на солнце. Они легко акклиматизируются в Англии и Америке (как их предшественники в Арктике) и вскоре обрастают зимней шерстью.

Трудно, например, поверить, что басенжи и комнатная померанская собачка~--- близкие родственники. Но собаки отличаются друг от друга, так же как и голуби. Селекционное выведение до того их изменило, что они порой не похожи даже на своего прародителя, жившего всего пятьдесят лет назад. Однако же, хотя другие породы и изменились, басенжи остается все такой же.

Многие века басенжи жили и процветали в непроходимых конголезских лесах, хотя слава Египта давно уже закатилась. А когда они были открыты вновь, то все еще оставались собаками времен фараонов.

Я благодарен наблюдательному французскому врачу, который так хорошо знал Африку и ее обитателей и который помог мне увидеть басенжи среди ее родных лесов.

Люди любят собак и тайны. Только этим можно объяснить, почему родезийский риджбек стал самой популярной собакой в Южной Африке. Ее отличительный признак~--- полоса вдоль спины с шерстью, торчащей вперед. Когда бы я ни смотрел на нее, я всегда мысленно пытаюсь проникнуть в тайны времени, чтобы выяснить происхождение этой изумительной желтовато-коричневой собаки.

Многие владельцы риджбеков считают, что предки этих собак в давние времена прибыли с острова Фу-Куок, расположенного в Сиамском заливе. Нигде в мире, кроме Южной Африки и острова Фу-Куок, нет собак с таким щетинистым ремнем на спине. Более того, южноафриканский риджбек поразительно напоминает фу-куока по сложению, росту, весу и окраске. Правда, можно заметить, что ремень у фу-куока более отчетлив, но это деталь. У этих собак общий предок, а ремень свидетельствует о их прямом кровном родстве, хотя они и разделены многими тысячами миль морского пространства.

На каждого непосвященного, кто вторгается в священную обитель породистых собак и пытается опровергнуть твердые устои, обязательно набросится рычащая свора собаководов, норовящая схватить его за ногу. Я иду на этот риск и твердо заявляю, что наш драгоценный риджбек родом не из Индокитая. Все специалисты утверждают обратное. И все они ошибаются. Это романтическая история, далекая история с сюрпризами и неожиданными поворотами.

По мнению специалистов, жители Фу-Куока вывели своих собак для охоты и продавали их на материк в Индокитай. Финикийские купцы или другие ранние мореплаватели завезли этих собак с Фу-Куока в Южную Африку. Таким образом и появился ремень у многих собак готтентотов.

Капитан Т.~К.~Хоулей и Д.~К.~Драй в брошюре, изданной в 1949 году Трансваальско-Родезийским риджбек-клубом, пишут, что готтентоты, внешне похожие на азиатов, прибыли в Африку из Азии сухопутным путем и привезли с собой и фукуокских собак. Будь это так, это было бы весьма знаменательное путешествие как для людей, так и для собак. Но такого путешествия на самом деле не было.

Владельцы риджбеков считают своих собак подлинно южноафриканскими, единственной местной чистокровной породой. Я надеюсь доказать, что риджбеки даже еще более южноафриканская порода, чем это могут представить себе их гордые владельцы. Я беру на себя смелость заявить, что риджбек не имеет ни капли крови индокитайских собак.

Прежде всего, что такое риджбек? Эта порода настолько молода, что ее отличительные признаки еще не вполне определились. И если мое описание не совпадает с мнением специалистов, меня, конечно, разорвут на части (что поделаешь, собачий мир ревнив!). Однако один владелец, компетенции которого я полностью доверяю, говорил мне, что самым важным признаком породы является загадочный ремень на спине и собака без ярко выраженного ремня не может быть отнесена к этой породе. Ремень должен симметрично суживаться и иметь две одинаковые <<короны>>, одну против другой.

\begin{figure}[ht!]
\centering
\includegraphics[width=90mm]{000014.jpg}
\caption{Африканская собака риджбек}
\label{overflow}
\end{figure}


Кроме того, он должен начинаться сразу за холкой и доходить до ложбинки крупа.

Несколько лет тому назад я осмелился дать описание этой породы, настаивая на том, что нос у идеального риджбека должен быть черный. За это я получил нагоняй от владельца одного из шести лучших в Южной Африке риджбеков, по экстерьеру почти чемпиона, но имевшего коричневый нос. Владелец заявил мне, что главное в чистокровном риджбеке~--- это сила и активность, большая выносливость и быстрота бега. Южноафриканская полиция несколько лет назад стала применять риджбеков в розыскной службе, и вполне возможно, что эта порода сумеет вытеснить доберманов, пока что лучших ищеек. В вельде риджбеки преследуют зверя по следу со скоростью пятнадцать миль в час. К тому же они отличные сторожевые собаки.

У риджбека довольно длинная голова с плоским черепом и мощной мордой. Шерсть короткая, предпочтительно пшенично-желтого цвета, хотя допускается коричневый и желтовато-коричневый с белыми пятнами. Хорошая выставочная собака должна иметь глубокую грудь и сильные лапы.

А теперь посмотрим, как выглядел риджбек до выставок, когда он был просто дворняжкой. Португальцы, высадившись в Южной Африке, увидели на побережье собак. Среди них, вероятно, были и риджбеки. Может быть, вы помните сообщение Васко да Гамы: <<У готтентотов залива Святой Елены много собак, которые напоминают португальских и лают так же, как они>>. Тил, историк, писавший в самом начале нашего века, за много лет до того, как риджбек вошел в моду, отмечал: <<Главной собственностью готтентотов были овцы и рогатый скот. Кроме них единственным домашним животным была собака. Это некрасивое существо по виду напоминает шакала, а шерсть по середине его спины направлена вперед. Но это очень преданная и полезная собака>>.

Установлено, что готтентоты переселились в Южную Африку в четырнадцатом веке и, по всей вероятности, привели с собой риджбеков. Можно предположить, что такие охотники, как бушмены, тоже не могли не иметь собак. Бушмены пришли на юг за несколько веков до готтентотов, и в их полной опасностей жизни собаки могли бы оказаться неоценимыми помощниками. Пещерные люди многих стран приручили волка и лисицу, шакала и койота. Это было величайшее открытие человечества, так же как каменные орудия и огонь. Первобытные люди ловили щенят диких собак и обучали их охранять свои семьи, предупреждать об опасности и преследовать раненую дичь. Возможно, существовали какие-то неизвестные нам виды диких собак, ныне вымершие, от которых и произошли наши домашние собаки. Или же прав Дарвин, утверждавший, что родоначальником наших собак был волк. Там, где еще существуют первобытные охотники, их собаки всегда очень похожи на диких собак этих областей. Даже в цивилизованных странах часто остается такое сходство. Собаки Египта были явно похожи на египетского волка, а многие собаки бушменов Южной Африки~--- родственники черноспинных шакалов.

И все же я считаю, что бушменам не удалось приручить собаку и что они получили ее от готтентотов. Кости собак находят в древнейших пещерах Европы. Пещерные художники Франции и Испании изобразили этого первого друга человека во многих охотничьих сценах. Пещеры бушменов в Южной Африке дали много необычных реликвий, но, как ни странно, останки собак или их изображения там ни разу не были найдены (если не считать находок, относящихся к сравнительно молодым средним векам).

Никогда не узнать, где и когда у риджбеков появился щетинистый ремень. Самой известной собакой Египта была борзая, и сейчас еще у риджбека осталось что-то от борзой. Но нет ни малейшего доказательства, что в Египте когда-то существовал хоть какой-нибудь вид собаки с подобным ремнем на спине. Этот характерный ремень мог появиться и во время великого переселения, и после того, как племена готтентотов осели в Юго-Западной и Южной Африке.

По-видимому, готтентотская собака (так раньше назывался риджбек) не вызывала никакого интереса у европейских поселенцев в Южной Африке до середины прошлого века, когда фермеры в районе Свеллендама занялись выведением новой породы собак для охоты в горах. Это была помесь бурхонда (дворняжки, как я определил по словарю африкаанс) и готтентотской собаки с незначительным добавлением крови ирландского терьера. Эти собаки имели квадратную челюсть и отличались бесстрашием. Теперь эта порода исчезла.

Одним из первых миссионеров в нынешней Родезии был Чарльз Хелм, который прибыл туда в фургоне в 1875 году из Свеллендама с женой и дочерью. Впоследствии его дочь (Джесси Лавмор) рассказывала, что какой-то доброжелатель подарил в Свеллендаме ее отцу двух риджбеков. Эти-то собаки Хелма и считаются прародителями современных родезийских риджбеков. Друг Селуса и сам охотник на крупную дичь Корнелис ван Ройен взял у Хелма этих собак и вырастил потом целую свору для охоты. Ван Ройен прибыл в Матабелеленд вместе с отцом и в четырнадцать лет стал охотником на слонов, добыв в первый же сезон восемь животных. <<Умный, приятный человек, хорошо говоривший по-английски,~--- характеризовали ван Ройена.~--- Большой поклонник Селуса, хотя и сам замечательный охотник>>. Вот каким был этот человек, основавший породу родезийских риджбеков.

Первое время с риджбеками охотились на львов. Стоило посмотреть, как бесстрашная свора этих собак преследовала льва и в конце концов одолевала властелина вельда. Сейчас, однако, для охоты на львов используется другая порода. Это собаки величиной с крупную немецкую овчарку и с пышной гривой, как у льва. Щетинистый ремень у них почти незаметен, или же его нет совсем.

Первые европейские поселенцы в Родезии считали риджбека очень полезной собакой. Известно, что в конце прошлого века один горный подрядчик привез сюда из Сереса еще двух риджбеков. В 1896 году в Родезию переселился Й.~Н.~Р.~Лабушань. Через три года он отправился в Бейру, где на борту немецкого корабля купил за двадцать фунтов огромную черную собаку по кличке Форман. У этой собаки неизвестного происхождения был на спине щетинистый ремень. Затем Лабушань купил рыжего пойнтера (суку) и от этих собак вывел риджбеков, распространившихся в районе Чипинги. Однако еще лет тридцать в Родезии смотрели на всех этих собак как на бесполезных дворняжек. И лишь на выставке в Солсбери в 1927 году риджбеки были заявлены как самостоятельная порода.

Не удивительно, что родезийцы гордятся своими рид-жбеками, поскольку они выросли в суровых условиях. Эти собаки охраняли лагерь первых переселенцев, отгоняли от него гиен, шакалов и даже львов. Немало собак, осмелившихся отойти далеко от костра, становилось добычей леопардов. Эти огромные кошки любят полакомиться собачьим мясом.

Риджбеки преследовали слонов, носорогов, буйволов. Самые храбрые собаки других пород, впервые учуяв льва, в страхе отступали. Но риджбеки держались крепко, они набрасывались на льва и отвлекали его внимание от своего хозяина.

Словом, отважный риджбек вернулся в Южную Африку из-за реки Лимпопо в новом облике. Для чистокровной собаки установлен рост в 26 дюймов и вес до 75 фунтов. В 1945 году в Трансваале был создан клуб, призванный сохранить чистоту риджбека в ЮАР. За последнее время риджбек вытеснил колли, кокер-спаниеля, бульдога и другие распространенные в Южной Африке породы и с легкостью отстаивает свои позиции как типичная и самая любимая южноафриканская собака. Рост популярности риджбека особенно примечателен, если учесть, что всего лишь в 1920 году две собаки этой породы демонстрировались в зоопарке Претории как диковинка.

В Англии риджбеков впервые увидели в 1928 году, когда миссис Фолджемб привезла двух собак этой породы. Они вызвали огромный интерес на выставке в Кристалл-палас. Но и до сих пор риджбеки редко бывают на выставках в Англии. Однажды английской королеве была подарена четырехмесячная сучка Банши, а во время своего визита в Родезию в 1947 году принцесса Елизавета получила щенка Холи. Это была великолепная пара родезийских риджбеков темно-пшеничного цвета.

Канадцы завезли к себе риджбеков для охоты на пуму, или <<горного льва>>. В Соединенных Штатах сейчас сотни риджбеков. В 1955 году риджбек был занесен в родословную книгу Американского клуба любителей собак. Это была первая порода за предыдущие десять лет и сто двенадцатая вообще, которая удостоилась такой чести. Некоторым породам потребовались тысячелетия, чтобы их ввели в избранное собачье общество. Риджбек достиг элиты за несколько десятков лет.

А между тем некоторые специалисты заявляют, что старая готтентотская собака с ремнем на спине выродилась. Сомневаюсь, чтобы она вымерла полностью, поскольку в 1936 году видел нескольких собак в резервации бушменов в Калахари. Мне их показал мой старый друг, охотник и проводник, покойный Дональд Бэйн. <<Бушмены никогда не продают своих собак,~--- сказал Бэйн.~--- Бушменский риджбек становится редкостью. Я думаю, что это лучшая охотничья собака в мире. Она может настигнуть шакала через двадцать ярдов, а лает лишь в тех случаях, если к лагерю близко подходит лев. Когда, по бушменскому обычаю, стариков оставляют умирать в пустыне, собаки остаются с ними и охраняют их до конца>>.

А теперь вернемся к фу-куоку на другом конце земного шара. Что это за собака? Я видел лишь фотографии фу-куока, так как это очень редкая собака и, говорят, вымирающая. Маркиз Бартелеми, получивший в свое время от французского правительства в концессию остров Фу-Куок, как-то послал трех собак этой породы в Парижский зоопарк. Но любители собак в Англии и Америке впервые прочитали подробное описание фу-куока незадолго до второй мировой войны. Клиффорд Хаббард, автор солидных исследований о собаках и их происхождении, объявил тогда фу-куока прародителем риджбека.

У вас тоже может легко сложиться такое впечатление. Капитан Р.~Д.~С.~Гуоткин, один из крупнейших южноафриканских специалистов по риджбекам, писал в 1933 году, что фу-куок с его длинной головой, красноватыми глазами, стоячими ушами, рыжевато-коричневой шерстью, более темной на спине, и подтянутым животом поразительно напоминает шакала. И такой щетинистый ремень на спине есть еще лишь у готтентотской собаки.

Гуоткин заявил, что некоторые породы восточных собак произошли, видимо, от египетского шакала. Эти породы дали китайскую собаку чау-чау (китайцы ее едят) и других родственных ей собак. По мнению Гуоткина, щетинистый ремень появился у одной из этих восточных пород. И он утверждает, что фу-куок завезен в Южную Африку, поскольку азиатские народы были мореплавателями, а готтентоты нет. В доказательство он приводит не очень убедительный факт, что на африканский берег у Порт-Элизабета была выброшена малайская лодка.

В такую же ловушку попали и другие специалисты. Хаббард, например, считает, что эту собаку в Африку привезли финикийцы, причем привезли ее морем, иначе риджбеки были бы и в других районах на всем пути их следования через материк. Другим возможным предком риджбека Хаббард считает ищеек, которые завозились в Южную Африку для преследования бежавших рабов.

Я нигде не могу найти доказательств древности породы фу-куок. И почему африканские <<охотники на львов>> должны быть выходцами с острова в Сиамском заливе? Да этого и не было. А финикийцы, возможно, плавали вокруг Африки за шестьсот лет до нашей эры, но ни у одного историка нет свидетельства, что финикийцы сначала совершили путешествие в Сиам, а потом в Южную Африку.

Арабы перевозили рабов на своих одномачтовых парусниках из Восточной Африки в Китай еще за тысячу лет до наших дней. Возможно, что именно арабский или португальский мореплаватель или же голландский шкипер всего лишь несколько веков тому назад и высадил риджбеков на остров Фу-Куок. Несомненно, что эта порода совершила путешествие с запада на восток. В Африке риджбеки всегда были в большом количестве, тогда как на острове Фу-Куок, по всем свидетельствам, их всегда было мало. Специалисты, которые отдают предпочтение азиатскому происхождению рид-жбека, никогда не задумывались над этим фактом. Они не были историками и хотели заставить хвост вильнуть собакой.

На острове Фу-Куок риджбек сохранил почти все свои африканские признаки. Должно быть, он смешался с чау-чау, поскольку у фу-куока черноватый язык, а иногда и черная пасть. Во всем же остальном это старый африканский чистопородный риджбек, который мог сохраниться лишь в изоляции.

Называйте его как угодно~--- рифругхонд или прон-круг, лееухонд или салругхонд, готтентотская собака или родезийский риджбек. Движения его грациозны, он всегда ласков с детьми, послушен и предан своему хозяину. Риджбек не восточная, а подлинно южноафриканская собака.

А как же быть с загадочным щетинистым ремнем на спине, отличительным признаком породы? Возможно, это наследие очень далеких времен. И мы никогда уж не сможем определить, от какого животного перешел к риджбеку этот признак.

\chapter{Тайны диких животных}

Надеюсь, что с жирафами Африка распростится не скоро. До сих пор от Сахары до Бечуаналенда и Юго-Западной Африки они бродят целыми стадами. Эти кроткие создания, самые высокие из всех млекопитающих, тоже загадка для нас. Очень бы хотелось встретить хоть одного натуралиста, который смог бы мне растолковать, откуда у жирафы такая длинная шея.

Пытаясь решить этот вопрос, Чарлз Дарвин совершил одну из своих наиболее серьезных ошибок. Он указывает, что все тело жирафы отлично приспособлено для ощипывания молодых побегов на высоких деревьях. Выживали только те жирафы, которые в засуху могли добраться до листьев на дюйм-два выше других. Скрещиваясь между собой, животные с более длинными шеями передавали потомству те же самые признаки~--- длинные ноги и шею. А жирафы с короткой шеей вымирали.

Более внимательное изучение убедило бы Дарвина прежде всего в том, что жирафа по существу не пустынное животное и что она вовсе не обязана своим существованием тому, что может объедать высокие акации. Там, где есть акации, обязательно есть трава и низкий кустарник. Некоторые виды кустарника так засухоустойчивы, что даже во время очень сильной засухи на них остаются листья, тогда как акации их теряют.

Жирафы отлично себя чувствуют и размножаются в зоопарках. Несомненно, Дарвин удивился бы, узнав, что статистика (которой в его время не было) доказала, что пара жираф с особенно длинными шеями вовсе не обязательно дает потомство с шеями длиннее обычных. Жирафы <<разводятся не по правилам>>. Таким образом, гипотеза Дарвина отпадает, как и многие другие возникшие за те сто лет, которые прошли со времени опубликования дарвиновской теории эволюции.

Французский зоолог Ламарк также изучал жирафу. Когда Дарвин начал свою карьеру, Ламарк был уже стариком. Но до сих пор некоторые ученые предпочитают ламаркизм дарвинизму. Ламарк утверждает, что в случае необходимости у животных могут появиться дополнительные органы. В связи с тем что жирафа постоянно тянулась за пищей вверх, у нее выросли длинные передние ноги и, чтобы сохранить пропорцию в строении, удлинилась и шея. Понаблюдайте за жирафой, когда она нагибается, чтобы напиться или пощипать травы, и вы увидите, что ее шея не совсем пропорциональна ногам. Лишь широко расставив передние ноги, жирафа может дотянуться до земли. В этом вопросе Ламарк подошел к истине значительно ближе, чем Дарвин.

\begin{figure}[ht!]
\centering
\includegraphics[width=90mm]{000015.jpg}
\caption{Жирафы}
\label{overflow}
\end{figure}


Однако и Дарвин, и Ламарк основывали свои гипотезы на ошибочной предпосылке, что жирафа~--- обитатель пустынь. Так что тайна длинной шеи жирафы все еще не разгадана. Некоторые современные натуралисты высказали предположение, что жирафа становилась все выше и выше из-за того, что ей все время приходилось высматривать льва, своего извечного врага. Другие ученые допускают, что у жирафы могли появиться длинные ноги из-за этой постоянной угрозы, но они ей были нужны для того, чтобы убегать от опасности. Я не принимаю этого довода, потому что при всей своей длинноногости жирафа может бежать со скоростью чуть выше тридцати миль в час, тогда как скорость льва достигает пятидесяти миль.

Северные жирафы выше южных. Самая высокая жирафа, ростом в девятнадцать футов, была встречена в Кении. Среди своеобразных правил на различных африканских железных дорогах есть пункт, гласящий: <<Не подлежат транспортировке жирафы ростом более тринадцати футов>>. И это вполне понятно. Не одна пойманная жирафа, которую перевозили на поезде в далекий зоопарк, нашла смерть в туннеле или под мостом, не вовремя подняв свою длинную шею. Именно так погибла жирафа у самого Сесиля Джона Родса, чудесная родезийская жирафа, которая украсила бы его частный зоопарк в Гроте-Шуре. Но люди, перевозившие ее по железной дороге из Родезии, совершенно забыли о туннеле на реке Хекс, так что жирафа оказалась не в зоопарке, а в Южноафриканском музее.

Чтобы избежать чуть ли не ежедневных происшествий, пришлось поднять телеграфные столбы вдоль железнодорожной линии Кения~--- Уганда, где пассажиры особенно часто видят жираф. Поезда все еще давят этих животных, которых привлекает на рельсы яркий свет фар и ослепляет их.

Как правило, жирафы~--- стадные животные. Если по соседству нет сородичей, они пасутся вместе с антилопами и зебрами. Говорят, самка может кормить и своего и чужого жирафенка. Хотя это и было принято с усмешкой, но однажды видели жирафу в окружении де-вяти детенышей, и на расстоянии целой мили от этой очаровательной картины не было больше ни одной взрослой жирафы.

Во всех старых книгах охотников и натуралистов сообщается, что жирафы глухи. Некоторые говорят, что у жирафы нет и голоса. Однако в последние годы это всеобщее убеждение было поколеблено. Мне удалось найти запись полицейского о поимке молодой жирафы близ Нуругаса, в округе Гротфонтейн. Когда двое полицейских повалили жирафу, она замычала, как теленок, и стала отбиваться передними ногами.

Есть еще более убедительные сведения, полученные из других районов Африки. Логан Хук из Наньюки (Кения) вел на своей ферме наблюдения над сорока пятью жирафами. Он заявил: <<Бесспорно, что с рождения и до девяти месяцев жирафы мычат, как телята, а затем становятся безмолвными>>. Художник и натуралист К.~Т.~Аслей Маберли слышал в северо-восточном районе Трансвааля, как около его автомобиля одна или две жирафы издавали странные звуки. Другие очевидцы говорят, что жирафы издают звуки, похожие на мычание, вскрик, свист, раскатистое хрюканье, блеяние и пыхтение.

Еще один опытный охотник, Кальман Киттенбергер, долгие годы занимавшийся в Восточной Африке сбором зоологических материалов по поручению Венгерского национального музея, тоже считал жирафу глухой. Он говорил, что жирафа, даже перенося боль, не издает ни звука. Однако один из африканцев, которых он брал с собой на охоту, уверял его, что молодые голодные жирафы издают иногда звуки, похожие на блеяние овцы.

Видимо, не все жирафы издают звуки, иначе бы люди не были так твердо убеждены, что у них нет голоса. Селус сказал однажды, что никогда не слышал, что-бы жирафа издавала звуки. Известный в Уганде инспектор заповедника полковник К. Питмэн придерживался такого же мнения. <<Никто, вероятно, не станет спорить, что жирафа, как правило, безголоса,~--- писал Питмэн.~--- Я ни разу не слышал, чтобы старая или молодая жирафа издала хоть звук>>.

Все специалисты неверно судили о голосе жирафы. Дарвин заблуждался относительно ее шеи, но он был прав, классифицируя жирафу как <<живое ископаемое>>. Жирафа~--- фантастический пережиток того животного мира, который существовал десять миллионов лет назад, когда у жирафы была короткая шея и разветвленные рога. Единственный из сохранившихся родственников жирафы~--- окапи, таинственный обитатель конголезских лесов, которого европейцы узнали только в начале нашего века.

Куда бы вы ни поехали в Африке, особенно в отдаленные районы, вы всюду услышите рассказы о странных сказочных животных, или <<ископаемых>>, вернувшихся к жизни. Естественно, к подобным сообщениям зоологи относятся скептически и часто воспринимают их с улыбкой. Они заявляют, что окапи~--- последнее загадочное животное Африки и больше неизвестных крупных животных не осталось.

Находка нового подвида даже мелкого млекопитающего рассматривается среди натуралистов как триумф. У меня есть все основания вспомнить об одном таком открытии во время экспедиции в отдаленный район Юго-Западной Африки. Один из сотрудников музея открыл тогда нового мелкого грызуна и назвал его моим именем!

Но, подогреваемые волнующими сообщениями, поиски новых удивительных животных продолжаются. Иногда рассказам, услышанным от вполне надежных людей, можно найти объяснение. Чаще же тайна остается неразгаданной. Легенда о единороге~--- одна из величайших тайн Южной Африки первой половины прошлого века. Несколько известных путешественников пытались в нее проникнуть. Я собрал и изучил все факты и выдвигаю свою гипотезу.

Впервые единорог упоминается в <<Естественной истории>>, написанной Плинием Старшим тысячу девятьсот лет назад. Единорог был первым загадочным животным, и слава о нем дошла почти до наших дней. (Кто знает, может быть, он вновь появится на сцене?) О единороге упоминает Ветхий завет: <<Захочет ли единорог служить тебе или же будет пребывать в твоем стойле? Сможешь ли ты запрячь единорога и направить его в борозду? Или же он будет бороздить долины вслед за тобою?>>

Очевидно, упоминание о единороге связано с ошибкой в переводе, грубой ошибкой, допущенной семьюдесятью толковниками, которые некогда перевели Ветхий завет на греческий язык с древнееврейского и ошибочно назвали единорогом дикого быка. Более поздние переводчики увековечили эту ошибку, так что для миллионов людей единорог стал реальным животным.

Существует ли где-нибудь свирепое однорогое животное, похожее на единорога? Конечно, я исключаю носорога, поскольку даже Марко Поло и древние путешественники знали, что носорог~--- это не легендарный единорог. И все же легенда о единороге не может не иметь известного основания. Во многих уголках земли были свои единороги. И сообщения о южноафриканском единороге увлекли читателей.

В начале прошлого века английский путешественник и дипломат Джон Барроу организовал специальную экспедицию для поисков единорога в районе Граф-Рейнет. На него произвел сильное впечатление рассказ Адриана Версфельда, фермера из Камдебу, о не известном ни одному колонисту животном, убитом в Бамбусберге. Животное это похоже на кваггу, но крупнее ее и имеет желтоватую окраску с черными полосами. Посреди его лба десятидюймовый нарост твердого костяного вещества, покрытого шерстью.

Фермеры рассказали Барроу о пещере с бушменской живописью в Бамбусберге, где был изображен единорог. Барроу нашел эту пещеру, расчистил кустарник, так чтобы свет падал на рисунок однорогого зверя. К сожалению, видна была лишь часть единорога, потому что поверх него был нарисован слон.

Немецкий врач Лихтенштейн, путешествовавший по Капской провинции в одно время с Барроу, также упоминает о единороге. Свеллендамский фермер Ломбард сообщил ему, что в 1790 году он принимал участие в поисках людей с погибшего корабля <<Гросвенор>>. В то время он услышал от африканцев о единороге. Ломбард верил в существование единорога, а Лихтенштейн нет. Губернатор Янссенс хотел во что бы то ни стало раскрыть эту тайну. Он обещал новый крепкий фургон и упряжку быков каждому, кто доставит целую шкуру с рогом и черепом единорога в Кейптаун. Но претендентов на награду не нашлось.

Миссионер Вандеркемп слышал о единороге на <<северо-востоке Кафирленда>>. Люди из племени имбо описали ему единорога как очень страшное и очень свирепое животное. Оно опрокидывает краали и разрушает хижины. Имбо отлично знают носорога и уверяли Вандеркемпа, что единорог совершенно другой зверь.

В 1840 году капитан Грэйсон передал слухи о на-тальском единороге. Какой-то охотник нарисовал единорога, и все местные жители подтвердили, что такое животное существует. <<Фана ихаше, мпондо эние>>,~--- сказали они. <<Как лошадь, с одним рогом>>. Примерно сто лет назад зулусы сообщили поселенцу из Наталя Осборну, что на каком-то болоте в Драконовых горах они спугнули шесть темно-бурых животных величиной с белолобую антилопу. Они были очень свирепы и убили несколько зулусов. У каждого животного на лбу был длинный прямой рог.

Таковы свидетельства. Я не собираюсь отвергать их как простую выдумку. По-моему, многие сообщения можно объяснить капризами природы и случайностью. Увидев антилопу бейзу, потерявшую один рог, ее можно издали принять за легендарного единорога. Несколько лет назад в Юго-Западной Африке была убита бей-за с переплетенными рогами. Кто возьмется утверждать, что среди миллионов рождавшихся в Африке антилоп не было небольшого процента однорогих уродов?

Если этого мало, сошлюсь на старинные и современные свидетельства о существовании среди некоторых африканских племен обычая выводить однорогий скот. Плиний и другие древние авторы говорили об однорогих коровах и быках в стране муров и в Эфиопии. Я слышал, что нилотское племя динка и некоторые племена в Южной Африке до сих пор проделывают простую операцию: у теленка вырезают зачатки обоих рогов, выравнивают их и вместе вставляют в надрез на лобных костях. Дальше они растут уже как один рог.

Такие быки обычно становятся вожаками стада. Все это помогает нам объяснить легенду, которая не могла бы просуществовать целых девятнадцать веков, не будь в природе хоть какого-то единорога.

Динозавры и птеродактили вымерли в Африке примерно сто миллионов лет назад. Я никак не могу поверить в существование таких <<животных-ископаемых>>. Но я не могу и объяснить некоторые необычные рассказы о достоисторических чудовищах и гигантских летающих ящерах, о которых сообщали в последние годы заслуживающие доверия лица.

Мелкие летающие змейки водятся в Азии, но не в Африке. Змеи эти скорее планируют, чем летают. Они напрягают тело, втягивают живот и бросаются с дерева. Приземляются они мягко. И все же это не полет, как у птеродактиля. А теперь послушайте рассказ о летающей змее из Киррис-Уэста~--- фермы, расположенной примерно в шестидесяти милях к востоку от Кетмансхопа в Юго-Западной Африке.

В начале 1942 года сержант Л.~О.~Хониборн, один из наиболее наблюдательных и опытных полицейских этого района, дежурил на участке в Кетмансхопе, когда раздался телефонный звонок из Киррис-Уэста. Сержанта попросили приехать на ферму. Когда он туда прибыл, все говорили о драконе, который напал на шестнадцатилетнего мальчика. Ему показали следы, оставленные чудовищем,~--- петли и ложбинки, напоминающие след мчавшегося на большой скорости и неожиданно затормозившего автомобиля.

Хониборн собрал все показания с дотошностью старого служаки. Прежде всего пастух из племени овамбо сообщил о гигантской летающей змее толщиной с бедро человека. <<Она перелетала со скалы на скалу и пугала каракулевых овец>>,~--- сказал пастух.

Спустя несколько недель змею увидел Майкл Эстерхьюзе, шестнадцатилетний сын фермера. Она лежала в расщелине, причем видна была ее голова и около двух футов тела. Когда Майкл бросил в нее камень, она зарычала по-собачьи. Майкл убежал и рассказал дома об этом происшествии.

На следующий день его отец Рас Эстерхьюзе и старший барт взяли ружья и отправились на поиски змеи. Но нашли они только ее следы. Через несколько дней Майкл еще раз увидел змею, когда собирал дикий мед на холмах. Но самая драматическая встреча произошла 13 января у подножия небольшого холма, где Майкл пас овец.

<<Я услышал звук, напоминающий завывание ветра в трубе, и неожиданно ко мне подлетела змея,~--- рассказывал Майкл.~--- С глухим стуком она ударилась о землю, а я отскочил в сторону. Змея затормозила, разбрасывая во все стороны гравий. Затем она снова поднялась в воздух, пролетела над невысоким деревцем и вернулась на холм>>.

Майкл бросился бежать, ноу него закружилась голова, и он свалился в кусты. Овцы вернулись в крааль без него. Отец и брат Майкла побежали на поиски, не надеясь застать его в живых. <<Мы вспомнили о змее и почти ни на что не надеялись, как вдруг увидели его в вельде,~--- рассказывал отец сержанту Хониборну.~--- Он лежал на земле. Вид у него был ужасный: безумные глаза и стиснутые зубы. Лишь через несколько часов он смог говорить>>.

На следующий день Рас Эстерхьюзе устроил облаву. Пришел Гермиас Штраус и другие соседи. На поиски змеи отправилось с десяток вооруженных мужчин. Среди холмов они нашли ее лежбище и кости растерзанных ягнят и антилоп. На седьмом или восьмом холме был виден след, который говорил о ее необычайных размерах. Но змеи в тот день они не видели. Место, где она напала на Майкла Эстерхьюзе, огородили колючим кустарником наподобие крааля, чтобы сохранить следы.

Хониборн осмотрел все как следует и решил, что змея должна быть не меньше двадцати пяти футов в длину. Холм, с которого она <<слетела>> на Майкла, был около трехсот футов высоты. Гравий на месте приземления змеи оказался разрытым на целый дюйм.

Что же это была за змея, которая так напугала Майкла Эстерхьюзе и сбила с толку сержанта Хониборна? Я могу лишь вообразить, что это был необычайно крупный иероглифовый питон. На пустынных плесах реки Оранжевой (примерно в ста семидесяти милях к югу от этой фермы) водится много иероглифовых питонов, иногда достигающих двадцати пяти футов в длину.

Питон нападает с огромной быстротой, и вполне может показаться, будто он подлетает. Он несется к добыче, как выпущенная из лука стрела, так что наблюдающий видит лишь расплывчатые очертания его тела. Иногда питоны бросаются с деревьев. Это-то Майкл Эстерхьюзе и увидел. В такой ужасный миг ему вполне могло показаться, что питон слетал с вершины соседнего холма. И так же быстро питоны отступают, что тоже производит впечатление полета.

Уязвимое место моих рассуждений~--- расстояние между рекой Оранжевой и районом этого происшествия. Питоны всегда держатся у воды и в глубине пустыни не водятся. Однако этот питон (если это был питон) мог подняться вверх по притоку Оранжевой в сезон дождей и задержаться в холмах, наводя ужас на тех, кто его видел.

За последнее время сообщения о летающих чудовищах приходят из различных районов тропической Африки. Опытный натуралист Иван Сандерсон описал встречу в Камеруне с животным, похожим на птеродактиля. Полковник Питмэн (его высказывания о жирафе я уже приводил) слышал о птеродактиле в Северной Родезии и записал эти сведения. Он был поражен, что африканец с такой точностью описал животное, исчезнувшее очень давно.

Доктор М.~Д.~У.~Джеффрис из Витватерсрандского университета, долго проживший в тропической Африке, специально изучал рассказы местных жителей о таких животных, которых нет в музеях. В Северной Родезии африканцам показывали рисунок птеродактиля. Они сразу узнали это животное и сказали, как оно у них называется. Джеффрис пытался объяснить это <<родовой памятью>>, видением, унаследованным от тех далеких дней, когда их предки скрывались от чудовищ.

Птеродактили были очень разные по величине -ч от небольшой, с воробья, летающей ящерицы до страшилищ с размахом крыльев в двадцать пять футов. Мой друг Рой Смизерс, директор Национального музея в Булавайо, считает, что за птеродактиля принимают китоглавов. У этой птицы вид доисторического создания, обитает она в глухих болотистых районах, откуда и поступают слухи об этих странных животных.

Известный немецкий звероторговец Карл Гагенбек снарядил однажды экспедицию к озеру Бангвеоло в Северной Родезии на поиски чудовища, известного африканцам как чипекве. Он получил сведения о чипекве из двух разных источников. Описывалось это животное как <<полудракон, полуслон>>.

Одно сообщение поступило от охотника-англичанина, в словах которого у Гагенбека не было оснований сомневаться. Другое~--- от его собственного зверолова Джозефа Менгеса, человека с многолетним опытом. В свое время Менгес открыл в Сомали новый вид дикого осла и доставил один живой экземпляр в Европу.

Гагенбека поразила одна деталь в сообщении Менгеса. Менгес отмечал, что бушмены изображали чипекве на своих наскальных рисунках. Все африканцы были твердо уверены в существовании чипекве. Эту трудную задачу Гагенбек возложил на Ганса Шомбургка, старого опытного зверолова, который когда-то разыскивал в Либерии легендарных карликовых бегемотов и нашел их.

Поймать бегемотов было совсем нелегко. Насколько же труднее добыть живого дракона, если это вообще возможно. Было это в начале нашего века. Во время путешествия на Шамбургка напали африканцы, но в конце концов он добрался до озера Бангвеоло. Местные жители подтвердили историю о чипекве и сказали, что он сожрал в озере всех бегемотов. Шомбургк хотел было организовать охоту на чудовище, но было как раз самое нездоровое время года в этом ужасном районе Африки. Малярия свалила и его, и его помощников, так что поиски были прекращены.

Торговец и охотник Д.~Э.~Хьюз, который приезжал на Бангвеоло в течение восемнадцати лет, упорно шел по следам слухов и назначил большую премию (тюк одежды) любому африканцу, который принесет ему кость или шкуру чипекве или же покажет следы этого животного. В конце концов он встретил африканца, который утверждал, что его дед участвовал в охоте на чипекве на реке Луапула. Эта памятная охота стала легендой племени. В охоте, длившейся целый день, принимали участие все лучшие охотники племени. Они метали в чипекве копья, пытаясь нащупать уязвимое место. Африканец сказал, что у чипекве гладкая темная шкура и один светлый рог, как у носорога.

Хьюз поверил этим рассказам, так как хорошо знал людей с этого озера и умел отличить правду от вымысла. Но он считал, что чудовище исчезло на памяти живущего поколения. Если бы чипекве обитал в Бангвеоло, то за время такого длительного пребывания здесь Хью-за оно дало бы о себе знать.

\chapter{Кладбище слонов}

Где бы вы ни охотились, вы непременно услышите неумирающую легенду о кладбище слонов, огромном кладбище, заваленном слоновой костью. Сюда влечет слонов инстинкт, когда они почуют приближение смерти. С трудом, напрягая последние силы, они добредают до этого места и здесь умирают.

Рассказы о <<долинах слоновой кости>> я слышал и в Южной Африке, и во многих районах тропической Африки. Вскоре после первой мировой войны группа охотников, разбившая лагерь в нижнем течении реки Сандис в Капской провинции, заметила узкое ущелье, где возвышалась груда разлагающихся слоновых костей. Такие же залежи слоновых костей и бивней эти охотники нашли в районе Бусакс-Клоф на ферме Глен-Ролло. Сюда из дальних мест за костями приходят африканцы. Они перемалывают эти кости и добавляют детям в кашу, чтобы те стали такими же отважными, как слоны.

Лет двадцать назад любитель-палеонтолог Лаурен-сон взял меня с собой к лагуне Милнертон на побережье залива Тэйбл и показал место, которое он называл кладбищем мамонтов. Много лет Лауренсон вел раскопки окаменелых бивней и зубов мамонтов близ Кейптауна. Эти мамонты известны науке как архидискодонты~--- гигантские предки слона. Лауренсон сказал, что там находятся огромные кладбища доисторических животных, особенно слонов. Он считал, что залив Тэйбл был когда-то внутренним озером, на берегах которого обитали необычайные животные. Многие из них и погибли в лагуне Милнертон.

Когда в Африке появились первые голландские поселенцы, слоны встречались на всем Капском полуострове, но последний слон близ их поселения был убит через пятьдесят лет. Однако в начале прошлого века в районе Граф-Рейнет еще встречались стада слонов, насчитывавшие до пятисот голов. И еще долгое время тут шла бойкая торговля слоновой костью. В 1876 году в лесах Книсна обитали сотни слонов. Никто не мог тогда представить, что слоны здесь когда-нибудь почти совсем исчезнут. Легенда о кладбище слонов вошла в фольклор жителей всех лесных районов, где водятся слоны.

Меня всегда удивляло, что Селус и другие знаменитые старые охотники верили в эту легенду. И по сей день многие специалисты обращают серьезное внимание на факты, которые свидетельствуют о существовании кладбищ слонов. Интересно, что такой осторожный ученый, как доктор Морис Бэртон, приводит в доказательство случай, когда слоны всю ночь тащили через джунгли своего мертвого товарища. Доктор Бэртон изучил много легенд о животных и считает, что некоторые из них соответствуют действительности. Да и все остальные легенды имеют в своей основе нечто реальное.

Как возникла эта наиболее живучая из африканских легенд? Первое упоминание об этом я встретил в книге Эндрю Бэттела, англичанина, путешествовавшего по Анголе в начале семнадцатого века. Бэттел писал, что португальцы находят в буше кучи бивней. Возможно, это и породило легенду, которую другие жаждущие ценной слоновой кости превратили в рассказы о сокровищах. Но все же я думаю, что в основе этой легенды лежат реальные факты, которые Морис Бэртон находил во многих замечательных рассказах о животных.

Легенда о кладбище слонов основывается на двух известных фактах. Во-первых, африканцы до сих пор все еще приносят много ценных бивней, по всей видимости взятых не у только что убитых слонов. <<Мы нашли их в буше>>,~--- неизменно говорят африканцы. Но ничто не заставит их показать европейцу место, где они берут слоновую кость.

Во-вторых, люди чрезвычайно редко встречают трупы слонов, если не считать убитых и попавших в западню. А ведь это животное не так-то просто спрятать. В некоторых районах Африки слонов все еще очень много. Их гигантские следы встречаются у каждого источника воды на многие мили вокруг. Издали движущееся стадо слонов похоже на экспресс. Но трупы их попадаются очень редко. На воле слоны живут около пятидесяти лет. Таким образом, в больших стадах слоны должны умирать ежемесячно. Куда же деваются умирающие слоны? Как говорит легенда, они чувствуют приближение смерти и, издав предсмертный рев, отправляются в уединенную долину, где белеют на солнце огромные скелеты их предшественников.

Некоторую правдивость легенде придает тот факт, что слоны обычно, убив человека, забрасывают его тело травой. Почему бы таким умным животным не хоронить и своих мертвых сородичей? Кроме того, охотники наблюдали, как слонихи оказывают помощь раненому слону. Они поят его водой, прикрывают ветками от жары и насекомых и долгое время караулят тело умершего слона.

Бывший окружной комиссар в Судане полковник Д.~Л.~Ф.~Твиди пытался объяснить эту тайну, основываясь на собственных наблюдениях. Как-то в его резиденцию принесли слоновую кость, чтобы взвесить ее и зарегистрировать. Твиди обратил внимание, что одна партия была сильно обгорелой. Оказалось, что африканские охотники выследили стадо слонов в пересохшем болоте. Окружив стадо, они подожгли со всех сторон высокую траву, и все перепуганные животные погибли в пламени. Полковник Твиди заметил, что путешественник, который через много лет обнаружит эти скелеты, вполне может подумать, что наткнулся на кладбище слонов.

Бывший губернатор Уганды сэр Уильям Доуэрс подсчитал, что ежегодно в Африке естественной смертью умирает около двух тысяч слонов. Однако ему приходилось видеть мертвых слонов лишь в тех случаях, когда они попадали в ловушки, гибли от пуль и копий или же сваливались со скалы. Несколько лет назад в Северной Родезии погибло три слона от удара молнии. Иногда они умирают от укусов змей. Сэр Уильям Доуэрс был противником гипотезы о кладбище слонов. Он считал, что, почуяв приближение смерти, слоны уходят в реки и болота. Его мнение поддержал один инженер, строивший мост через Голубой Нил в Хартуме. Он наткнулся на кости слонов под двадцатифутовым слоем речных наносов.

Другие противники этой легенды считают, что отсутствие трупов слонов нетрудно объяснить. Ослабевший, умирающий слон становится легкой добычей львов. Затем появляются гиены. Они разгрызают и растаскивают даже самые крупные кости. В конце концов кустарник скрывает от глаз последние остатки. По берегам рек к пиршеству присоединяются крокодилы. Таким образом огромные животные растаскиваются по кускам и исчезают бесследно.

Немецкий путешественник и изыскатель шестидесятых годов прошлого века Карл Маух, исследовавший район Зимбабве, был сторонником легенды. Как-то в Бечуаналенде он с местным проводником вышел к узкой расселине, которая переходила в большое глубокое ущелье, на многие мили устланное костями слонов. Правда, бивней там не было. Маух описывал это место как кладбище слонов Бечуаналенда.

Не знаю, видел ли еще кто-нибудь это кладбище после Мауха. Но в 1936 году, находясь в Бечуаналенде с экспедицией, я услышал калахарское предание, связанное с легендой о кладбище слонов. Говорят, что где-то в Калахари есть огромный кратер вулкана, ставший ловушкой для людей, животных и фургонов. Немецкий охотник Эрлангер утверждал, что однажды он свалился в эту огромную яму вместе с фургоном. Эрлангер не разбился, отделавшись лишь ушибами. На дне кратера он нашел источник, а в фургоне у него было достаточно продуктов. Он принялся исследовать кратер и наткнулся на множество скелетов слонов вместе с бивнями. Прожив в этой необычной тюрьме несколько дней, Эрлангер вбил в отвесную стену деревянные колья и выбрался наружу. Бушмены вывели его из пустыни. Несомненно, что тяжелое испытание, выпавшее на долю Эрлангера, со временем было приукрашено. Однако в Калахари столько географических причуд, что такой кратер вполне может существовать.

Говорят, что один из крупнейших работорговцев Занзибара, Типпо Тиб, нашел в Восточной Африке огромную <<долину слоновой кости>>, куда уходили умирать тысячи слонов. Англичане предложили ему якобы десять процентов стоимости всей слоновой кости, если он покажет этот склад. Но Тиб не верил ни одному европейцу, так что тайна умерла вместе с ним. Однако у Тиба был племянник Мохамед Абдулла, такой же проходимец, как и его дядя. В 1927 году Мохамед Абдулла сумел заинтересовать правительство Уганды, рассказав о слоновой кости, погребенной задолго до прихода европейцев. Он отказался сообщить подробности и только сказал, что получил слоновую кость от покойного друга. Был ли это склад Типпо Тиба? Мохамед Абдулла добился нужных гарантий и доставил больше сотни замечательных бивней.

Один из первых охотников в Британской Восточной Африке майор П.~Г.~Д.~Поуэлл-Коттон всегда говорил, что нашел самое настоящее кладбище слонов на землях племени туркана, <<место, куда слоны приходят умирать>>. Он уверял меня, что их скосила не болезнь. Когда слоны почувствуют слабость, они издалека стремятся сюда, чтобы сложить здесь свои кости. Место это хорошо знают люди племени туркана, которые постоянно приходят сюда за бивнями.

Еще более убедительно звучат слова майора Д.~Ф.~Камминга, окружного комиссара в Судане, который несколько лет назад поместил в научном журнале статью о кладбище слонов. На берегу Верхнего Нила майор Камминг застрелил в стаде одного слона и вернулся на другой день, чтобы вырезать бивни. Но животное исчезло. Оно оказалось под восемнадцатидюймовым слоем земли, причем всюду были видны следы бивней слонов, зарывавших труп.

Как-то во время второй мировой войны я увидел с самолета настоящее кладбище слонов. Один из моих коллег, который раньше бывал в этом месте, рассказал мне о нем. Сквозь дымную завесу горящего леса я увидел большое таинственное озеро. Это было озеро Банг-веоло в Северной Родезии. Тысяча шестьсот квадратных миль мелкой воды и извилистых проток. Сквозь топи проложены дорожки из бревен. Лодки, огромные бананы и тростник. Мириады птиц~--- утки, цапли, журавли и марабу. Товарищ сказал мне, что на озере есть острова, где живут одни слоны. В период дождей эти острова совсем отрезаны от остального мира.

Здесь слоны рождались и умирали. Охотники не осмеливались преследовать их, потому что тростник и заросли делали это убежище слонов опасным для человека. Как гласит местная легенда, где-то у западного побережья Бангвеоло есть кладбище слонов, настоящий склад слоновой кости. Я был рад увидеть это место рождения и смерти слонов, но мне хотелось, чтобы наш <<Лодстар>> летел не так быстро. Я испытывал танталовы муки, когда появившееся на мгновение под крылом самолета кладбище тут же исчезло из поля зрения.

Можно лишь с уверенностью сказать, что больной, умирающий слон испытывает огромную жажду. Поэтому многие слоны в свои последние часы направляются по одному и тому же пути~--- по тропе, ведущей к ближайшему источнику воды. Напившись в последний раз, огромные слоны погружаются в болото или омут, которые могут стать могилой целого стада. Здесь они и исчезают, пока в сильную засуху на высохшем дне вновь не появятся их скелеты и бивни.

<<Он отправился умирать,~--- говорят африканцы, когда старый слон, которого они хорошо знают, вдруг исчезает с их территории,~--- Откуда нам знать, куда он ушел?>>

Может быть, это просто выдумка, но мне хотелось бы думать, что в долине Слоновой Кости из года в год растет холм из костей слонов.

\chapter{В лесах горилл}

Взгляните на карту Западной Африки, и вы увидите на французской территории выступ, известный как мыс Лопес. Он прикрывает огромный залив и гавань Порт-Жантиль. Когда-то Порт-Жантиль был пристанищем старого авантюриста по кличке Торговец Хорн. И вот сюда в свой компаунд привел меня современный торговец и показал шкуру огромной черной гориллы с оскаленной пастью. Даже мертвая она выглядела свирепой.

---~Такие вещи все еще стоят денег,~--- заметил он.~--- Но я предпочел бы видеть их распяленными на солнце в ожидании отправки в музей, как эта горилла, а не встречаться с ними в лесу. Некоторые утверждают, что горилла~--- безобидный монстр, котррый никогда не нападает на человека. Не верьте им.

Удивительные существа рождаются в огромных экваториальных лесах, которые начинаются у Порт-Жантиля и простираются на тысячи миль до озер. Но из всех населяющих этот мир мрака животных самое достопримечательное~--- горилла. Не только потому, что горилла~--- самая крупная человекообразная обезьяна, которая нередко достигает шести футов роста и весит четыреста фунтов. Но и потому, что, как утверждают сэр Артур Кейт и другие, она самый ближайший родственник человека. Предполагают, что, когда в древности начали исчезать леса, крупные обезьяны, лишившись укрытий, должны были стать более хитрыми, чтобы выжить на открытом пространстве. Так и возникло человечество. Но здесь, в экваториальной Африке, все еще достаточно лесов, дающих убежище. И здесь водятся гориллы и шимпанзе.

Это их мир. Сырой девственный лес, где красное дерево и гигантские сейбы возвышаются над пальмами и древовидными папоротниками, над бананами и вьющимися растениями. Мрачный мир, где лишь изредка встречается какой-нибудь ориентир или тропинка. Зеленый, насыщенный испарениями мир~--- гигантская теплица экватора. Вы, может быть, помните впечатления Стэнли: <<Мы, конечно, и раньше видели леса, но этот стал целой эпохой в нашей жизни, мучительной эпохой, которую никогда не забудешь. Мы вынуждены были ползать на четвереньках, как дикие звери>>.

Этот первобытный лес был настолько недоступен, что еще девяносто лет назад горилла оставалась легендарным животным. Слухи о гориллах проскальзывали задолго до этого. В начале семнадцатого века англичанин Эндрю Бэттелл писал о <<понго>>, живущем в этом районе. <<Во всех отношениях понго похож на человека, правда, скорее на скульптуру гиганта, чем на человека. Он очень высок, у него человеческое лицо с глубоко сидящими глазами и длинными волосами на бровях. Понго нельзя поймать живым. У них такая сила, что десять мужчин не в состоянии удержать одного понго. Однако молодых понго часто поражают отравленными стрелами>>.

Рассказ Бэттелла был принят за выдумку. В середине прошлого века американский миссионер доктор Томас Сэвидж собрал в Габоне несколько черепов горилл, которых съели африканцы, и отправил их в Нью-Йорк. Но сам он никогда не видел живой гориллы.

Затем появляется молодой Поль Дю Шайю, сын французского торговца в Габоне, но обосновавшийся в Америке. Дю Шайю после Бэттелла был первым белым человеком, видевшим живую гориллу. Он отправился через леса в сопровождении африканцев. Дю Шайю знал габонский диалект, много слышал о гориллах и был полон решимости застрелить одно животное и увезти в Нью-Йорк его шкуру. Вот как он сам описывает свое открытие;

<<Мои люди кое-что заметили, и это привело всех нас в состояние величайшего возбуждения. Тростник повсюду был смят и вырван с корнями. Я догадался, что это свежие следы гориллы, и мое сердце радостно забилось. Вот сейчас я, может быть, встречусь лицом к лицу с монстром, о свирепости, силе и хитрости которого так много рассказывали мне африканцы. Встречусь с животным, почти неизвестным цивилизованному миру, на которого до этого не охотился ни один белый человек. Мое сердце билось так, что я испугался, как бы его громкие удары не спугнули гориллу. И тут я вздрогнул от странного, резкого, получеловеческого крика и увидел четырех молодых горилл, убегающих в густой лес. Они бежали на задних конечностях и поразительно напоминали волосатых людей. Опущенная голова, наклоненное вперед туловище~--- весь их вид напоминал людей, бегущих от опасности>>.

Первая горилла, убитая Дю Шайю, была самка, которая кормилась ягодами в нескольких футах от своего детеныша. Дю Шайю поймал малыша и посадил его в клетку, но тот вскоре умер. Эта первая экспедиция длилась четыре года. Дю Шайю привез не только коллекцию горилл и шимпанзе, но и около тысячи других животных и две тысячи птиц. Он отправил их шкуры в Филадельфийскую академию естественных наук, Британский музей и Лондонский королевский хирургический колледж.

Затем он поехал в Англию и Америку читать лекции, которые принесли ему лишь славу второго Мюнхгаузена. Скептически настроенная аудитория не приняла его точного и яркого описания огромного самца гориллы, бьющего себя в грудь, как в барабан. Он рассказывал о проделках стада горилл, опустошавших сады и банановые плантации африканцев, а аудитория смеялась. Он говорил о гнездах, которые самки-гориллы сооружают для своих детенышей на деревьях, тогда как старые самцы остаются внизу и охраняют покой семьи. А цивилизованные люди не могли в это поверить.

Дю Шайю был блестящим молодым исследователем, но ему не хватало опыта лекционной работы. Его манера изложения была неубедительна, и люди уходили с лекции, уверенные, что слушали шарлатана. Печальная награда абсолютно честному человеку, который так много страдал от лихорадки и голода и преодолел столько трудностей. Но как это типично для недальновидной публики, провозглашающей шарлатанов героями и высмеивающей такого человека, как Дю Шайю, который для доказательства привез с собой целую груду шкур горилл.

Жители Габона верят, что души умерших людей поселяются в самых крупных гориллах. Эти гориллы обладают умом человека и свирепой силой зверя. Страстно тоскуя по человеческому обществу, они похищают туземных женщин и уносят их в лес.

Дю Шайю знал эту легенду и записал рассказ двух женщин из племени мбондемо, которые были похищены гориллой. Одной из них вскоре удалось убежать, другая вернулась в деревню на несколько дней позже. Она сказала, что самец-горилла дурно с ней обошелся, но все же она сумела освободиться. Дю Шайю отнесся к этой истории осторожно, считая ее <<суеверным представлением>>. Другие авторы приукрасили эту историю, и она превратилась наконец в одну из таких же бессмертных легенд, как и предание о дереве-людоеде мадагаскарских лесов.

Есть ли в этой истории хоть доля правды? Я искренне надеюсь, что нет, хотя более поздние путешественники по Габону собрали некоторые сведения, подтверждающие эту историю. В 1920 году член Королевского географического общества Ф.~У.~Г.~Мигод, крупный чиновник английской администрации в Западной Африке, вернувшись из путешествия по Габону, сообщил, что многие габонские женщины курят, когда отправляются в лес. Эта мера предосторожности помогает им избежать опасной встречи с гориллами, которые не выносят запаха табака. <<Достоверные случаи нападения на женщин горилл очень редки>>,~--- замечает Мигод. Он обсуждал вопрос с тогдашним английским вице-консулом в Габоне В. Тилом. Тил сказал, что знает одну такую женщину. Она была изгнана из своей деревни и нашла приют в другом племени.

Американский ученый доктор Фред Палстоун, путешественник времен Стэнли, слышал от местных жителей, что гориллы похищают детей и воспитывают их. Он не смог установить, насколько достоверна эта история, но многие жители уверяли его, что она правдива.

Местные женщины действительно выкармливали детенышей горилл, но вот эту историю в духе Тарзана, которая изумила доктора Палстоуна, нельзя принимать на веру.

Торговец Хорн в своих красочных воспоминаниях тоже, конечно, писал об этом. Он рассказывает об одном опыте, который проделал работорговец в Порт-Жантиле. <<В клетку с огромным самцом гориллы посадил девушку-рабыню. Но ничего не произошло. Горилла мрачно сидела в одном углу, тогда как бедная девушка горько плакала в другом>>.

Габонское племя фанг разделяет людей на три расы~--- белых, черных и горилл. Говорят, что днем гориллы безобидны и становятся опасными лишь ночью, когда охраняют свои гаремы на ветвях деревьев. Однако, когда несколько лет назад Александр Йорк шел с партией изыскателей через леса Габона, на них то и дело нападали гориллы даже среди белого дня. Им пришлось перебить множество горилл, чтобы спокойно продолжать путешествие. Но горилла не хищник, она не смотрит на человека как на добычу. Никто еще ни разу не сообщал, чтобы гориллы ели людей.

Африканцы же едят горилл. Их темно-красное мясо хотя и жестко, но прекрасно на вкус и ценится очень высоко. Чтобы добыть этот деликатес, местные охотники идут на невероятный риск. С плохонькими ружьишками в руках они ждут, когда горилла схватится за ствол и начнет его кусать. В это время они и стреляют. Очень часто охотники на горилл гибнут во время такой охоты.

Видимо, первая живая горилла была доставлена из Западной Африки в Европу в семидесятых годах прошлого века. Ее достал в Порт-Жантиле капитан парохода <<Ангола>> Томпсон и продал одному зоопарку в Германии за пятьсот фунтов. Лишь в 1887 году гориллу получил Лондонский зоопарк, но она прожила в неволе недолго.

Многие годы ученые считали, что по умственным способностям шимпанзе превосходит гориллу. Но Джон Даниэль~\rom{1} и Джон Даниэль~\rom{2}~--- две гориллы, пойманные в лесах Порт-Жантиля,~--- заставили ученых изменить свое мнение, показав, как ошибочно делать выводы без достаточного количества сравнительных опытов. Многие годы научные эксперименты велись в основном на более дешевых и выносливых шимпанзе. Те же немногие гориллы, с которыми работали ученые, оказались малосообразительными и быстро умирали.

Но вот в 1918 году француз из Порт-Жантиля продал звероторговцу в Лондоне за шестьдесят фунтов Джона Даниэля I. Крупный магазин в Кенсингтоне купил гориллу для рекламы в витрине, однако двухлетняя горилла стала чахнуть. В это время мисс Элис Каннингэм предложила продать обезьяну ей. Сделка состоялась, и Каннингэм приступила к дрессировке гориллы. Вскоре Джон Даниэль стал заметной фигурой не только в Лондонском зоопарке. Джон жил в квартире Каннингэм и отлично приспособился к человеческому образу жизни. У него была своя постель, он умел держать себя за столом и пользовался ванной, никогда не забывая закрыть после купания кран.

Целые толпы собирались у окна квартиры, чтобы посмотреть на проделки Джона, а отряд королевской стражи, проходивший мимо дома, еле мог сохранять стройность своих рядов. В конце концов полиция обратилась к Элис Каннингэм с просьбой держать гориллу подальше от окна, так как из-за нее нарушалось уличное движение.

Все, что было под силу шимпанзе, с неменьшим успехом мог сделать и Джон. Его любимым номером было притворяться слепым и, натыкаясь на мебель, ходить по комнате, все время отлично сознавая, как забавляет он публику. Джон во всем проявлял сообразительность. Он заметил, что мисс Каннингэм всегда сгоняет его с колен, когда на ней нарядное платье. И вот однажды он принес газету и расстелил ее на коленях хозяйки, прежде чем сесть.

В 1921 году Каннингэм продала Джона в один из американских цирков с условием, что его не заставят исполнять непосильные трюки. Вскоре горилла умерла. Джон Даниэль так привязался к мисс Каннингэм, что разлука с ней разбила его сердце.

Через два года Элис Каннингэм выехала с братом в леса Порт-Жантиля за другой молодой гориллой. Один мой друг, торговец, хорошо помнил их приезд. Каннингэм сказал ему, что один американский зоопарк готов уплатить две тысячи фунтов за живого <<карликового слона>>, и спросил, может ли он найти ему такого слона.

---~Безнадежное дело,~--- ответил торговец.~--- Единственно, что вы можете сделать, это поймать слоненка, давать ему побольше джина, чтобы задержать рост, и поскорее отправить в Америку, пока он не вырос. Только так можно получить эти деньги.

Каннингэм и его сестра вернулись из лесов в Порт-Жантиль со слоненком и молодой гориллой, которой суждено было прославиться как Джон Даниэль~\rom{2}. Эта горилла, такая же смышленая, как и Джон Даниэль~\rom{1}, демонстрировалась в Лондоне, на континенте и в Соединенных Штатах. Джон~II прожил в неволе с 1923 по 1927 год.

В отличие от шимпанзе горилла не может долго переносить неволю. Это в равной степени относится к обоим подвидам~--- прибрежной (известной в науке как \textit{Gorilla gorilla gorilla}) и горной (\textit{Gorilla gorilla beringei}). Горные гориллы водятся в самом восточном районе экваториальных лесов. Интересно отметить, что долгое время это громадное животное не было известно натуралистам. Лишь в начале нашего века капитан фон Беринг привез в Европу первую убитую горную гориллу.

Время от времени появлялись сообщения еще и о красной горилле. Но в настоящее время натуралисты не верят в существование других видов помимо тех двух, о которых я упоминал. У горилл косматые черные волосы, хотя с возрастом старые самцы становятся серыми или серебристыми. У горной гориллы на голове могут быть и красноватые волосы, но в общем это черные как смоль животные. Единственное исключение из правила~--- альбиносы, но они встречаются крайне редко. Много лет назад в верховьях реки Луалли, к северу от устья реки Конго, доктор Палстоун поймал белую гориллу.

Умеют ли гориллы говорить? Немецкий охотник и писатель Герман Фрейберг утверждает, что он слышал, как местный знахарь долго разговаривал в лесу с гориллой. В конце горилла умоляла сохранить ей жизнь и была отпущена с миром. Когда я думаю об этом сообщении, я чувствую себя в некотором роде одним из слушателей бедного Поля Дю Шайю, который так и не сумел их ни в чем убедить. По ночам гориллы издают жуткие вопли и стоны. Особым звуком они предупреждают друг друга об опасности, у них есть сигнал сбора и отправления в путь и другие примитивные голосовые сигналы. Но ни Джон Даниэль~\rom{1}, ни его преемник не показали ни малейшего признака, что у горилл есть разговорный язык. Несколько лет назад немецкий ученый Швидецки написал книгу о языке шимпанзе. Но специалисты восприняли ее очень сдержанно. Шимпанзе более <<говорливы>>, чем гориллы, и могут издать много звуков~--- от громких криков до хныканья, что и подогревает воображение впечатлительных лингвистов.

Одну из удивительных историй о человекообразной обезьяне рассказал сотрудник Музея естественной истории в Саут-Кенсингтоне доктор А.~Е.~Ансорге. Он плыл на речном пароходе через леса к югу от Порт-Жантиля. На ночь пароход причалил к уединенной торговой фактории. На дорожке от дома торговца к пароходу показался фонарь. Ансорге увидел, что его несет молодой шимпанзе, который взял его потом за руку и отвел к дому. Хозяин сказал, что он обучил шимпанзе встречать все речные суда и приветствовать гостей.

Торговец страдал от одиночества в этой нездоровой местности, и Ансорге стал уговаривать его бросить шимпанзе и поискать общества людей. Но тот ответил, что предпочитает дружбу обезьян. Через некоторое время доктор Ансорге узнал, что шимпанзе стал очень злобным и в конце концов убил торговца.

Еще труднее угадать, как поведет себя горилла. А ее гигантские размеры и сила делают ее еще более опасной, чем шимпанзе. Гориллы стоят дороже и к тому же не размножаются в неволе. Один из крупнейших знатоков шимпанзе, американский психолог Р.~М.~Йеркес, считает, что горилла изучена совсем недостаточно. Даже о предельном возрасте гориллы судят лишь предположительно.

Но нет никакого сомнения в злобном характере гориллы. Несколько лет в Порт-Жантиле жил один француз, бывший адвокат, который питал ненависть к гориллам. Я не могу назвать причины этой ненависти, но знаю, что он поселился в этом месте лишь для того, чтобы иметь возможность убивать горилл.

Надо еще упомянуть о профессоре Гарнере, который приехал в Порт-Жантиль, чтобы изучать горилл. Он решил не подвергать себя риску и захватил с собой несколько больших клеток. Одна из них упала в море и утонула. Но другие были благополучно доставлены к нужному месту в лесу. Гарнер закрывался в клетке и в полной безопасности вел свои наблюдения.

Хотя на горилл и охотятся ради мяса, большой опасности, что они будут истреблены, все же, видимо, нет. Доктор Н.~А.~Дайс Шарп провел <<перепись>> горилл на площади тридцать квадратных миль и насчитал их больше двухсот. По его мнению, во всей экваториальной Африке насчитывается много десятков тысяч горилл. Доктор Дайс Шарп считает, что горилла~--- самое опасное животное тропической Африки, единственное животное, которое нападает на человека.

\chapter*{Послесловие}

Книга Лоуренса Грина <<Последние тайны старой Африки>> необычна. Это не путевой очерк и не научное исследование. Личные впечатления переплетаются здесь со сведениями, которые автор слышал от других людей, читал в книгах. Она не посвящена какой-либо определенной теме. В ней уживаются главы о народной медицине и о тайнах перелета птиц, об африканских барабанах и о дереве-людоеде, об особенностях быта некоторых африканских народностей и о кладбищах слонов. Все эти столь различные сюжеты собрала воедино любовь автора <<ко всему случайному, странному и непостижимому>>, происшедшему к тому же в Африке, которую он, по его же словам, знает <<лучше, чем любой другой континент>>.

<<\ldotsЯ не доверяю тем, кто разъезжает по свету в поисках различий. Ибо главное состоит в установлении сходства>>,~--- писал шведский писатель Пер Вестберг, разоблачивший в своей книге <<Запретная зона>>\footnote{Пер Вестберг. Запретная зона. М., Географгиз, 1963.} расистские порядки Южной Родезии. Лоуренс Грин ставил перед собой иную цель, и средства у него соответственно иные. Относясь со всей возможной снисходительностью к автору <<Последних тайн>> в силу названных выше причин, последуем, однако, совету Вестберга и посмотрим на только что прочитанную книгу с максимальной критичностью. Тем более что она действительно в этом нуждается. Именно такой подход может помочь превратить ее из сборника занимательных фактов, не всегда достаточно полно и убедительно объясняемых автором, в источник, дающий толчок уму, пищу для размышлений.

Читая эту книгу, нужно иметь в виду, что ее автор~--- уроженец Южной Африки, гражданин одного из самых реакционных, расистских государств. И тем знаменательнее, что Лоуренс Грин, описывая коренных жителей Черного континента, говорит о них с большой симпатией и даже с восхищением. Правда, он пользуется распространенной в обиходе и печати капиталистического мира терминологией.

Лоуренс Грин неоднократно сам подчеркивает свою некомпетентность в объяснении многих из тех загадочных явлений, о которых он пишет. <<Может быть, это так, а может быть, и иначе>>~--- такой мотив звучит со многих страниц книги. Для любого научного труда такая позиция автора заслуживала бы клейма <<агностицизм>>. Но подойти таким образом к <<Последним тайнам>> все равно что человека, воскликнувшего <<Боже мой!>>, обвинить в распространении религиозного дурмана. Грин совершенно не претендует на научность, не пытается навязать своих взглядов. <<Вот то, что я знаю. Кое-что могу объяснить. А уж вы, будьте добры, проверьте, прав ли я, и найдите то объяснение, которое вам покажется правильным>>,~--- как бы говорит Грин своим читателям. Впрочем, сам он чаще высказывает сомнение в правильности существующих объяснений для тех или иных таинственных явлений, чем выдвигает свои собственные. А в книге Лоуренса Грина собраны, как правило, явления малоизученные, остающиеся действительно загадочными. Например, в главе <<Испытание огнем>> Грин объясняет способность некоторых людей переносить без вреда для себя температуры, при которых гибнет живая ткань, с помощью <<эффекта Лейденфроста>>: <<Просто ноги выделяют пот и образующиеся шарики жидкости предохраняют их от ожогов. Точно так же слюна предохраняет рот глотателя огня>>. Эта гипотеза действительно имеет основание. Известно, например, что благодаря подобному процессу можно обмакнуть в расплавленный свинец палец и не обжечь его. Однако вряд ли с ее помощью можно объяснить все известные случаи <<испытания огнем>>. В этой же главе рассказывается о колдуне из племени вакимбо, который засовывал голову в ямку с раскаленными углями и держал ее там до двадцати минут. В статье <<Ходящие по огню>>, напечатанной в журнале <<Вокруг света>> № 9 за 1965 год, рассказывается, что, выполняя ритуальный обряд, бытующий на одном из островов Тихого океана, человек становится на раскаленный докрасна нож мачете и стоит на нем до тех пор, пока железо не остынет. При длительном воздействии высокой температуры на тело человека гипотеза о <<паровой рубашке>> оказывается недостаточной. О причине этого говорит и сам Грин: <<Дело в том, что при каждом шаге подошва находится в соприкосновении с углями менее полсекунды, каждая же часть подошвы, которой переступают с пятки на носок, и того менее~--- какую-то долю секунды. В этом и заключается спасение ходящих по огню>>. В случаях же более длительного воздействия огня на ткани, по-видимому, важна тренировка. Известно, например, что руки кузнеца, имеющего дело с разогретым металлом, менее чувствительны к высокой температуре, чем у других людей. Далеко не до конца также изучена способность человеческого организма к терморегуляции.

Загадочный характер некоторых описываемых в книге явлений, отсутствие в настоящее время достоверных научных объяснений для них могут создать впечатление о склонности автора к мистицизму. На самом деле Грин подходит ко всем этим загадкам если не всегда достаточно научно, то по крайней мере с позиций здравого смысла. Наиболее ярко это демонстрирует 
глава <<Говорящий дым>>.

В этой главе автор рассказывает об одном из видов <<беспроволочного телеграфа>> отсталых народов~--- дымовой сигнализации. Лоуренс Грин, как и многие его предшественники, был поражен тем, как много, порой весьма детальной, информации могут передать бушмены, пользуясь дымом костра. В случае, свидетелем которого был сам автор, группа бушменов-охотников, находившаяся на большом расстоянии от лагеря путешественника, сигнализировала не только об успешности охоты, но, если верить Грину, даже и о виде дичи и возрасте убитых животных. По его словам, дымовая сигнализация позволяет передать гораздо более широкую информацию, чем азбука Морзе. Ссылаясь на .свидетельство бушменов, автор утверждает, что каждый из них <<знает то, что хочет передать в дыму>> их соплеменник. Не добившись более вразумительного объяснения сущности дымовой сигнализации, вспоминая, что подобная же связь существует и у австралийских аборигенов, Грин пытается разрешить загадку, призывая на помощь телепатию и телекинетику.

О телепатии много писалось и дискутировалось в широкой печати. В обиходе это явление трактуют как <<передачу мысли>> на расстояние. На самом деле все приводимые обычно в описаниях случаи представляют собой либо образное восприятие, принимаемое одним человеком от другого, либо восприятие эмоционального состояния человека-телетранслятора. Телекинетика~--- явление того же порядка, но представляет собой передачу материальной, например мускульной, силы на расстояние и способность подобным <<психическим>> воздействием передвигать предметы. Грин полагает, что подробность дымовой информации у бушменов и австралийских аборигенов можно объяснить тем, что человек, подающий сигнал, мысленно воздействует на столб дыма, вызывая его колебания и используя как своеобразный отражатель своих образных восприятий. Грин называет этот способ передачи информации волнами мысли.

Безусловно, это очень натянутое и неправдоподобное объяснение. Дымовая сигнализация~--- один из древнейших видов связи. Уже у раннего человека столб дыма от костра вдали должен был свидетельствовать о присутствии соседнего человеческого стада, а в случае если этот столб дыма поднимался там, где не было постоянной стоянки, он говорил об удачной охоте и приготовлении пищи. В дальнейшем дым от костра стали использовать для передачи какой-либо информации специально. Стоит вспомнить только казачьи сторожевые посты в Диком поле на Украине в эпоху татарских набегов. Легкий, светлый столб дыма мог ничего не означать, но при появлении вражеского отряда в костер подбрасывалось влажное топливо, и густые черные клубы дыма означали тревогу. Таким образом, уже густота и цвет дыма, а также форма дымового столба могут служить своеобразной азбукой Морзе. С другой стороны, в пустынных областях существует строго определенная сезонность расцвета той или иной растительности и жизненных циклов животных. Скажем, весной, когда в пустыне больше растительности и воды, появляются определенные животные, которые уходят с наступлением сухого сезона. Естественно, что и сбор тех или иных съедобных растений, и охота на тех или иных животных также носят строго определенный сезонный характер. Поэтому бушмен, которого расспрашивал Грин, с полным правом мог заявить, что он и его сородичи <<всегда знают, что передает дым>>, так как он действительно всегда знал, на кого в данное время года обычно охотятся и какие съедобные растения собирают.

Тем не менее автор книги приводит и примеры телепатической связи и ясновидения у коренного африканского (особенно бушменского) населения. Он полагает, что <<африканцы обладают шестым чувством>>, получая информацию и возвещая о своем присутствии и действиях своим сородичам телепатическим путем.

Как явления телепатии, так в еще большей степени и явления телекинеза с точки зрения современной науки более чем сомнительны. Прежде всего потому, что нет ни одного случая научно зарегистрированного бесспорного факта телепатической передачи и приема. Более того, есть сколько угодно примеров, что подобные случаи~--- результат исключительной наблюдательности, как это демонстрировал в свое время Вольф Мессинг, или иллюзионного искусства, либо, наконец (что бывает чаще всего), прямого мошенничества. Однако обыденная практика, не имеющая силы научного факта, говорит о противоположном. Именно поэтому вопрос о телепатии вызывает широкий интерес и интригует читающую публику, именно поэтому он так живуч, несмотря на то, что множество авторитетов всячески опровергают возможность существования телепатических явлений, хотя по сути дела никто еще и никогда не дал себе труда по-настоящему научно поставить исследование этого вопроса.

Теоретически как возникновение телепатических способностей человеческого организма, так и механизм его действия вполне можно себе представить и объяснить. Мы подчеркиваем: именно человеческого организма, хотя некоторые исследователи, как, например, Б.~Б.~Кажинский, считали возможным искать корни этого явления в организме животных. В энергетическом отношении более высокоорганизованные группы животных превосходят более низкоорганизованные. Так, энергетика организмов высших млекопитающих~--- плацентарных~--- выше, чем энергетика низших млекопитающих, например сумчатых. В свою очередь организм человека производит и потребляет почти в четыре раза больше энергии, чем организм любых других плацентарных, включая человекоподобных обезьян. И львиная доля этой энергии идет на питание мощного человеческого мозга, так что если бы организм непосредственных обезьяньих предков человека не выработал бы способности производить большее количество энергии, то не смог бы развиться и крупный сложный человеческий мозг. Можно полагать, что энергетика организма пред-людей и ранних людей увеличивалась быстрее и раньше (хотя бы за счет повышения калорийности пищи при переходе на мясное питание), чем увеличение и усложнение мозга, поэтому в организме древнейших людей имелись избытки энергии. Это и могло создать предпосылки к возможности возникновения телепатической связи между членами одного человеческого стада.

Древнейшие и древние люди существовали за счет собирательства и охоты. Плохо вооруженные, медленно передвигавшиеся по земле, малочисленные, они могли выжить только при условии дружных и согласованных коллективных действий. Согласованность действий особенно была необходима во время охоты. Но как раз во время охоты действия коллектива должны были быть особенно гибкими в зависимости от изменения поведения преследуемых животных. При отсутствии членораздельной речи с помощью одних сигнальных звуков, хотя и более многообразных, чем у обезьян, но не могущих охватить все возможные ситуации, возникающие во время охоты, успешные действия коллектива охотников требовали более совершенных средств связи. Таким средством и могла быть телепатическая передача образов и эмоций..

В дальнейшем, по мере усложнения образа жизни и взаимоотношений между людьми, такая чувственно-образная связь уже не могла удовлетворить потребностей человека, так как с ее помощью невозможно было передавать понятия, особенно абстрактные. Начала развиваться членораздельная речь, которая, как более совершенное средство связи, вытеснила телепатическую связь. К тому же развитие мозга, у человека постепенно ликвидировало первоначальные возможные излишки вырабатываемой организмом энергии, что сильно сократило телепатические возможности. И в настоящее время способность к телепатии, если она существует, должна носить характер своеобразного атавизма.

Все нами сказанное о телепатии отнюдь не означает, что мы утверждаем реальность этого явления. Мы хотим только сказать, что теоретически оно вполне может существовать и возможно объяснение не только его природы, но и причин его возникновения и исчезновения, что наука пока не имеет права ни утверждать, ни отрицать этого явления и что строки, посвященные Грином телепатии,~--- это, пожалуй, единственная настоящая загадка из всех тех, о которых он говорит в своей книге. Его материалы только подтверждают вывод об атавистическом возможном характере телепатических способностей, так как хотя он и полагает, что у всех африканцев сохранилось это <<шестое>> чувство, но по сути дела он упоминает только одного-двух коренных жителей, у которых эти способности ярко выражены.

Совершенно реалистические объяснения Лоуренс Грин дает колдовским ритуалам различных африканских народов, например, ритуалам, связанным с вызыванием дождя: <<\ldotsЗдесь не может быть и речи ни о колдовстве, ни о предчувствии погоды. Эти люди просто отличные знатоки животного и растительного мира>>. Но он слишком большое значение придает процессу внушения, говоря о колдовстве, имеющем целью смерть человека, против которого это колдовство направлено. Правда, не следует отрицать, что многие знахари и жрецы отсталых (а также и древних) народов использовали гипноз и внушение.

Однако история колдовства знает много примеров, когда заклинаниям колдуна часто помогал яд или нож. Кроме того, не нужно упускать из вида силу психологического воздействия на суеверных людей самого факта колдовства, направленного против них. Во всяком случае, такое колдовство обычно получает широкую огласку и сведения о нем всегда доходят до ушей жертвы. У человека, глубоко верящего в <<силу>> колдуна, это известие, естественно, должно вызвать страх и ужас, и могут быть случаи, когда этот ужас убивает жертву.

Особое значение здесь имеют тайные общества, вроде общества <<людей-леопардов>>. Возникновение подобных обществ связано с обрядами инициации~--- посвящения юношей в охотники или воины,~--- в эпоху первобытнообщинного и родового строя. В процессе распада родового строя эти общества в дальнейшем были приспособлены феодально-жреческой верхушкой для укрепления своего влияния и власти и для истребления недовольных и опасных лиц. В ряде случаев дело доходило до прямого террора, как в случае <<общества Леопарда>>.

Сам подход автора к <<тайнам>>, о которых он говорит,~--- ограниченный географическими рамками одного континента~--- безусловно, в значительной степени сужает его возможности искать разгадки <<последних тайн>>. Эта географическая ограниченность может иной раз даже создать неправильное представление о какой-то исключительности Африки.

Некоторые данные, приводимые автором, в наши дни устарели. К ним относится, например, характеристика жизни египтян, поскольку описывается Египет 20~-- 30-х годов, а не послереволюционный, в котором исчезли или сильно сократились пороки колониально-феодального времени (хотя бы преступность).

В главе <<Цыгане Нила>> дается яркая характеристика разложения феодальной верхушки дореволюционного Египта. Безземелие народных масс, отсутствие промышленности имело следствием весьма низкий уровень жизни населения Египта, в частности, египетских цыган, которые вынуждены были добывать средства существования способами часто предосудительными.

Важной научной проблемой является открытие и изучение следов древних цивилизаций в Африке. О прошлом африканского континента современная наука знает еще мало. Это обстоятельство используется некоторыми буржуазными историками для оправдания так называемой <<цивилизаторской миссии>> белого человека. Следам исчезнувших цивилизаций Африки посвящены две главы~--- <<Затерянный город пустыни Калахари>> и <<Никто не знает Сахары>>. В первой из них Грин передает рассказы об остатках погибшего города в Калахари, которого, впрочем, ему, как и другим путешественникам, не удалось разыскать.

Пустыня Калахари возникла гораздо раньше Сахары. Пока что нет данных, не в пример последней, что еще несколько тысячелетий назад Калахари была более влажной, чем сейчас: археологические материалы, в частности, упоминаемые самим Грином в первой главе, показывают, что вряд ли на территории Калахари можно ожидать остатков древних высоких городских цивилизаций. Но следов цивилизаций более молодых исключать нельзя. В относительно близком соседстве с Калахари обнаружены развалины средневековых городов, среди которых наибольшей известностью пользуется Зимбабве, основанный в шестом веке нашей эры. Возможно, что на торговых путях через Калахари к западному побережью Южной Африки в погибших оазисах в средние века и существовали перевалочные пункты, иногда городского типа. И кажется вероятным, что город, обнаруженный американцем Фарини, впоследствии был погребен песками.

Зато безусловно следует ожидать большого количества археологических находок в Сахаре, где Анри Лот обнаружил следы крито-микенцев и где всего лишь несколько тысячелетий назад была саванна и даже леса и на протяжении многих сотен тысяч лет жил человек.

Особо следует остановиться на антропологических сведениях и проблемах, затрагиваемых Лоуренсом Грином в главах <<Властелины пустыни>>, <<Гиганты и пигмеи>>, <<Цыгане Нила>>. Здесь, пожалуй, в наибольшей степени сказался его африканский патриотизм, любительский подход к проблемам происхождения народов и психологическая ограниченность белого уроженца Южной Африки.

Несколько слов, прежде всего, следует сказать о происхождении пигмеев. Грин правильно отмечает, что малый рост, <<худосочность>> малорослых народов~--- результат того, что в тяжелых условиях пустынь или дождевых заболоченных тропических лесов человеческий организм недоразвивается, вырастают слабосильные, малорослые люди. Однако он вряд ли ясно представляет себе истинные причины возникновения малорослых народов. Предки бушменов были люди нормального роста и заселяли большую часть Африки, так как их костные остатки находят по всей Восточной Африке, в Западной Африке, в бассейне Конго и, конечно, в Южной Африке. Несколько тысячелетий назад, когда в Сахаре саванны стали сменяться пустыней, в бушменскую Африку с севера вторглись народы, сформировавшиеся на побережье Средиземного моря. Потом здесь появились индо-малайские народы (об этом, как раз и свидетельствуют риджбеки), а из Западной Азии пришли племена, родственные арабам. Предки бушменов частично смешались с пришельцами, частично были истреблены, а некоторая их часть оказалась оттеснена в наименее удобные для жизни районы, где они из поколения в поколение жили в условиях постоянного недоедания, с трудом добывая себе средства существования. Результатом и было возникновение пигмоидных и пигмейских племен. Особенно ухудшилось их положение в эпоху вторжения в Африку арабов, а затем~--- с появлением европейских колонизаторов. Европейцы просто предприняли массовое уничтожение местного населения. Описание варварства португальцев и немцев прекрасно сделано Грином. Автор, правда, умалчивает, что англичане поступали ничуть не лучше, однако он все же говорит о <<войне>> некоторых белых фермеров с бушменами, когда фермеры запрещают исконным владельцам земли охотиться на территории ферм. В таких строках ясно выступает свободолюбивый дух бушменов.

Очень интересны сведения, приводимые Лоуренсом Грином о высокорослых племенах Африки, полезен для читателя рассказ о существующих гипотезах происхождения цыган. Но особенно ценны, хотя и весьма скудные, данные о наличии среди коренных жителей долгоживу-щих людей: до сих пор в мировой литературе такие данные почти не публиковались, в то время как на других континентах ведется учет долгожителей.

Любопытны главы, посвященные животному миру Африки. Проблемы миграций газелей и птиц, как и других животных, до сих пор до конца не разрешены наукой. Все же научные данные более определенны, чем это представляется Грину. Миграции любых животных вызываются наступлением неблагоприятных условий для их жизни, причем, безусловно, вызываются они ухудшением не одного какого-либо фактора среды, а их совокупностью. Отсюда и кажущееся несоответствие некоторых гипотез, учитывающих только одну какую-либо причину. Здесь наиболее обнаженно проявляются законы диалектики. Что касается причин перелетов птиц, то давно наука предположила, что перелеты направлены в области, где некогда сформировался данный род или вид. Исследования последних лет с большой определенностью установили, что при перелетах многие виды птиц ориентируются по солнцу или по звездам, в зависимости от того~--- днем или ночью они находятся в полете. Опыты с малиновками в ГДР также показали, что <<знание>> птицами звездного неба передается по наследству и здесь мы встречаемся с генетической родовой памятью. В то же время кажется, что некоторые птицы руководствуются направлением геомагнитных линий и обладают своеобразным <<шестым чувством>>. Очень возможно, что низшие птицы, близкие, например, куриным, руководствуются магнитными полями, а высшие, вроде врано-вых,~--- астрономическими ориентирами.

Лоуренс Грин, по-видимому, человек глубоко верующий, поэтому он скептически относится к теории эволюции Дарвина. Отсюда и появляются нотки иронии, когда он говорит о причинах появления у жираф высоких передних ног и длинных шей. Этому способствует ошибочное представление многих биологов о том, что жирафовые всегда были жителями саванн. На самом деле жирафовые возникли около 20 млн. лет назад в тропических дождевых лесах Индии и не менее 10 млн. лет оставались обитателями постепенно разреживающихся этих лесов. Для тропических дождевых лесов характерно, что почва их~--- совершенно голая, заболоченная, пересеченная досковидными корнями, для передвижения там удобны длинные ноги; в то же время стволы деревьев голы, листва начинается высоко. В таких условиях и могли возникнуть такие формы, как жирафы. Сильное сокращение площади тропических лесов около 7~-- 10~млн.~лет назад, замена их на огромных пространствах саванной, заставило жирафовых приспособиться к новой обстановке: вымереть они не могли, так как грубой растительной пищи в саванне достаточно.

Вполне соответствует новейшим данным характеристика горилл, как ближайших к человеку ныне живущих человекоподобных обезьян. Согласно исследованиям последних лет, продолжительность жизни горилл - 80~-- 100 лет, их детеныши, как и у людей, достигают полной зрелости к 14 (у самок) - 18 (самцы горных горилл) годам; продолжительность беременности самок~--- такая же как у человека; более того, в составе хромосом горилл имеются хромосомы, типичные для человека; максимальный объем мозга горилл (752~см$^3$) лишь немногим меньше минимального объема мозга древнейших питекантропов (775~см$^3$), а их сообразительность не только не ниже, но даже выше, чем у шимпанзе. К этому можно добавить, что у представителей переходного между обезьянами и человеком звена, остатки которых найдены на о. Яве и в Восточной Африке (<<Homo habilis>> Луиса Лики), установлено очень много гориллоидных черт. Если сопоставить со всем этим факт, что до сих пор найдены ископаемые третичные остатки всех человекоподобных, кроме горилл, то вполне логичным будет заключение о том, что гориллы~--- это деградировавшая и вторично специализированная ветвь переходного между обезьяной и человеком звена. Впрочем, устарела характеристика горилл, как злобных и опасных животных. Уже после выхода в свет книги Грина опубликовали свои наблюдения зоологи Шаллер и Эмлен. Шаллер почти год вел наблюдения за горными гориллами, порой даже ночуя вместе с ними. Оказалось, что эти сильнейшие из сухопутных животных очень мягки и миролюбивы~--- не следует только вести себя вызывающе и враждебно. Их <<свирепость>>~--- всего лишь защитная реакция. Книга Лоуренса Грина может представить интерес н.е только для широкого читателя, некоторые ее страницы небезынтересно будет прочесть и научным работникам (например, данные о долгоживущих африканцах).

\cleardoublepage
\vspace*{3cm}

\begin{center}
\noindent \textbf{Old Africa's last secrets\\
by Lawrence G. Green\\
London, 1961}

\vspace{2cm}

\noindent Перевод с английского К.~Гришечкина и С.~Картузова\vspace{0.5cm}

\noindent Послесловие Ю.~К.~Назаркина и Ю.~Г.~Решетова\vspace{0.5cm}

\noindent В подготовке книги к изданию принимали участие Институт этнографии и Институт Африки Академии наук СССР\vspace{0.3cm}

\noindent Книга печатается с некоторыми сокращениями\vspace{0.3cm}

\noindent Художник Н.~И.~Гришин\vspace{1cm}\end{center}

\noindent2-8-3\\
238-66\vspace{2cm}

\begin{center}\small\noindent Редактор Л.~А.~Деревянкина

\noindent Младший редактор В.~А.~Мартынова

\noindent Художественный редактор А.~Г.~Шикин

\noindent Технический редактор В.~Н.~Корнилова

\noindent Корректоры Л.~А.~Рубина, Е.~Д.~Левина\vspace{1cm}

\noindent Сдано в набор 2 июля 1966 г. Подписано в печать 18 октября 1966 г. Формат бумаги 84х1081/32, № 2. Бумажных листов 4,25. Печатных листов 14,28. Учетно-издательских листов 14,11. Тираж 30 000 экз. Цена 60 коп. Заказ № 442. Темплан 1966 г.~--- № 238. Издательство <<Мысль>>. Москва, В-71, Ленинский проспект, 15. <<Мысль>>. 1966. 269 с, с илл.\vspace{2cm}

\noindent Книжная фабрика № 1 Росглавполиграфпрома Комитета по печати при Совете Министров РСФСР, г. Электросталь, Московской обл., Школьная, 25.

\end{center}
\end{document}




